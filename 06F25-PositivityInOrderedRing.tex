\documentclass[12pt]{article}
\usepackage{pmmeta}
\pmcanonicalname{PositivityInOrderedRing}
\pmcreated{2013-03-22 14:46:40}
\pmmodified{2013-03-22 14:46:40}
\pmowner{pahio}{2872}
\pmmodifier{pahio}{2872}
\pmtitle{positivity in ordered ring}
\pmrecord{12}{36424}
\pmprivacy{1}
\pmauthor{pahio}{2872}
\pmtype{Theorem}
\pmcomment{trigger rebuild}
\pmclassification{msc}{06F25}
\pmclassification{msc}{12J15}
\pmclassification{msc}{13J25}
\pmrelated{PositiveCone}
\pmrelated{TopicEntryOnRealNumbers}

% this is the default PlanetMath preamble.  as your knowledge
% of TeX increases, you will probably want to edit this, but
% it should be fine as is for beginners.

% almost certainly you want these
\usepackage{amssymb}
\usepackage{amsmath}
\usepackage{amsfonts}

% used for TeXing text within eps files
%\usepackage{psfrag}
% need this for including graphics (\includegraphics)
%\usepackage{graphicx}
% for neatly defining theorems and propositions
 \usepackage{amsthm}
% making logically defined graphics
%%%\usepackage{xypic}

% there are many more packages, add them here as you need them

% define commands here
\theoremstyle{definition}
\newtheorem*{thmplain}{Theorem}
\begin{document}
\begin{thmplain}
\, If\, $(R,\,\leq)$\, is an ordered ring, then it contains a subset $R_+$ having the following \PMlinkescapetext{properties}:
\begin{itemize}
 \item $R_+$ is \PMlinkescapetext{closed} under ring addition and, supposing that the ring contains no zero divisors, also under ring multiplication.
 \item Every element $r$ of $R$ satisfies exactly one of the conditions\,\, 
$(1)\,\,r = 0$,\,\,\, $(2)\,\,r\in R_+$,\,\,\, $(3)\,\,-r\in R_+$.
\end{itemize}
\end{thmplain}

{\em Proof.}\, We take\, $R_+ = \{r\in R:\,\, 0 < r\} 
= \{r\in R:\,\, 0\leq r\, \wedge \,0 \neq r\}$.\, Let\, $a,\,b \in R_+$.\, Then \, $0 < a$,\, $0 < b$, and therefore we have\, $0 < a\!+\!0 < a\!+\!b$,\, i.e.\, $a\!+\!b \in R_+$.\, If $R$ has no zero-divisors, then also\, $ab \neq 0$\, and\, $0 = a0 < ab$,\, i.e.\, $ab\in R_+$.\, Let $r$ be an arbitrary non-zero element of $R$.\, Then we must have either\, $0 < r$\, or\, $r < 0$\, (not both) because $R$ is totally ordered.\, The latter alternative gives that\, $0 = -r\!+\!r < -r\!+\!0 = -r$.\, The both cases \PMlinkescapetext{mean} that either\, $r\in R_+$\, or\, $-r\in R_+$.
%%%%%
%%%%%
\end{document}
