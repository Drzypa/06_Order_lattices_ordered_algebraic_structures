\documentclass[12pt]{article}
\usepackage{pmmeta}
\pmcanonicalname{BirkhoffRecurrenceTheorem}
\pmcreated{2015-03-20 0:56:48}
\pmmodified{2015-03-20 0:56:48}
\pmowner{Filipe}{28191}
\pmmodifier{Filipe}{28191}
\pmtitle{Birkhoff Recurrence Theorem}
\pmrecord{2}{88234}
\pmprivacy{1}
\pmauthor{Filipe}{28191}
\pmtype{Theorem}
%\pmkeywords{Recurrence; Dynamical Systems}
\pmrelated{Poincaré Recurrence Theorem}

\endmetadata

% this is the default PlanetMath preamble.  as your knowledge
% of TeX increases, you will probably want to edit this, but
% it should be fine as is for beginners.

% almost certainly you want these
\usepackage{amssymb}
\usepackage{amsmath}
\usepackage{amsfonts}

% need this for including graphics (\includegraphics)
\usepackage{graphicx}
% for neatly defining theorems and propositions
\usepackage{amsthm}

% making logically defined graphics
%\usepackage{xypic}
% used for TeXing text within eps files
%\usepackage{psfrag}

% there are many more packages, add them here as you need them

% define commands here

\begin{document}
Let $T:X\rightarrow X$ be a continuous tranformation in a compact metric space $X$. Then, there exists some point $x \in X$ that is recurrent to $T$, that is, there exists a sequence $(n_k)_k$ such that $T^{n_k}(x)\rightarrow x$ when $k\rightarrow \infty$.

Several proofs of this theorem are available. It may be obtained from topological arguments together with Zorn's lemma. It is also a consequence of Krylov-Bogolyubov theorem, or existence of invariant probability measures theorem, which asserts that every continuous transformation in a compact metric space admits an invariant probability measure, and an application of Poincaré Recurrence theorem to that invariant probability measure yields Birkhoff Recurrence theorem.

There is also a generalization of Birkhoff recurrence theorem for multiple commuting transformations, known as Birkhoff Multiple Recurrence theorem.
\end{document}
