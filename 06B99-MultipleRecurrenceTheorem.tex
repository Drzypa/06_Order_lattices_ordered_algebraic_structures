\documentclass[12pt]{article}
\usepackage{pmmeta}
\pmcanonicalname{MultipleRecurrenceTheorem}
\pmcreated{2015-03-20 0:29:34}
\pmmodified{2015-03-20 0:29:34}
\pmowner{Filipe}{28191}
\pmmodifier{Filipe}{28191}
\pmtitle{Multiple Recurrence Theorem}
\pmrecord{1}{}
\pmprivacy{1}
\pmauthor{Filipe}{28191}
\pmtype{Theorem}
\pmsynonym{Poincaré Multiple Recurrence Theorem; Fürstenberg Recurrence theorem}{MultipleRecurrenceTheorem}
%\pmkeywords{Recurrence}
\pmrelated{Poincaré Recurrence Theorem}

\endmetadata

% this is the default PlanetMath preamble.  as your knowledge
% of TeX increases, you will probably want to edit this, but
% it should be fine as is for beginners.

% almost certainly you want these
\usepackage{amssymb}
\usepackage{amsmath}
\usepackage{amsfonts}

% need this for including graphics (\includegraphics)
\usepackage{graphicx}
% for neatly defining theorems and propositions
\usepackage{amsthm}

% making logically defined graphics
%\usepackage{xypic}
% used for TeXing text within eps files
%\usepackage{psfrag}

% there are many more packages, add them here as you need them

% define commands here

\begin{document}
Let $(X, \mathcal{B}, \mu)$ be a probability space, and let $T_i:X \rightarrow X$ be measure-preserving transformations, for $i$ between $1$ and $q$. Assume that all the transformations $T_i$ commute. If $E\subset X$ is a positive measure set $\mu(E)>0$, then, there exists $n \in \mathbb{N}$ such that 
$$\mu (E \cap T_1^{-n}(E) \cap \cdots \cap T_q^{-n}(E))>0$$
In other words there exist a certain time $n$ such that the subset of $E$ for which all elements return to $E$ simultaneously for all transformations $T_i$ is a subset of $E$ with positive measure.
Observe that the theorem may be applied again to the set $G=E \cap T_1^{-n}(E) \cap \cdots \cap T_q^{-n}(E)$, obtaining the existence of $m\in \mathbb{N}$ such that
$$\mu (G \cap T_1^{-m}(G) \cap \cdots \cap T_q^{-m}(G))>0$$
so that $$\mu (E \cap T_1^{-(m+n)}(E) \cap \cdots \cap T_q^{-(m+n)}(E))\geq \mu (G \cap T_1^{-m}(G) \cap \cdots \cap T_q^{-m}(G)) >0$$
So we may conclude that, when $E$ has positive measure, there are infinite times for which there is a simultaneous return for a subset of $E$ with positive measure.

As a corollary, since the powers $T,T^2 \cdots T^q$ of a transformation $T$ commute, we have that, for $E$ with positive measure there exists $n \in \mathbb{N}$ such that
$$\mu (E\cap T^{-n}(E) \cap \cdots \cap T^{-qn}(E))>0$$ 
As a consequence of the multiple recurrence theorem one may prove Szemerédi's Theorem about arithmetic progressions.
\end{document}
