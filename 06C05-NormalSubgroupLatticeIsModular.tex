\documentclass[12pt]{article}
\usepackage{pmmeta}
\pmcanonicalname{NormalSubgroupLatticeIsModular}
\pmcreated{2013-03-22 15:50:33}
\pmmodified{2013-03-22 15:50:33}
\pmowner{CWoo}{3771}
\pmmodifier{CWoo}{3771}
\pmtitle{normal subgroup lattice is modular}
\pmrecord{9}{37821}
\pmprivacy{1}
\pmauthor{CWoo}{3771}
\pmtype{Derivation}
\pmcomment{trigger rebuild}
\pmclassification{msc}{06C05}
\pmclassification{msc}{20E25}

\endmetadata

\usepackage{amssymb,amscd}
\usepackage{amsmath}
\usepackage{amsfonts}

% used for TeXing text within eps files
%\usepackage{psfrag}
% need this for including graphics (\includegraphics)
%\usepackage{graphicx}
% for neatly defining theorems and propositions
\usepackage{amsthm}
% making logically defined graphics
%%%\usepackage{xypic}

% define commands here
\begin{document}
The fact that the normal subgroups of a group $G$ form a lattice (call it $N(G)$) is proved \PMlinkname{here}{NormalSubgroupsFormSublatticeOfASubgroupLattice}.  The only remaining item is to show that $N(G)$ is \PMlinkname{modular}{ModularLattice}.  This means, for any normal subgroups $H,K,L$ of $G$ such that $L\subseteq K$, 
$$L\vee(H\wedge K)=(L\vee H)\wedge K,$$
where the meet operation $A\wedge B$ denotes set intersection $A\cap B$, and the join operation $A \vee B$ denotes the subgroup generated by $A\cup B$.
\begin{proof}
First, we show that $L\vee(H\wedge K)\subseteq (L\vee H)\wedge K$.  It is easy to see that 
\begin{enumerate}
\item $L\subseteq K\wedge(L\vee H)$: $L\subseteq K$ is assumed and $L\subseteq L\vee H$ follows from the definition of $\vee$, and
\item $H\wedge K \subseteq K\wedge(L\vee H)$: $H\wedge K\subseteq K$ follows from the definition of $\wedge$, and $H\wedge K\subseteq H\subseteq L\vee H$.  
\end{enumerate}
As a result, $L\vee(H\wedge K)\subseteq K\wedge(L\vee H)=(L\vee H)\wedge K$.

Before proving the other inclusion, we shall derive a small lemma concerning $L\vee H$ where $L,H$ are normal subgroups of $G$:
$$L\vee H=\lbrace \ell h\mid \ell\in L\mbox{ and }h\in H\rbrace.$$
\begin{proof}  One direction is obvious, so we will just show $L\vee H \subseteq \lbrace \ell h\mid \ell\in L\mbox{ and }h\in H\rbrace$.  Any element of $L\vee H$ can be expressed as a finite product of elements from $L$ or $H$.  This finite product representation can be reduced so that no two adjacent elements belong to the same group.  Next, $h\ell = (h\ell h^{-1})h$, where $h\ell h^{-1}\in L$ since $L$ is normal, showing that an $\ell$ on the right of the product can be ``filtered'' to the left of the product (of course, this ``filtering'' changes $\ell$ to another element of $L$, but it is the form of the product, not the elements in the product, that we are interested in).  This implies that the finite product representation can further be reduced (by an induction argument) so it has the final form $\ell h$.
\end{proof}

Now back to the main proof.  Take any $g\in (L\vee H)\wedge K$.  Then $g\in K$ and $g\in L\vee H$ and so $g=\ell h$ for some $\ell\in L$ and $h\in H$ by the lemma just shown.  Since $g\in K$, this means $h=\ell^{-1}g\in LK\subseteq K$.  So $h\in H\wedge K$.  We have just expressed $g$ as a product of $\ell\in L$ and $h\in H\wedge K$, and so $g\in L\vee (H\wedge K)$.
\end{proof}
%%%%%
%%%%%
\end{document}
