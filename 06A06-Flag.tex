\documentclass[12pt]{article}
\usepackage{pmmeta}
\pmcanonicalname{Flag}
\pmcreated{2013-03-22 12:42:35}
\pmmodified{2013-03-22 12:42:35}
\pmowner{rmilson}{146}
\pmmodifier{rmilson}{146}
\pmtitle{flag}
\pmrecord{9}{32994}
\pmprivacy{1}
\pmauthor{rmilson}{146}
\pmtype{Definition}
\pmcomment{trigger rebuild}
\pmclassification{msc}{06A06}
\pmclassification{msc}{15A03}
\pmdefines{adapted basis}
\pmdefines{complete flag}

\endmetadata

\usepackage{amsmath}
\usepackage{amsfonts}
\usepackage{amssymb}
\newcommand{\reals}{\mathbb{R}}
\newcommand{\natnums}{\mathbb{N}}
\newcommand{\cnums}{\mathbb{C}}
\newcommand{\znums}{\mathbb{Z}}
\newcommand{\lp}{\left(}
\newcommand{\rp}{\right)}
\newcommand{\lb}{\left[}
\newcommand{\rb}{\right]}
\newcommand{\supth}{^{\text{th}}}
\newtheorem{proposition}{Proposition}
\newtheorem{definition}[proposition]{Definition}
\newcommand{\nl}[1]{\PMlinkescapetext{{#1}}}
\newcommand{\pln}[2]{\PMlinkname{#1}{#2}}
\begin{document}
Let $V$ be a finite-dimensional vector space.  A filtration of
subspaces
$$V_1\subset V_2\subset\cdots \subset V_n= V$$
is called a \emph{flag} in $V$.
We speak of a {\em complete flag} when
$$\dim V_i = i$$
for each $i=1,\ldots,n$.

Next, putting
$$d_k = \dim V_k,\quad k=1,\ldots n,$$
we say that a list of vectors
$(u_1,\ldots,u_{d_n})$ is an \emph{adapted basis} relative to the flag, if
the first $d_1$ vectors give a basis of $V_1$, the first $d_2$ vectors
give a basis of $V_2$, etc.  Thus, an alternate characterization of a
complete flag, is that the first $k$ elements of an adapted basis are
a basis of $V_k$.

\paragraph{Example}
Let us consider $\reals^n$.  For each $k=1,\ldots,n$ let $V_k$ be the
span of $e_1,\ldots,e_k$, where $e_j$ denotes the $j\supth$ basic
vector, i.e. the column vector with $1$ in the $j\supth$ position and
zeros everywhere else.  The $V_k$ give a complete flag in $\reals^n$ .
The list $(e_1,e_2,\ldots, e_n)$ is an adapted basis relative to this
flag, but the list $(e_2,e_1,\ldots,e_n)$ is not.

\paragraph{Generalizations.}
More generally, a flag can be defined as a maximal chain in a partially ordered set.   If one considers the poset consisting of subspaces of a (finite dimensional) vector space, one recovers the definition given above.

%%%%%
%%%%%
\end{document}
