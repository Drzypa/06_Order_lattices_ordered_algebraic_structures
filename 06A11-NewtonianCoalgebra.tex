\documentclass[12pt]{article}
\usepackage{pmmeta}
\pmcanonicalname{NewtonianCoalgebra}
\pmcreated{2013-03-22 16:48:36}
\pmmodified{2013-03-22 16:48:36}
\pmowner{mps}{409}
\pmmodifier{mps}{409}
\pmtitle{Newtonian coalgebra}
\pmrecord{6}{39044}
\pmprivacy{1}
\pmauthor{mps}{409}
\pmtype{Definition}
\pmcomment{trigger rebuild}
\pmclassification{msc}{06A11}
\pmclassification{msc}{16W30}

% this is the default PlanetMath preamble.  as your knowledge
% of TeX increases, you will probably want to edit this, but
% it should be fine as is for beginners.

% almost certainly you want these
\usepackage{amssymb}
\usepackage{amsmath}
\usepackage{amsfonts}

% used for TeXing text within eps files
%\usepackage{psfrag}
% need this for including graphics (\includegraphics)
%\usepackage{graphicx}
% for neatly defining theorems and propositions
%\usepackage{amsthm}
% making logically defined graphics
%%%\usepackage{xypic}

% there are many more packages, add them here as you need them

% define commands here

\begin{document}
Let $R$ be a commutative ring.  A \emph{Newtonian coalgebra} over $R$ is an $R$-module $C$ which is simultaneously a coalgebra with comultiplication $\Delta\colon C\to C\otimes C$ and an algebra with multiplication $\cdot\colon C\otimes C\to C$ such that $\Delta$ is a derivation over $\cdot$, that is, such that the identity
\[
\Delta(u\cdot v) = \Delta(u)\cdot v + u\cdot\Delta(v)
\]
holds for any $u$ and $v$ in $C$.  Newtonian coalgebras were introduced by Joni and Rota in~\cite{JoRo}, where they were called \emph{infinitesimal coalgebras}.     They reserved the term ``Newtonian coalgebra'' for the special case of the coalgebra of divided differences.  This example was studied in more detail by Hirschhorn and Raphael~\cite{HiRa}.  Joni and Rota also showed that Newtonian coalgebras provide a language which can explain iterated differentiation of trigonometric functions as well as Fa\`{a} di Bruno's formula.  See also the paper of Nichols and Sweedler~\cite{NiSw} for more on trigonometric coalgebras.

A Newtonian coalgebra cannot have both a unit and a counit, so no Newtonian coalgebra is a Hopf algebra.  However, Aguiar~\cite{Ag} developed a notion of antipode that makes sense for Newtonian coalgebras, leading to what he calls an infinitesimal Hopf algebra.  Ehrenborg and Readdy~\cite{EhRe_coproducts} used Newtonian coalgebras to give an algebraic structure to the \PMlinkname{$\mathbf{cd}$-index}{CdIndex}, a poset invariant generalizing the $f$-vector of polytopes.

One example of a Newtonian coalgebra is the free associative algebra $R\langle\mathbf{a},\mathbf{b}\rangle$ of polynomials on the noncommuting variables $\mathbf{a}$ and $\mathbf{b}$ with coefficients in $R$.  The product is the ordinary noncommutative polynomial product, and the comultiplication is defined by setting
\[
\Delta(u_1\cdots u_n) = \sum_{j\in[n]} u_1\cdots u_{i-1}\otimes u_{i+1}\cdots u_n
\]
for each monomial and extending by linearity.

% I've commented out some of the references because I don't yet refer to
% them, and right now the bibliography is longer than the rest of the 
% entry.
%
\begin{thebibliography}{99}
\bibitem{Ag}
M.~Aguiar, Infinitesimal Hopf algebras.  {\it New trends in Hopf algebra theory: (La Falda, 1999)},  1--29, Contemp.~Math., 267, Amer.~Math.~Soc., Providence, RI, 2000. 

\bibitem{Ag2}
M.~Aguiar, Infinitesimal Hopf algebras and the $\mathbf{cd}$-index of polytopes.  {\it Discrete~Comput.~Geom.}, 27 (2002), no. 1, 3--28. 

%\bibitem{BaBi}
%M.~Bayer and L.~Billera, Generalized Dehn-Sommerville relations for polytopes, %spheres and Eulerian partially ordered sets.  {\it Invent.~Math.}, 79 (1985), %143--157.

%\bibitem{BaKl}
%M.~Bayer and A.~Klapper, A new index for polytopes,  {\it Discrete %Comput.~Geom.}, 6 (1991), 33--47.

\bibitem{EhRe_coproducts}
R.~Ehrenborg and M.~Readdy, Coproducts and the $\mathbf{cd}$-index,  {\it J.~Algebr.~Comb.}, 8 (1998), 273--299.

%\bibitem{EhRe_homology}
%R.~Ehrenborg and M.~Readdy, Homology of Newtonian coalgebras.  {\it European 
%J.~Combin.} 23 (2002), no. 8, 919--927.

\bibitem{HiRa}
P.~S.~Hirschhorn and L.~A.~Raphael, Coalgebraic foundation of the method of divided differences,  {\it Adv.~Math.}, 91 (1992), 75--135.

\bibitem{JoRo}
S.~A.~Joni and G.-C.~Rota, {\it Coalgebras and bialgebras in combinatorics}, Stud.~Appl.~Math., 61 (1979), pp. 93--139.

\bibitem{NiSw}
W.~Nichols and M.~Sweedler, {\it Hopf algebras and combinatorics}, in {\it Proceedings of the conference on umbral calculus and Hopf algebras}, ed. R.~Morris, AMS, 1982.

%\bibitem{Sw}
%M.~Sweedler, {\it Hopf algebras}, Benjamin, New York, 1969.
\end{thebibliography}

% bleh
\PMlinkescapeword{formula}
\PMlinkescapeword{identity}
\PMlinkescapeword{infinitesimal}
\PMlinkescapeword{language}
\PMlinkescapeword{term}
%%%%%
%%%%%
\end{document}
