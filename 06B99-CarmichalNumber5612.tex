\documentclass[12pt]{article}
\usepackage{pmmeta}
\pmcanonicalname{CarmichalNumber561}
\pmcreated{2015-07-12 6:12:20}
\pmmodified{2015-07-12 6:12:20}
\pmowner{akdevaraj}{13230}
\pmmodifier{akdevaraj}{13230}
\pmtitle{Carmichal  number   561}
\pmrecord{1}{}
\pmprivacy{1}
\pmauthor{akdevaraj}{13230}
\pmtype{Definition}

\endmetadata

% this is the default PlanetMath preamble.  as your knowledge
% of TeX increases, you will probably want to edit this, but
% it should be fine as is for beginners.

% almost certainly you want these
\usepackage{amssymb}
\usepackage{amsmath}
\usepackage{amsfonts}

% need this for including graphics (\includegraphics)
\usepackage{graphicx}
% for neatly defining theorems and propositions
\usepackage{amsthm}

% making logically defined graphics
%\usepackage{xypic}
% used for TeXing text within eps files
%\usepackage{psfrag}

% there are many more packages, add them here as you need them

% define commands here

\begin{document}
561  is  the  smallest  Carmichael  number  in  the  ring of
integers. However  it is  only  a  pseudoprime  in the ring
of  Gaussian  integers.  We  can  use  pari to  find
the  bases  to  which  561  is  a  pseudoprime.  I  
found it  is   pseudoprime to bases  12  +  i,  22  +i,
33  +  i  and  44  +  i.  Needless  to  say  it is  a
pseudoprime to  other  bases  too. More  on  this later.
\end{document}
