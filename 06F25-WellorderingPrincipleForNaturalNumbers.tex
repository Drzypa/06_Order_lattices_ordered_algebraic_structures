\documentclass[12pt]{article}
\usepackage{pmmeta}
\pmcanonicalname{WellorderingPrincipleForNaturalNumbers}
\pmcreated{2013-03-22 11:46:38}
\pmmodified{2013-03-22 11:46:38}
\pmowner{CWoo}{3771}
\pmmodifier{CWoo}{3771}
\pmtitle{well-ordering principle for natural numbers}
\pmrecord{18}{30244}
\pmprivacy{1}
\pmauthor{CWoo}{3771}
\pmtype{Axiom}
\pmcomment{trigger rebuild}
\pmclassification{msc}{06F25}
\pmclassification{msc}{65A05}
\pmclassification{msc}{11Y70}
\pmrelated{MaximalityPrinciple}
\pmrelated{WellOrderedSet}
\pmrelated{ExistenceAndUniquenessOfTheGcdOfTwoIntegers}

\endmetadata

\usepackage{amssymb}
\usepackage{amsmath}
\usepackage{amsfonts}
\usepackage{graphicx}
%%%%\usepackage{xypic}
\begin{document}
\PMlinkescapeword{equivalent}
Every nonempty set $S$ of natural numbers contains a least element; that is, there is some number $a$ in $S$ such that $a \leq b$ for all $b$ belonging to $S$.\\

Beware that there is another statement (which is equivalent to the axiom of choice) called the \emph{well-ordering principle}. It asserts that every set can be well-ordered.

Note that the well-ordering principle for natural numbers is equivalent to the principle of mathematical induction (or, the principle of finite induction).
%%%%%
%%%%%
%%%%%
%%%%%
\end{document}
