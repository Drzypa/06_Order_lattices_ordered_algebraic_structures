\documentclass[12pt]{article}
\usepackage{pmmeta}
\pmcanonicalname{Fence}
\pmcreated{2013-03-22 17:08:00}
\pmmodified{2013-03-22 17:08:00}
\pmowner{CWoo}{3771}
\pmmodifier{CWoo}{3771}
\pmtitle{fence}
\pmrecord{7}{39439}
\pmprivacy{1}
\pmauthor{CWoo}{3771}
\pmtype{Definition}
\pmcomment{trigger rebuild}
\pmclassification{msc}{06A06}
\pmdefines{even fence}
\pmdefines{odd fence}

\endmetadata

\usepackage{amssymb,amscd}
\usepackage{amsmath}
\usepackage{amsfonts}
\usepackage{mathrsfs}

% used for TeXing text within eps files
%\usepackage{psfrag}
% need this for including graphics (\includegraphics)
%\usepackage{graphicx}
% for neatly defining theorems and propositions
\usepackage{amsthm}
% making logically defined graphics
%%\usepackage{xypic}
\usepackage{pst-plot}
\usepackage{psfrag}

% define commands here
\newtheorem{prop}{Proposition}
\newtheorem{thm}{Theorem}
\newtheorem{ex}{Example}
\newcommand{\real}{\mathbb{R}}
\newcommand{\pdiff}[2]{\frac{\partial #1}{\partial #2}}
\newcommand{\mpdiff}[3]{\frac{\partial^#1 #2}{\partial #3^#1}}
\begin{document}
A finite poset is said to be a \emph{fence} if it has a Hasse diagram that looks like one of the following four types:
\begin{enumerate}
\item
\begin{equation*}
\xymatrix @C-=2pt @R=6pt {
& & b_1 \ar@{-}[lldd] \ar@{-}[rrdd] & & & & b_2 \ar@{-}[lldd] \ar@{-}[rd] \\
& & & & & & & \\
a_1 & & & & a_2 & & }
\xymatrix @C-=2pt @R=6pt  { {} \\ {\cdots} \\ }
\xymatrix @C-=2pt @R=6pt  {
& & & b_{m-1} \ar@{-}[lldd] \ar@{-}[rrdd] & & & & b_m \ar@{-}[lldd] \\
\ar@{-}[rd] & & & & & & & & \\
& a_{m-1} & & & & a_m & & }
\end{equation*}
\item
\begin{equation*}
\xymatrix @C-=2pt @R=6pt {
b_1 \ar@{-}[rrdd] & & & & b_2 \ar@{-}[lldd] \ar@{-}[rrdd] & & \\
& & & & & & & \ar@{-}[ld] \\
& & a_1 & & & & a_2 }
\xymatrix @C-=2pt @R=6pt  { {} \\ {\cdots} \\ }
\xymatrix @C-=2pt @R=6pt  {
& b_{n-1} \ar@{-}[ld] \ar@{-}[rrdd] & & & & b_n \ar@{-}[lldd] \ar@{-}[rrdd] & & \\
& & & & & & & & \\
& & & a_{n-1} & & & & a_n }
\end{equation*}
\item
\begin{equation*}
\xymatrix @C-=2pt @R=6pt {
& & b_1 \ar@{-}[lldd] \ar@{-}[rrdd] & & & & b_2 \ar@{-}[lldd] \ar@{-}[rd] \\
& & & & & & & \\
a_1 & & & & a_2 & & }
\xymatrix @C-=2pt @R=6pt  { {} \\ {\cdots} \\ }
\xymatrix @C-=2pt @R=6pt  {
& & & b_{p-1} \ar@{-}[lldd] \ar@{-}[rrdd] & & & & b_p \ar@{-}[lldd] \ar@{-}[rrdd] & & \\
\ar@{-}[rd] & & & & & & & & & & \\
& a_{p-1} & & & & a_p & & & & a_{p+1}}
\end{equation*}
\item
\begin{equation*}
\xymatrix @C-=2pt @R=6pt {
b_1 \ar@{-}[rrdd] & & & & b_2 \ar@{-}[lldd] \ar@{-}[rrdd] & & \\
& & & & & & & \ar@{-}[ld] \\
& & a_1 & & & & a_2 }
\xymatrix @C-=2pt @R=6pt  { {} \\ {\cdots} \\ }
\xymatrix @C-=2pt @R=6pt  {
& b_{q-1} \ar@{-}[ld] \ar@{-}[rrdd] & & & & b_q \ar@{-}[lldd] \ar@{-}[rrdd] & & & & b_{q+1} \ar@{-}[lldd] \\
& & & & & & & & & & \\
& & & a_{q-1} & & & & a_q & &}
\end{equation*}
\end{enumerate}
where $m,n,p,q$ are positive integers.  

When no pairs of $m,n,p,q$ are the same, no two types are order isomorphic.  A fence of type 1 or type 2 are called an \emph{even fence}, for the obvious reason that it contains an even number of elements.  The other two types are the \emph{odd fences}.  

When $m=n$, the two even fences are isomorphic, simply by mapping $c_i$ in the first fence to $c_{n-i}$ in the second one, where $c$ is either $a$ or $b$.  When $p=q$, the two odd fences are dually isomorphic to one another (the dual of one fence is isomorphic to the other fence).

Another property of a fence is that it is \PMlinkname{connected}{ConnectedPoset} and is minimal in the sense that, among all partial orderings on the underlying set, a fence has the least number of elements in its partial ordering.  Of course, the converse is not true, as the following example illustrates
\begin{equation*}
\xymatrix {
b_1 \ar@{-}[rrd] & b_2 \ar@{-}[rd] & \cdots & b_{n-1} \ar@{-}[ld] & b_n \ar@{-}[lld] \\
& & a_1 & & }
\end{equation*}

It can be shown that the number of lower sets in a fence is a Fibonacci number.  In other words, if $f(n)$ is the number of lower sets in a fence of cardinality $n$ ($n$ at least $2$), then $f(n+1)=f(n)+f(n-1)$ and $f(2)=2$.

\textbf{Remark}.  One may extend the definition of a fence to the infinitely countable case.  In this case, it has a Hasse diagram that looks like
\begin{equation*} 
\xymatrix { {} \ar@{}[d]_-{\displaystyle{\stackrel{}{\cdots}}} \\ {} } 
\xymatrix { 
\ar@{-}[rd] & & b_{n-1} \ar@{-}[rd] \ar@{-}[ld] & & b_n \ar@{-}[ld] \ar@{-}[rd] & & b_{n+1} \ar@{-}[ld] \ar@{-}[rd] & \\ 
& a_{n-1} & & a_n & & a_{n+1} & & } 
\xymatrix { {} \ar@{}[d]_-{\displaystyle{\stackrel{}{\cdots}}} \\ {} } 
\end{equation*} 
Its underlying set is the disjoint union of the two countably infinite sets $A=\lbrace a_i\mid i\in \mathbb{Z}\rbrace$ and $B=\lbrace b_i\mid i\in \mathbb{Z}\rbrace$, and its partial order is the disjoint union of the three sets $\lbrace (a_i,b_i)\mid i\in \mathbb{Z}\rbrace$, $\lbrace (a_i,b_{i-1})\mid i\in \mathbb{Z}\rbrace$, and $\lbrace (c,c)\mid c\in A\cup B\rbrace$.

\begin{thebibliography}{6}
\bibitem{dp} B. A. Davey, H. A. Priestley, {\it Introduction to Lattices and Order}, 2nd Edition, Cambridge (2003)
\end{thebibliography}
%%%%%
%%%%%
\end{document}
