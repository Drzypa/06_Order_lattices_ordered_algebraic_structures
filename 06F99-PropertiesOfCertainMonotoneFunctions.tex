\documentclass[12pt]{article}
\usepackage{pmmeta}
\pmcanonicalname{PropertiesOfCertainMonotoneFunctions}
\pmcreated{2013-03-22 19:03:31}
\pmmodified{2013-03-22 19:03:31}
\pmowner{CWoo}{3771}
\pmmodifier{CWoo}{3771}
\pmtitle{properties of certain monotone functions}
\pmrecord{9}{41942}
\pmprivacy{1}
\pmauthor{CWoo}{3771}
\pmtype{Result}
\pmcomment{trigger rebuild}
\pmclassification{msc}{06F99}
\pmclassification{msc}{08C99}
\pmclassification{msc}{08A99}

\usepackage{amssymb,amscd}
\usepackage{amsmath}
\usepackage{amsfonts}
\usepackage{mathrsfs}

% used for TeXing text within eps files
%\usepackage{psfrag}
% need this for including graphics (\includegraphics)
%\usepackage{graphicx}
% for neatly defining theorems and propositions
\usepackage{amsthm}
% making logically defined graphics
%%\usepackage{xypic}
\usepackage{pst-plot}

% define commands here
\newcommand*{\abs}[1]{\left\lvert #1\right\rvert}
\newtheorem{prop}{Proposition}
\newtheorem{thm}{Theorem}
\newtheorem{ex}{Example}
\newcommand{\real}{\mathbb{R}}
\newcommand{\pdiff}[2]{\frac{\partial #1}{\partial #2}}
\newcommand{\mpdiff}[3]{\frac{\partial^#1 #2}{\partial #3^#1}}
\begin{document}
In the definitions of some partially ordered algebraic systems such as po-groups and po-rings, the multiplication is set to be compatible with the partial ordering on the universe in the following sense:
$$ab \le ac \quad\mbox{iff}\quad b \le c \qquad\mbox{ and } \qquad ab \le cb \quad\mbox{iff}\quad a \le c$$
This is no coincidence.  In fact, these ``definitions'' are actually consequences of properties concerning monotone functions satisfying certain algebraic rules, which is the focus of this entry.

Recall that an $n$-ary function $f$ on a set $A$ is said to be monotone if it is monotone in each of its variables.  In other words, for every $i=1,2,\ldots,n$, the function $f(a_1,\ldots,a_{i-1},x,a_{i+1},\ldots,a_n)$ is monotone in $x$, where each of the $a_j$ is a fixed but arbitrary element of $A$.  We use the notation $\uparrow,\downarrow,\updownarrow$ to denote the monotonicity of each variable in $f$.  For example, $(\uparrow,\uparrow,\uparrow)$ denotes a ternary isotone function, whereas $(\downarrow,\updownarrow)$ denotes a binary function which is antitone with respect to its first variable, and both isotone/antitone with respect to the second.

\begin{prop}  Let $f$ be an $n$-ary commutative monotone operation on a set $A$.  Then $f$ is either isotone or antitone.
\end{prop}
\begin{proof}  Suppose $f$ is isotone (or antitone) in its first variable.  Since $f(x,a_1,\ldots,a_{n-1}) = f(a_1,x,\ldots, a_{n-1}) = \cdots = f(a_1,\ldots,a_{n-1},x)$, $f$ is isotone (or antitone) in each of its remaining variables.
\end{proof}

\begin{prop}  Let $f$ be an $n$-ary monotone operation on a set $A$ with an identity element $e\in A$.  In other words, $f(x,e,\ldots, e) = f(e,x,\ldots,e) = \cdots = f(e,e,\ldots, x)=x$.  Then $f$ is either strictly isotone or strictly antitone.
\end{prop}
\begin{proof}  The proof is the same as the one before.  Furthermore, if $f$ is isotone and $a < b$, then $f(a,e,\ldots,e)=a<b=f(b,e,\ldots,e)$, so the strict ordering is preserved.  The same holds true if $f$ is antitone.
\end{proof}

\begin{prop}  Let $f$ be a binary monotone operation on a set $A$ such that it is isotone (antitone) with respect to its first variable.  Suppose $g$ is a unary operation on $A$ such that $f(x,g(x))$ is a fixed element of $A$.  Then $g$ is antitone (isotone).
\end{prop}

\begin{prop}  Let $f$ be an $n$-ary associative monotone operation on a set $A$.  Then
\begin{itemize}
\item $f$ is isotone if $n$ is even
\item $f$ is either isotone, or is $(\underbrace{\uparrow,\ldots,\uparrow}_{m}, \downarrow,\underbrace{\uparrow,\ldots,\uparrow}_{m})$, if $n$ is odd, say $n=2m+1$.
\end{itemize}
\end{prop}
\begin{proof}
Suppose first that $n=2m$, $i\le m$, and $g(x)=f(a_1,\ldots, a_{i-1},x,\ldots, a_{m+1}, \ldots, a_{2m})$ is antitone.  Then $g(g(x))$ is isotone.  By the associativity of $f$, $g(g(x))$ is
\begin{eqnarray*}
&& f(a_1,\ldots, a_{i-1},f(a_1,\ldots, a_{i-1},x,a_{i+1},\ldots, a_m, \ldots, a_{2m}),\ldots, a_{m+1}, \ldots, a_{2m})\\
&=& f(a_1,\ldots, a_{i-1},a_1,\ldots, a_{i-1},x,a_{i+1},\ldots, f(a_{2m-i+1}, \ldots, a_{2m},a_{i+1},\ldots, a_{m+1}, \ldots, a_{2m})).
\end{eqnarray*}
In the second expression, the position of $x$ is $2i-2\le 2m-1 < 2m$, therefore implying that $g(g(x))$ is antitone, which is a contradiction!  Therefore, $g(x)$ is isotone.
Now, if $i>m$, and $h(x)=f(b_1,\ldots, b_m, \ldots, b_{i-1}, x, b_{i+1},\ldots, b_{2m})$ is antitone, then $h(h(x))$ is isotone.  But 
\begin{eqnarray*}
&& f(b_1,\ldots, b_m, \ldots, b_{i-1}, f(b_1,\ldots, b_m, \ldots, b_{i-1}, x, b_{i+1},\ldots, b_{2m}), b_{i+1}, \ldots, b_{2m})\\
&=& f(f(b_1,\ldots, b_m, \ldots, b_{i-1},b_1,\ldots, b_{2m-i+1}), \ldots, b_{i-1},x,b_{i+1},\ldots, b_{2m}, b_{i+1}, \ldots, b_{2m})),
\end{eqnarray*}
and the position of $x$ is the second expression is $(i-1)-(2m-i+1)+2=2i-2m>1$, therefore implying that $h(h(x))$ is antitone, again a contradiction.  As a result, $f$ is isotone for all $i=1,\ldots, n$.

The argument above also works when $n$ is odd, say $n=2m+1$ and $i\ne m+1$.  Finally, since $f$ is monotone, it is monotone with respect to the $i$-th variable when $i=m+1$, so $f$ is one of the following three forms: 
$$(\underbrace{\uparrow,\ldots,\uparrow}_{2m+1}), \qquad (\underbrace{\uparrow,\ldots,\uparrow}_{m}, \updownarrow,\underbrace{\uparrow,\ldots,\uparrow}_{m}), \qquad (\underbrace{\uparrow,\ldots,\uparrow}_{m}, \downarrow,\underbrace{\uparrow,\ldots,\uparrow}_{m}),$$ the first two of which imply that $f$ is isotone.
\end{proof}
An example of an associative function that is, say $(\uparrow,\downarrow,\uparrow)$, is given by $$f:\mathbb{Z}^3 \to \mathbb{Z}\qquad\mbox{where}\qquad f(x,y,z)=x-y+z.$$
$f$ is associative since $f(f(r,s,t),u,v)=f(r,f(s,t,u),v)=f(r,s,f(t,u,v))=r-s+t-u+v$.
%%%%%
%%%%%
\end{document}
