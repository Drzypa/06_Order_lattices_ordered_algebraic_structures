\documentclass[12pt]{article}
\usepackage{pmmeta}
\pmcanonicalname{CompleteSemilattice}
\pmcreated{2013-03-22 17:44:49}
\pmmodified{2013-03-22 17:44:49}
\pmowner{CWoo}{3771}
\pmmodifier{CWoo}{3771}
\pmtitle{complete semilattice}
\pmrecord{13}{40197}
\pmprivacy{1}
\pmauthor{CWoo}{3771}
\pmtype{Definition}
\pmcomment{trigger rebuild}
\pmclassification{msc}{06A12}
\pmclassification{msc}{06B23}
\pmsynonym{countably complete upper-semilattice}{CompleteSemilattice}
\pmsynonym{countably complete lower-semilattice}{CompleteSemilattice}
\pmsynonym{complete upper-semilattice homomorphism}{CompleteSemilattice}
\pmsynonym{complete lower-semilattice homomorphism}{CompleteSemilattice}
\pmrelated{CompleteLattice}
\pmrelated{Semilattice}
\pmrelated{ArbitraryJoin}
\pmdefines{countably complete join-semilattice}
\pmdefines{countably complete meet-semilattice}
\pmdefines{complete join-semilattice homomorphism}
\pmdefines{complete meet-semilattice homomorphism}

\endmetadata

\usepackage{amssymb,amscd}
\usepackage{amsmath}
\usepackage{amsfonts}
\usepackage{mathrsfs}

% used for TeXing text within eps files
%\usepackage{psfrag}
% need this for including graphics (\includegraphics)
%\usepackage{graphicx}
% for neatly defining theorems and propositions
\usepackage{amsthm}
% making logically defined graphics
%%\usepackage{xypic}
\usepackage{pst-plot}

% define commands here
\newcommand*{\abs}[1]{\left\lvert #1\right\rvert}
\newtheorem{prop}{Proposition}
\newtheorem{thm}{Theorem}
\newtheorem{ex}{Example}
\newcommand{\real}{\mathbb{R}}
\newcommand{\pdiff}[2]{\frac{\partial #1}{\partial #2}}
\newcommand{\mpdiff}[3]{\frac{\partial^#1 #2}{\partial #3^#1}}
\begin{document}
A complete join-semilattice is a join-semilattice $L$ such that for any subset $A\subseteq L$, $\bigvee A$, the arbitrary join operation on $A$, exists.  Dually, a complete meet-semilattice is a meet-semilattice such that $\bigwedge A$ exists for any $A\subseteq L$.  Because there are no restrictions placed on the subset $A$, it turns out that a 
complete join-semilattice is a complete meet-semilattice, and therefore a complete lattice.  In other words, by dropping the arbitrary join (meet) operation from a complete lattice, we end up with nothing new.  For a proof of this, see \PMlinkname{here}{CriteriaForAPosetToBeACompleteLattice}.  The crux of the matter lies in the fact that $\bigvee$ ($\bigwedge$) applies to \emph{any} set, including $L$ itself, and the empty set $\varnothing$, so that $L$ always contains has a top and a bottom.

\textbf{Variations}.  To obtain new objects, one looks for variations in the definition of ``complete''.  For example, if we require that any $A\subseteq L$ to be countable, we get what is a called a \emph{countably complete join-semilattice} (or dually, a \emph{countably complete meet-semilattice}).  More generally, if $\kappa$ is any cardinal, then a $\kappa$-complete join-semilattice is a semilattice $L$ such that for any set $A\subseteq L$ such that $|A|\le \kappa$, $\bigvee A$ exists.  If $\kappa$ is finite, then $L$ is just a join-semilattice.  When $\kappa=\infty$,  the only requirement on $A\subseteq L$ is that it be non-empty.  In [1], a complete semilattice is defined to be a poset $L$ such that for any non-empty $A\subseteq L$, $\bigwedge A$ exists, and any directed set $D\subseteq L$, $\bigvee D$ exists.

\textbf{Example}.  Let $A$ and $B$ be two isomorphic complete chains (a chain that is a complete lattice) whose cardinality is $\kappa$.  Combine the two chains to form a lattice $L$ by joining the top of $A$ with the top of $B$, and the bottom of $A$ with the bottom of $B$, so that 
\begin{itemize}
\item if $a\le b$ in $A$, then $a\le b$ in $L$
\item if $c\le d$ in $B$, then $c\le d$ in $L$
\item if $a\in A$, $c\in B$, then $a\le c$ iff $a$ is the bottom of $A$ and $c$ is the top of $B$
\item if $a\in A$, $c\in B$, then $c\le a$ iff $a$ is the top of $A$ and $c$ is the bottom of $B$
\end{itemize}
Now, $L$ can be easily seen to be a $\kappa$-complete lattice.  Next, remove the bottom element of $L$ to obtain $L'$.  Since, the meet operation no longer works on all pairs of elements of $L'$ while $\vee$ still works, $L'$ is a join-semilattice that is not a lattice.  In fact, $\bigvee$ works on all subsets of $L'$.  Since $|L'|=\kappa$, we see that $L'$ is a $\kappa$-complete join-semilattice.

\textbf{Remark}.  Although a complete semilattice is the same as a complete lattice, a homomorphism $f$ between, say, two complete join-semilattices $L_1$ and $L_2$, may fail to be a homomorphism between $L_1$ and $L_2$ as complete lattices.  Formally, a \emph{complete join-semilattice homomorphism} between two complete join-semilattices $L_1$ and $L_2$ is a function $f:L_1\to L_2$ such that for any subset $A\subseteq L_1$, we have $$f(\bigvee A)=\bigvee f(A)$$ where $f(A)=\lbrace f(a)\mid a\in A\rbrace$.  Note that it is not required that $f(\bigwedge A)=\bigwedge f(A)$, so that $f$ needs not be a complete lattice homomorphism.

To give a concrete example where a complete join-semilattice homomorphism $f$ fails to be complete lattice homomorphism, take $L$ from the example above, and define $f:L\to L$ by $f(a)=1$ if $a\ne 0$ and $f(0)=0$.  Then for any $A\subseteq L$, it is evident that $f(\bigvee A)=\bigvee f(A)$.  However, if we take two incomparable elements $a,b\in L$, then $f(a\wedge b)=f(0)=0$, while $f(a)\wedge f(b)= 1\wedge 1=1$.

\begin{thebibliography}{8}
\bibitem{ghklms} G. Gierz, K. H. Hofmann, K. Keimel, J. D. Lawson, M. W. Mislove, D. S. Scott, {\em Continuous Lattices and Domains}, Cambridge University Press, Cambridge (2003).
\bibitem{johnstone} P. T. Johnstone, \emph{Stone Spaces}, Cambridge University Press (1982).
\end{thebibliography}
%%%%%
%%%%%
\end{document}
