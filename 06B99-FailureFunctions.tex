\documentclass[12pt]{article}
\usepackage{pmmeta}
\pmcanonicalname{FailureFunctions}
\pmcreated{2015-05-27 5:00:48}
\pmmodified{2015-05-27 5:00:48}
\pmowner{akdevaraj}{13230}
\pmmodifier{akdevaraj}{13230}
\pmtitle{Failure  functions}
\pmrecord{1}{}
\pmprivacy{1}
\pmauthor{akdevaraj}{13230}
\pmtype{Definition}

\endmetadata

% this is the default PlanetMath preamble.  as your knowledge
% of TeX increases, you will probably want to edit this, but
% it should be fine as is for beginners.

% almost certainly you want these
\usepackage{amssymb}
\usepackage{amsmath}
\usepackage{amsfonts}

% need this for including graphics (\includegraphics)
\usepackage{graphicx}
% for neatly defining theorems and propositions
\usepackage{amsthm}

% making logically defined graphics
%\usepackage{xypic}
% used for TeXing text within eps files
%\usepackage{psfrag}

% there are many more packages, add them here as you need them

% define commands here

\begin{document}
Background:  see  messages.
Abstract  definition:  Let  $\phi(x)$  be  a  function  of
$x$.  Then $x$  =  $\psi(x_0)$  is  a  failure  function  IF  $\phi(psi(x_0))$  is
a   failure  in  accordance  with  our  definition  of  a  failure. Here  $x_0$  is
a  specific  function  of  $x$.   Note  the dual  role  played  by  $x$. It  is  a  
variable  in  $\phi(x)$  and  a  function of  a  specific value  of  $x_0$  in $\psi(x_0)$.
              \para.
              Examples:
              \line  1)  Let  our  definition  of  a  failure  be  a  composite  number.Let the
              parent  function, $\phi(x)$,    be a  polynomial ring in which  $ x$ belongs to 
              $Z$.  Then  $x  = psi(x_0)  =  x_0  +  k*\psi(x_0)$  is  a  failure function.Here
              $k$  belongs to  $Z$.
               
\end{document}
