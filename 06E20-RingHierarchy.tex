\documentclass[12pt]{article}
\usepackage{pmmeta}
\pmcanonicalname{RingHierarchy}
\pmcreated{2013-03-22 16:01:16}
\pmmodified{2013-03-22 16:01:16}
\pmowner{Algeboy}{12884}
\pmmodifier{Algeboy}{12884}
\pmtitle{ring hierarchy}
\pmrecord{8}{38060}
\pmprivacy{1}
\pmauthor{Algeboy}{12884}
\pmtype{Topic}
\pmcomment{trigger rebuild}
\pmclassification{msc}{06E20}

\endmetadata

\usepackage{latexsym}
\usepackage{amssymb}
\usepackage{amsmath}
\usepackage{amsfonts}
\usepackage{amsthm}

\usepackage{graphicx}

%%\usepackage{xypic}

%-----------------------------------------------------

%       Standard theoremlike environments.

%       Stolen directly from AMSLaTeX sample

%-----------------------------------------------------

%% \theoremstyle{plain} %% This is the default

\newtheorem{thm}{Theorem}

\newtheorem{coro}[thm]{Corollary}

\newtheorem{lem}[thm]{Lemma}

\newtheorem{lemma}[thm]{Lemma}

\newtheorem{prop}[thm]{Proposition}

\newtheorem{conjecture}[thm]{Conjecture}

\newtheorem{conj}[thm]{Conjecture}

\newtheorem{defn}[thm]{Definition}

\newtheorem{remark}[thm]{Remark}

\newtheorem{ex}[thm]{Example}



%\countstyle[equation]{thm}



%--------------------------------------------------

%       Item references.

%--------------------------------------------------


\newcommand{\exref}[1]{Example-\ref{#1}}

\newcommand{\thmref}[1]{Theorem-\ref{#1}}

\newcommand{\defref}[1]{Definition-\ref{#1}}

\newcommand{\eqnref}[1]{(\ref{#1})}

\newcommand{\secref}[1]{Section-\ref{#1}}

\newcommand{\lemref}[1]{Lemma-\ref{#1}}

\newcommand{\propref}[1]{Prop\-o\-si\-tion-\ref{#1}}

\newcommand{\corref}[1]{Cor\-ol\-lary-\ref{#1}}

\newcommand{\figref}[1]{Fig\-ure-\ref{#1}}

\newcommand{\conjref}[1]{Conjecture-\ref{#1}}


% Normal subgroup or equal.

\providecommand{\normaleq}{\unlhd}

% Normal subgroup.

\providecommand{\normal}{\lhd}

\providecommand{\rnormal}{\rhd}
% Divides, does not divide.

\providecommand{\divides}{\mid}

\providecommand{\ndivides}{\nmid}


\providecommand{\union}{\cup}

\providecommand{\bigunion}{\bigcup}

\providecommand{\intersect}{\cap}

\providecommand{\bigintersect}{\bigcap}










\begin{document}
\begin{figure}
\includegraphics{RingChart}
\caption{Diagram of the hierarchy of rings.}
\end{figure}

The objects in the diagram reflect many of the common rings encountered in ring theory.

\begin{itemize}
\item Every ring considered here has a 1.
\item   When one class of rings is connected to another class by a line, then the lower class is a subclass of the higher placed class. 
\item If a class has more than one parent in the graph it is not always the case that this class represents the strict intersection of these two classes, but it is certainly contained in this intersection.
\item Many of these containments are trivial in the sense that they are defined as subclasses of one another.  For instance, principal ideal domain is by definition a domain.  
\item However some subclasses are the result of deep theorems.  For example, every artinian ring is also noetherian.  
\end{itemize}

\Large{List of common rings}

\begin{enumerate}
\item Ring.
\item Commutative ring.
\item \PMlinkname{Noetherian ring}{Noetherian}.
\item Jacobson semisimple ring.
\item Local ring.
\item Integral domain.
\item \PMlinkname{Artinian ring}{Artinian}.
\item Primitive ring.
\item Unique factorization domain (UFD).
\item Dedekind domain.
\item Semisimple ring.
\item \PMlinkname{Principal ideal domain (PID)}{PrincipalIdealDomain}.
\item Simple ring.
\item \PMlinkname{Discrete valuation domain (DVD)}{DiscreteValuationRing} (Also
called a Discrete valuation ring).
\item Euclidean domain.
\item Division ring.
\item Field.
\end{enumerate}

The following containments are definitional:
\begin{itemize}
\item Ring $>$ commutative ring, noetherian ring and Jacobson semisimple ring.
\item Commutative ring $>$ local ring and integral domain.
\item Integral domain $>$ unique factorization domain and Dedekind domain.
\item Semisimple rings $>$ simple rings.
\item Local rings $>$ Discrete valuation domains.
\item Principal ideal domains $>$ Discrete valuation domains.
\item Division rings $>$ fields.
\end{itemize}

The following containments are due to theorems:
\begin{enumerate}
\item Jacobson semisimple rings $>$ primitive rings \cite[p. 571]{Rot}.
\item Noetherian rings $>$ artinian rings [Hopkins-Levitzki] \cite[Theorem 8.46]{Rot}.
\item Noetherian rings $>$ Dedekind domain \cite[Theorem VIII.6.10]{Hungerford}.
\item Artinian rings $>$ semisimple rings, [Wedderburn-Artin theorem].
\footnote{Some definitions semisimple make this containment part of the definition.  Otherwise the result is part of the Wedderburn-Artin theorem.}
\item Jacobson semisimple $>$ semisimple rings.[Wedderburn-Artin theorem].\footnote{Also depends on the definition of semisimple.}
\item Dedekind domain $>$ Principal ideal domain \cite[p. 401]{Hungerford}.
\item Principal ideal domains $>$ euclidean domains \cite[Theorem 3.60]{Rot}.
\item Simple rings $>$ division rings.
\end{enumerate}

\bibliographystyle{amsplain}
\providecommand{\bysame}{\leavevmode\hbox to3em{\hrulefill}\thinspace}
\providecommand{\MR}{\relax\ifhmode\unskip\space\fi MR }
% \MRhref is called by the amsart/book/proc definition of \MR.
\providecommand{\MRhref}[2]{%
\href{http://www.ams.org/mathscinet-getitem?mr=#1}{#2}
}
\providecommand{\href}[2]{#2}
\begin{thebibliography}{10}

\bibitem{Hungerford}
Hungerford, Thomas W.
\emph{Algebra}, Graduate Texts in Mathematics, 73
Springer-Verlag, New York, (1980), pp. xxiii+502.

\bibitem{Rot}
Rotman, Joseph J.
\emph{Advanced modern algebra},
Prentice Hall Inc.,Upper Saddle River, NJ, (2002), pp {xvi+1012+A8+B6+I14}.
\end{thebibliography}

%%%%%
%%%%%
\end{document}
