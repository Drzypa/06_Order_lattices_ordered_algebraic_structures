\documentclass[12pt]{article}
\usepackage{pmmeta}
\pmcanonicalname{MeetContinuous}
\pmcreated{2013-03-22 16:36:41}
\pmmodified{2013-03-22 16:36:41}
\pmowner{CWoo}{3771}
\pmmodifier{CWoo}{3771}
\pmtitle{meet continuous}
\pmrecord{12}{38808}
\pmprivacy{1}
\pmauthor{CWoo}{3771}
\pmtype{Definition}
\pmcomment{trigger rebuild}
\pmclassification{msc}{06A12}
\pmclassification{msc}{06B35}
\pmsynonym{order convergence}{MeetContinuous}
\pmrelated{CriteriaForAPosetToBeACompleteLattice}
\pmrelated{JoinInfiniteDistributive}
\pmdefines{join continuous}
\pmdefines{order converges}

\endmetadata

\usepackage{amssymb,amscd}
\usepackage{amsmath}
\usepackage{amsfonts}

% used for TeXing text within eps files
%\usepackage{psfrag}
% need this for including graphics (\includegraphics)
%\usepackage{graphicx}
% for neatly defining theorems and propositions
%\usepackage{amsthm}
% making logically defined graphics
%%\usepackage{xypic}
\usepackage{pst-plot}
\usepackage{psfrag}

% define commands here

\begin{document}
Let $L$ be a meet semilattice.  We say that $L$ is \emph{meet continuous} if
\begin{enumerate}
\item for any monotone net $D=\lbrace x_i \mid i\in I\rbrace$ in $L$, its supremum $\bigvee D$ exists, and 
\item for any $a\in L$ and any monotone net $\lbrace x_i\mid i\in I\rbrace$,  $$a\wedge \bigvee \lbrace x_i \mid i\in I \rbrace = \bigvee \lbrace a\wedge x_i\mid i\in I \rbrace.$$ 
\end{enumerate}

A monotone net $\lbrace x_i\mid i\in I\rbrace$ is a net $x:I\to L$ such that $x$ is a non-decreasing function; that is, for any $i\le j$ in $I$, $x_i\le x_j$ in $L$.

Note that we could have replaced the first condition by saying simply that $D\subseteq L$ is a directed set.  (A monotone net is a directed set, and a directed set is a trivially a monotone net, by considering the identity function as the net).  It's not hard to see that if $D$ is a directed subset of $L$, then $a\wedge D:=\lbrace a\wedge x\mid x\in D\rbrace$ is also directed, so that the right hand side of the second condition makes sense.

Dually, a join semilattice $L$ is \emph{join continuous} if its dual (as a meet semilattice) is meet continuous.  In other words, for any antitone net $D=\lbrace x_i\mid i\in I\rbrace$, its infimum $\bigwedge D$ exists and that 
$$a\vee \bigwedge \lbrace x_i\mid i\in I\rbrace =\bigwedge \lbrace a\vee x_i\mid i\in I\rbrace.$$
An antitone net is just a net $x:I\to L$ such that for $i\le j$ in $I$, $x_j\le x_i$ in $L$.

\textbf{Remarks}.
\begin{itemize}
\item A meet continuous lattice is a complete lattice, since a poset such that finite joins and directed joins exist is a complete lattice (see the link below for a proof of this).
\item 
Let a lattice $L$ be both meet continuous and join continuous.  Let $\lbrace x_i\mid i\in I\rbrace$ be any net in $L$.  We define the following:
$$\overline{\lim}\ x_i = \bigwedge_{j\in I} \lbrace \bigvee_{j\le i} x_i\rbrace\qquad\mbox{ and }\qquad\underline{\lim}\ x_i = \bigvee_{j\in I} \lbrace \bigwedge_{i\le j} x_i\rbrace$$
If there is an $x\in L$ such that $\overline{\lim}\ x_i=x=\underline{\lim}\ x_i$, then we say that the net $\lbrace x_i\rbrace$ \emph{order converges} to $x$, and we write $x_i\to x$, or $x=\lim\ x_i$.  Now, define a subset $C\subseteq L$ to be \emph{closed} (in $L$) if for any net $\lbrace x_i\rbrace$ in $C$ such that $x_i\to x$ implies that $x\in C$, and \emph{open} if its set complement is closed, then $L$ becomes a topological lattice.  With respect to this topology, meet $\wedge$ and join $\vee$ are easily seen to be continuous.
\end{itemize}

\begin{thebibliography}{8}
\bibitem{gb} G. Birkhoff, {\em Lattice Theory}, 3rd Edition, Volume 25, AMS, Providence (1967).
\bibitem{ghklms} G. Gierz, K. H. Hofmann, K. Keimel, J. D. Lawson, M. W. Mislove, D. S. Scott, {\em Continuous Lattices and Domains}, Cambridge University Press, Cambridge (2003).
\bibitem{gg} G. Gr\"{a}tzer, {\em General Lattice Theory}, 2nd Edition, Birkh\"{a}user (1998).
\end{thebibliography}
%%%%%
%%%%%
\end{document}
