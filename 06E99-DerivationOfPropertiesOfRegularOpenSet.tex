\documentclass[12pt]{article}
\usepackage{pmmeta}
\pmcanonicalname{DerivationOfPropertiesOfRegularOpenSet}
\pmcreated{2013-03-22 17:59:24}
\pmmodified{2013-03-22 17:59:24}
\pmowner{CWoo}{3771}
\pmmodifier{CWoo}{3771}
\pmtitle{derivation of properties of regular open set}
\pmrecord{6}{40501}
\pmprivacy{1}
\pmauthor{CWoo}{3771}
\pmtype{Derivation}
\pmcomment{trigger rebuild}
\pmclassification{msc}{06E99}

\endmetadata

\usepackage{amssymb,amscd}
\usepackage{amsmath}
\usepackage{amsfonts}
\usepackage{mathrsfs}

% used for TeXing text within eps files
%\usepackage{psfrag}
% need this for including graphics (\includegraphics)
%\usepackage{graphicx}
% for neatly defining theorems and propositions
\usepackage{amsthm}
% making logically defined graphics
%%\usepackage{xypic}
\usepackage{pst-plot}

% define commands here
\newcommand*{\abs}[1]{\left\lvert #1\right\rvert}
\newtheorem{prop}{Proposition}
\newtheorem{thm}{Theorem}
\newtheorem{ex}{Example}
\newcommand{\real}{\mathbb{R}}
\newcommand{\pdiff}[2]{\frac{\partial #1}{\partial #2}}
\newcommand{\mpdiff}[3]{\frac{\partial^#1 #2}{\partial #3^#1}}
\begin{document}
Recall that a subset $A$ of a topological space $X$ is regular open if it is equal to the interior of the closure of itself.

To facilitate further analysis of regular open sets, define the operation $^{\bot}$ as follows:
$$A^{\bot}:=X-\overline{A}.$$

Some of the properties of $^\bot$ and regular openness are listed and derived:
\begin{enumerate}
\item For any $A\subseteq X$, $A^{\bot}$ is open.  This is obvious.
\item $^\bot$ reverses inclusion.  This is also obvious.
\item $\varnothing^{\bot}=X$ and $X^{\bot}=\varnothing$.  This too is clear.
\item $A\cap A^{\bot}=\varnothing$, because $A\cap A^{\bot}\subseteq A\cap (X-A)=\varnothing$.
\item $A\cup A^{\bot}$ is dense in $X$, because $X=\overline{A}\cup A^{\bot} \subseteq \overline{A}\cup \overline{A^\bot} =\overline{A\cup A^\bot}$.
\item $A^{\bot}\cup B^{\bot}\subseteq(A\cap B)^{\bot}$.  To see this, first note that $A\cap B\subseteq A$, so that $A^\bot \subseteq (A\cap B)^\bot$.  Similarly, $A^\bot \subseteq (A\cap B)^\bot$.  Take the union of the two inclusions and the result follows.
\item $A^{\bot}\cap B^{\bot}=(A\cup B)^{\bot}$.  This can be verified by direct calculation: $$A^{\bot}\cap B^{\bot}= (X- \overline{A})\cap (X-\overline{B})=X-(\overline{A}\cup \overline{B})=X-\overline{A\cup B}=(A\cup B)^{\bot}.$$
\item $A$ is regular open iff $A=A^{\bot\bot}$.  See the remark at the end of \PMlinkname{this entry}{DerivationOfPropertiesOnInteriorOperation}.
\item If $A$ is open, then $A^{\bot}$ is regular open.
\begin{proof}
By the previous property, we want to show that $A^{\bot\bot\bot}=A^\bot$ if $A$ is open.  For notational convenience, let us write $A^-$ for the closure of $A$ and $A^c$ for the complement of $A$.  As $^\bot=^{-c}$, the equation now becomes $A^{-c-c-c}=A^{-c}$ for any open set $A$.

Since $A\subseteq A^-$ for any set, $A^{-c}\subseteq A^c$.  This means $A^{-c-}\subseteq A^{c-}$.  Since $A$ is open, $A^c$ is closed, so that $A^{c-}=A^c$.  The last inclusion becomes $A^{-c-}\subseteq A^c$.  Taking complement again, we have 
\begin{equation}
A\subseteq A^{-c-c}.
\end{equation}
Since $^\bot=^{-c}$ reverses inclusion, we have $A^{-c-c-c}\subseteq A^{-c}$, which is one of the inclusions.  On the other hand, the inclusion (1) above applies to \emph{any} open set, and because $A^{-c}$ is open, $A^{-c}\subseteq A^{-c-c-c}$, which is the other inclusion.
\end{proof}
\item If $A$ and $B$ are regular open, then so is $A\cap B$.
\begin{proof}  Since $A,B$ are regular open, $(A\cap B)^{\bot\bot}= (A^{\bot\bot}\cap B^{\bot\bot})^{\bot\bot}$, which is equal to $(A^\bot \cup B^\bot)^{\bot\bot\bot}$ by property 7 above.  Since $A^\bot \cup B^\bot$ is open, the last expression becomes $(A^\bot \cup B^\bot)^\bot$ by property 9, or $A\cap B$ by property 7 again.
\end{proof}
\end{enumerate}

\textbf{Remark}.  All of the properties above can be dualized for regular closed sets.  If fact, proving a property about regular closedness can be easily accomplished once we have the following: 
\begin{quote}\begin{center}
$(*)$ $A$ is regular open iff $X-A$ is regular closed.
\end{center}\end{quote}
\begin{proof}
Suppose first that $A$ is regular open.  Then $\overline{\operatorname{int}(X-A)} = \overline{X-\overline{A}}=X-\operatorname{int}(\overline{A})=X-A$.  The converse is proved similarly.
\end{proof}
As a corollary, for example, we have: if $A$ is closed, then $\overline{X-A}$ is regular closed.
\begin{proof}
If $A$ is closed, then $X-A$ is open, so that $(X-A)^\bot=X-\overline{X-A}$ is regular open by property 9 above, which implies that $X-(X-A)^\bot = \overline{X-A}$ is regular closed by $(*)$.
\end{proof}
%%%%%
%%%%%
\end{document}
