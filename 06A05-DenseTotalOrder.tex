\documentclass[12pt]{article}
\usepackage{pmmeta}
\pmcanonicalname{DenseTotalOrder}
\pmcreated{2013-03-22 16:40:48}
\pmmodified{2013-03-22 16:40:48}
\pmowner{mps}{409}
\pmmodifier{mps}{409}
\pmtitle{dense total order}
\pmrecord{8}{38888}
\pmprivacy{1}
\pmauthor{mps}{409}
\pmtype{Definition}
\pmcomment{trigger rebuild}
\pmclassification{msc}{06A05}
\pmsynonym{dense linear order}{DenseTotalOrder}
\pmrelated{LinearContinuum}
\pmdefines{dense}
\pmdefines{dense in}
\pmdefines{dense in itself}

\endmetadata

% this is the default PlanetMath preamble.  as your knowledge
% of TeX increases, you will probably want to edit this, but
% it should be fine as is for beginners.

% almost certainly you want these
\usepackage{amssymb}
\usepackage{amsmath}
\usepackage{amsfonts}

% used for TeXing text within eps files
%\usepackage{psfrag}
% need this for including graphics (\includegraphics)
%\usepackage{graphicx}
% for neatly defining theorems and propositions
%\usepackage{amsthm}
% making logically defined graphics
%%%\usepackage{xypic}

% there are many more packages, add them here as you need them

% define commands here

\begin{document}
A total order $(S,<)$ is \emph{dense} if whenever $x < z$ in $S$, there exists at least one element $y$ of $S$ such that $x < y < z$.  That is, each nontrivial closed interval has nonempty interior.

A subset $T$ of a total order $S$ is \emph{dense in} $S$ if for every $x,z\in S$ such that $x<z$, there exists some $y\in T$ such that $x<y<z$.  Because of this, a dense total order $S$ is sometimes said to be \emph{dense in itself}.

For example, the integers with the usual order are not dense, since there is no integer strictly between $0$ and $1$.  On the other hand, the rationals $\mathbb{Q}$ are dense, since whenever $r$ and $s$ are rational numbers, it follows that $(r+s)/2$ is a rational number strictly between $r$ and $s$.  
Also, both $\mathbb{Q}$ and the irrationals $\mathbb{R}\setminus\mathbb{Q}$ are dense in $\mathbb{R}$.

It is usually convenient to assume that a dense order has at least two elements.  This allows one to avoid the trivial cases of the one-point order and the empty order.
%%%%%
%%%%%
\end{document}
