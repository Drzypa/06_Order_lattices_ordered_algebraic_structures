\documentclass[12pt]{article}
\usepackage{pmmeta}
\pmcanonicalname{RedundancyOfTwosidednessInDefinitionOfGroup}
\pmcreated{2015-01-20 17:28:03}
\pmmodified{2015-01-20 17:28:03}
\pmowner{pahio}{2872}
\pmmodifier{pahio}{2872}
\pmtitle{redundancy of two-sidedness in definition of group}
\pmrecord{3}{88222}
\pmprivacy{1}
\pmauthor{pahio}{2872}
\pmtype{Definition}

% this is the default PlanetMath preamble.  as your knowledge
% of TeX increases, you will probably want to edit this, but
% it should be fine as is for beginners.

% almost certainly you want these
\usepackage{amssymb}
\usepackage{amsmath}
\usepackage{amsfonts}

% need this for including graphics (\includegraphics)
\usepackage{graphicx}
% for neatly defining theorems and propositions
\usepackage{amsthm}

% making logically defined graphics
%\usepackage{xypic}
% used for TeXing text within eps files
%\usepackage{psfrag}

% there are many more packages, add them here as you need them

% define commands here

\begin{document}
In the definition of group, one usually supposes that there is a two-sided identity element and that 
any element has a two-sided inverse (cf. \PMlinkname{group}{Group}).

The group may also be defined without the two-sidednesses:\\

A {\it group} is a pair of a non-empty set $G$ and its associative binary operation 
$(x,y) \mapsto xy$ such that

1) the operation has a right identity element $e$;

2) any element $x$ of $G$ has a right inverse $x^{-1}$.\\

We have to show that the right identity $e$ is also a left identity 
and that any right inverse is also a left inverse.

Let the above assumptions on $G$ be true.\, If $a^{-1}$ is the right 
inverse of an arbitrary element $a$ of $G$, the calculation
$$a^{-1}a = a^{-1}ae = a^{-1}aa^{-1}(a^{-1})^{-1} = 
a^{-1}e(a^{-1})^{-1} = a^{-1}(a^{-1})^{-1} = e$$
shows that it is also the left inverse of $a$.\, Using this result, 
we then can write
$$ea = (aa^{-1})a = a(a^{-1}a) = ae = a,$$
whence $e$ is a left identity element, too.

\end{document}
