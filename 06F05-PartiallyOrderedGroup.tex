\documentclass[12pt]{article}
\usepackage{pmmeta}
\pmcanonicalname{PartiallyOrderedGroup}
\pmcreated{2013-03-22 16:42:25}
\pmmodified{2013-03-22 16:42:25}
\pmowner{CWoo}{3771}
\pmmodifier{CWoo}{3771}
\pmtitle{partially ordered group}
\pmrecord{14}{38922}
\pmprivacy{1}
\pmauthor{CWoo}{3771}
\pmtype{Definition}
\pmcomment{trigger rebuild}
\pmclassification{msc}{06F05}
\pmclassification{msc}{06F20}
\pmclassification{msc}{06F15}
\pmclassification{msc}{20F60}
\pmsynonym{po-group}{PartiallyOrderedGroup}
\pmsynonym{l-group}{PartiallyOrderedGroup}
\pmsynonym{Archimedean po-group}{PartiallyOrderedGroup}
\pmsynonym{integrally closed po-group}{PartiallyOrderedGroup}
\pmsynonym{po-semigroup}{PartiallyOrderedGroup}
\pmsynonym{lattice-ordered group}{PartiallyOrderedGroup}
\pmsynonym{l-semigroup}{PartiallyOrderedGroup}
\pmrelated{OrderedGroup}
\pmdefines{directed group}
\pmdefines{positive element}
\pmdefines{positive cone}
\pmdefines{lattice ordered group}
\pmdefines{Archimedean partially ordered group}
\pmdefines{integrally closed group}
\pmdefines{integrally closed partially ordered group}
\pmdefines{partially ordered semigroup}
\pmdefines{lattice ordered semigroup}
\pmdefines{Archimedean}

\endmetadata

\usepackage{amssymb,amscd}
\usepackage{amsmath}
\usepackage{amsfonts}

% used for TeXing text within eps files
%\usepackage{psfrag}
% need this for including graphics (\includegraphics)
%\usepackage{graphicx}
% for neatly defining theorems and propositions
\usepackage{amsthm}
% making logically defined graphics
%%\usepackage{xypic}
\usepackage{pst-plot}
\usepackage{psfrag}

% define commands here

\begin{document}
A \emph{partially ordered group} is a group $G$ that is a poset at the same time, such that if $a,b\in G$ and $a\le b$, then
\begin{enumerate}
\item $ac\le bc$, and
\item $ca\le cb$,
\end{enumerate}
for any $c\in G$.  The two conditions are equivalent to the one condition $cad\le cbd$ for all $c,d\in G$.  A partially ordered group is also called a \emph{po-group} for short.

\textbf{Remarks}.
\begin{itemize}
\item
One of the immediate properties of a po-group is this: if $a\le b$, then $b^{-1}\le a^{-1}$.  To see this, left multiply by the first inequality by $a^{-1}$ on both sides to obtain $e\le a^{-1}b$.  Then right multiply the resulting inequality on both sides by $b^{-1}$ to obtain the desired inequality: $b^{-1}\le a^{-1}$.
\item
If can be seen that for every $a\in G$, the automorphisms $L_a,R_a:G\to G$ also preserve order, and hence are order automorphisms as well.  For instance, if $b\le c$, then $L_a(b)=ab\le ac = L_a(c)$.
\item
A element $a$ in a po-group $G$ is said to be \emph{positive} if $e\le a$, where $e$ is the identity element of $G$.  The set of positive elements in $G$ is called the \emph{positive cone} of $G$.
\item (special po-groups)
\begin{enumerate}
\item A po-group whose underlying poset is a directed set is called a \emph{directed group}.  
\begin{itemize}
\item
If $G$ is a directed group, then $G$ is also a filtered set: if $a,b\in G$, then there is a $c\in G$ such that $a\le c$ and $b\le c$, so that $ac^{-1}b\le a$ and $ac^{-1}b\le b$ as well.
\item
Also, if $G$ is directed, then $G=\langle G^+\rangle$: for any $x\in G$, let $a$ be the upper bound of $\lbrace x,e\rbrace$ and let $b=ax^{-1}$.  Then $e\le b$ and $x=a^{-1}b\in \langle G^+\rangle$.
\end{itemize}
\item A po-group whose underlying poset is a lattice is called a \emph{lattice ordered group}, or an \emph{l-group}.
\item If the partial order on a po-group $G$ is a linear order, then $G$ is called a totally ordered group, or simply an ordered group.
\item A po-group is said to be \emph{Archimedean} if $a^n\le b$ for all $n\in \mathbb{Z}$, then $a=e$.  Equivalently, if $a\ne e$, then for any $b\in G$, there is some $n\in \mathbb{Z}$ such that $b<a^n$.  This is a generalization of the Archimedean property on the reals: if $r\in \mathbb{R}$, then there is some $n\in \mathbb{N}$ such that $r<n$.  To see this, pick $b=r$, and $a=1$.
\item A po-group is said to be \emph{integrally closed} if $a^n\le b$ for all $n\ge 1$, then $a\le e$.  An integrally closed group is Archimedean: if $a^n\le b$ for all $n\in\mathbb{Z}$, then $a\le e$ and $e\le b$.  Since we also have $(a^{-1})^{-n}\le b$ for all $n<0$, this implies $a^{-1}\le e$, or $e\le a$.  Hence $a=e$.  In fact, an directed integrally closed group is an Abelian po-group.
\end{enumerate}
\item Since the definition above does not involve any specific group axioms, one can more generally introduce partial ordering on a semigroup in the same fashion.  The result is called a partially ordered semigroup, or a po-semigroup for short.  A \emph{lattice ordered semigroup} is defined similarly.
\end{itemize}
%%%%%
%%%%%
\end{document}
