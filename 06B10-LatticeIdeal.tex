\documentclass[12pt]{article}
\usepackage{pmmeta}
\pmcanonicalname{LatticeIdeal}
\pmcreated{2013-03-22 15:48:58}
\pmmodified{2013-03-22 15:48:58}
\pmowner{CWoo}{3771}
\pmmodifier{CWoo}{3771}
\pmtitle{lattice ideal}
\pmrecord{17}{37781}
\pmprivacy{1}
\pmauthor{CWoo}{3771}
\pmtype{Definition}
\pmcomment{trigger rebuild}
\pmclassification{msc}{06B10}
\pmsynonym{prime lattice ideal}{LatticeIdeal}
\pmsynonym{maximal lattice ideal}{LatticeIdeal}
\pmrelated{LatticeFilter}
\pmrelated{UpperSet}
\pmrelated{OrderIdeal}
\pmrelated{LatticeOfIdeals}
\pmdefines{ideal}
\pmdefines{proper ideal}
\pmdefines{prime ideal}
\pmdefines{sublattice generated by}
\pmdefines{ideal generated by}
\pmdefines{principal ideal}
\pmdefines{maximal ideal}
\pmdefines{non-trivial ideal}

\endmetadata

\usepackage{amssymb,amscd}
\usepackage{amsmath}
\usepackage{amsfonts}

% used for TeXing text within eps files
%\usepackage{psfrag}
% need this for including graphics (\includegraphics)
%\usepackage{graphicx}
% for neatly defining theorems and propositions
\usepackage{amsthm}
% making logically defined graphics
%%\usepackage{xypic}

% define commands here
\newcommand{\up}{\uparrow\!\!}
\newcommand{\down}{\downarrow\!\!}
\begin{document}
Let $L$ be a lattice.  An \emph{ideal} $I$ of $L$ is a non-empty subset of $L$ such that\
\begin{enumerate}
\item $I$ is a sublattice of $L$, and
\item for any $a\in I$ and $b\in L$, $a\wedge b\in I$.
\end{enumerate}

Note the similarity between this definition and the definition of an \PMlinkname{ideal}{Ideal} in a ring (except in a ring with 1, an ideal is almost never a subring)

Since the fact that $a\wedge b\in I$ for $a,b\in I$ in the first condition is already implied by the second condition, we can replace the first condition by a weaker one: \begin{quote} for any $a,b\in I$, $a\vee b\in I$.\end{quote}

Another equivalent characterization of an ideal $I$ in a lattice $L$ is 
\begin{enumerate}
\item for any $a,b\in I$, $a\vee b\in I$, and
\item for any $a\in I$, if $b\le a$, then $b\in I$.
\end{enumerate}

Here's a quick proof.  In fact, all we need to show is that the two second conditions are equivalent for $I$.  First assume that for any $a\in I$ and $b\in L$, $a\wedge b\in I$.  If $b\le a$, then $b=a\wedge b\in I$.  Conversely, since $a\wedge b\le a\in I$, $a\wedge b\in I$ as well.

\textbf{Special Ideals}.  Let $I$ be an ideal of a lattice $L$.  Below are some common types of ideals found in lattice theory.
\begin{itemize}
\item $I$ is \emph{proper} if $I\ne L$.
\item If $L$ contains $0$, $I$ is said to be \emph{non-trivial} if $I\ne 0$.
\item $I$ is a \emph{prime ideal} if it is proper, and for any $a\wedge b\in I$, either $a\in I$ or $b\in I$.
\item $I$ is a \emph{maximal ideal} of $L$ if $I$ is proper and the only ideal having $I$ as a proper subset is $L$. 
\item \textbf{ideal generated by a set}.  Let $X$ be a subset of a lattice $L$.  Let $S$ be the set of all ideals of $L$ containing $X$.  Since $S\ne\varnothing$ ($L\in S$), the intersection $M$ of all elements in $S$, is also an ideal of $L$ that contains $X$.  $M$ is called the \emph{ideal generated by} $X$, written $(X]$.  If $X$ is a singleton $\lbrace x\rbrace$, then $M$ is said to be a \emph{principal ideal} generated by $x$, written $(x]$.  (Note that this construction can be easily carried over to the construction of a \emph{sublattice generated by} a subset of a lattice).
\end{itemize}
\textbf{Remarks}.  Let $L$ be a lattice.
\begin{enumerate}
\item Given any subset $X\subset L$, let $X'$ be the set consisting of all finite joins of elements of $X$, which is clearly a directed set.  Then $\down X'$, the down set of $X'$, is $(X]$.  Any element of $(X]$ is less than or equal to a finite join of elements of $X$.
\item If $L$ is a distributive lattice, every maximal ideal is prime.  Suppose $I\subseteq L$ is maximal and $a\wedge b\in I$ with $a\notin I$.  Then the ideal generated by $I$ and $a$ must be $L$, so that $b\le p\vee a$ for some $p\in I$.  Then $b=b\wedge b\le (p\vee a)\wedge b=(p\wedge b)\vee (a\wedge b)\in I$, which means $b\in I$.  So $I$ is prime.
\item If $L$ is a complemented lattice, every prime ideal is maximal.  Suppose $I\subseteq L$ is prime and $a\notin I$.  Let $b$ be a complement of $a$, then $b\in I$, for otherwise, $0=a\wedge b\notin I$, a contradiction.  Let $J$ be the ideal generated by $I$ and $a$, then $1\le b\vee a\in J$, so $J=L$.
\item Combining the two results above, in a Boolean algebra, an ideal is prime iff it is maximal.
\end{enumerate}

\textbf{Examples}.  In the lattice $L$ below,

\begin{equation*}
\xymatrix{
& 1 \ar@{-}[ld] \ar@{-}[rd] \\
a \ar@{-}[rd] & & b \ar@{-}[ld] \\
& c \ar@{-}[d] & \\
& d \ar@{-}[ld] \ar@{-}[rd] & \\
e \ar@{-}[rd] & & f \ar@{-}[ld] \\
& 0
}
\end{equation*}

Besides $L$ and $\lbrace 0\rbrace$, below are all proper ideals of $L$:
\begin{itemize}
\item $M=\lbrace a, c, d, e, f, 0\rbrace$, 
\item $N=\lbrace b, c, d, e, f, 0\rbrace$, 
\item $R=\lbrace c, d, e, f, 0\rbrace$, 
\item $S=\lbrace d, e, f, 0\rbrace$, 
\item $T=\lbrace e, 0\rbrace$, and 
\item $U=\lbrace f, 0\rbrace$.
\end{itemize}
Out of these, $M,N,S,T,U$ are prime, and $M,N$ are maximal.  The ideal generated by, say $\lbrace c,e\rbrace$, is $R$.  Looking more closely, we see that $R$ can actually be generated by $c$, and so is principal.  In fact, all ideals in $L$ are principal, generated by their maximal elements.  It is not hard to see, that in a lattice $L$ with acc (ascending chain condition), all ideals are principal:

\begin{proof}.  First, let's show that an ideal $I$ in a lattice $L$ with acc has at least one maximal element.  Suppose $a\in I$.  If $a$ is not maximal in $I$, there is a $a_1\in I$ such that $a\le a_1$.  If $a_1$ is not maximal in $I$, repeat the process above so we get a chain $a\le a_1\le a_2\le\ldots$ in $I$.  Eventually this chain terminates $a_n=a_{n+1}=\cdots$.  Thus $b=a_n$ is maximal in $I$.  Next, suppose that $I$ has two distinct maximal elements.  Then their join is again in $I$, contradicting maximality.  So $b$ is unique and all elements $c$ such that $c\le b$ must be in $I$.  Therefore, $I=(b]$.\end{proof}

Finally, an example of a sublattice that is not an ideal is the subset $\lbrace b, c, d, e, 0\rbrace$.
%%%%%
%%%%%
\end{document}
