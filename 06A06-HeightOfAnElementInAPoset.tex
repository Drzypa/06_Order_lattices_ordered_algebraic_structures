\documentclass[12pt]{article}
\usepackage{pmmeta}
\pmcanonicalname{HeightOfAnElementInAPoset}
\pmcreated{2013-03-22 16:31:32}
\pmmodified{2013-03-22 16:31:32}
\pmowner{CWoo}{3771}
\pmmodifier{CWoo}{3771}
\pmtitle{height of an element in a poset}
\pmrecord{10}{38705}
\pmprivacy{1}
\pmauthor{CWoo}{3771}
\pmtype{Definition}
\pmcomment{trigger rebuild}
\pmclassification{msc}{06A06}
\pmrelated{GradedPoset}
\pmdefines{Jordan-Dedekind chain condition}

\endmetadata

\usepackage{amssymb,amscd}
\usepackage{amsmath}
\usepackage{amsfonts}

% used for TeXing text within eps files
%\usepackage{psfrag}
% need this for including graphics (\includegraphics)
%\usepackage{graphicx}
% for neatly defining theorems and propositions
\usepackage{amsthm}
% making logically defined graphics
%%\usepackage{xypic}
\usepackage{pst-plot}
\usepackage{psfrag}

% define commands here
\newcommand{\up}{\uparrow\!\!}
\newcommand{\down}{\downarrow\!\!}
\begin{document}
Let $P$ be a poset.  Given any $a\in P$, the lower set $\down a$ of $a$ is a subposet of $P$.  Call the height of $\down a$ less 1 the \emph{height} of $a$.  Let's denote $h(a)$ the height of $a$, so $$h(a)=\operatorname{height}(\down a)-1.$$  From this definition, we see that $h(a)=0$ iff $a$ is minimal and $h(a)=1$ iff $a$ is an atom.  Also, $h$ partitions $P$ into equivalence classes, so that $a$ is equivalent to $b$ in $P$ iff $h(a)=h(b)$.  Two distinct elements in the same equivalence class are necessarily incomparable.  In other words, the equivalence classes are antichains.  Furthermore, given any two equivalence classes $[a],[b]$, set $[a]\le[b]$ iff $h(a)\le h(b)$, then the set of equivalence classes form a chain.

The height function of a poset $P$ looks remarkably like the rank function of a graded poset: $h$ is constant on the set of all minimal elements, and $h$ is isotone (preserves order).  When is $h$ a rank function (the additional condition being the preservation of the covering relation)?  The answer is given by a chain condition imposed on $P$, called the \emph{Jordan-Dedekind chain condition}:
\begin{quote}
(*) In a poset, the cardinalities of two maximal chains between common end points must be the same.
\end{quote}
Suppose for each $a\in P$, $h(a)$ is finite and $P$ has a unique minimal element $0$.  Then $P$ can be graded by $h$ iff (*) is satisfied.  More generally, if we drop the assumption of the uniqueness of a minimal element, then $P$ can be graded by $h$ iff any two maximal chains ending at the same end point have the same length.

%%%%%
%%%%%
\end{document}
