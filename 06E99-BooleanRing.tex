\documentclass[12pt]{article}
\usepackage{pmmeta}
\pmcanonicalname{BooleanRing}
\pmcreated{2013-03-22 12:27:28}
\pmmodified{2013-03-22 12:27:28}
\pmowner{yark}{2760}
\pmmodifier{yark}{2760}
\pmtitle{Boolean ring}
\pmrecord{24}{32602}
\pmprivacy{1}
\pmauthor{yark}{2760}
\pmtype{Definition}
\pmcomment{trigger rebuild}
\pmclassification{msc}{06E99}
\pmclassification{msc}{03G05}
\pmrelated{Idempotency}
\pmrelated{BooleanLattice}
\pmrelated{BooleanIdeal}

\endmetadata

\usepackage{amssymb}
\usepackage{amsmath}
\usepackage{amsfonts}

\newcommand{\Z}{\mathbb{Z}}
\begin{document}
\PMlinkescapeword{boolean}
\PMlinkescapeword{commutative}
\PMlinkescapeword{equivalent}
\PMlinkescapeword{isomorphic}

A \emph{Boolean ring} is a ring $R$ that has a multiplicative identity,
and in which every element is idempotent, that is, 
$$x^2=x\text{ for all }x\in R.$$
Boolean rings are necessarily \PMlinkname{commutative}{CommutativeRing}.
Also, if $R$ is a Boolean ring, then $x=-x$ for each $x\in R$.

Boolean rings are equivalent to Boolean algebras (or \PMlinkname{Boolean lattices}{BooleanLattice}).
Given a Boolean ring $R$, 
define $x \land y = xy$ and $x \lor y = x + y + xy$ and $x'=x+1$
for all $x,y\in R$, 
then $(R,\land,\lor,\phantom{i}',0,1)$ is a Boolean algebra.
Given a Boolean algebra $(L,\land,\lor,\phantom{i}',0,1)$,
define $x\cdot y = x \land y$ and $x + y = (x' \land y) \lor (x \land y')$,
then $(L,\cdot,+)$ is a Boolean ring.
In particular, the category of Boolean rings is isomorphic to the category of Boolean lattices.

\section*{Examples}

As mentioned above, every Boolean algebra can be considered as a Boolean ring. In particular, if $X$ is any set, then the power set ${\cal P}(X)$ forms a Boolean ring, with intersection as multiplication and symmetric difference as addition.

Let $R$ be the ring $\Z_2\times\Z_2$ with the operations being coordinate-wise.
Then we can check:
\begin{eqnarray*}
(1,1)\times(1,1)&=&(1,1)\\
(1,0)\times(1,0)&=&(1,0)\\
(0,1)\times(0,1)&=&(0,1)\\
(0,0)\times(0,0)&=&(0,0)
\end{eqnarray*}
the four elements that form the ring are idempotent. So $R$ is Boolean.
%%%%%
%%%%%
\end{document}
