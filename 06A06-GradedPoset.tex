\documentclass[12pt]{article}
\usepackage{pmmeta}
\pmcanonicalname{GradedPoset}
\pmcreated{2013-03-22 14:09:12}
\pmmodified{2013-03-22 14:09:12}
\pmowner{mps}{409}
\pmmodifier{mps}{409}
\pmtitle{graded poset}
\pmrecord{9}{35571}
\pmprivacy{1}
\pmauthor{mps}{409}
\pmtype{Definition}
\pmcomment{trigger rebuild}
\pmclassification{msc}{06A06}
\pmclassification{msc}{05B35}
%\pmkeywords{poset}
%\pmkeywords{graded}
%\pmkeywords{rank}
\pmrelated{EulerianPoset}
\pmrelated{StarProduct}
\pmrelated{HeightOfAnElementInAPoset}
\pmdefines{rank function}
\pmdefines{saturated chain}

\endmetadata

% this is the default PlanetMath preamble.  as your knowledge
% of TeX increases, you will probably want to edit this, but
% it should be fine as is for beginners.

% almost certainly you want these
\usepackage{amssymb}
\usepackage{amsmath}
\usepackage{amsfonts}

% used for TeXing text within eps files
%\usepackage{psfrag}
% need this for including graphics (\includegraphics)
%\usepackage{graphicx}
% for neatly defining theorems and propositions
%\usepackage{amsthm}
% making logically defined graphics
%%\usepackage{xypic}

% there are many more packages, add them here as you need them

% define commands here
\begin{document}
\PMlinkescapeword{satisfy}
\PMlinkescapeword{properties}
\PMlinkescapeword{rank}
\PMlinkescapeword{theory}
\PMlinkescapeword{saturated}
\PMlinkescapeword{property}

A \emph{graded poset} is a poset $P$ that is equipped with a \emph{rank function} $\rho$, which is a function from $P$ to $\mathbb{Z}$, satisfying the following three conditions:  
\begin{enumerate}
\item $\rho$ is constant on all minimal elements of $P$, usually with value $-1$ or $0$
\item $\rho$ is \emph{isotone}, that is, if $a\le b$, then $\rho(a)\le\rho(b)$, and
\item $\rho$ preserves covering relations: if $a\prec b$, then $\rho(a)+1=\rho(b)$.
\end{enumerate}

Equivalently, a poset $P$ is graded if it admits a partition into
maximal antichains $\{A_n\mid n\in\mathbb{N}\}$ such that 
for each $x\in A_n$,
all of the elements covering $x$ are in $A_{n+1}$ and all the elements
covered by $x$ are in $A_{n-1}$.

A poset $P$ \emph{can be graded} if one can define a rank function $\rho$ on $P$ so $(P,\rho)$ is a graded poset.  Below is a poset that can not be graded:

\begin{equation*}
\entrymodifiers={[o]} \xymatrix @!=1pt {
& & \circ \ar@{-}[ld] \ar@{-}[rd] & \\
& \ar@{-}[ld] & & \circ \ar@{-}[d] \\
\circ \ar@{-}[rd] & & & \ar@{-}[d] \\
& \ar@{-}[rd] & & \circ \ar@{-}[ld] \\
& & \circ & }
\end{equation*}

\subsection*{Generalized rank functions}
Since certain common posets such as the face lattice of a polytope are most naturally graded by \PMlinkname{dimension}{Dimension2}, the rank of a minimal element is sometimes required to be $-1$.  

More generally, given a chain $C$, one can define \emph{$C$-graded posets}.  A poset $P$ is $C$-graded provided that there is a poset map $\rho\colon P\to C$ that preserves covers and is constant on minimal elements of $P$.  Such a rank function is unique up to choice of the rank of minimal elements.  In practice, however, the term graded is only used to indicate $\mathbb{N}$-grading, $\mathbb{N}\cup\{-1\}$-grading, or $\mathbb{Z}$-grading.

\subsection*{Maximal chains in graded posets}
Let $P$ be a graded poset with rank function $\rho$.  A chain $C$ in $P$ is said to be a \emph{saturated chain} provided that $\rho(C)=\rho(P)$.  If $C$ is saturated in $P$, then each cover relation in $C$ is also a cover relation in $P$; thus a saturated chain is also a maximal chain.  

It is a property of graded posets that all saturated chains have the same cardinality.  As a partial converse, if $P$ is a finite \PMlinkname{bounded poset}{BoundedLattice} and each maximal chain has the same cardinality, then $P$ is graded.
%%%%%
%%%%%
\end{document}
