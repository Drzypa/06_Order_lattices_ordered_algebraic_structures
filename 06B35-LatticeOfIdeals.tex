\documentclass[12pt]{article}
\usepackage{pmmeta}
\pmcanonicalname{LatticeOfIdeals}
\pmcreated{2013-03-22 16:59:40}
\pmmodified{2013-03-22 16:59:40}
\pmowner{CWoo}{3771}
\pmmodifier{CWoo}{3771}
\pmtitle{lattice of ideals}
\pmrecord{13}{39275}
\pmprivacy{1}
\pmauthor{CWoo}{3771}
\pmtype{Definition}
\pmcomment{trigger rebuild}
\pmclassification{msc}{06B35}
\pmclassification{msc}{14K99}
\pmclassification{msc}{16D25}
\pmclassification{msc}{11N80}
\pmclassification{msc}{13A15}
\pmrelated{SumOfIdeals}
\pmrelated{LatticeIdeal}
\pmrelated{IdealCompletionOfAPoset}

\endmetadata

\usepackage{amssymb,amscd}
\usepackage{amsmath}
\usepackage{amsfonts}
\usepackage{mathrsfs}

% used for TeXing text within eps files
%\usepackage{psfrag}
% need this for including graphics (\includegraphics)
%\usepackage{graphicx}
% for neatly defining theorems and propositions
\usepackage{amsthm}
% making logically defined graphics
%%\usepackage{xypic}
\usepackage{pst-plot}
\usepackage{psfrag}

% define commands here
\newtheorem{prop}{Proposition}
\newtheorem{thm}{Theorem}
\newtheorem{cor}{Corollary}
\newtheorem{ex}{Example}
\newcommand{\real}{\mathbb{R}}
\newcommand{\pdiff}[2]{\frac{\partial #1}{\partial #2}}
\newcommand{\mpdiff}[3]{\frac{\partial^#1 #2}{\partial #3^#1}}
\begin{document}
Let $R$ be a ring.  Consider the set $L(R)$ of all left ideals of $R$.  Order this set by inclusion, and we have a partially ordered set.  In fact, we have the following:

\begin{prop}  $L(R)$ is a complete lattice. \end{prop}
\begin{proof}
For any collection $S=\lbrace J_i\mid i\in I\rbrace$ of (left) ideals of $R$ ($I$ is an index set), define $$\bigwedge S:=\bigcap S\qquad\mbox{and}\qquad\bigvee S=\sum_i J_i,$$ the sum of ideals $J_i$.  We assert that $\bigwedge S$ is the greatest lower bound of the $J_i$, and $\bigvee S$ the least upper bound of the $J_i$, and we show these facts separately
\begin{itemize}
\item
First, $\bigwedge S$ is a left ideal of $R$: if $a,b\in \bigwedge S$, then $a,b\in J_i$ for all $i\in I$.  Consequently, $a-b\in J_i$ and so $a-b\in \bigwedge S$.  Furthermore, if $r\in R$, then $ra\in J_i$ for any $i\in I$, so $ra\in \bigwedge S$ also.  Hence $\bigwedge S$ is a left ideal.  By construction, $\bigwedge S$ is clearly contained in all of $J_i$, and is clearly the largest such ideal.
\item
For the second part, we want to show that $\bigvee S$ actually exists for arbitrary $S$.  We know the existence of $\bigvee S$ if $S$ is finite.  Suppose now $S$ is infinite.  Define $J$ to be the set of finite sums of elements of $\bigcup_i J_i$.  If $a,b\in J$, then $a+b$, being a finite sum itself, clearly belongs to $J$.  Also, $-a\in J$ as well, since the additive inverse of each of the additive components of $a$ is an element of $\bigcup_i J_i$.  Now, if $r\in R$, then $ra\in J$ too, since multiplying each additive component of $a$ by $r$ (on the left) lands back in $\bigcup_i J_i$.  So $J$ is a left ideal.  It is evident that $J_i\subseteq J$.  Also, if $M$ is a left ideal containing each $J_i$, then any finite sum of elements of $J_i$ must also be in $M$, hence $J\subseteq M$.  This implies that $J$ is the smallest ideal containing each of the $J_i$.  Therefore $S$ exists and is equal to $J$.
\end{itemize}
In summary, both $\bigvee S$ and $\bigwedge S$ are well-defined, and exist for finite $S$, so $L(R)$ is a lattice.  Additionally, both operations work for arbitrary $S$, so $L(R)$ is complete.
\end{proof}

From the above proof, we see that the sum $S$ of ideals $J_i$ can be equivalently interpreted as
\begin{itemize}
\item the ``ideal'' of finite sums of the elements of $J_i$, or
\item the ``ideal'' generated by (elements of) $J_i$, or
\item the join of ideals $J_i$.
\end{itemize}

A special sublattice of $L(R)$ is the lattice of finitely generated ideals of $R$.  It is not hard to see that this sublattice comprises precisely the compact elements in $L(R)$.

Looking more closely at the above proof, we also have the following:
\begin{cor}  $L(R)$ is an algebraic lattice. \end{cor}
\begin{proof}
As we have already shown, $L(R)$ is a complete lattice.  If $J$ is any (left) ideal of $R$, by the previous remark, each $J$ is the sum (or join) of ideals generated by individual elements of $J$.  Since these ideals are principal ideals (generated by a single element), they are compact, and therefore $L(R)$ is algebraic.
\end{proof}

\textbf{Remarks}.  
\begin{itemize}
\item
One can easily reconstruct all of the above, if $L(R)$ is the set of \emph{right ideals}, or even \emph{two-sided ideals} of $R$.  We may distinguish the three notions: $l.L(R),r.L(R),$ and $L(R)$ as the lattices of left, right, and two-sided ideals of $R$.
\item
When $R$ is commutative, $l.L(R)=r.L(R)=L(R)$.  Furthermore, it can also be shown that $L(R)$ has the additional structure of a quantale.
\item
There is also a related result on lattice theory: the set $\operatorname{Id}(L)$ of lattice ideals in a upper semilattice $L$ with bottom $0$ forms a complete lattice.  For a proof of this, see \PMlinkname{this entry}{IdealCompletionOfAPoset}.
\item
However, the more general case is not true: the set of order ideals in a poset is a dcpo.
\end{itemize}
%%%%%
%%%%%
\end{document}
