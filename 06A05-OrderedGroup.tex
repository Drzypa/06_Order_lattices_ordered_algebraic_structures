\documentclass[12pt]{article}
\usepackage{pmmeta}
\pmcanonicalname{OrderedGroup}
\pmcreated{2013-03-22 14:54:36}
\pmmodified{2013-03-22 14:54:36}
\pmowner{pahio}{2872}
\pmmodifier{pahio}{2872}
\pmtitle{ordered group}
\pmrecord{16}{36595}
\pmprivacy{1}
\pmauthor{pahio}{2872}
\pmtype{Definition}
\pmcomment{trigger rebuild}
\pmclassification{msc}{06A05}
\pmclassification{msc}{20F60}
\pmrelated{KrullValuation}
\pmrelated{PartiallyOrderedGroup}
\pmrelated{PraeclarumTheorema}
\pmdefines{ordered group equipped with zero}

\endmetadata

% this is the default PlanetMath preamble.  as your knowledge
% of TeX increases, you will probably want to edit this, but
% it should be fine as is for beginners.

% almost certainly you want these
\usepackage{amssymb}
\usepackage{amsmath}
\usepackage{amsfonts}

% used for TeXing text within eps files
%\usepackage{psfrag}
% need this for including graphics (\includegraphics)
%\usepackage{graphicx}
% for neatly defining theorems and propositions
 \usepackage{amsthm}
% making logically defined graphics
%%%\usepackage{xypic}

% there are many more packages, add them here as you need them

% define commands here

\theoremstyle{definition}
\newtheorem{thmplain}{Theorem}
\begin{document}
\textbf{Definition 1.}\, We say that the subsemigroup $S$ of the group $G$ (with the operation denoted multiplicatively) defines an \PMlinkescapetext{{\em order for the group}} $G$, if 
\begin{itemize}
 \item $a^{-1}Sa \subseteq S \quad \forall a\in G,$
 \item $G = S\cup \{1\} \cup S^{-1}$\,\, where \,$S^{-1} = \{s^{-1}: \,s\in S\}$\, and the members of the union are pairwise disjoint.
\end{itemize}


The order ``$<$'' of the group $G$ is explicitly given by setting in $G$:
$$a < b \,\, \Leftrightarrow \,\,ab^{-1}\in S$$
Then we speak of the {\em ordered group}\, $(G,\,<)$,\, or simply $G$.\\

\begin{thmplain}
\,\,The order ``$<$'' defined by the subsemigroup $S$ of the group $G$ has the following properties.
\begin{enumerate}
 \item For all\, $a,\,b\in G$, exactly one of the conditions\,\, $a < b,\,\,a = b,\,\,b < a$\,\, holds.
 \item $a < b \,\land\, b < c \,\,\Rightarrow\,\,a < c$
 \item $a < b \,\,\Rightarrow\,\, ac < bc \,\land\, ca < cb$ 
 \item $a < b \,\land\, c < d \,\,\Rightarrow\,\, ac < bd$
 \item $a < b \,\,\Leftrightarrow\,\, b^{-1} < a^{-1}$
 \item $a < 1 \,\,\Leftrightarrow\,\, a\in S$
\end{enumerate}
\end{thmplain}


\textbf{Definition 2.}\, The set $G$ is an {\em ordered group equipped with zero} 0, if the set $G^*$ of its elements distinct from its element 0 forms an ordered group\, $(G^*,\,<)$\, and if
\begin{itemize}
 \item $0a = a0 = 0 \quad\forall a\in G,$
 \item $0 < a \quad\forall a\in G^*.$
\end{itemize}

Cf. 7 in examples of semigroups.

\begin{thebibliography}{9}
\bibitem{artin} {\sc Emil Artin}: {\em Theory of Algebraic Numbers}.\, Lecture notes. \,Mathematisches Institut, G\"ottingen (1959).
\bibitem{Jaffard} {\sc Paul Jaffard}: {\em Les syst\`emes d'id\'eaux}.\, Dunod, Paris (1960).
\end{thebibliography}
%%%%%
%%%%%
\end{document}
