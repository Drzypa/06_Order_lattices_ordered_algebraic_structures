\documentclass[12pt]{article}
\usepackage{pmmeta}
\pmcanonicalname{PurePoset}
\pmcreated{2013-03-22 17:19:47}
\pmmodified{2013-03-22 17:19:47}
\pmowner{mps}{409}
\pmmodifier{mps}{409}
\pmtitle{pure poset}
\pmrecord{4}{39682}
\pmprivacy{1}
\pmauthor{mps}{409}
\pmtype{Definition}
\pmcomment{trigger rebuild}
\pmclassification{msc}{06A06}

\endmetadata

% this is the default PlanetMath preamble.  as your knowledge
% of TeX increases, you will probably want to edit this, but
% it should be fine as is for beginners.

% almost certainly you want these
\usepackage{amssymb}
\usepackage{amsmath}
\usepackage{amsfonts}

% used for TeXing text within eps files
%\usepackage{psfrag}
% need this for including graphics (\includegraphics)
%\usepackage{graphicx}
% for neatly defining theorems and propositions
%\usepackage{amsthm}
% making logically defined graphics
%%%\usepackage{xypic}

% there are many more packages, add them here as you need them

% define commands here
\newcommand{\fm}[1]{{\it #1}}
\begin{document}
A poset is \emph{pure} if it is finite and every maximal chain has the same length.
If \fm{P} is a pure poset, we can create a rank function \fm{r} on \fm{P} by defining
\fm{r}(\fm{x}) to be the length of a maximal chain bounded above by \fm{x}.  Every
interval of a pure poset is a graded poset, and every graded poset is pure.  Moreover,
the closure of a pure poset, formed by adjoining an artificial minimum element and 
an artificial maximum element, is always graded.

The face poset of a pure simplicial complex is pure as a poset.

% add more stuff

\PMlinkescapeword{closure}
\PMlinkescapeword{face} % need to create face poset
%%%%%
%%%%%
\end{document}
