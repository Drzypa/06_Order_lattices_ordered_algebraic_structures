\documentclass[12pt]{article}
\usepackage{pmmeta}
\pmcanonicalname{KuroshOreTheorem}
\pmcreated{2013-03-22 18:10:11}
\pmmodified{2013-03-22 18:10:11}
\pmowner{CWoo}{3771}
\pmmodifier{CWoo}{3771}
\pmtitle{Kurosh-Ore theorem}
\pmrecord{6}{40731}
\pmprivacy{1}
\pmauthor{CWoo}{3771}
\pmtype{Theorem}
\pmcomment{trigger rebuild}
\pmclassification{msc}{06D05}
\pmclassification{msc}{06C05}
\pmclassification{msc}{06B05}

\usepackage{amssymb,amscd}
\usepackage{amsmath}
\usepackage{amsfonts}
\usepackage{mathrsfs}

% used for TeXing text within eps files
%\usepackage{psfrag}
% need this for including graphics (\includegraphics)
%\usepackage{graphicx}
% for neatly defining theorems and propositions
\usepackage{amsthm}
% making logically defined graphics
%%\usepackage{xypic}
\usepackage{pst-plot}

% define commands here
\newcommand*{\abs}[1]{\left\lvert #1\right\rvert}
\newtheorem{prop}{Proposition}
\newtheorem{thm}{Theorem}
\newtheorem{ex}{Example}
\newcommand{\real}{\mathbb{R}}
\newcommand{\pdiff}[2]{\frac{\partial #1}{\partial #2}}
\newcommand{\mpdiff}[3]{\frac{\partial^#1 #2}{\partial #3^#1}}
\begin{document}
\begin{thm}[Kurosh-Ore]  Let $L$ be a modular lattice and suppose that $a\in L$ has two irredundant decompositions of joins of join-irreducible elements:
$$a=x_1\vee \cdots \vee x_m = y_1 \vee \cdots \vee y_n.$$
Then
\begin{enumerate}
\item $m=n$, and
\item every $x_i$ can be replaced by some $y_j$, so that $$a= x_1\vee \cdots \vee x_{i-1} \vee y_j \vee x_{i+1} \vee \cdots \vee x_m.$$
\end{enumerate}
\end{thm}

There is also a dual statement of the above theorem in terms of meets.

\textbf{Remark}.  Additionally, if $L$ is a distributive lattice, then the second property above (known the \emph{replacement property}) can be strengthened: each $x_i$ is equal to some $y_j$.  In other words, except for the re-ordering of elements in the decomposition, the above join is unique.
%%%%%
%%%%%
\end{document}
