\documentclass[12pt]{article}
\usepackage{pmmeta}
\pmcanonicalname{DualSpaceOfABooleanAlgebra}
\pmcreated{2013-03-22 19:08:35}
\pmmodified{2013-03-22 19:08:35}
\pmowner{CWoo}{3771}
\pmmodifier{CWoo}{3771}
\pmtitle{dual space of a Boolean algebra}
\pmrecord{6}{42043}
\pmprivacy{1}
\pmauthor{CWoo}{3771}
\pmtype{Definition}
\pmcomment{trigger rebuild}
\pmclassification{msc}{06E05}
\pmclassification{msc}{03G05}
\pmclassification{msc}{06B20}
\pmclassification{msc}{03G10}
\pmclassification{msc}{06E20}
\pmrelated{StoneRepresentationTheorem}
\pmrelated{MHStonesRepresentationTheorem}

\usepackage{amssymb,amscd}
\usepackage{amsmath}
\usepackage{amsfonts}
\usepackage{mathrsfs}

% used for TeXing text within eps files
%\usepackage{psfrag}
% need this for including graphics (\includegraphics)
%\usepackage{graphicx}
% for neatly defining theorems and propositions
\usepackage{amsthm}
% making logically defined graphics
%%\usepackage{xypic}
\usepackage{pst-plot}

% define commands here
\newcommand*{\abs}[1]{\left\lvert #1\right\rvert}
\newtheorem{prop}{Proposition}
\newtheorem{thm}{Theorem}
\newtheorem{ex}{Example}
\newcommand{\real}{\mathbb{R}}
\newcommand{\pdiff}[2]{\frac{\partial #1}{\partial #2}}
\newcommand{\mpdiff}[3]{\frac{\partial^#1 #2}{\partial #3^#1}}
\begin{document}
Let $B$ be a Boolean algebra, and $B^*$ the set of all maximal ideals of $B$.  In this entry, we will equip $B^*$ with a topology so it is a Boolean space.

\textbf{Definition}.  For any $a\in B$, define $M(a):=\lbrace M\in B^* \mid a\notin M\rbrace$, and $\mathcal{B}:=\lbrace M(a)\mid a\in B\rbrace$.

It is know that in a Boolean algebra, maximal ideals and prime ideals coincide.  From \PMlinkname{this entry}{RepresentingABooleanLatticeByFieldOfSets}, we have the three following properties concerning $M(a)$:
$$M(a)\cap M(b)=M(a\wedge b), \qquad M(a)\cup M(b)=M(a\vee b), \qquad B^*-M(a)=M(a').$$
Furthermore, if $M(a)=M(b)$, then $a=b$.

From these properties, we see that $M(0)=\varnothing$ and $M(1)=B^*$.  As a result, we see that
\begin{prop} $B^*$ is a topological space, whose topology $\mathcal{T}$ is generated by the basis $\mathcal{B}$. \end{prop}
\begin{proof} $\varnothing$ and $B^*$ are both open, as they are $M(0)$ and $M(1)$ respectively.  Also, the intersection of open sets $M(a)$ and $M(b)$ is again open, since it is $M(a\wedge b)$. \end{proof}

We may in fact treat $\mathcal{B}$ as a subbasis for $\mathcal{T}$, since finite intersections of elements of $\mathcal{B}$ remain in $\mathcal{B}$.

\begin{prop} Each member of $\mathcal{B}$ is closed, hence $\mathcal{T}$ is generated by a basis of clopen sets.  In other words, $B^*$ is zero-dimensional.  \end{prop}
\begin{proof} Each $M(a)$ is open, by definition, and closed, since it is the complement of the open set $M(a')$.  \end{proof}

\begin{prop} $B^*$ is Hausdorff.  \end{prop}
\begin{proof} If $M,N\in B^*$ such that $M\ne N$, then there is some $a\in B$ such that $a\in M$ and $a\notin N$.  This means that $N\in M(a)$ and $M\notin M(a)$, which means that $M\in B^*-M(a)=M(a')$.  Since $M(a)$ and $M(a')$ are open and disjoint, with $N\in M(a)$ and $M\in M(a')$, we see that $B^*$ is Hausdorff. \end{proof}

Now, based on a topological fact, every zero-dimensional Hausdorff space is totally disconnected.  Hence $B^*$ is totally disconnected.

\begin{prop} $B^*$ is compact. \end{prop}
\begin{proof} Suppose $\lbrace U_i \mid i\in I\rbrace$ is a collection of open sets whose union is $B^*$.  Since each $U_i$ is a union of elements of $\mathcal{B}$, we might as well assume that $B^*$ is covered by elements of $\mathcal{B}$.  In other words, we may assume that each $U_i$ is some $M(a_i)\in \mathcal{B}$.  

Let $J$ be the ideal generated by the set $\lbrace a_i\mid i\in I\rbrace$.  If $J\ne B$, then $J$ can be extended to a maximal ideal $M$.  Since each $a_i\in M$, we see that $M\notin M(a_i)$, so that $M \notin \bigcup \lbrace M(a_i) \mid i\in I\rbrace = B^*$, which is a contradiction.  Therefore, $J=B$.  In particular, $1\in J$, which means that $1$ can be expressed as the join of a finite number of the $a_i$'s: $$1= \bigvee \lbrace a_i \mid i \in K\rbrace,$$ where $K$ is a finite subset of $J$.  As a result, we have $$\bigcup \lbrace M(a_i)\mid i\in K\rbrace = M(\bigvee \lbrace a_i \mid i\in K \rbrace )=M(1)=B^*.$$
So $B^*$ has a finite subcover, and hence is compact.
\end{proof}

Collecting the last three results, we see that $B^*$ is a Boolean space.

\textbf{Remark}.  It can be shown that $B$ is isomorphic to the Boolean algebra of clopen sets in $B^*$.  This is the famous Stone representation theorem.
%%%%%
%%%%%
\end{document}
