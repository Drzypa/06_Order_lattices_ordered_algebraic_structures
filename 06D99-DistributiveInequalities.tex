\documentclass[12pt]{article}
\usepackage{pmmeta}
\pmcanonicalname{DistributiveInequalities}
\pmcreated{2013-03-22 16:37:48}
\pmmodified{2013-03-22 16:37:48}
\pmowner{CWoo}{3771}
\pmmodifier{CWoo}{3771}
\pmtitle{distributive inequalities}
\pmrecord{5}{38830}
\pmprivacy{1}
\pmauthor{CWoo}{3771}
\pmtype{Derivation}
\pmcomment{trigger rebuild}
\pmclassification{msc}{06D99}
\pmrelated{ModularInequality}

\usepackage{amssymb,amscd}
\usepackage{amsmath}
\usepackage{amsfonts}

% used for TeXing text within eps files
%\usepackage{psfrag}
% need this for including graphics (\includegraphics)
%\usepackage{graphicx}
% for neatly defining theorems and propositions
\usepackage{amsthm}
% making logically defined graphics
%%\usepackage{xypic}
\usepackage{pst-plot}
\usepackage{psfrag}

% define commands here

\begin{document}
Let $L$ be a lattice.  Then for $a,b,c\in L$, we have the following inequalities:

\begin{enumerate}
\item $a\vee (b\wedge c)\le (a\vee b)\wedge (a\vee c)$,
\item $(a\wedge b)\vee (a\wedge c)\le a\wedge (b\vee c)$.
\end{enumerate}

\begin{proof}
Since $a\le a\vee b$ and $a\le a\vee c$, $a\le (a\vee b)\wedge (a\vee c)$.
Similarly, $b\wedge c \le b\le a\vee b$ and $b\wedge c\le c\le a\vee c$ imply 
$b\wedge c\le (a\vee b)\wedge (a\vee c)$.
Together, we have $a\vee (b\wedge c)\le (a\vee b)\wedge (a\vee c)$.

The second inequality is the dual of the first one.
\end{proof}

The two inequalities above are called the \emph{distributive inequalities}.

\textbf{Proposition}  A lattice $L$ is a distributive lattice if one of the following inequalities holds:
\begin{enumerate}
\item $(a\vee b)\wedge (a\vee c)\le a\vee (b\wedge c)$,
\item $a\wedge (b\vee c)\le (a\wedge b)\vee (a\wedge c)$.
\end{enumerate}

\begin{proof}
By the distributive inequalities, all we need to show is that 1. implies 2.  (that 2. implies 1. is just the dual statement).  So suppose 1. holds.  Then 
\begin{alignat*}{2}
(a\wedge b)\vee (a\wedge c) &\ge ((a\wedge b)\vee a)\wedge ((a\wedge b)\vee c) & \qquad\mbox{by assumption}\\ 
&= a \wedge ((a\wedge b)\vee c) & \qquad\mbox{by absorption}\\ 
&\ge a\wedge ((c\vee a)\wedge (c\vee b)) & \qquad\mbox{by assumption}\\ 
&= (a\wedge (c\vee a))\wedge (c\vee b) & \qquad\mbox{meet associativity}\\ 
&= a \wedge (c\vee b). & \qquad\mbox{by absorption}
\end{alignat*}
\end{proof}
%%%%%
%%%%%
\end{document}
