\documentclass[12pt]{article}
\usepackage{pmmeta}
\pmcanonicalname{DescendingOrder}
\pmcreated{2013-03-22 16:06:49}
\pmmodified{2013-03-22 16:06:49}
\pmowner{CompositeFan}{12809}
\pmmodifier{CompositeFan}{12809}
\pmtitle{descending order}
\pmrecord{5}{38179}
\pmprivacy{1}
\pmauthor{CompositeFan}{12809}
\pmtype{Definition}
\pmcomment{trigger rebuild}
\pmclassification{msc}{06A99}
\pmrelated{AscendingOrder}
\pmdefines{strictly descending order}

\endmetadata

% this is the default PlanetMath preamble.  as your knowledge
% of TeX increases, you will probably want to edit this, but
% it should be fine as is for beginners.

% almost certainly you want these
\usepackage{amssymb}
\usepackage{amsmath}
\usepackage{amsfonts}

% used for TeXing text within eps files
%\usepackage{psfrag}
% need this for including graphics (\includegraphics)
%\usepackage{graphicx}
% for neatly defining theorems and propositions
%\usepackage{amsthm}
% making logically defined graphics
%%%\usepackage{xypic}

% there are many more packages, add them here as you need them

% define commands here

\begin{document}
A sequence or arbitrary ordered set or one-dimensional array of numbers, $a$, is said to be in {\em descending order} if each $a_i \ge a_{i + 1}$. For example, the aliquot sequence of 259 is in descending order: 45, 33, 15, 9, 4, 3, 1, 0, 0, 0 ... The aliquot sequence starting at 60, however, is not in descending order: 108, 172, 136, 134, 70, 74, 40, 50, 43, 1, 0, 0, 0 ...

In a trivial sense, the sequence of values of the sign function multiplied by -1 is in descending order: ... 1, 1, 1, 0, --1, --1, --1... When each $a_i > a_{i + 1}$ in the sequence, set or array, then it can be said to be in {\em strictly descending order}.
%%%%%
%%%%%
\end{document}
