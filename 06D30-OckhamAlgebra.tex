\documentclass[12pt]{article}
\usepackage{pmmeta}
\pmcanonicalname{OckhamAlgebra}
\pmcreated{2013-03-22 17:08:34}
\pmmodified{2013-03-22 17:08:34}
\pmowner{CWoo}{3771}
\pmmodifier{CWoo}{3771}
\pmtitle{Ockham algebra}
\pmrecord{9}{39450}
\pmprivacy{1}
\pmauthor{CWoo}{3771}
\pmtype{Definition}
\pmcomment{trigger rebuild}
\pmclassification{msc}{06D30}

\usepackage{amssymb,amscd}
\usepackage{amsmath}
\usepackage{amsfonts}
\usepackage{mathrsfs}

% used for TeXing text within eps files
%\usepackage{psfrag}
% need this for including graphics (\includegraphics)
%\usepackage{graphicx}
% for neatly defining theorems and propositions
\usepackage{amsthm}
% making logically defined graphics
%%\usepackage{xypic}
\usepackage{pst-plot}
\usepackage{psfrag}

% define commands here
\newtheorem{prop}{Proposition}
\newtheorem{thm}{Theorem}
\newtheorem{ex}{Example}
\newcommand{\real}{\mathbb{R}}
\newcommand{\pdiff}[2]{\frac{\partial #1}{\partial #2}}
\newcommand{\mpdiff}[3]{\frac{\partial^#1 #2}{\partial #3^#1}}
\begin{document}
A lattice $L$ is called an \emph{Ockham algebra} if 
\begin{enumerate}
\item $L$ is distributive
\item $L$ is bounded, with $0$ as the bottom and $1$ as the top
\item there is a unary operator $\neg$ on $L$ with the following properties:
\begin{enumerate}
\item $\neg$ satisfies the de Morgan's laws; this means that:
\begin{itemize}
\item $\neg (a\vee b)=\neg a\wedge \neg b$ and
\item $\neg (a\wedge b)=\neg a\vee \neg b$
\end{itemize}
\item $\neg 0=1$ and $\neg 1=0$
\end{enumerate}
\end{enumerate}
Such a unary operator is an example of a dual endomorphism.  When applied, $\neg$ interchanges the operations of $\vee$ and $\wedge$, and $0$ and $1$.


An Ockham algebra is a generalization of a Boolean algebra, in the sense that $\neg$ replaces $'$, the complement operator, on a Boolean algebra.

\textbf{Remarks}. 
\begin{itemize}
\item
An intermediate concept is that of a De Morgan algebra, which is an Ockham algebra with the additional requirement that $\neg (\neg a)=a$.
\item
In the category of Ockham algebras, the morphism between any two objects is a $\lbrace 0,1\rbrace$-\PMlinkname{lattice homomorphism}{LatticeHomomorphism} $f$ that preserves $\neg$: $f(\neg a)=\neg f(a)$.  In fact, $f(0)=f(\neg 1)=\neg f(1)=\neg 1=0$, so that it is safe to drop the assumption that $f$ preserves $0$.
\end{itemize}

\begin{thebibliography}{8}
\bibitem{bv} T.S. Blyth, J.C. Varlet, {\em Ockham Algebras}, Oxford University Press, (1994).
\bibitem{tsb} T.S. Blyth, {\em Lattices and Ordered Algebraic Structures}, Springer, New York (2005).
\end{thebibliography}
%%%%%
%%%%%
\end{document}
