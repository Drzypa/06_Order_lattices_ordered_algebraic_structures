\documentclass[12pt]{article}
\usepackage{pmmeta}
\pmcanonicalname{BirkhoffPrimeIdealTheorem}
\pmcreated{2013-03-22 17:02:18}
\pmmodified{2013-03-22 17:02:18}
\pmowner{CWoo}{3771}
\pmmodifier{CWoo}{3771}
\pmtitle{Birkhoff prime ideal theorem}
\pmrecord{11}{39325}
\pmprivacy{1}
\pmauthor{CWoo}{3771}
\pmtype{Theorem}
\pmcomment{trigger rebuild}
\pmclassification{msc}{06D05}
\pmclassification{msc}{03E25}
\pmrelated{DistributiveLattice}

\usepackage{amssymb,amscd}
\usepackage{amsmath}
\usepackage{amsfonts}
\usepackage{mathrsfs}

% used for TeXing text within eps files
%\usepackage{psfrag}
% need this for including graphics (\includegraphics)
%\usepackage{graphicx}
% for neatly defining theorems and propositions
\usepackage{amsthm}
% making logically defined graphics
%%\usepackage{xypic}
\usepackage{pst-plot}
\usepackage{psfrag}

% define commands here
\newtheorem{prop}{Proposition}
\newtheorem{thm}{Theorem}
\newtheorem{ex}{Example}
\newcommand{\real}{\mathbb{R}}
\newcommand{\pdiff}[2]{\frac{\partial #1}{\partial #2}}
\newcommand{\mpdiff}[3]{\frac{\partial^#1 #2}{\partial #3^#1}}
\begin{document}
\textbf{Birkhoff Prime Ideal Theorem}.  Let $L$ be a distributive lattice and $I$ a proper lattice ideal of $L$.  Pick any element $a\notin I$.  Then there is a prime ideal $P$ in $L$ such that $I\subseteq P$ and $a\notin P$.

\begin{proof}
If $I$ is prime, then we are done.  Let $S:=\lbrace J\mid J\mbox{ is an ideal in }L\mbox{, and }a\notin J\rbrace$.  Then $I\in S$. Order $S$ by inclusion.  This turns $S$ into a poset.  Let $C$ be a chain in $S$.  Let $K=\bigcup C$.  If $x,y\in K$, then $x\in J_1$ and $y\in J_2$ for some ideals $J_1,J_2\in C$.  Since $C$ is a chain, we may assume that $J_1\subseteq J_2$, so that $x\in J_2$ as well.  This means $x\vee y\in J_2\subseteq K$.  Next, assume $x\in K$ and $y\le x$.  Then $x\in J$ for some ideal $J\in C$, so that $y\in J\subseteq K$ also.  This shows that $K$ is an ideal.  If $a\in K$, then $a\in J$ for some $J\in C\subseteq S$, contradicting the definition of $S$.  So $a\notin K$ and $K\in S$ also.  This shows that every chain in $S$ has an upper bound.  We can now appeal to Zorn's lemma, and conclude that $S$ has a maximal element, say $P$.

We now want to show that $P$ is the candidate that we are seeking: $P$ is a prime ideal in $L$ and $a\notin P$.  Since $P\in S$, $P$ is an ideal such that $a\notin P$.  So the only thing left to prove is that $P$ is prime.  This amounts to showing that if $x\wedge y\in P$, then $x\in P$ or $y\in P$.  Suppose not: $x,y\notin P$.  Let $Q_1$ be the ideal generated by elements of $P$ and $x$, and $Q_2$ the ideal generated by $P$ and $y$.  Since $Q_1$ and $Q_2$ properly contain $P$, $a\in Q_1$ and $a\in Q_2$.  Write $a\le p_1\vee x$ and $a\le p_2\vee y$, where $p_1,p_2\in P$.  Then $a\vee p_2\le (p_1\vee p_2)\vee x$ and $a\vee p_1\le (p_1\vee p_2)\vee y$.  Take the meet of these two expressions, and we obtain $(a\vee p_2) \wedge (a\vee p_1)\le ((p_1\vee p_2)\vee x)\wedge((p_1\vee p_2)\vee y)$.  Since $L$ is distributive, on the left hand side, we get $a\vee (p_1\wedge p_2)$.  On the right hand side, we have $(p_1\vee p_2)\vee (x\wedge y)\in P$.  As the left hand side is less than or equal to the right hand side, we get that $a\vee (p_1\wedge p_2)\in P$.  Since $a\le a\vee (p_1\wedge p_2)\in P$, $a\in P$, a contradiction.  Therefore, $P$ is prime and the proof is complete.
\end{proof}

In the proof, we use the fact that, an element $a\in L$ belongs to the ideal generated by ideals $I_k$ iff $a$ is less than or equal to a finite join of elements, each of which belongs to some $I_k$.


\textbf{Remarks}.  
\begin{enumerate}
\item
The theorem can be generalized: if we use a subset $S\cap I=\varnothing$ instead of an element $a\notin I$, there is a prime ideal $P$ containing $I$ but excluding $S$.
\item
Birkhoff's prime ideal theorem has been shown to be equivalent to the axiom of choice, under ZF.
\end{enumerate}
%%%%%
%%%%%
\end{document}
