\documentclass[12pt]{article}
\usepackage{pmmeta}
\pmcanonicalname{PropertiesOfWellorderedSets}
\pmcreated{2013-03-22 15:23:42}
\pmmodified{2013-03-22 15:23:42}
\pmowner{GrafZahl}{9234}
\pmmodifier{GrafZahl}{9234}
\pmtitle{properties of well-ordered sets}
\pmrecord{10}{37231}
\pmprivacy{1}
\pmauthor{GrafZahl}{9234}
\pmtype{Theorem}
\pmcomment{trigger rebuild}
\pmclassification{msc}{06A05}
\pmsynonym{initial segment}{PropertiesOfWellorderedSets}
\pmdefines{section}

\endmetadata

% this is the default PlanetMath preamble.  as your knowledge
% of TeX increases, you will probably want to edit this, but
% it should be fine as is for beginners.

% almost certainly you want these
\usepackage{amssymb}
\usepackage{amsmath}
\usepackage{amsfonts}

% used for TeXing text within eps files
%\usepackage{psfrag}
% need this for including graphics (\includegraphics)
%\usepackage{graphicx}
% for neatly defining theorems and propositions
\usepackage{amsthm}
% making logically defined graphics
%%%\usepackage{xypic}

% there are many more packages, add them here as you need them

% define commands here
\newcommand{\<}{\langle}
\renewcommand{\>}{\rangle}
\newcommand{\Bigcup}{\bigcup\limits}
\newcommand{\DirectSum}{\bigoplus\limits}
\newcommand{\Prod}{\prod\limits}
\newcommand{\Sum}{\sum\limits}
\newcommand{\h}{\widehat}
\newcommand{\mbb}{\mathbb}
\newcommand{\mbf}{\mathbf}
\newcommand{\mc}{\mathcal}
\newcommand{\mmm}[9]{\left(\begin{array}{rrr}#1&#2&#3\\#4&#5&#6\\#7&#8&#9\end{array}\right)}
\newcommand{\mf}{\mathfrak}
\newcommand{\ol}{\overline}

% Math Operators/functions
\DeclareMathOperator{\Aut}{Aut}
\DeclareMathOperator{\End}{End}
\DeclareMathOperator{\Frob}{Frob}
\DeclareMathOperator{\cwe}{cwe}
\DeclareMathOperator{\id}{id}
\DeclareMathOperator{\mult}{mult}
\DeclareMathOperator{\we}{we}
\DeclareMathOperator{\wt}{wt}
\begin{document}
\PMlinkescapeword{induced}
\newtheorem{thm}{Theorem}
\theoremstyle{definition}
\newtheorem{defn}{Definition}
The purpose of this entry is to collect properties of well-ordered
sets. We denote all orderings uniformly by $\leq$. If you are
interested in history, have a look at~\cite{C}.

The following properties are easy to see:
\begin{itemize}
\item If $A$ is a totally ordered set such that for every subset
  $B\subseteq A$ the set of all elements of $A$ strictly greater than
  the elements of $B$,
\begin{equation*}
B_>:=\{a\in A\setminus B\mid b\leq a\text{ for all }b\in B\},
\end{equation*}
is either empty or has a least element, then $A$ is well-ordered.
\item Every subset of a well-ordered set is well-ordered.
\item If $A$ is well-ordered and $B$ is a poset such that there is a
  bijective order morphism $\varphi\colon A\to B$, then $B$ is
  well-ordered.
\end{itemize}

Now we define an important ingredient for understanding the structure
of well-ordered sets.

\begin{defn}[section]
Let $A$ be well-ordered. Then for every $a\in A$ we define the
\emph{section} of $a$:
\begin{equation*}
\h{a}:=\{b\in A\mid b\leq a\}.
\end{equation*}
A section is also known as an \emph{initial segment}.  We denote the set of all sections of $A$ by $\h{A}$. This set is
ordered by inclusion.
\end{defn}

\begin{thm}
\label{thm:AtosectionsofA}
Let $A$ be a well-ordered set.  Then the mapping $\h{\cdot}\colon
A\to\h{A}$ defined by $a\mapsto\h{a}$ is a bijective order morphism.
In particular, $\h{A}$ is well-ordered.
\end{thm}
\begin{proof}
Let $a,b\in A$ with $a\leq b$. Then $\h{a}\subseteq\h{b}$, so
$\h{\cdot}$ is an order morphism. Now assume that $\h{a}=\h{b}$. If
$a$ didn't equal $b$, we would have $b\notin\h{a}$, leading to a
contradiction. Therefore $\h{\cdot}$ is injective. Now let
$C\in\h{A}$, then there exists a $c\in A$ such that $\h{c}=C$, so
$\h{\cdot}$ is surjective.
\end{proof}

\begin{thm}
Let $A$ and $B$ be well-ordered sets and $\varphi\colon A\to B$ a
bijective order morphism. Then there exists a bijective order morphism
$\h{\varphi}\colon\h{A}\to\h{B}$ such that for all $a\in A$
\begin{equation*}
\h{\varphi}(\h{a})=\h{\varphi(a)}.
\end{equation*}
\end{thm}
\begin{proof}
Setting $\h{\varphi}(\h{a}):=\h{\varphi(a)}$ is well-defined by
Theorem~\ref{thm:AtosectionsofA}. The rest of the theorem follows
since $\h{\cdot}$ and $\varphi$ are bijective order morphisms.
\end{proof}

\begin{thm}
\label{thm:similarsection}
Let $A$ be a well-ordered set and $a\in A$ such that there is an
injective order morphism $\varphi\colon A\to\h{a}$. Then $A=\h{a}$.
\end{thm}
\begin{proof}
The image of a section of $A$ under $\varphi$ has a maximal element
which in turn defines a smaller section of $A$. We may therefore
define the following two monotonically decreasing sequences of sets:
\begin{align*}
B_0&:=\varphi(\h{a}),&A_0&:=\text{section defined by maximal element of
}B_0,\\
B_n&:=\varphi(A_{n-1}),&A_n&:=\text{section defined by maximal element
of }B_n.
\end{align*}
Now $\h{A}$ is well-ordered, so the set defined by the elements of the
sequence $(A_n)$ has a minimal element, that is $A_N=A_{N+1}$ and
hence $B_N=B_{N+1}$ for some sufficiently large $N$. Applying
$\varphi^{-1}$ $N+1$ times to the latter equation yields $\h{a}=A_0$,
that is $a$ is the maximal element of $B$, and thus of $A$.
\end{proof}

\begin{thm}
\label{thm:uniquemorphism}
Let $A$ and $B$ be well-ordered sets. Then there exists at most one
bijective order morphism $\varphi\colon A\to B$.
\end{thm}
\begin{proof}
Let $\varphi,\psi\colon A\to B$ be two bijective order morphisms. Let
$a\in A$ and set $b:=\varphi(a)$ and $c:=\psi(a)$. Then the
restrictions $\left.\varphi\right|_{\h{a}}\colon\h{a}\to\h{b}$ and
$\left.\psi\right|_{\h{a}}\colon\h{a}\to\h{c}$ are bijective order
morphisms, so the restriction of $\psi\varphi^{-1}$ to $\h{b}$ is a
bijective order morphism to $\h{c}$. Now either $\h{b}\subseteq\h{c}$
or $\h{c}\subseteq\h{b}$, so by Theorem~\ref{thm:similarsection}
$\h{b}=\h{c}$, hence $b=c$ and thus $\varphi=\psi$.
\end{proof}

\begin{thm}
\label{thm:sectiontosection}
Let $A$ and $B$ be well-ordered sets such that for every section
$\h{a}\in\h{A}$ there is a bijective order morphism to a section
$\h{b}\in\h{B}$ and vice-versa, then there is a bijective order
morphism $\varphi\colon A\to B$.
\end{thm}
\begin{proof}
Let $a\in A$ and let $\h{b}\in\h{B}$ be a section such that there is a
bijective order morphism $\psi_a\colon\h{a}\to\h{b}$. By
Theorem~\ref{thm:similarsection}, $\h{b}$ is unique, and so is
$\psi_a$ by Theorem~\ref{thm:uniquemorphism}. Defining
$\varphi\colon A\to B$ by setting $\varphi(a)=b$ gives therefore a
well-defined (by Theorem~\ref{thm:AtosectionsofA}) and injective order
morphism. But $\varphi$ is also surjective, since any $b\in B$ maps
uniquely to $A$ via $\h{b}\to\h{a}$, and back again by $\varphi$.
\end{proof}

\begin{thm}
\label{thm:comparable}
Let $A$ and $B$ be well-ordered sets. Then there is an injective order
morphism $\iota\colon A\to B$ or $\iota\colon B\to A$. If $\iota$
cannot be chosen bijective, then it can at least be chosen such that
its image is a section.
\end{thm}
\begin{proof}
Let $\h{A}_0$ be the set of sections of $A$ from which there is an
injective order morphism to $B$. If $\h{A}_0$ is the empty set, then
$B$ must be empty, since otherwise we could map the least element of
$A$ to $B$. If $\h{A}_0$ is not empty, we may consider the set
$A_0:=\cap\h{A}_0$. If $A_0=A$, nothing remains to be shown. Otherwise
the set $A\setminus A_0$ is nonempty an hence has a least element
$a_0\in A$. By construction, there is no injective order morphism from
$\h{a}_0$ to $B$, but there is an injective order morphism from
$\varphi_a\colon\h{a}\to B$ for every element $a\in A$ which is
strictly smaller than $a_0$. Now assume there is an element $b\in B$
such that there is no injective order morphism from $\h{b}\to
A$. Then we can similarly construct a least element $b_0\in B$ for
which there is no injective order morphism $\h{b}_0\to A$. Surely,
$b_0$ is greater than all the elements from the images of the
functions $\varphi_a$, but then there is a bijective order
morphism from $\h{a}_0$ to $\h{b}_0$ by
Theorem~\ref{thm:sectiontosection} which is a
contradiction. Therefore, all sections of $B$ and $B$ itself
map injectively and order-preserving to $A$.
\end{proof}

\begin{thm}
Let $A$ be a well-ordered set and $B\subseteq A$ a nonempty
subset. Then there is a bijective order morphism from $B$ to one of
the sets in $\h{A}\cup\{A\}$.
\end{thm}
\begin{proof}
The set $B$ is well-ordered with respect to the order induced by
$A$. Assume a bijective order morphism as stated by the theorem does
not exist. Then, by virtue of Theorem~\ref{thm:comparable}, there is
an injective but not surjective order morphism $\iota\colon A\to B$ whose
image is a section $\h{b}\in\h{B}$. The element $b$ defines a section
in $\h{A}$ which is identical to $A$ by
Theorem~\ref{thm:similarsection}. Thus $\iota$ is surjective which is
a contradiction. 
\end{proof}

\begin{thebibliography}{C}

\bibitem[C]{C} \textsc{G.~Cantor}, Beitr\"{a}ge zur Begr\"{u}ndung der
  transfiniten Mengenlehre (Zweiter Artikel), \emph{Math.\ Ann.}\ 49,
  207--246 (1897).

\end{thebibliography}
%%%%%
%%%%%
\end{document}
