\documentclass[12pt]{article}
\usepackage{pmmeta}
\pmcanonicalname{DualOfDilworthsTheorem}
\pmcreated{2013-03-22 15:01:49}
\pmmodified{2013-03-22 15:01:49}
\pmowner{justice}{4961}
\pmmodifier{justice}{4961}
\pmtitle{dual of Dilworth's theorem}
\pmrecord{7}{36740}
\pmprivacy{1}
\pmauthor{justice}{4961}
\pmtype{Theorem}
\pmcomment{trigger rebuild}
\pmclassification{msc}{06A06}
\pmrelated{DilworthsTheorem}

% this is the default PlanetMath preamble.  as your knowledge
% of TeX increases, you will probably want to edit this, but
% it should be fine as is for beginners.

% almost certainly you want these
\usepackage{amssymb}
\usepackage{amsmath}
\usepackage{amsfonts}

% used for TeXing text within eps files
%\usepackage{psfrag}
% need this for including graphics (\includegraphics)
%\usepackage{graphicx}
% for neatly defining theorems and propositions
\usepackage{amsthm}
\theoremstyle{plain}
\newtheorem{thm}{Theorem}
% making logically defined graphics
%%%\usepackage{xypic}

% there are many more packages, add them here as you need them

% define commands here
\begin{document}
\begin{thm}
Let $P$ be a poset of height $h$. Then $P$ can be partitioned into $h$ antichains and furthermore at least $h$ antichains are required.
\end{thm}

\begin{proof}
Induction on $h$. If $h=1$, then no elements of $P$ are comparable, so $P$ is an antichain. Now suppose that $P$ has height $h\geq 2$ and that the theorem is true for $h-1$. Let $A_1$ be the set of maximal elements of $P$. Then $A_1$ is an antichain in $P$ and $P-A_1$ has height $h-1$ since we have removed precisely one element from every chain. Hence, $P-A_1$ can be paritioned into $h-1$ antichains $A_2,A_3,\dots A_h$. Now we have the partition $A_1\cup A_2\cup\dots\cup A_h$ of $P$ into $h$ antichains as desired.

The necessity of $h$ antichains is trivial by the pigeonhole principle; since $P$ has height $h$, it has a chain of length $h$, and each element of this chain must be placed in a different antichain of our partition.
\end{proof}
%%%%%
%%%%%
\end{document}
