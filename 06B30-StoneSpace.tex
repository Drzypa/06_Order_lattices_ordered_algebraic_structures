\documentclass[12pt]{article}
\usepackage{pmmeta}
\pmcanonicalname{StoneSpace}
\pmcreated{2013-03-22 13:24:23}
\pmmodified{2013-03-22 13:24:23}
\pmowner{CWoo}{3771}
\pmmodifier{CWoo}{3771}
\pmtitle{Stone space}
\pmrecord{19}{33949}
\pmprivacy{1}
\pmauthor{CWoo}{3771}
\pmtype{Definition}
\pmcomment{trigger rebuild}
\pmclassification{msc}{06B30}
\pmclassification{msc}{06E15}
\pmsynonym{Boolean space}{StoneSpace}
\pmrelated{DualityInMathematics}
\pmrelated{DualOfStoneRepresentationTheorem}
\pmdefines{dual algebra}

\endmetadata

% this is the default PlanetMath preamble.  as your knowledge
% of TeX increases, you will probably want to edit this, but
% it should be fine as is for beginners.

% almost certainly you want these
\usepackage{amssymb}
\usepackage{amsmath}
\usepackage{amsfonts}

% used for TeXing text within eps files
%\usepackage{psfrag}
% need this for including graphics (\includegraphics)
%\usepackage{graphicx}
% for neatly defining theorems and propositions
%\usepackage{amsthm}
% making logically defined graphics
%%%\usepackage{xypic}

% there are many more packages, add them here as you need them

% define commands here

\begin{document}
 A \emph{Stone space}, also called a \emph{Boolean space}, is a topological space that is zero-dimensional, \PMlinkname{$T_{0}$}{T0Space} and compact.  Equivalently, a Stone space is a totally disconnected compact Hausdorff space.

Given a stone space $X$, one may associate a Boolean algebra $X^*$ by taking the set of all of its clopen sets.  The set theoretic operations of intersection, union, and complement makes $X^*$ a Boolean algebra.  $X^*$ is known as the \emph{dual algebra} of $X$.

The significance of Stone spaces stems from \emph{Stone duality}:
a pervasive equivalence between the algebraic notions and theorems
of Boolean algebras on one hand, and the topological notions and
theorems of Stone spaces on the other. This equivalence comprises
the content and consequences of M. H. Stone's representation
theorem.

There is a bijective correspondence between the following

\begin{enumerate}
\item The class consisting of all Boolean spaces
\item The class consisting of all Boolean algebras
\item The class consisting of all Boolean rings
\item The class consisting of all prime spectra of von Neumann regular rings
\end{enumerate}

In fact, viewing each class as a category equipped with the appropriate class of morphisms, the categories are naturally equivalent between one another.

More to come: partial proofs, constructions, categorical equivalence between the first three items due to Stone's Representation Theorem and references $\dots$.

\begin{thebibliography}{2}
\bibitem{halmos} Paul R. Halmos, \emph{Lectures on Boolean Algebras}, D. Van Nostrand Company, Inc., 1963.
\bibitem{johnstone} Peter T. Johnstone, \emph{Stone Spaces}, Cambridge University Press, 1982.
\end{thebibliography}
%%%%%
%%%%%
\end{document}
