\documentclass[12pt]{article}
\usepackage{pmmeta}
\pmcanonicalname{PartiallyOrderedAlgebraicSystem}
\pmcreated{2013-03-22 19:03:19}
\pmmodified{2013-03-22 19:03:19}
\pmowner{CWoo}{3771}
\pmmodifier{CWoo}{3771}
\pmtitle{partially ordered algebraic system}
\pmrecord{8}{41935}
\pmprivacy{1}
\pmauthor{CWoo}{3771}
\pmtype{Definition}
\pmcomment{trigger rebuild}
\pmclassification{msc}{06F99}
\pmclassification{msc}{08C99}
\pmclassification{msc}{08A99}
\pmrelated{AlgebraicSystem}

\usepackage{amssymb,amscd}
\usepackage{amsmath}
\usepackage{amsfonts}
\usepackage{mathrsfs}

% used for TeXing text within eps files
%\usepackage{psfrag}
% need this for including graphics (\includegraphics)
%\usepackage{graphicx}
% for neatly defining theorems and propositions
\usepackage{amsthm}
% making logically defined graphics
%%\usepackage{xypic}
\usepackage{pst-plot}

% define commands here
\newcommand*{\abs}[1]{\left\lvert #1\right\rvert}
\newtheorem{prop}{Proposition}
\newtheorem{thm}{Theorem}
\newtheorem{ex}{Example}
\newcommand{\real}{\mathbb{R}}
\newcommand{\pdiff}[2]{\frac{\partial #1}{\partial #2}}
\newcommand{\mpdiff}[3]{\frac{\partial^#1 #2}{\partial #3^#1}}
\begin{document}
Let $A$ be a poset.  Recall a function $f$ on $A$ is said to be 
\begin{itemize}
\item order-preserving (or isotone) provided that $f(a)\le f(b)$, or 
\item order-reversing (or antitone) provided that $f(a)\ge f(b)$, or
\end{itemize}
whenever $a\le b$.  Furthermore, $f$ is called monotone if $f$ is either isotone or antitone.  

For every function $f$ on $A$, we denote it to be $\uparrow$, $\downarrow$, or $\updownarrow$ according to whether it is isotone, antitone, or both.  The following are some easy consequences:
\begin{itemize}
\item $\uparrow \circ \downarrow = \downarrow \circ \uparrow = \downarrow$ (meaning that the composition of an isotone and an antitone maps is antitone), 
\item $\uparrow \circ \uparrow = \downarrow \circ \downarrow = \uparrow$ (meaning that the composition of two isotone or two antitone maps is isotone),
\item $f$ is $\updownarrow$ iff it is a constant on any chain in $A$, and if this is the case, for every $a\in A$, $f^{-1}(a)$ is a maximal chain in $A$.
\end{itemize}

The notion above can be generalized to $n$-ary operations on a poset $A$.  An $n$-ary operation $f$ on a poset $A$ is said to be \emph{isotone}, \emph{antitone}, or \emph{monotone} iff when $f$ is isotone, antitone, or monotone with respect to each of its $n$ variables.  We continue to use to arrow notations above to denote $n$-ary monotone functions.  For example, a ternary function that is $(\uparrow,\downarrow,\uparrow)$ is isotone with respect to its first and third variables, and antitone with respect to its second variable.

\textbf{Definition}.  A \emph{partially ordered algebraic system} is an algebraic system $\mathcal{A}=(A,O)$ such that $A$ is a poset, and every operation $f \in O$ on $A$ is monotone.  A partially ordered algebraic system is also called a partially ordered algebra, or a po-algebra for short.

Examples of po-algebras are po-groups, po-rings, and po-semigroups.  In all three cases, the multiplication operations are $(\uparrow,\uparrow)$, as well as the addition operation in a po-ring..  In the case of a po-group, the 
multiplicative inverse operation is $\downarrow$, as well as the additive inverse operation in a po-ring.

Another example is an ordered vector space $V$ over a field $k$.  The underlying universe is $V$ (not $k$).  Addition over $V$ is, like the other examples above, isotone.  Each element $r\in k$ acts as a unary operator on $V$, given by $r(v)=rv$, the scalar multiplication of $r$ and $v$.  As $k$ is itself a poset, it can be partitioned into three sets: the positive cone $P(k)$ of $k$, the negative cone $-P(k)$, and $\lbrace 0\rbrace$.  Then $r\in P(k)$ iff it is $\uparrow$ as a unary operator, $r\in -P(k)$ iff it is $\downarrow$, and $r=0$ iff it is $\updownarrow$.

\textbf{Remarks}
\begin{itemize}
\item A homomorphism from one po-algebra $\mathcal{A}$ to another $\mathcal{B}$ is an isotone map $\phi$ from posets $A$ to $B$ that is at the same time a homomorphism from the algebraic systems $\mathcal{A}$ to $\mathcal{B}$.
\item A partially ordered subalgebra of a po-algebra $\mathcal{A}$ is just a subalgebra of $\mathcal{A}$ viewed as an algebra, where the partial ordering on the universe of the subalgebra is inherited from the partial ordering on $A$.
\end{itemize}

\begin{thebibliography}{9}
\bibitem{LF} L. Fuchs, {\em Partially Ordered Algebraic Systems}, Addison-Wesley, (1963).
\end{thebibliography}
%%%%%
%%%%%
\end{document}
