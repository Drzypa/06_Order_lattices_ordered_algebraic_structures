\documentclass[12pt]{article}
\usepackage{pmmeta}
\pmcanonicalname{AscendingOrder}
\pmcreated{2013-03-22 16:06:39}
\pmmodified{2013-03-22 16:06:39}
\pmowner{CompositeFan}{12809}
\pmmodifier{CompositeFan}{12809}
\pmtitle{ascending order}
\pmrecord{7}{38176}
\pmprivacy{1}
\pmauthor{CompositeFan}{12809}
\pmtype{Definition}
\pmcomment{trigger rebuild}
\pmclassification{msc}{06A99}
\pmrelated{DescendingOrder}
\pmdefines{strictly ascending order}

% this is the default PlanetMath preamble.  as your knowledge
% of TeX increases, you will probably want to edit this, but
% it should be fine as is for beginners.

% almost certainly you want these
\usepackage{amssymb}
\usepackage{amsmath}
\usepackage{amsfonts}

% used for TeXing text within eps files
%\usepackage{psfrag}
% need this for including graphics (\includegraphics)
%\usepackage{graphicx}
% for neatly defining theorems and propositions
%\usepackage{amsthm}
% making logically defined graphics
%%%\usepackage{xypic}

% there are many more packages, add them here as you need them

% define commands here

\begin{document}
A sequence or arbitrary ordered set or one-dimensional array of numbers, $a$, is said to be in {\em ascending order} if each $a_i \le a_{i + 1}$. For example, the Fibonacci sequence is in ascending order: 1, 1, 2, 3, 5, 8, 13, 21 ... The Perrin sequence is not in ascending order: 3, 0, 2, 3, 2, 5, 5, 7, 10, 12, 17 ...

In a trivial sense, the sequence of values of the sign function is in ascending order: ... -1, -1, -1, 0, 1, 1, 1... When each $a_i < a_{i + 1}$ in the sequence, set or array, then it can be said to be in {\em strictly ascending order}.
%%%%%
%%%%%
\end{document}
