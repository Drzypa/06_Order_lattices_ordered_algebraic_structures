\documentclass[12pt]{article}
\usepackage{pmmeta}
\pmcanonicalname{ExampleOfANonlatticeHomomorphism}
\pmcreated{2013-03-22 16:58:31}
\pmmodified{2013-03-22 16:58:31}
\pmowner{Algeboy}{12884}
\pmmodifier{Algeboy}{12884}
\pmtitle{example of a non-lattice homomorphism}
\pmrecord{8}{39252}
\pmprivacy{1}
\pmauthor{Algeboy}{12884}
\pmtype{Example}
\pmcomment{trigger rebuild}
\pmclassification{msc}{06B23}

\endmetadata

\usepackage{latexsym}
\usepackage{amssymb}
\usepackage{amsmath}
\usepackage{amsfonts}
\usepackage{amsthm}

%%\usepackage{xypic}

%-----------------------------------------------------

%       Standard theoremlike environments.

%       Stolen directly from AMSLaTeX sample

%-----------------------------------------------------

%% \theoremstyle{plain} %% This is the default

\newtheorem{thm}{Theorem}

\newtheorem{coro}[thm]{Corollary}

\newtheorem{lem}[thm]{Lemma}

\newtheorem{lemma}[thm]{Lemma}

\newtheorem{prop}[thm]{Proposition}

\newtheorem{conjecture}[thm]{Conjecture}

\newtheorem{conj}[thm]{Conjecture}

\newtheorem{defn}[thm]{Definition}

\newtheorem{remark}[thm]{Remark}

\newtheorem{ex}[thm]{Example}



%\countstyle[equation]{thm}



%--------------------------------------------------

%       Item references.

%--------------------------------------------------


\newcommand{\exref}[1]{Example-\ref{#1}}

\newcommand{\thmref}[1]{Theorem-\ref{#1}}

\newcommand{\defref}[1]{Definition-\ref{#1}}

\newcommand{\eqnref}[1]{(\ref{#1})}

\newcommand{\secref}[1]{Section-\ref{#1}}

\newcommand{\lemref}[1]{Lemma-\ref{#1}}

\newcommand{\propref}[1]{Prop\-o\-si\-tion-\ref{#1}}

\newcommand{\corref}[1]{Cor\-ol\-lary-\ref{#1}}

\newcommand{\figref}[1]{Fig\-ure-\ref{#1}}

\newcommand{\conjref}[1]{Conjecture-\ref{#1}}


% Normal subgroup or equal.

\providecommand{\normaleq}{\unlhd}

% Normal subgroup.

\providecommand{\normal}{\lhd}

\providecommand{\rnormal}{\rhd}
% Divides, does not divide.

\providecommand{\divides}{\mid}

\providecommand{\ndivides}{\nmid}


\providecommand{\union}{\cup}

\providecommand{\bigunion}{\bigcup}

\providecommand{\intersect}{\cap}

\providecommand{\bigintersect}{\bigcap}










\begin{document}
Consider the Hasse diagram of the lattice of subgroups of the quaternion group 
of order $8$, $Q_8$.  [The use of $Q_8$ is only for a concrete realization of the lattice.]
\[
\begin{xy}<5mm,0mm>:<0mm,5mm>::
(0,3) +*{Q_8} = "Q8";
(-2,2) +*{\langle i\rangle} = "i";
(0,2) +*{\langle j\rangle} = "j";
(2,2) +*{\langle k\rangle} = "k";
(0,1) +*{\langle -1\rangle} = "-1";
(0,0) +*{\langle 1\rangle} = "1";
"1"; "-1" **@{-};
"-1"; "i" **@{-};
"-1"; "j" **@{-};
"-1"; "k" **@{-};
"i"; "Q8" **@{-};
"j"; "Q8" **@{-};
"k"; "Q8" **@{-};
\end{xy}
\]

To establish an order-preserving map which is not a lattice isomorphism
one can simply ``skip'' $\langle -1\rangle$, which we display graphically as:

\[
\begin{xy}<5mm,0mm>:<0mm,5mm>::
(-3,3) +*{Q_8} = "Q81";
(-5,2) +*{\langle i\rangle} = "i1";
(-3,2) +*{\langle j\rangle} = "j1";
(-1,2) +*{\langle k\rangle} = "k1";
(-3,1) +*{\langle -1\rangle} = "-11";
(-3,0) +*{\langle 1\rangle} = "11";
(3,2.5) +*{Q_8} = "Q82";
(1,1.5) +*{\langle i\rangle} = "i2";
(3,1.5) +*{\langle j\rangle} = "j2";
(5,1.5) +*{\langle k\rangle} = "k2";
(3,0.5) +*{\langle -1\rangle} = "-12";
(3,-0.5) +*{\langle 1\rangle} = "12";
"11"; "-11" **@{-};
"-11"; "i1" **@{-};
"-11"; "j1" **@{-};
"-11"; "k1" **@{-};
"i1"; "Q81" **@{-};
"j1"; "Q81" **@{-};
"k1"; "Q81" **@{-};
"12"; "-12" **@{-};
"-12"; "i2" **@{-};
"-12"; "j2" **@{-};
"-12"; "k2" **@{-};
"i2"; "Q82" **@{-};
"j2"; "Q82" **@{-};
"k2"; "Q82" **@{-};
"Q81"; "Q82" **@{..};
"i1"; "i2" **@{..};
"j1"; "j2" **@{..};
"k1"; "k2" **@{..};
"-11"; "12" **@{..};
"11"; "12" **@{..};
\end{xy}
\]

Since containment is still preserved the map is order-preserving.  However, the intersection (meet) of
$\langle i\rangle$ and $\langle j\rangle$, which is $\langle -1\rangle$, is not perserved under this
map.  Thus it is not a lattice homomorphism.


%%%%%
%%%%%
\end{document}
