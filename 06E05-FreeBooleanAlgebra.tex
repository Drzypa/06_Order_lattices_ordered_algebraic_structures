\documentclass[12pt]{article}
\usepackage{pmmeta}
\pmcanonicalname{FreeBooleanAlgebra}
\pmcreated{2013-03-22 18:01:23}
\pmmodified{2013-03-22 18:01:23}
\pmowner{CWoo}{3771}
\pmmodifier{CWoo}{3771}
\pmtitle{free Boolean algebra}
\pmrecord{12}{40540}
\pmprivacy{1}
\pmauthor{CWoo}{3771}
\pmtype{Definition}
\pmcomment{trigger rebuild}
\pmclassification{msc}{06E05}
\pmclassification{msc}{03G05}
\pmclassification{msc}{06B20}
\pmclassification{msc}{03G10}

\endmetadata

\usepackage{amssymb,amscd}
\usepackage{amsmath}
\usepackage{amsfonts}
\usepackage{mathrsfs}

% used for TeXing text within eps files
%\usepackage{psfrag}
% need this for including graphics (\includegraphics)
%\usepackage{graphicx}
% for neatly defining theorems and propositions
\usepackage{amsthm}
% making logically defined graphics
%%\usepackage{xypic}
\usepackage{pst-plot}

% define commands here
\newcommand*{\abs}[1]{\left\lvert #1\right\rvert}
\newtheorem{prop}{Proposition}
\newtheorem{thm}{Theorem}
\newtheorem{ex}{Example}
\newcommand{\real}{\mathbb{R}}
\newcommand{\pdiff}[2]{\frac{\partial #1}{\partial #2}}
\newcommand{\mpdiff}[3]{\frac{\partial^#1 #2}{\partial #3^#1}}
\begin{document}
Let $A$ be a Boolean algebra and $X\subseteq A$ such that $\langle X\rangle = A$.  In other words, $X$ is a set of generators of $A$.  $A$ is said to be \emph{freely generated by $X$}, or that $X$ is a \emph{free set of generators of $A$}, if $\langle X\rangle =A$, and every function $f$ from $X$ to some Boolean algebra $B$ can be extended to a Boolean algebra homomorphism $g$ from $A$ to $B$, as illustrated by the commutative diagram below:

\begin{center}
$
\xymatrix@R-=2pt{
X\ar[dr]^f\ar[dd]_i\\
&B\\
A\ar[ur]_g
}
$
\end{center}
where $i: X\to A$ is the inclusion map.  By extension of $f$ to $g$ we mean that $g(x)=f(x)$ for every $x\in X$.  Any subset $X\subseteq A$ containing $0$ (or $1$) can never be a free generating set for any subalgebra of $A$, for any function $f:X\to B$ such that $f(0)\ne 0$ can never be extended to a Boolean homomorphism.

A Boolean algebra is said to be \emph{free} if it has a free set of generators.  If $A$ has $X$ as a free set of generators, $A$ is said to be \emph{free on} $X$.  If $A$ and $B$ are both free on $X$, then $A$ and $B$ are isomorphic.  This means that free algebras are uniquely determined by its free generating set, up to isomorphisms.

A simple example of a free Boolean algebra is the one freely generated by one element.  Let $X$ be a singleton consisting of $a$.  Then the set $A=\lbrace 0,a,a',1\rbrace$ is a Boolean algebra, with the obvious Boolean operations identified.  Every function from $X$ to a Boolean algebra $B$ singles out an element $b\in B$ corresponding to $a$.  Then the function $g: A\to B$ given by $g(a)=b$, $g(a')=b'$, $g(0)=0$, and $g(1)=1$ is clearly Boolean.

The two-element algebra $\lbrace 0,1\rbrace$ is also free, its free generating set being $\varnothing$, the empty set, since the only function on $\varnothing$ is $\varnothing$, and thus can be extended to any function.

In general, if $X$ is finite, then the Boolean algebra freely generated by $X$ has cardinality $2^{2^{|X|}}$, where $|X|$ is the cardinality of $X$.  If $X$ is infinite, then the cardinality of the Boolean algebra freely generated by $X$ is $|X|$.
%%%%%
%%%%%
\end{document}
