\documentclass[12pt]{article}
\usepackage{pmmeta}
\pmcanonicalname{ExampleOfLawOfTrichotomyOn}
\pmcreated{2013-03-22 17:15:07}
\pmmodified{2013-03-22 17:15:07}
\pmowner{me_and}{17092}
\pmmodifier{me_and}{17092}
\pmtitle{example of law of trichotomy on <}
\pmrecord{5}{39586}
\pmprivacy{1}
\pmauthor{me_and}{17092}
\pmtype{Example}
\pmcomment{trigger rebuild}
\pmclassification{msc}{06A05}
\pmclassification{msc}{03E20}

\endmetadata

\usepackage{amssymb}
%\usepackage{amsmath}
%\usepackage{amsfonts}
%\usepackage{amsthm}

%Named sets
\newcommand{\R}{\mathbb{R}} %Real numbers
%\newcommand{\C}{\mathbb{C}} %Complex numbers

%Functions
%\newcommand{\modulus}[1]{\left|{#1}\right|} %|z|

%Numbers
%\newcommand{\I}{\mathrm{i}} %sqrt{-1}
%\newcommand{\e}{\mathrm{e}} $exponential

%Greek
%\newcommand{\ve}{\varepsilon} %nice epsilon
\begin{document}
From the axiomatic definition of the real numbers, ``$<$'' is a relation on
$\R$ which satisfies the law of trichotomy. That is, for all $a,b\in\R$,
exactly one of the following is true:
\begin{itemize}
\item $a<b$
\item $b<a$
\item $a=b$
\end{itemize}
As an axiom, this is sometimes expressed with $b=0$. That is, for all $a\in\R$, $a$ is either positive, negative, or zero.
%%%%%
%%%%%
\end{document}
