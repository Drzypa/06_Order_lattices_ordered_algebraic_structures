\documentclass[12pt]{article}
\usepackage{pmmeta}
\pmcanonicalname{OrderedIntegralDomainWithWellorderedPositiveElements}
\pmcreated{2013-03-22 14:46:43}
\pmmodified{2013-03-22 14:46:43}
\pmowner{Wkbj79}{1863}
\pmmodifier{Wkbj79}{1863}
\pmtitle{ordered integral domain with well-ordered positive elements}
\pmrecord{11}{36425}
\pmprivacy{1}
\pmauthor{Wkbj79}{1863}
\pmtype{Theorem}
\pmcomment{trigger rebuild}
\pmclassification{msc}{06F25}
\pmclassification{msc}{12J15}
\pmclassification{msc}{13J25}
\pmrelated{TotalOrder}
\pmrelated{OrderedRing}
\pmdefines{positive element}

% this is the default PlanetMath preamble.  as your knowledge
% of TeX increases, you will probably want to edit this, but
% it should be fine as is for beginners.

% almost certainly you want these
\usepackage{amssymb}
\usepackage{amsmath}
\usepackage{amsfonts}

% used for TeXing text within eps files
%\usepackage{psfrag}
% need this for including graphics (\includegraphics)
%\usepackage{graphicx}
% for neatly defining theorems and propositions
 \usepackage{amsthm}
% making logically defined graphics
%%%\usepackage{xypic}

% there are many more packages, add them here as you need them

% define commands here
\theoremstyle{definition}
\newtheorem*{thmplain}{Theorem}
\begin{document}
\begin{thmplain}
\,\,If \,$(R,\,\leq)$\, is an \PMlinkname{ordered}{OrderedRing} integral domain and if the set\, $R_+ = \{r\in R: \,\,0 < r\}$\, of its \PMlinkname{positive elements}{PositivityInOrderedRing} is well-ordered, then $R$ and $R_+$ can be expressed as sets of multiples of the unity as follows:
\begin{itemize}
 \item $R   = \{m\cdot 1: \,\,m\in\mathbb{Z}\}$,
 \item $R_+ = \{n\cdot 1: \,\,n\in\mathbb{Z}_+\}$.
\end{itemize}
\end{thmplain}

The theorem may be interpreted so that such an integral domain is isomorphic with the ordered ring $\mathbb{Z}$ of rational integers.
%%%%%
%%%%%
\end{document}
