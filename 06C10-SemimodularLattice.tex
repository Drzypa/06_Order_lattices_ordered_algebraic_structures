\documentclass[12pt]{article}
\usepackage{pmmeta}
\pmcanonicalname{SemimodularLattice}
\pmcreated{2013-03-22 15:26:20}
\pmmodified{2013-03-22 15:26:20}
\pmowner{mps}{409}
\pmmodifier{mps}{409}
\pmtitle{semimodular lattice}
\pmrecord{9}{37286}
\pmprivacy{1}
\pmauthor{mps}{409}
\pmtype{Definition}
\pmcomment{trigger rebuild}
\pmclassification{msc}{06C10}
\pmsynonym{upper semimodular lattice}{SemimodularLattice}
\pmsynonym{lower semimodular lattice}{SemimodularLattice}
\pmrelated{ModularLattice}
\pmrelated{IncidenceGeometry}

% this is the default PlanetMath preamble.  as your knowledge
% of TeX increases, you will probably want to edit this, but
% it should be fine as is for beginners.

% almost certainly you want these
\usepackage{amssymb}
\usepackage{amsmath}
\usepackage{amsfonts}

% used for TeXing text within eps files
%\usepackage{psfrag}
% need this for including graphics (\includegraphics)
%\usepackage{graphicx}
% for neatly defining theorems and propositions
%\usepackage{amsthm}

% making logically defined graphics
%%\usepackage{xypic}

% there are many more packages, add them here as you need them

% define commands here
\begin{document}
A lattice $L$ is {\em semimodular}
\footnote{Or {\em upper semimodular}, if one wants to stress the 
distinction with lower semimodular lattices.}
if for any $a$ and $b\in L$,
\[
  a \wedge b \prec a
  \quad\text{implies}\quad
  b \prec a \vee b,
\]
where $\prec$ denotes the covering relation in $L$.
Dually, a lattice $L$ is said to be \emph{lower semimodular} 
if for any $a$ and $b\in L$,
\[
  b \prec a \vee b
  \quad\text{implies}\quad
  a \wedge b \prec a.
\]
A chain finite lattice is \PMlinkname{modular}{ModularLattice} 
if and only if it is both semimodular and lower semimodular.


The smallest lattice which is semimodular but not modular is
\[\xymatrix{
    & 1 \ar@{-}[ld] \ar@{-}[d] \ar@{-}[rd] & \\
    a \ar@{-}[d] & b \ar@{-}[ld] \ar@{-}[rd] & c \ar@{-}[d] \\
    d \ar@{-}[rd] & & e \ar@{-}[ld] \\
    & 0 &
}\]
since $d \le a$ but $a \wedge (c \vee d) \neq (a \wedge c) \vee d$.
%%%%%
%%%%%
\end{document}
