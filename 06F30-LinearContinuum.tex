\documentclass[12pt]{article}
\usepackage{pmmeta}
\pmcanonicalname{LinearContinuum}
\pmcreated{2013-03-22 17:17:40}
\pmmodified{2013-03-22 17:17:40}
\pmowner{azdbacks4234}{14155}
\pmmodifier{azdbacks4234}{14155}
\pmtitle{linear continuum}
\pmrecord{11}{39638}
\pmprivacy{1}
\pmauthor{azdbacks4234}{14155}
\pmtype{Definition}
\pmcomment{trigger rebuild}
\pmclassification{msc}{06F30}
\pmclassification{msc}{54B99}
%\pmkeywords{linear order}
%\pmkeywords{total order}
%\pmkeywords{least upper bound}
%\pmkeywords{least upper bound property}
%\pmkeywords{supremum}
%\pmkeywords{order topology}
\pmrelated{DenseTotalOrder}
\pmrelated{TotalOrder}
\pmrelated{Supremum}
\pmrelated{LowestUpperBound}
\pmrelated{OrderTopology}
\pmrelated{ASpaceIsConnectedUnderTheOrderedTopologyIfAndOnlyIfItIsALinearContinuum}
\pmdefines{linear continuum}

%%packages
\usepackage{amssymb}
\usepackage{amsmath}
\usepackage{amsfonts}
\usepackage{amsthm}
%%theorem environments
\theoremstyle{plain}
\newtheorem*{theorem*}{Theorem}
\newtheorem*{lemma*}{Lemma}
\newtheorem*{corollary*}{Corollary}
\newtheorem*{proposition*}{Proposition}
%math operators and commands
\newcommand{\set}[1]{\{#1\}}
\newcommand{\medset}[1]{\left\{#1\right\}}
\newcommand{\bigset}[1]{\bigg\{#1\bigg\}}
\newcommand{\abs}[1]{\vert#1\vert}
\newcommand{\medabs}[1]{\left\vert#1\right\vert}
\newcommand{\bigabs}[1]{\bigg\vert#1\bigg\vert}
\newcommand{\norm}[1]{\Vert#1\Vert}
\newcommand{\mednorm}[1]{\left\Vert#1\right\Vert}
\newcommand{\bignorm}[1]{\bigg\Vert#1\bigg\Vert}
\DeclareMathOperator{\Aut}{Aut}
\DeclareMathOperator{\End}{End}
\DeclareMathOperator{\Inn}{Inn}
\DeclareMathOperator{\supp}{supp}

\begin{document}
Let $X$ be a totally-ordered set under an order \PMlinkescapetext{relation} $<$ having at least two distinct points. Then $X$ is said to be a \emph{linear continuum} if the following two conditions are satisfied:
\begin{enumerate}
\item The order relation $<$ is a dense total order (i.e., for every $x,y\in X$ with $x<y$ there exists $z\in X$ such that $x<z<y$).
\item Every non-empty subset of $X$ that is bounded above has a least upper bound (i.e., $X$ has the least upper bound property).
\end{enumerate}

Some examples of ordered sets that are linear continua include $\mathbb{R}$, the set $[0,1]\times[0,1]$ in the dictionary order, and the so-called long line $\Omega\times[0,1)$ in the dictionary topology. (The third example is a special case of a general result on well-ordered sets and linear continua.)

\begin{proposition*}
If $X$ is a well-ordered set, then the set $X\times[0,1)$ is a linear continua in the dictionary order topology.
\end{proposition*}

Linear continua are of special interest when they are made into topological spaces under the order topology, and the following two \PMlinkescapeword{propositions} establish some useful properties of such spaces:

\begin{proposition*}
If $X$ is a linear continuum in the order topology, then $X$ is \PMlinkid{connected}{941} and so are intervals in $X$.
\end{proposition*}

As a corollary of the preceding \PMlinkescapeword{proposition}, we obtain the result that $\mathbb{R}$ is \PMlinkescapetext{connected} in its usual topology, as are the intervals $[a,b]$ and $(a,b)$, where $a<b\in\mathbb{R}$. 

\begin{proposition*}
If $X$ is a linear continuum in the order topology, then every closed interval in $X$ is compact.
\end{proposition*}
\begin{proof}
This is essentially a slightly generalized version of the Heine-Borel Theorem for $\mathbb{R}$, and the proof is almost identical.
\end{proof}




%%%%%
%%%%%
\end{document}
