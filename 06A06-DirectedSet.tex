\documentclass[12pt]{article}
\usepackage{pmmeta}
\pmcanonicalname{DirectedSet}
\pmcreated{2013-03-22 12:54:00}
\pmmodified{2013-03-22 12:54:00}
\pmowner{yark}{2760}
\pmmodifier{yark}{2760}
\pmtitle{directed set}
\pmrecord{11}{33249}
\pmprivacy{1}
\pmauthor{yark}{2760}
\pmtype{Definition}
\pmcomment{trigger rebuild}
\pmclassification{msc}{06A06}
\pmsynonym{upward-directed set}{DirectedSet}
\pmsynonym{upward directed set}{DirectedSet}
\pmrelated{Cofinality}
\pmrelated{AccumulationPointsAndConvergentSubnets}
\pmdefines{residual}
\pmdefines{cofinal}
\pmdefines{downward-directed set}
\pmdefines{downward directed set}
\pmdefines{filtered set}

\usepackage{amssymb}
\usepackage{amsmath}
\usepackage{amsfonts}

\begin{document}
\PMlinkescapeword{property}

A \emph{directed set} is a partially ordered set $(A, \leq)$ such that whenever $a,b\in A$ there is an $x\in A$ such that $a\leq x$ and $b\leq x$.

A subset $B\subseteq A$ is said to be \emph{residual} if there is $a\in A$ such that $b\in B$ whenever $a\leq b$, and \emph{cofinal} if for each $a\in A$ there is $b\in B$ such that $a\leq b$.

A directed set is sometimes called an \emph{upward-directed set}.
We may also define the dual notion:
a \emph{downward-directed set} (or \emph{filtered set}) is a partially ordered set $(A, \leq)$ such that whenever $a,b\in A$ there is an $x\in A$ such that $x\leq a$ and $x\leq b$.

Note: Many authors do not require $\leq$ to be antisymmetric,
so that it is only a pre-order (rather than a partial order)
with the given property.
Also, it is common to require $A$ to be non-empty.


%%%%%
%%%%%
\end{document}
