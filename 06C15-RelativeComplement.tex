\documentclass[12pt]{article}
\usepackage{pmmeta}
\pmcanonicalname{RelativeComplement}
\pmcreated{2013-03-22 15:51:45}
\pmmodified{2013-03-22 15:51:45}
\pmowner{CWoo}{3771}
\pmmodifier{CWoo}{3771}
\pmtitle{relative complement}
\pmrecord{11}{37852}
\pmprivacy{1}
\pmauthor{CWoo}{3771}
\pmtype{Definition}
\pmcomment{trigger rebuild}
\pmclassification{msc}{06C15}
\pmrelated{RelativePseudocomplement}
\pmrelated{BrouwerianLattice}
\pmdefines{relatively complemented lattice}
\pmdefines{relatively complemented}

\usepackage{amssymb,amscd}
\usepackage{amsmath}
\usepackage{amsfonts}

% used for TeXing text within eps files
%\usepackage{psfrag}
% need this for including graphics (\includegraphics)
%\usepackage{graphicx}
% for neatly defining theorems and propositions
%\usepackage{amsthm}
% making logically defined graphics
%%%\usepackage{xypic}

% define commands here
\begin{document}
\PMlinkescapeword{interval}
\PMlinkescapeword{bounded}

A complement of an element in a lattice is only defined when the lattice in question is \PMlinkname{bounded}{BoundedLattice}.  In general, a lattice is not bounded and there are no complements to speak of.  Nevertheless, if the sublattice of a lattice is bounded, we can speak of complements of an element \emph{relative} to that sublattice.

Let $L$ be a lattice, $a$ an element of $L$, and $I=[b,c]$ an \PMlinkname{interval}{LatticeInterval} in $L$.  An element\, $d\in L$\, is said to be a complement of $a$ \emph{relative} to $I$ if
$$a\vee d=c\,\mbox{ and }\,a\wedge d=b.$$

It is easy to see that $a\le c$ and $b\le a$,\, so\, $a\in I$.  Similarly, $d\in I$.

An element $a\in L$ is said to be \emph{relatively complemented} if for every interval $I$ in $L$ with $a\in I$, it has a complement relative to $I$.  The lattice $L$ itself is called a \emph{relatively complemented lattice} if every element of $L$ is relatively complemented.  Equivalently, $L$ is relatively complemented iff each of its interval is a complemented lattice.

\textbf{Remarks}.  
\begin{itemize}
\item A relatively complemented lattice is complemented if it is bounded.  Conversely, a complemented lattice is relatively complemented if it is 
\PMlinkname{modular}{ModularLattice}.
\item The notion of a relative complement of an element in a lattice has nothing to do with that found in set theory: let $U$ be a set and $A,B$ subsets of $U$, the relative complement of $A$ in $B$ is the set theoretic difference $B-A$.  While the relative difference is necessarily a subset of $B$, $A$ does not have to be a subset of $B$.
\end{itemize}
%%%%%
%%%%%
\end{document}
