\documentclass[12pt]{article}
\usepackage{pmmeta}
\pmcanonicalname{PartialOrder}
\pmcreated{2013-03-22 11:43:32}
\pmmodified{2013-03-22 11:43:32}
\pmowner{mathcam}{2727}
\pmmodifier{mathcam}{2727}
\pmtitle{partial order}
\pmrecord{24}{30123}
\pmprivacy{1}
\pmauthor{mathcam}{2727}
\pmtype{Definition}
\pmcomment{trigger rebuild}
\pmclassification{msc}{06A06}
\pmclassification{msc}{35C10}
\pmclassification{msc}{35C15}
\pmclassification{msc}{55-01}
\pmclassification{msc}{55-00}
\pmsynonym{order}{PartialOrder}
\pmsynonym{partial ordering}{PartialOrder}
\pmsynonym{ordering}{PartialOrder}
%\pmkeywords{relation}
%\pmkeywords{total order}
%\pmkeywords{transitivity}
%\pmkeywords{reflexivity}
%\pmkeywords{antisymmetry}
\pmrelated{Relation}
\pmrelated{TotalOrder}
\pmrelated{Poset}
\pmrelated{BinarySearch}
\pmrelated{SortingProblem}
\pmrelated{ChainCondition}
\pmrelated{PartialOrderWithChainConditionDoesNotCollapseCardinals}
\pmrelated{QuasiOrder}
\pmrelated{CategoryAssociatedToAPartialOrder}
\pmrelated{OrderingRelation}
\pmrelated{HasseDiagram}
\pmrelated{NetsAndClosuresOfSubspaces}

\endmetadata

\usepackage{amssymb}
\usepackage{amsmath}
\usepackage{amsfonts}
\usepackage{graphicx}
%%%%%%%%\usepackage{xypic}
\begin{document}
A \emph{partial order} (often simply referred to as an \emph{order} or \emph{ordering}) is a relation $\leq\:\subset A\times A$ that satisfies the following three properties:

\begin{enumerate}
    \item Reflexivity: $a \leq a$ for all $a\in A$
    \item Antisymmetry: If $a \leq b$ and $b \leq a$ for any $a, b\in A$, then $a = b$
    \item Transitivity: If $a \leq b$ and $b \leq c$ for any $a, b, c\in A$, then $a \leq c$
\end{enumerate}

A \emph{total order} is a partial order that satisfies a fourth property known as \emph{comparability}:

\begin{itemize}
\item Comparability:  For any $a,b\in A$, either $a\leq b$ or $b\leq a$.
\end{itemize}

A set and a partial order on that set define a poset.

\textbf{Remark}.  In some literature, especially those dealing with the foundations of mathematics, a partial order $\le$ is defined as a transitive irreflexive binary relation (on a set).  As a result, if $a\le b$, then $b \nleq a$, and therefore $\le$ is antisymmetric.
%%%%%
%%%%%
%%%%%
%%%%%
%%%%%
%%%%%
%%%%%
%%%%%
\end{document}
