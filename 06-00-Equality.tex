\documentclass[12pt]{article}
\usepackage{pmmeta}
\pmcanonicalname{Equality}
\pmcreated{2013-03-22 18:01:26}
\pmmodified{2013-03-22 18:01:26}
\pmowner{pahio}{2872}
\pmmodifier{pahio}{2872}
\pmtitle{equality}
\pmrecord{14}{40541}
\pmprivacy{1}
\pmauthor{pahio}{2872}
\pmtype{Topic}
\pmcomment{trigger rebuild}
\pmclassification{msc}{06-00}
\pmrelated{Equation}
\pmdefines{equality relation}
\pmdefines{identity}
\pmdefines{identic equation}

% this is the default PlanetMath preamble.  as your knowledge
% of TeX increases, you will probably want to edit this, but
% it should be fine as is for beginners.

% almost certainly you want these
\usepackage{amssymb}
\usepackage{amsmath}
\usepackage{amsfonts}

% used for TeXing text within eps files
%\usepackage{psfrag}
% need this for including graphics (\includegraphics)
%\usepackage{graphicx}
% for neatly defining theorems and propositions
 \usepackage{amsthm}
% making logically defined graphics
%%%\usepackage{xypic}

% there are many more packages, add them here as you need them

% define commands here

\theoremstyle{definition}
\newtheorem*{thmplain}{Theorem}

\begin{document}
In any set $S$, the {\em equality}, denoted by ``$=$'', is a binary relation which is reflexive, symmetric, transitive and antisymmetric, i.e. it is an antisymmetric equivalence relation on $S$, or which is the same thing, the equality is a symmetric partial order on $S$.

In fact, for any set $S$, the smallest equivalence relation on $S$ is the equality (by smallest we \PMlinkescapetext{mean} that it is contained in every equivalence relation on $S$).  This offers a definition of ``equality''.  From this, it is clear that there is only one equality relation on $S$.\, Its equivalence classes are all singletons $\{x\}$ where\, $x \in S$.

The concept of equality is essential in almost all branches of mathematics.  A few examples will suffice:
\begin{eqnarray*}
1 + 1 & = & 2 \\
e^{i \pi} & = & -1 \\
\mathbb{R}[i] & = & \mathbb{C}
\end{eqnarray*}
(The second example is Euler's identity.)\\

\textbf{Remark 1.}\, Although the four characterising \PMlinkescapetext{properties}, reflexivity, \PMlinkname{symmetry}{Symmetric}, transitivity and \PMlinkname{antisymmetry}{Antisymmetric}, determine the equality on $S$ uniquely, they cannot be thought to form the definition of the equality, since the concept of antisymmetry already \PMlinkescapetext{contains} the equality.\\

\textbf{Remark 2.}\, An equality (equation) in a set $S$ may be true regardless to the values of the variables involved in the equality; then one speaks of an {\em identity} or {\em identic equation} in this set.\, E.g.\, $(x+y)^2 = x^2+y^2$\, is an identity in a field with \PMlinkname{characteristic}{Characteristic} $2$.

%%%%%
%%%%%
\end{document}
