\documentclass[12pt]{article}
\usepackage{pmmeta}
\pmcanonicalname{CenterOfALattice}
\pmcreated{2013-03-22 17:31:50}
\pmmodified{2013-03-22 17:31:50}
\pmowner{CWoo}{3771}
\pmmodifier{CWoo}{3771}
\pmtitle{center of a lattice}
\pmrecord{5}{39927}
\pmprivacy{1}
\pmauthor{CWoo}{3771}
\pmtype{Definition}
\pmcomment{trigger rebuild}
\pmclassification{msc}{06B05}
\pmdefines{central element}

\usepackage{amssymb,amscd}
\usepackage{amsmath}
\usepackage{amsfonts}
\usepackage{mathrsfs}

% used for TeXing text within eps files
%\usepackage{psfrag}
% need this for including graphics (\includegraphics)
%\usepackage{graphicx}
% for neatly defining theorems and propositions
\usepackage{amsthm}
% making logically defined graphics
%%\usepackage{xypic}
\usepackage{pst-plot}
\usepackage{psfrag}

% define commands here
\newtheorem{prop}{Proposition}
\newtheorem{thm}{Theorem}
\newtheorem{ex}{Example}
\newcommand{\real}{\mathbb{R}}
\newcommand{\pdiff}[2]{\frac{\partial #1}{\partial #2}}
\newcommand{\mpdiff}[3]{\frac{\partial^#1 #2}{\partial #3^#1}}
\begin{document}
\PMlinkescapeword{central}
\PMlinkescapeword{center}

Let $L$ be a bounded lattice.  An element $a\in L$ is said to be \emph{central} if $a$ is \PMlinkname{complemented}{ComplementedLattice} and \PMlinkname{neutral}{SpecialElementsInALattice}.  The \emph{center} of $L$, denoted $\operatorname{Cen}(L)$, is the set of all central elements of $L$.

\textbf{Remarks}.
\begin{itemize}
\item $0$ and $1$ are central: they are complements of one another, both distributive and dually distributive, and satisfying the property $$a\wedge b=a\wedge c\mbox{ and }a\vee b=a\vee c\mbox{ imply }b=c\mbox{ for all }b,c\in L$$ where $a\in \lbrace 0,1\rbrace$, and therefore neutral.
\item $\operatorname{Cen}(L)$ is a sublattice of $L$.
\item $\operatorname{Cen}(L)$ is a Boolean algebra.
\end{itemize}

\begin{thebibliography}{8}
\bibitem{gg} G. Gr\"{a}tzer, {\em General Lattice Theory}, 2nd Edition, Birkh\"{a}user (1998).
\end{thebibliography}
%%%%%
%%%%%
\end{document}
