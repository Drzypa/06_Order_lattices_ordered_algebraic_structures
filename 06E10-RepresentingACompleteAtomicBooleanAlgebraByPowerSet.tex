\documentclass[12pt]{article}
\usepackage{pmmeta}
\pmcanonicalname{RepresentingACompleteAtomicBooleanAlgebraByPowerSet}
\pmcreated{2013-03-22 19:08:30}
\pmmodified{2013-03-22 19:08:30}
\pmowner{CWoo}{3771}
\pmmodifier{CWoo}{3771}
\pmtitle{representing a complete atomic Boolean algebra by power set}
\pmrecord{8}{42041}
\pmprivacy{1}
\pmauthor{CWoo}{3771}
\pmtype{Definition}
\pmcomment{trigger rebuild}
\pmclassification{msc}{06E10}
\pmrelated{RepresentingABooleanLatticeByFieldOfSets}

\endmetadata

\usepackage{amssymb,amscd}
\usepackage{amsmath}
\usepackage{amsfonts}
\usepackage{mathrsfs}

% used for TeXing text within eps files
%\usepackage{psfrag}
% need this for including graphics (\includegraphics)
%\usepackage{graphicx}
% for neatly defining theorems and propositions
\usepackage{amsthm}
% making logically defined graphics
%%\usepackage{xypic}
\usepackage{pst-plot}

% define commands here
\newcommand*{\abs}[1]{\left\lvert #1\right\rvert}
\newtheorem{prop}{Proposition}
\newtheorem{thm}{Theorem}
\newtheorem{cor}{Corollary}
\newtheorem{ex}{Example}
\newcommand{\real}{\mathbb{R}}
\newcommand{\pdiff}[2]{\frac{\partial #1}{\partial #2}}
\newcommand{\mpdiff}[3]{\frac{\partial^#1 #2}{\partial #3^#1}}
\begin{document}
It is a known fact that every Boolean algebra is isomorphic to a field of sets (of some set) (proof \PMlinkname{here}{RepresentingABooleanLatticeByFieldOfSets}).  In this entry, we show that, furthermore, if a Boolean algebra is atomic and complete, then it is isomorphic to \emph{the} field of sets of some set, in other words, the powerset of some set, viewed as a Boolean algebra via the usual set-theoretic operations of union, intersection, and complement.  

The proof is based on the following function, defined for any atomic Boolean algebra:

\textbf{Definition}.  Let $B$ be an atomic Boolean algebra, and $X$ \emph{the} set of its atoms.  Define $f: B\to P(X)$ by $$f(x):=\lbrace a \mid a\le x \rbrace.$$

It is easy to see that $f(x)=\lbrace x\rbrace$ iff $x$ is an atom of $B$.

\begin{prop} $f(x)$ and $f(x')$ are complement of one another in $X$. \end{prop}
\begin{proof}  For any $a\in X$, $a\le 1=x\vee x'$, so that $a\le x$ or $a\le x'$, or $a\in f(x)$ or $a\in f(x')$.  This shows that $f(x)\cup f(x')=X$.  If $a\in f(x)\cap f(x')$, then $a\le x$ and $a\le x'$, so that $a\le x\wedge x'=0$, which is impossible, since $a$ is an atom, and by definition, must be greater than $0$.
\end{proof}

\begin{prop} $f$ is a Boolean algebra homomorphism.  \end{prop}
\begin{proof} First, $f(x') = X-f(x)$ by the last proposition.

Next, $f(x\vee y)= \lbrace a \mid a\le x\vee y \rbrace = \lbrace a \mid a\le x \mbox{ or } a\le y\rbrace$ since $a$ is an atom.  But the right hand side equals $\lbrace a\mid a\le x\rbrace \cup \lbrace a\mid a\le y \rbrace = f(x)\cup f(y)$, we see that $f$ preserves $\vee$. 

Finally, $f(0)=\lbrace a\mid a\le 0\rbrace = \varnothing$ since any atom must be greater than $0$.

Hence, $f$ is a Boolean algebra homomorphism.
\end{proof}

\begin{prop} $f$ is an injection. \end{prop}
\begin{proof} Suppose $f(x)=\varnothing$.  If $x\ne 0$, then there must be some atom $a$ such that $a\le x$.  But this implies that $f(x)\ne \varnothing$, a contradiction.  Hence $x=0$ and $f$ is injective.  \end{proof}

\begin{prop} $f$ is conditionally complete, in the sense that if $\bigvee A$ is defined for any $A\subseteq B$, then $$f(\bigvee A)=\bigcup \lbrace f(x)\mid x\in A\rbrace.$$ \end{prop}
\begin{proof}  Suppose $y=\bigvee A$ and $Y=f(y)$.  Let $Z=\bigcup \lbrace f(x) \mid x\in A \rbrace$.  We want to show that $Y=Z$.  If $a\in Y$, then $a\le y$, or $a\le x$ for some $x \in A$, since $a$ is an atom.  So $a\in f(x)\subseteq Z$.  Conversely, if $a\in Z$, then $a\in f(x)$, or $a\le x$ for some $x\in A$.  This means that $a\le x \le \bigvee A =y$, and therefore $a\in f(y)=Y$.
\end{proof}

\begin{prop} If $B$ is complete, so is $f$.  Moreover, $f$ is surjective.  \end{prop}
\begin{proof} The first sentence is a direct consequence of the previous proposition.  For the second setnence, let $Y\in P(X)$.  Let $y=\bigvee Y$.  $x$ exists because $B$ is complete.  So $f(y)=f(\bigvee Y) = \bigcup \lbrace f(x) \mid x \in Y \rbrace = \bigcup \lbrace \lbrace x\rbrace \mid x\in Y\rbrace = Y$, since each $x\in Y$ is an atom.  \end{proof}

Rewording the above proposition, we have
\begin{thm} Any complete atomic Boolean algebra is isomorphic (as complete Boolean algebras) to the powerset of some set, namely, the set of all of its atoms.  \end{thm}

A useful application of this representation theorem is the following:
\begin{cor} The cardinality of a finite Boolean algebra is a power of $2$. \end{cor}
\begin{proof} Every finite Boolean algebra is complete and atomic, and hence isomorphic to the powerset of a set, which is also finite, and the result follows. \end{proof}

\textbf{Remark}.  The proof does not depend on the representation of a Boolean algebra by a field of sets.
%%%%%
%%%%%
\end{document}
