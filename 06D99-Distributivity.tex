\documentclass[12pt]{article}
\usepackage{pmmeta}
\pmcanonicalname{Distributivity}
\pmcreated{2013-03-22 13:47:00}
\pmmodified{2013-03-22 13:47:00}
\pmowner{yark}{2760}
\pmmodifier{yark}{2760}
\pmtitle{distributivity}
\pmrecord{15}{34493}
\pmprivacy{1}
\pmauthor{yark}{2760}
\pmtype{Definition}
\pmcomment{trigger rebuild}
\pmclassification{msc}{06D99}
\pmclassification{msc}{16-00}
\pmclassification{msc}{13-00}
\pmclassification{msc}{17-00}
\pmsynonym{distributive law}{Distributivity}
\pmsynonym{distributive property}{Distributivity}
\pmrelated{Ring}
\pmrelated{DistributiveLattice}
\pmrelated{NearRing}
\pmdefines{distributive}
\pmdefines{left distributive}
\pmdefines{right distributive}
\pmdefines{left-distributive}
\pmdefines{right-distributive}
\pmdefines{distributes over}
\pmdefines{left distributivity}
\pmdefines{right distributivity}
\pmdefines{left distributes over}
\pmdefines{left distributive law}
\pmdefines{right distributive law}

\endmetadata

\usepackage{amssymb}
\usepackage{amsmath}
\usepackage{amsfonts}
\begin{document}
Given a \PMlinkname{set}{Set} $S$ with two binary operations $+\colon S \times S \to S$ and $\cdot\colon S \times S \to S$, we say that $\cdot$ is {\em right distributive} over $+$ if
$$(a+b) \cdot c = (a \cdot c) + (b \cdot c)\mathrm{~for~all~} a,b,c \in S$$
and {\em left distributive} over $+$ if
$$a \cdot (b+c) = (a \cdot b) + (a \cdot c)\mathrm{~for~all~}a,b,c \in S.$$
If $\cdot$ is both left and right distributive over $+$, then it is said to be {\em distributive} over $+$ (or, alternatively, we may say that $\cdot$ {\em distributes over} $+$).
%%%%%
%%%%%
\end{document}
