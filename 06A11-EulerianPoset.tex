\documentclass[12pt]{article}
\usepackage{pmmeta}
\pmcanonicalname{EulerianPoset}
\pmcreated{2013-03-22 14:08:16}
\pmmodified{2013-03-22 14:08:16}
\pmowner{mps}{409}
\pmmodifier{mps}{409}
\pmtitle{Eulerian poset}
\pmrecord{9}{35552}
\pmprivacy{1}
\pmauthor{mps}{409}
\pmtype{Definition}
\pmcomment{trigger rebuild}
\pmclassification{msc}{06A11}
\pmclassification{msc}{06A07}
\pmrelated{GradedPoset}

\endmetadata

% this is the default PlanetMath preamble.  as your knowledge
% of TeX increases, you will probably want to edit this, but
% it should be fine as is for beginners.

% almost certainly you want these
\usepackage{amssymb}
\usepackage{amsmath}
\usepackage{amsfonts}

% used for TeXing text within eps files
%\usepackage{psfrag}
% need this for including graphics (\includegraphics)
%\usepackage{graphicx}
% for neatly defining theorems and propositions
%\usepackage{amsthm}
% making logically defined graphics
%%\usepackage{xypic}

% there are many more packages, add them here as you need them

% define commands here
\begin{document}
A graded poset $P$ is {\it Eulerian} if for any $x\le y$ in $P$, the 
\PMlinkname{M\"{o}bius function}{MoebiusInversionFormula}
 of the interval $[x,y]$ is given by
\[\mu_P(x,y)=(-1)^{\rho(y)-\rho(x)},\]
where $\rho$ is the rank function of $P$.  If $P$ is finite and
graded with strictly
positive rank, then we can equivalently and more simply say that $P$ is
Eulerian if and only if each nontrivial interval of $P$ has the same number of
elements of even rank as odd rank.

For example, the \PMlinkname{lattice}{Lattice} $M_3$ is not Eulerian, because it
has 3 elements of rank 1 but only 2 elements of ranks 0 or 2.
\[\xymatrix{
             & \widehat{1}\ar@{-}[dl]\ar@{-}[d]\ar@{-}[dr] & \\
a\ar@{-}[dr] & b\ar@{-}[d] & c\ar@{-}[dl] \\
             & \widehat{0} &
}\]
On the other hand, the Boolean lattice $B_n$ is always Eulerian.
\[\xymatrix{
             & \widehat{1}\ar@{-}[dl]\ar@{-}[d]\ar@{-}[dr] & \\
ab\ar@{-}[d]\ar@{-}[dr] & ac\ar@{-}[dl]\ar@{-}[dr] & bc\ar@{-}[dl]\ar@{-}[d] \\
a\ar@{-}[dr] & b\ar@{-}[d] & c\ar@{-}[dl] \\
             & \widehat{0} &
}\]
Each interval of $B_n$ is isomorphic to a smaller Boolean lattice, so
we can give a proof by induction that $B_n$ is Eulerian.  Now $B_0$ is
a single point and so there is only one interval, which has M\"{o}bius
function 1.  So to show that $B_n$ is Eulerian, we just need to show
that the M\"obius function of $B_n$ is $(-1)^n$.  Recall that by definition,
\begin{equation}
\mu_{B_n}(B_n) = - \sum_{S\subset\{1,\dots,n\}}\mu(\emptyset,S)\tag{*}
\end{equation}
Thus $\sum_{S\subseteq\{1,\dots,n\}}\mu_{B_n}(\emptyset,S)=0$.  We can apply
the binomial theorem to $0$ to see that
$0=(-1+1)^n=\sum_{k=0}^n\binom{n}{k}(-1)^k$
and thus
\begin{align*}
(-1)^n        &= - \sum_{k=0}^{n-1}\binom{n}{k}(-1)^k \\
              &= - \sum_{S\subset\{1,\dots,n\}}\mu(\emptyset,S)\tag{**}
\end{align*}
It then follows from Equation~$(*)$ and Equation~$(**)$ that $\mu(B_n)=(-1)^n$.

Eulerian posets are generalizations of certain special posets, such as
face lattices of polytopes.

\section*{Eulerian posets of small rank}

In this section we classify bounded Eulerian posets of rank less than 4.

For $n<2$, there is exactly one bounded poset of rank $n$, the chain,
and it is Eulerian.

A bounded poset of rank $2$ is determined precisely by its number of
coatoms.  The ab-index of a rank $2$ bounded Eulerian poset with $r$
coatoms is $a + (r-1)b$, which is a cd-index if and only if $r=2$.
Since every Eulerian poset has a cd-index, this implies that every
rank $2$ bounded Eulerian poset has exactly two coatoms.  Thus up to
isomorphism $B_2$ is the only rank $2$ bounded Eulerian poset.

Let $P$ be a rank $3$ bounded Eulerian poset.  Since $P$ has as many
elements of odd rank as even rank, it has exactly the same number of
atoms as coatoms.  Since every rank $2$ interval in $P$ is a copy of
$B_2$, it follows that $P$ is not a chain and so has $n\ge 2$ atoms.
Moreover, every atom is covered by exactly two coatoms, and every
coatom covers exactly two atoms.  That is, the Hasse diagram of the
interior of $P$ is a 2-regular bipartite graph.  This graph admits a
decomposition into even cycles.  In each cycle, half of the nodes are
atoms and half are coatoms.  The non-induced subposet of $P$
consisting of the points on a cycle of length $2k$ plus the minimum
and maximum elements of $P$ is the face poset of a $k$-gon.  Hence $P$
itself is the face poset of the disjoint union of one or more $k$-gons
for various $k\ge 2$.

\begin{thebibliography}{1}
\bibitem{cite:RS}
Stanley, R., \emph{Enumerative Combinatorics}, vol. 1, 2nd ed., Cambridge
University Press, Cambridge, 1996.
\end{thebibliography}

\PMlinkescapeword{bounded}
\PMlinkescapeword{function}
\PMlinkescapeword{rank}
\PMlinkescapeword{ranks}
\PMlinkescapeword{section}
%%%%%
%%%%%
\end{document}
