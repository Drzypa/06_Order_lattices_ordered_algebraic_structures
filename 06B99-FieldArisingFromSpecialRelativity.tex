\documentclass[12pt]{article}
\usepackage{pmmeta}
\pmcanonicalname{FieldArisingFromSpecialRelativity}
\pmcreated{2016-04-20 13:42:53}
\pmmodified{2016-04-20 13:42:53}
\pmowner{pahio}{2872}
\pmmodifier{pahio}{2872}
\pmtitle{field arising from special relativity}
\pmrecord{8}{88254}
\pmprivacy{1}
\pmauthor{pahio}{2872}
\pmtype{Topic}

\endmetadata

% this is the default PlanetMath preamble.  as your knowledge
% of TeX increases, you will probably want to edit this, but
% it should be fine as is for beginners.

% almost certainly you want these
\usepackage{amssymb}
\usepackage{amsmath}
\usepackage{amsfonts}

% need this for including graphics (\includegraphics)
\usepackage{graphicx}
% for neatly defining theorems and propositions
\usepackage{amsthm}

% making logically defined graphics
%\usepackage{xypic}
% used for TeXing text within eps files
%\usepackage{psfrag}

% there are many more packages, add them here as you need them

% define commands here

\begin{document}
The velocities $u$ and $v$ of two bodies moving along a line 
obey, by the special theory of relativity, the addition rule
\begin{align} 
u\oplus v \;:=\; \frac{u+v}{1+\frac{uv}{c^2}},
\end{align}
  where $c$ is the velocity of light.\, As $c$ is unreachable for any material body, it 
  plays for the velocities of the bodies the role of the infinity.\, These velocities $v$
  thus satisfy always
      $$|v| \;<\; c.$$
 By (1) we get
 $$c\oplus c \;=\; c, \quad c \oplus v \;=\; c$$
  for $|v| < c$;\, so $c$ behaves like the infinity.
  
  One can define the \PMlinkname{mapping}{mapping}\,  $f:\; \mathbb{R} \to (-c,\,c) = S$\; by setting
  \begin{align}
  f(x) \;:=\; c\;\tanh{x}
  \end{align}
  which is easily seen to be a bijection.
  
  Define also the \PMlinkname{binary operation}{binaryoperation} $\odot$  
  for the \PMlinkname{numbers}{number} $u,\,v$ of the 
  \PMlinkname{open interval}{interval}\, 
  $(-c,\,c)$ by
  \begin{align}
  u \odot v \;=\; c\; 
  \tanh\left[\left(\mbox{artanh}\frac{u}{c}\right)
  \left(\mbox{artanh}\frac{v}{c}\right)\right].
  \end{align}
  Then the system $(S,\oplus,\odot)$ may be checked to be a ring 
  and the bijective mapping (2) to be 
  \PMlinkname{homomorphic}{structurehomomorphism}:
  $$f(x+y) \;=\; f(x)\oplus f(y), \quad 
  f(xy) \;=\; f(x)\odot f(y)$$
  
  
  Consequently, the system $(S,\oplus,\odot)$, as the 
  \PMlinkname{homomorphic image}{homomorphicimageofgroup} of 
  the field   $(\mathbb{R},+,\cdot)$, also itself is a field.
  
  Baker [1] calls the numbers of the set $S$, i.e. $(-c,\,c)$, 
  the {\it Einstein numbers}.\\
  
  
\begin{thebibliography}{8}
\bibitem{baker}{\sc G. A. Baker, Jr.}: ``Einstein numbers''. --\emph{Amer. Math. Monthly} \textbf{61} 
(1954), 39--41.
\bibitem{davis}{\sc H. T. Davis}: \emph{College algebra}. Prentice-Hall, N.Y. (1940), 351.
\bibitem{gregor}{\sc T. Gregor \& J. Halu\v{s}ka}: \emph{Two-dimensional Einstein 
numbers and associativity}.  
\PMlinkexternal{arXiv}{http://arxiv.org/abs/1309.0660} (2013)
\end{thebibliography}\\

\end{document}
