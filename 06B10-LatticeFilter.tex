\documentclass[12pt]{article}
\usepackage{pmmeta}
\pmcanonicalname{LatticeFilter}
\pmcreated{2013-03-22 15:49:01}
\pmmodified{2013-03-22 15:49:01}
\pmowner{CWoo}{3771}
\pmmodifier{CWoo}{3771}
\pmtitle{lattice filter}
\pmrecord{9}{37782}
\pmprivacy{1}
\pmauthor{CWoo}{3771}
\pmtype{Definition}
\pmcomment{trigger rebuild}
\pmclassification{msc}{06B10}
\pmsynonym{ultra filter}{LatticeFilter}
\pmsynonym{ultra-filter}{LatticeFilter}
\pmsynonym{maximal filter}{LatticeFilter}
\pmrelated{Ultrafilter}
\pmrelated{UpperSet}
\pmrelated{LatticeIdeal}
\pmrelated{OrderIdeal}
\pmdefines{filter}
\pmdefines{prime filter}
\pmdefines{ultrafilter}
\pmdefines{filter generated by}
\pmdefines{principal filter}

\endmetadata

\usepackage{amssymb,amscd}
\usepackage{amsmath}
\usepackage{amsfonts}

% used for TeXing text within eps files
%\usepackage{psfrag}
% need this for including graphics (\includegraphics)
%\usepackage{graphicx}
% for neatly defining theorems and propositions
%\usepackage{amsthm}
% making logically defined graphics
%%%\usepackage{xypic}

% define commands here
\begin{document}
Let $L$ be a lattice.  A \emph{filter} (of $L$) is the dual concept of an \PMlinkname{ideal}{LatticeIdeal}.  Specifically, a filter $F$ of $L$ is a non-empty subset of $L$ such that
\begin{enumerate}
\item $F$ is a sublattice of $L$, and
\item for any $a\in F$ and $b\in L$, $a\vee b\in F$.
\end{enumerate}

The first condition can be replaced by a weaker one: 
for any $a,b\in F$, $a\wedge b\in F$.

An equivalent characterization of a filter $I$ in a lattice $L$ is
\begin{enumerate}
\item for any $a,b\in F$, $a\wedge b\in F$, and
\item for any $a\in F$, if $a\le b$, then $b\in F$.
\end{enumerate}

Note that the dualization switches the meet and join operations, as well as reversing the ordering relationship.

\textbf{Special Filters}. Let $F$ be a filter of a lattice $L$. Some of the common types of filters are defined below.
\begin{itemize}
\item $F$ is a \emph{proper filter} if $F\ne L$, and, if $L$ contains $0$, $F\ne 0$.
\item $F$ is a \emph{prime filter} if it is proper, and $a\vee b\in F$ implies that either $a\in F$ or $b\in F$.
\item $F$ is an \emph{ultrafilter} (or \emph{maximal filter}) of $L$ if $F$ is proper and the only filter properly contains $F$ is $L$.
\item \textbf{filter generated by a set}. Let $X$ be a subset of a lattice $L$. Let $T$ be the set of all filters of $L$ containing $X$. Since $T\ne\varnothing$ ($L\in T$), the intersection $N$ of all elements in $T$, is also a filter of $L$ that contains $X$. $N$ is called the \emph{filter generated by} $X$, written $[X)$. If $X$ is a singleton $\lbrace x\rbrace$, then $N$ is said to be a \emph{principal filter} generated by $x$, written $[x)$.
\end{itemize}

\textbf{Examples}.
\begin{enumerate}
\item Consider the positive integers, with meet and join defined by the greatest common divisor and the least common multiple operations.  Then the positive even numbers form a filter, generated by $2$.  If we toss in $3$ as an additional element, then $1=2\wedge 3\in[\lbrace 2,3\rbrace)$ and consequently any positive integer $i\in[\lbrace 2,3\rbrace)$, since $1\le i$.  In general, if $p,q$ are relatively prime, then $[\lbrace p,q\rbrace)=\mathbb{Z}^{+}$.  In fact, any proper filter in $\mathbb{Z}^{+}$ is principal.  When the generator is prime, the filter is prime, which is also maximal.  So prime filters and ultrafilters coincide in $\mathbb{Z}^{+}$.
\item Let $A$ be a set and $2^A$ the power set of $A$.  If the set inclusion is the ordering defined on $2^A$, then the definition of a filter here coincides with the ususal definition of a \PMlinkname{filter}{Filter} on a set in general.
\end{enumerate}

\textbf{Remark}.  If $F$ is both a filter and an ideal of a lattice $L$, then $F=L$.
%%%%%
%%%%%
\end{document}
