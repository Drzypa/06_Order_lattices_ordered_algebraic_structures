\documentclass[12pt]{article}
\usepackage{pmmeta}
\pmcanonicalname{RationalRankOfAGroup}
\pmcreated{2013-03-22 16:53:25}
\pmmodified{2013-03-22 16:53:25}
\pmowner{polarbear}{3475}
\pmmodifier{polarbear}{3475}
\pmtitle{rational rank of a group}
\pmrecord{9}{39146}
\pmprivacy{1}
\pmauthor{polarbear}{3475}
\pmtype{Definition}
\pmcomment{trigger rebuild}
\pmclassification{msc}{06F20}
\pmclassification{msc}{20K20}
\pmclassification{msc}{20K15}
\pmclassification{msc}{20K99}
\pmdefines{rationally independent}
\pmdefines{rational rank}
\pmdefines{divisible hull}

% this is the default PlanetMath preamble.  as your knowledge
% of TeX increases, you will probably want to edit this, but
% it should be fine as is for beginners.

% almost certainly you want these
\usepackage{amssymb}
\usepackage{amsmath}
\usepackage{amsfonts}

% used for TeXing text within eps files
%\usepackage{psfrag}
% need this for including graphics (\includegraphics)
%\usepackage{graphicx}
% for neatly defining theorems and propositions
\usepackage{amsthm}
% making logically defined graphics
%%%\usepackage{xypic}
\newtheorem{defn}{Definition}
\newtheorem{prop}{Proposition}

% there are many more packages, add them here as you need them

% define commands here

\begin{document}
 In the following, $G$ is an abelian group.
\begin{defn} The group $\mathbb{Q} \otimes_{\mathbb{Z}} G$ is called the divisible hull of $G$. \end{defn} It is a $\mathbb{Q}$-vector space such that the scalar $\mathbb{Z}$-multiplication of $G$ is extended to $\mathbb{Q}$.
\begin{defn} The elements $g_1, g_2, ... , g_r \in G$ are called rationally independent if they are linearly independent over $\mathbb{Z}$, i.e. for all $n_1, ... , n_r \in Z$:\begin{equation*}
n_1 g_1 + ... + n_r g_r = 0 \Rightarrow n_1 = ... = n_r = 0. \end{equation*}\end{defn} 
\begin{defn} The dimension of  $\mathbb{Q} \otimes_{\mathbb{Z}} G$ over $\mathbb{Q}$ is called the rational rank of $G$.\end{defn} We denote the rational rank of $G$ by $r(G)$.\newline
\textbf{Example}:\newline
$r(\mathbb{Z} \times \mathbb{Z}) = 2$ because $\mathbb{Q} \otimes_\mathbb{Z} (\mathbb{Z} \times \mathbb{Z}) = (\mathbb{Q} \otimes_{\mathbb{Z}} \mathbb{Z}) \times (\mathbb{Q} \otimes_{\mathbb{Z}} \mathbb{Z}) = \mathbb{Q} \times \mathbb{Q}$.\newline
\textbf{Properties}:
\begin{itemize}
\item If $H$ is a subgroup of $G$ then we have:\begin{equation*}
r(G) = r(H) + r(G/H). \end{equation*} It results from the fact that $_\mathbb{Z}\mathbb{Q}$ is a flat module.
\item The rational rank of the group $G$ can be defined as the least upper bound (finite or infinite) of the cardinals $r$ such that there exist $r$ rationally independent elements in $G$.
\end{itemize}
  

%%%%%
%%%%%
\end{document}
