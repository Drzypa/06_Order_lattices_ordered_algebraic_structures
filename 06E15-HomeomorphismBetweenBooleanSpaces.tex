\documentclass[12pt]{article}
\usepackage{pmmeta}
\pmcanonicalname{HomeomorphismBetweenBooleanSpaces}
\pmcreated{2013-03-22 19:09:04}
\pmmodified{2013-03-22 19:09:04}
\pmowner{CWoo}{3771}
\pmmodifier{CWoo}{3771}
\pmtitle{homeomorphism between Boolean spaces}
\pmrecord{4}{42055}
\pmprivacy{1}
\pmauthor{CWoo}{3771}
\pmtype{Result}
\pmcomment{trigger rebuild}
\pmclassification{msc}{06E15}
\pmclassification{msc}{06B30}
\pmrelated{DualOfStoneRepresentationTheorem}

\endmetadata

\usepackage{amssymb,amscd}
\usepackage{amsmath}
\usepackage{amsfonts}
\usepackage{mathrsfs}

% used for TeXing text within eps files
%\usepackage{psfrag}
% need this for including graphics (\includegraphics)
%\usepackage{graphicx}
% for neatly defining theorems and propositions
\usepackage{amsthm}
% making logically defined graphics
%%\usepackage{xypic}
\usepackage{pst-plot}

% define commands here
\newcommand*{\abs}[1]{\left\lvert #1\right\rvert}
\newtheorem{prop}{Proposition}
\newtheorem{thm}{Theorem}
\newtheorem{lem}{Lemma}
\newtheorem{ex}{Example}
\newcommand{\real}{\mathbb{R}}
\newcommand{\pdiff}[2]{\frac{\partial #1}{\partial #2}}
\newcommand{\mpdiff}[3]{\frac{\partial^#1 #2}{\partial #3^#1}}
\begin{document}
In this entry, we derive a test for deciding when a bijection between two Boolean spaces is a homeomorphism.

We start with two general remarks.

\begin{lem}  If $Y$ is zero-dimensional, then $f:X\to Y$ is continuous provided that $f^{-1}(U)$ is open for every clopen set $U$ in $Y$. \end{lem}
\begin{proof}  Since $Y$ is zero-dimensional, $Y$ has a basis of clopen sets.  To check the continuity of $f$, it is enough to check that $f^{-1}(U)$ is open for each member of the basis, which is true by assumption.  Hence $f$ is continuous.
\end{proof}

\begin{lem} If $X$ is compact and $Y$ is Hausdorff, and $f$ is a bijection, then $f$ is a homeomorphism iff it is continuous. \end{lem}
\begin{proof}  One direction is obvious.  We want to show that $f^{-1}$ is continuous, or equivalently, for any closed set $U$ in $X$, $f(U)$ is closed in $Y$.  Since $X$ is compact, $U$ is compact, and therefore $f(U)$ is compact since $f$ is continuous.  But $Y$ is Hausdorff, so $f(U)$ is closed.
\end{proof}

\begin{prop} If $X,Y$ are Boolean spaces, then a bijection $f:X\to Y$ is homeomorphism iff it maps clopen sets to clopen sets.  \end{prop}
\begin{proof} Once more, one direction is clear.  Now, suppose $f$ maps clopen sets to clopen sets.  Since $X$ is zero-dimensional, $f^{-1}:Y\to X$ is continuous by the first proposition.  Since $Y$ is compact and $X$ Hausdorff, $f^{-1}$ is a homeomorphism by the second proposition.
\end{proof}
%%%%%
%%%%%
\end{document}
