\documentclass[12pt]{article}
\usepackage{pmmeta}
\pmcanonicalname{CommutativityRelationInAnOrthocomplementedLattice}
\pmcreated{2013-03-22 16:43:22}
\pmmodified{2013-03-22 16:43:22}
\pmowner{CWoo}{3771}
\pmmodifier{CWoo}{3771}
\pmtitle{commutativity relation in an orthocomplemented lattice}
\pmrecord{6}{38943}
\pmprivacy{1}
\pmauthor{CWoo}{3771}
\pmtype{Definition}
\pmcomment{trigger rebuild}
\pmclassification{msc}{06C15}
\pmclassification{msc}{03G12}
\pmdefines{dually commute}
\pmdefines{orthogonally commute}

\usepackage{amssymb,amscd}
\usepackage{amsmath}
\usepackage{amsfonts}

% used for TeXing text within eps files
%\usepackage{psfrag}
% need this for including graphics (\includegraphics)
%\usepackage{graphicx}
% for neatly defining theorems and propositions
\usepackage{amsthm}
% making logically defined graphics
%%\usepackage{xypic}
\usepackage{pst-plot}
\usepackage{psfrag}

% define commands here
\newtheorem{prop}{Proposition}
\newtheorem{thm}{Theorem}
\newtheorem{ex}{Example}
\newcommand{\com}{\operatorname{C}}
\newcommand{\dcom}{\operatorname{D}}
\newcommand{\mcom}{\operatorname{M}}
\begin{document}
Let $L$ be an orthocomplemented lattice with $a,b\in L$.  We say that $a$ \emph{commutes} with $b$ if $a=(a\wedge b)\vee (a\wedge b^{\perp})$.  When $a$ commutes with $b$, we write $a\com b$.  Dualize everything, we have that $a$ \emph{dually commutes} with $b$, written $a\dcom b$, if $a=(a\vee b)\wedge (a\vee b^{\perp})$.

\textbf{Some properties}.  Below are some properties of the commutativity relations $\com$ and $\dcom$.
\begin{enumerate}
\item $\com$ is reflexive.
\item $a \com b$ iff $a \com b^{\perp}$.
\item $a \com b$ iff $a^{\perp} \dcom b^{\perp}$.
\item if $a\le b$ or $a\le b^{\perp}$, then $a\com b$.
\item $a$ is said to \emph{orthogonally commute} with $b$ if $a \com b$ and $b\com a$.  In this case, we write $a \mcom b$.  The terminology comes from the following fact: $a \mcom b$ iff there are $x,y,z,t\in L$, with:
\begin{enumerate}
\item $x\perp y$ ($x$ is orthogonal to $y$, or $x\le y^{\perp}$), 
\item $z\perp t$, 
\item $x\perp z$,
\item $a=x\vee y$, and 
\item $b=z\vee t$.
\end{enumerate}
\item $\com$ is symmetric iff $\dcom=\com (=\mcom)$ iff $L$ is an orthomodular lattice.
\item $\com$ is an equivalence relation iff $\com=L\times L$ iff $L$ is a Boolean algebra.
\end{enumerate}

\textbf{Remark}.  More generally, one can define commutativity $\com$ on an orthomodular poset $P$: for $a,b\in P$, $a \com b$ iff $a\wedge b$, $a\wedge b^{\perp}$, and $(a\wedge b)\vee (a\wedge b^{\perp})$ exist, and $(a\wedge b)\vee (a\wedge b^{\perp})=a$.  Dual commutativity and mutual commutativity in an orthomodular poset are defined similarly (with the provision that the binary operations on the pair of elements are meaningful).

\begin{thebibliography}{8}
\bibitem{lb} L. Beran, {\em Orthomodular Lattices, Algebraic Approach}, Mathematics and Its Applications (East European Series), D. Reidel Publishing Company, Dordrecht, Holland (1985).
\end{thebibliography}
%%%%%
%%%%%
\end{document}
