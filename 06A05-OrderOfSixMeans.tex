\documentclass[12pt]{article}
\usepackage{pmmeta}
\pmcanonicalname{OrderOfSixMeans}
\pmcreated{2013-03-22 18:45:28}
\pmmodified{2013-03-22 18:45:28}
\pmowner{pahio}{2872}
\pmmodifier{pahio}{2872}
\pmtitle{order of six means}
\pmrecord{4}{41537}
\pmprivacy{1}
\pmauthor{pahio}{2872}
\pmtype{Theorem}
\pmcomment{trigger rebuild}
\pmclassification{msc}{06A05}
\pmclassification{msc}{26B35}
\pmclassification{msc}{26D07}
\pmrelated{Mean3}
\pmrelated{ComparisonOfPythagoreanMeans}
\pmrelated{InequalityWithAbsoluteValues}
\pmrelated{LehmerMean}

% this is the default PlanetMath preamble.  as your knowledge
% of TeX increases, you will probably want to edit this, but
% it should be fine as is for beginners.

% almost certainly you want these
\usepackage{amssymb}
\usepackage{amsmath}
\usepackage{amsfonts}

% used for TeXing text within eps files
%\usepackage{psfrag}
% need this for including graphics (\includegraphics)
%\usepackage{graphicx}
% for neatly defining theorems and propositions
 \usepackage{amsthm}
% making logically defined graphics
%%%\usepackage{xypic}

% there are many more packages, add them here as you need them

% define commands here

\theoremstyle{definition}
\newtheorem*{thmplain}{Theorem}

\begin{document}
The \PMlinkescapetext{size order} of the six usual means of two positive numbers ($a$ and $b$) is from the least to the greatest one
\begin{enumerate}
\item harmonic mean,
\item geometric mean,
\item Heronian mean,
\item arithmetic mean,
\item quadratic mean,
\item contraharmonic mean,
\end{enumerate}
i. e. 
$$\frac{2ab}{a\!+\!b} \;\leqq\; \sqrt{ab} \;\leqq\; \frac{a\!+\!\sqrt{ab}\!+\!b}{3} \;\leqq\; \frac{a\!+\!b}{2} 
\;\leqq\; \sqrt{\frac{a^2\!+\!b^2}{2}} \;\leqq\; \frac{a^2\!+\!b^2}{a\!+\!b}.$$
The equality signs are valid iff \,$a = b$.\\

{\em Proof.}\; If\, $x^2-y^2 \geqq 0$\, for nonnegative $x$ and $y$,\, then\, $x \geqq y$.

``$1\leqq2$'':\\
$\displaystyle\left(\sqrt{ab}\right)^2-\left(\frac{a\!+\!b}{2}\right)^2 = ab\!-\!\frac{4a^2b^2}{(a\!+\!b)^2} 
= ab\left(1\!-\!\frac{4ab}{(a\!+\!b)^2}\right) = ab\cdot\frac{(a\!+\!b)^2-4ab}{(a\!+\!b)^2} 
= \frac{ab(a\!-\!b)^2}{(a+b)^2} \geqq 0$\\

``$2\leqq3$'' and ``$3\leqq4$'': proven in\, Heronian mean is between geometric and arithmetic mean

``$4\leqq5$'':\\
$\displaystyle\left(\sqrt{\frac{a^2\!+\!b^2}{2}}\right)^2-\left(\frac{a\!+\!b}{2}\right)^2 
= \frac{2a^2\!+\!2b^2\!-\!a^2\!-\!2ab\!-\!b^2}{4} = \left(\frac{a\!-\!b}{2}\right)^2 \geqq 0$\\

``$5\leqq6$'':\\
$\displaystyle\left(\frac{a^2\!+\!b^2}{a\!+\!b}\right)^2-\left(\sqrt{\frac{a^2\!+\!b^2}{2}}\right)^2 
= \frac{2(a^2\!+\!b^2)^2-(a^2\!+\!b^2)(a\!+\!b)^2}{2(a\!+\!b)^2} = 
\frac{(a^2\!+\!b^2)(2a^2\!+\!2b^2\!-\!a^2\!-\!2ab\!-\!b^2)}{2(a\!+\!b)^2}\\ 
= \frac{(a^2\!+\!b^2)(a\!-\!b)^2}{2(a\!+\!b)^2} \geqq 0$
%%%%%
%%%%%
\end{document}
