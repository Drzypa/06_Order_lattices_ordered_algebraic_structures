\documentclass[12pt]{article}
\usepackage{pmmeta}
\pmcanonicalname{DistributivityInPogroups}
\pmcreated{2013-03-22 17:05:12}
\pmmodified{2013-03-22 17:05:12}
\pmowner{CWoo}{3771}
\pmmodifier{CWoo}{3771}
\pmtitle{distributivity in po-groups}
\pmrecord{6}{39379}
\pmprivacy{1}
\pmauthor{CWoo}{3771}
\pmtype{Definition}
\pmcomment{trigger rebuild}
\pmclassification{msc}{06F05}
\pmclassification{msc}{06F20}
\pmclassification{msc}{06F15}
\pmclassification{msc}{20F60}

\endmetadata

\usepackage{amssymb,amscd}
\usepackage{amsmath}
\usepackage{amsfonts}
\usepackage{mathrsfs}

% used for TeXing text within eps files
%\usepackage{psfrag}
% need this for including graphics (\includegraphics)
%\usepackage{graphicx}
% for neatly defining theorems and propositions
\usepackage{amsthm}
% making logically defined graphics
%%\usepackage{xypic}
\usepackage{pst-plot}
\usepackage{psfrag}

% define commands here
\newtheorem{prop}{Proposition}
\newtheorem{thm}{Theorem}
\newtheorem{ex}{Example}
\newcommand{\real}{\mathbb{R}}
\newcommand{\pdiff}[2]{\frac{\partial #1}{\partial #2}}
\newcommand{\mpdiff}[3]{\frac{\partial^#1 #2}{\partial #3^#1}}
\begin{document}
Let $G$ be a po-group and $A$ be a set of elements of $G$.  Denote the supremum of elements of $A$, if it exists, by $\bigvee A$.  Similarly, denote the infimum of elements of $A$, if it exists, by $\bigwedge A$.  Furthermore, let $A^{-1}=\lbrace a^{-1}\mid a\in A\rbrace$, and for any $g\in G$, let $gA=\lbrace ga\mid a\in A\rbrace$ and $Ag=\lbrace ag\mid a\in A\rbrace$.
\begin{enumerate}
\item If $\bigvee A$ exists, so do $\bigvee gA$ and $\bigvee Ag$.
\item If 1. is true, then $g\bigvee A=\bigvee gA =\bigvee Ag$.
\item $\bigvee A$ exists iff $\bigwedge A^{-1}$ exists; when this is the case, $\bigwedge A^{-1}=(\bigvee A)^{-1}$.  
\item If $\bigwedge A$ exists, so do $\bigwedge gA$, and $\bigwedge Ag$.
\item If 4. is true, then $g\bigwedge A=\bigwedge gA=\bigwedge Ag$.
\item If 1. is true and $A=\lbrace a,b\rbrace$, then $a\wedge b$ exists and is equal to $a(a\vee b)^{-1}b$.
\end{enumerate}

\begin{proof}  Suppose $\bigvee A$ exists.
\begin{itemize}
\item (1. and 2.)  Clearly, for each $a\in A$, $a\le \bigvee A$, so that $ga\le g\bigvee A$, and therefore elements of $gA$ are bounded from above by $g\bigvee A$.  To show that $g\bigvee A$ is the least upper bound of elements of $gA$, suppose $b$ is the upper bound of elements of $gA$, that is, $ga\le b$ for all $a\in A$, this means that $a\le g^{-1}b$ for all $a\in A$.  Since $\bigvee A$ is the least upper bound of the $a$'s, $\bigvee A\le g^{-1}b$, so that $g\bigvee A \le b$.  This shows that $g\bigvee A$ is the supremum of elements of $gA$; in other words, $g\bigvee A=\bigvee gA$.  Similarly, $\bigvee Ag$ exists and $g\bigvee A=\bigvee Ag$ as well.
\item (3.)  Write $c=\bigvee A$.  Then $a\le c$ for each $a\in A$.  This means $c^{-1}\le a^{-1}$.  If $b\le a^{-1}$ for all $a\in A$, then $a\le b^{-1}$ for all $a\in A$, so that $c\le b^{-1}$, or $b\le c^{-1}$.  This shows that $c^{-1}$ is the greatest lower bound of elements of $A^{-1}$, or $(\bigvee A)^{-1}=\bigwedge A^{-1}$.  The converse is proved likewise.
\item (4. and 5.)  This is just the dual of 1. and 2., so the proof is omitted.
\item (6.)  If $A=\lbrace a,b\rbrace$, then $aA^{-1}b=A$, and the existence of $\bigwedge A$ is the same as the existence of $\bigwedge (aA^{-1}b)$, which is the same as the existence of $a (\bigwedge A^{-1}) b$ by 4 and 5 above.  Since $\bigvee A$ exists, so does $\bigwedge A^{-1}$, and hence $a (\bigwedge A^{-1}) b$, by 3 above.  Also by 3, we have the equality $a (\bigwedge A^{-1}) b=a(\bigvee A)^{-1} b$.  Putting everything together, we have the result: $a\wedge b=a(a\vee b)^{-1}b$.
\end{itemize}
This completes the proof.
\end{proof}

\textbf{Remark}.  From the above result, we see that group multiplication distributes over arbitrary joins and meets, if these joins and meets exist.

One can use this result to prove the following: every Dedekind complete po-group is an Archimedean po-group.
\begin{proof}
Suppose $a^n\le b$ for all integers $n$.  Let $A=\lbrace a^n\mid n\in \mathbb{Z}\rbrace$.  Then $A$ is bounded from above by $b$ so has least upper bound $\bigvee A$.  Then $a\bigvee A=\bigvee aA=\bigvee A$, since $aA=A$.  As a result, multiplying both sides by $(\bigvee A)^{-1}$, we get $a=e$.
\end{proof}

\textbf{Remark}.  The above is a generalization of a famous property of the real numbers: $\mathbb{R}$ has the Archimedean property.
%%%%%
%%%%%
\end{document}
