\documentclass[12pt]{article}
\usepackage{pmmeta}
\pmcanonicalname{ProofOfSchroederBernsteinTheoremUsingTarskiKnasterTheorem}
\pmcreated{2013-03-22 15:30:24}
\pmmodified{2013-03-22 15:30:24}
\pmowner{kompik}{10588}
\pmmodifier{kompik}{10588}
\pmtitle{proof of Schroeder-Bernstein theorem using Tarski-Knaster theorem}
\pmrecord{7}{37367}
\pmprivacy{1}
\pmauthor{kompik}{10588}
\pmtype{Proof}
\pmcomment{trigger rebuild}
\pmclassification{msc}{06B99}
\pmrelated{SchroederBernsteintheorem}
\pmrelated{TarskiKnastertheorem}
\pmrelated{TarskiKnasterTheorem}
\pmrelated{SchroederBernsteinTheorem}

% this is the default PlanetMath preamble.  as your knowledge
% of TeX increases, you will probably want to edit this, but
% it should be fine as is for beginners.

% almost certainly you want these
\usepackage{amssymb}
\usepackage{amsmath}
\usepackage{amsfonts}
\usepackage{amsthm}

% used for TeXing text within eps files
%\usepackage{psfrag}
% need this for including graphics (\includegraphics)
%\usepackage{graphicx}
% for neatly defining theorems and propositions
%\usepackage{amsthm}
% making logically defined graphics
%%%\usepackage{xypic}

% there are many more packages, add them here as you need them

% define commands here
\newcommand{\sR}[0]{\mathbb{R}}
\newcommand{\sC}[0]{\mathbb{C}}
\newcommand{\sN}[0]{\mathbb{N}}
\newcommand{\sZ}[0]{\mathbb{Z}}

\usepackage{bbm}
\newcommand{\Z}{\mathbbmss{Z}}
\newcommand{\C}{\mathbbmss{C}}
\newcommand{\R}{\mathbbmss{R}}
\newcommand{\Q}{\mathbbmss{Q}}



\newcommand*{\norm}[1]{\lVert #1 \rVert}
\newcommand*{\abs}[1]{| #1 |}


\newcommand{\Zobr}[3]{#1:#2\to#3}

\newcommand{\vp}{\varphi}

\newcommand{\Invimg}[2]{\inv{#1}(#2)}
\newcommand{\Img}[2]{#1[#2]}
\begin{document}
The Tarski-Knaster theorem can be used to give a short, elegant proof of the Schroeder-Bernstein theorem.

\begin{proof}
Suppose $f:S\to T$ and $g:T\to S$ are injective.  Define a function $\varphi\colon P(S)\to P(S)$
by $\varphi(X)=S\setminus g(T\setminus f(X))$.

If $X\subseteq Y\subseteq S$, then $S\setminus g(T\setminus f(X))\subseteq S\setminus g(T\setminus f(Y))$, and so $\varphi$ is monotone.  Since $P(S)$ is a complete lattice, we may apply the Tarski-Knaster theorem to conclude that the set of fixed points of $\varphi$ is a complete lattice and thus nonempty.

Let $C$ be a fixed point of $\varphi$.  We have
\[
S\setminus C=g(T\setminus f(C)).
\]
Hence $g|_{T\setminus f(C)}\colon T\setminus f(C)\to S\setminus C$
and $f|_C:C\to f(C)$ are bijections.  We can therefore construct the desired bijection $h\colon S\to T$ by defining
\[
h(x)=\begin{cases}
f(x)                           & \text{if\ }x\in C \\
(g|_{T\setminus f(C)})^{-1}(x) & \text{if\ }x\notin C.\qedhere
\end{cases}
\]
\end{proof}

The usual proof of Schroeder-Bernstein theorem explicitly constructs a fixed point of $\varphi$.

\begin{thebibliography}{1}
\bibitem{forster}
Thomas Forster, \emph{{Logic, induction and sets}}, {Cambridge University Press}, Cambridge, 2003.

\bibitem{kolibiar}
M.~Kolibiar, A.~Leg\'e\v{n}, T.~\v{S}al\'at, and \v{S}. Zn\'am, \emph{Algebra a pr\'\i buzn\'e discipl\'\i ny}, Alfa, Bratislava, 1992 (Slovak).
\end{thebibliography}
%%%%%
%%%%%
\end{document}
