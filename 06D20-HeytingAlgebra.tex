\documentclass[12pt]{article}
\usepackage{pmmeta}
\pmcanonicalname{HeytingAlgebra}
\pmcreated{2013-03-22 16:33:03}
\pmmodified{2013-03-22 16:33:03}
\pmowner{CWoo}{3771}
\pmmodifier{CWoo}{3771}
\pmtitle{Heyting algebra}
\pmrecord{20}{38734}
\pmprivacy{1}
\pmauthor{CWoo}{3771}
\pmtype{Definition}
\pmcomment{trigger rebuild}
\pmclassification{msc}{06D20}
\pmclassification{msc}{03G10}
\pmsynonym{pseudo-Boolean algebra}{HeytingAlgebra}
\pmrelated{QuantumTopos}
\pmrelated{Lattice}
\pmdefines{Heyting lattice}

\usepackage{amssymb,amscd}
\usepackage{amsmath}
\usepackage{amsfonts}

% used for TeXing text within eps files
%\usepackage{psfrag}
% need this for including graphics (\includegraphics)
%\usepackage{graphicx}
% for neatly defining theorems and propositions
\usepackage{amsthm}
% making logically defined graphics
%%\usepackage{xypic}
\usepackage{pst-plot}
\usepackage{psfrag}

% define commands here

\begin{document}
A \emph{Heyting lattice} $L$ is a Brouwerian lattice with a bottom element $0$.  Equivalently, $L$ is Heyting iff it is relatively pseudocomplemented and pseudocomplemented iff it is bounded and relatively pseudocomplemented.

Let $a^*$ denote the pseudocomplement of $a$ and $a\to b$ the pseudocomplement of $a$ relative to $b$. Then we have the following properties:

\begin{enumerate}
\item $a^*=a\to 0$ (equivalence of definitions)
\item $1^*=0$ (if $c=1\to 0$, then $c=c\wedge 1\le 0$ by the definition of $\to$.)
\item $a^*=1$ iff $a=0$ ($1=a\to 0$ implies that $c\wedge a\le 0$ whenever $c\le 1$.  In particular $a\le 1$, so $a=a\wedge a\le 0$ or $a=0$.  On the other hand, if $a=0$, then $a^*=0^*=0\to 0=1$.)
\item $a\le a^{**}$ and $a^*=a^{***}$ (already true in any pseudocomplemented lattice)
\item $a^*\le a\to b$ (since $a^*\wedge a=0\le b$)
\item $(a\to b)\wedge (a\to b^*)=a^*$ 
\begin{proof}
If $c\wedge a=0$, then $c\wedge a\le b$ so $c\le (a\to b)$, and $c\le (a\to b^*)$ likewise, so $c\le (a\to b)\wedge (a\to b^*)$.  This means precisely that $a^*=(a\to b)\wedge (a\to b^*)$.
\end{proof}
\item $a\to b\le b^*\to a^*$ (since $(a\to b)\wedge b^*\le (a\to b)\wedge (a\to b^*)=a^*)$
\item $a^*\vee b\le a\to b$ (since $b\wedge a\le b$ and $a^* \wedge a=0\le b$)
\end{enumerate}

Note that in property 4, $a\le a^{**}$, whereas $a^{**}\le a$ is in general not true, contrasting with the equality $a=a^{\prime\prime}$ in a Boolean lattice, where $^{\prime}$ is the complement operator.  It is easy to see that if $a^{**}\le a$ for all $a$ in a Heyting lattice $L$, then $L$ is a Boolean lattice.  In this case, the pseudocomplement coincides with the complement of an element $a^*=a^{\prime}$, and we have the equality in property 7: $a^*\vee b=a\to b$, meaning that the concept of \PMlinkname{relative pseudocomplementation}{RelativelyPseudocomplemented} coincides with the material implication in classical propositional logic.

A \emph{Heyting algebra} is a Heyting lattice $H$ such that $\to$ is a binary operator on $H$.  A Heyting algebra homomorphism between two Heyting algebras is a lattice homomorphism that preserves $0,1$, and $\to$.  In addition, if $f$ is a Heyting algebra homomorphism, $f$ preserves psudocomplementation: $f(a^*)=f(a\to 0)=f(a)\to f(0)=f(a)\to 0=f(a)^*$.

\textbf{Remarks}.  
\begin{itemize}
\item
In the literature, the assumption that a Heyting algebra contains $0$ is sometimes dropped.  Here, we call it a Brouwerian lattice instead.
\item
Heyting algebras are useful in modeling intuitionistic logic.  Every intuitionistic propositional logic can be modelled by a Heyting algebra, and every intuitionistic predicate logic can be modelled by a complete Heyting algebra.
\end{itemize}

%%%%%
%%%%%
\end{document}
