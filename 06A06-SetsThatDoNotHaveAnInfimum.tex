\documentclass[12pt]{article}
\usepackage{pmmeta}
\pmcanonicalname{SetsThatDoNotHaveAnInfimum}
\pmcreated{2013-03-22 13:09:51}
\pmmodified{2013-03-22 13:09:51}
\pmowner{sleske}{997}
\pmmodifier{sleske}{997}
\pmtitle{sets that do not have an infimum}
\pmrecord{9}{33607}
\pmprivacy{1}
\pmauthor{sleske}{997}
\pmtype{Example}
\pmcomment{trigger rebuild}
\pmclassification{msc}{06A06}
\pmrelated{InfimumAndSupremumForRealNumbers}

\usepackage{amssymb}
\usepackage{amsmath}
\usepackage{amsthm}
\usepackage{amsfonts}
\begin{document}
\PMlinkescapeword{complete}
\PMlinkescapeword{dense}
\PMlinkescapeword{natural}
Some examples for sets that do not have an infimum:

\begin{itemize}
\item The set $M_1:=\mathbb Q$ (as a subset of $\mathbb Q$) does not have
an infimum (nor a supremum).  Intuitively this is clear, as the set is
unbounded. The (easy) formal proof is left as an exercise for the reader.

\item A more interesting example: The set $M_2:=\{x \in \mathbb Q :
x^2\geq 2, x>0 \}$ (again as a subset of $\mathbb Q$) .
\begin{proof}
Clearly, $\inf(M_2)>0$. Assume $i>0$ is an infimum of $M_2$. Now we use
the fact that $\sqrt 2$ is not rational, and therefore $i<\sqrt 2$ or
$i>\sqrt 2$.

If $i<\sqrt 2$, choose any $j\in \mathbb Q$ from the interval $(i, \sqrt 2
) \subset \mathbb R$ (this is a real interval, but as the rational numbers
are \PMlinkname{dense}{Dense} in the real numbers, every nonempty interval in $\mathbb R$
contains a rational number, hence such a $j$ exists).

Then $j>i$, but $j< \sqrt 2$, hence $j^2<2$ and therefore $j$ is a lower
bound for $M_2$, which is a contradiction.

On the other hand, if $i>\sqrt 2$, the argument is very similar:
Choose any $j\in \mathbb Q$ from the interval $(\sqrt 2, i) \subset
\mathbb R$. Then $j<i$, but $j> \sqrt 2$, hence $j^2>2$ and therefore
$j\in M_2$. Thus $M_2$ contains an element $j$ smaller than $i$, which is
a contradiction to the assumption that $i=\inf(M_2)$
\end{proof}

Intuitively speaking, this example exploits the fact that $\mathbb Q$ does
not have ``enough elements''. More formally, $\mathbb Q$ as a metric space
is not \PMlinkname{complete}{Complete}. The $M_2$ defined above is the real interval
$M_2':=(\sqrt 2, \infty)\subset \mathbb R$ intersected with $\mathbb Q$.
$M_2'$ as a subset of $\mathbb R$ does have an infimum (namely $\sqrt 2$),
but as that is not an element of $\mathbb Q$, $M_2$ does not have an
infimum as a subset of $\mathbb Q$.

This example also makes it clear that it is important to clearly state the
superset one is working in when using the notion of infimum or supremum.

It also illustrates that the infimum is a natural generalization of the
minimum of a set, as a set that does not have a minimum may still have
an infimum (such as $M_2'$).

Of course all the ideas expressed here equally apply to the supremum, as the
two notions are completely analogous (just reverse all inequalities).
\end{itemize}
%%%%%
%%%%%
\end{document}
