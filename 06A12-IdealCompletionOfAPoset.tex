\documentclass[12pt]{article}
\usepackage{pmmeta}
\pmcanonicalname{IdealCompletionOfAPoset}
\pmcreated{2013-03-22 17:03:01}
\pmmodified{2013-03-22 17:03:01}
\pmowner{CWoo}{3771}
\pmmodifier{CWoo}{3771}
\pmtitle{ideal completion of a poset}
\pmrecord{8}{39340}
\pmprivacy{1}
\pmauthor{CWoo}{3771}
\pmtype{Definition}
\pmcomment{trigger rebuild}
\pmclassification{msc}{06A12}
\pmclassification{msc}{06A06}
\pmrelated{LatticeOfIdeals}
\pmdefines{ideal completion}

\usepackage{amssymb,amscd}
\usepackage{amsmath}
\usepackage{amsfonts}
\usepackage{mathrsfs}

% used for TeXing text within eps files
%\usepackage{psfrag}
% need this for including graphics (\includegraphics)
%\usepackage{graphicx}
% for neatly defining theorems and propositions
\usepackage{amsthm}
% making logically defined graphics
%%\usepackage{xypic}
\usepackage{pst-plot}
\usepackage{psfrag}

% define commands here
\newtheorem{prop}{Proposition}
\newtheorem{thm}{Theorem}
\newtheorem{ex}{Example}
\newcommand{\real}{\mathbb{R}}
\newcommand{\pdiff}[2]{\frac{\partial #1}{\partial #2}}
\newcommand{\mpdiff}[3]{\frac{\partial^#1 #2}{\partial #3^#1}}
\newcommand{\up}{\uparrow\!\!}
\newcommand{\down}{\downarrow\!\!}
\begin{document}
Let $P$ be a poset. Consider the set $\operatorname{Id}(P)$ of all order ideals of $P$.

\begin{thm} $\operatorname{Id}(P)$ is an algebraic dcpo, such that $P$ can be embedded in. \end{thm}
\begin{proof}
We shall list, and when necessary, prove the following series of facts which ultimately prove the main assertion.  For convenience, write $P'=\operatorname{Id}(P)$.
\begin{enumerate}
\item $P'$ is a poset with $\le$ defined by set theoretic inclusion.
\item For any $x\in P$, $\down x\in P'$.
\item $P$ can be embedded in $P'$.  The function $f:P\to P'$ defined by $f(x)=\down x$ is order preserving and one-to-one.  If $x\le y$, and $a\le x$, then $a\le y$, hence $\down x\subseteq \down y$.  If $\down x=\down y$, we have that $x\le y$ and $y\le x$, so $x=y$, since $\le$ is antisymmetric.
\item $P'$ is a dcpo.  Suppose $D$ is a directed set in $P'$.  Let $E=\bigcup D$.  For any $x,y\in E$, $x\in I$ and $y\in J$ for some ideals $I,J\in D$.  As $D$ is directed, there is $K\in D$ such that $I\subseteq K$ and $J\subseteq K$.  So $x,y\in K$ and hence there is $z\in K\subseteq E$ such that $x\le z$ and $y\le z$.  This shows that $E$ is directed.  Next, suppose $x\in E$ and $y\le x$.  Then $x\in I$ for some $I\in D$, so $y\in I\subseteq E$ as well.  This shows that $E$ is a down set.  So $E$ is an ideal of $P$: $\bigvee D=E\in P'$.
\item For every $x\in P$, $\down x$ is a compact element of $P'$.  If $\down x\le \bigvee D$, where $D$ is directed in $P'$, then $\down x\subseteq \bigcup D$, or $x\in \bigcup D$, which implies $x\in I$ for some ideal $I\in D$.  Therefore $\down x\subseteq I$, and $\down x$ is way below itself: $\down x$ is compact.
\item $P'$ is an algebraic dcpo.  Let $I\in P'$.  Let $C=\lbrace \down x\mid x\in I\rbrace$.  For any $x,y\in I$, there is $z\in I$ such that $x\le z$ and $y\le z$.  This shows that $\down x\le \down z$ and $\down y\le \down z$ in $C$, so that $C$ is directed.  It is easy to see that $I=\bigvee C$.  Since $I$ is a join of a directed set consisting of compact elements, $P'$ is algebraic.
\end{enumerate}
This completes the proof.
\end{proof}

\textbf{Definition}.  $\operatorname{Id}(P)$ is called the \emph{ideal completion} of $P$.

\textbf{Remarks}.  
\begin{itemize}
\item
In general, the ideal completion of a poset is not a complete lattice.  It is complete in the sense of being directed complete.  This is different from another type of completion, called the MacNeille completion of $P$, which is a complete lattice.  
\item
If $P$ is an upper semilattice, then so is $\operatorname{Id}(P)$.  In fact, the join of any non-empty family of ideals exists.  Furthermore, if $P$ has a bottom element $0$, then $\operatorname{Id}(P)$ is a complete lattice.
\begin{proof}
Let $S$ be a non-empty family of ideals in $P$.  Let $A$ be the set of $P$ consisting of all finite joins of elements of those ideals in $S$, and $B=\down A$.  Clearly, $B$ is a lower set.  For every $a,b\in B$, we have $c,d\in A$ such that $a\le c$ and $b\le d$.  Since $c$ and $d$ are both finite joins of elements of those ideals in $S$, so is $c\vee d$.  Since $a\le c\vee d$ and $b\le c\vee d$, $B$ is directed.  If $I$ is any ideal larger than any of the ideals in $S$, clearly $A\subseteq I$, since $I$ is directed.  So $B=\down A\subseteq \down I=I$.  Therefore, $B=\bigvee S$.

If $0\in P$, then $\langle 0\rangle$, the bottom of $\operatorname{Id}(P)$, is the join of the empty family of ideals in $P$.  By \PMlinkname{this entry}{CriteriaForAPosetToBeACompleteLattice}, $\operatorname{Id}(P)$ is a complete lattice.
\end{proof}
\item
If $P$ is a lower semilattice, then so is $\operatorname{Id}(P)$.
\begin{proof}
Let $I,J$ be two ideals in $P$ and $K=I\cap J$.  By definition, $I$ and $J$ are non-empty, so let $a\in I$ and $b\in J$.  As $P$ is a lower semilattice, $c:=a\wedge b$ exists and $c\le a$ and $c\le b$.  So $c \in I\cap J$, and that $K=I\cap J$ is non-empty.  If $x\le y\in K$, then $x\le y\in I$ or $x\in I$.  Similarly $x\in J$.  Therefore $x\in I\cap J=K$ and $K$ is a lower set.  If $r,s\in K$, then there is $u\in I$ and $v\in J$ such that $r,s\le u,v$.  So $r,s\le u\wedge v$ and $K$ is directed.  This means that $I\cap J\in \operatorname{Id}(P)$.
\end{proof}
\end{itemize}

\begin{thebibliography}{8}
\bibitem{ghklms} G. Gierz, K. H. Hofmann, K. Keimel, J. D. Lawson, M. W. Mislove, D. S. Scott, {\em Continuous Lattices and Domains}, Cambridge University Press, Cambridge (2003).
\end{thebibliography}
%%%%%
%%%%%
\end{document}
