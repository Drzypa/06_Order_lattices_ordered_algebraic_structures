\documentclass[12pt]{article}
\usepackage{pmmeta}
\pmcanonicalname{PartitionsFormALattice}
\pmcreated{2013-03-22 16:45:22}
\pmmodified{2013-03-22 16:45:22}
\pmowner{CWoo}{3771}
\pmmodifier{CWoo}{3771}
\pmtitle{partitions form a lattice}
\pmrecord{6}{38982}
\pmprivacy{1}
\pmauthor{CWoo}{3771}
\pmtype{Derivation}
\pmcomment{trigger rebuild}
\pmclassification{msc}{06B20}
\pmdefines{lattice of equivalence relations}

\usepackage{amssymb,amscd}
\usepackage{amsmath}
\usepackage{amsfonts}

% used for TeXing text within eps files
%\usepackage{psfrag}
% need this for including graphics (\includegraphics)
%\usepackage{graphicx}
% for neatly defining theorems and propositions
\usepackage{amsthm}
% making logically defined graphics
%%\usepackage{xypic}
\usepackage{pst-plot}
\usepackage{psfrag}

% define commands here
\newtheorem{prop}{Proposition}
\newtheorem{thm}{Theorem}
\newtheorem{ex}{Example}
\newcommand{\real}{\mathbb{R}}
\begin{document}
Let $S$ be a set.  Let $\operatorname{Part}(S)$ be the set of all partitions on $S$.  Since each partition is a cover of $S$, $\operatorname{Part}(S)$ is partially ordered by covering refinement relation, so that $P_1\preceq P_2$ if for every $a\in P_1$, there is a $b\in P_2$ such that $a\subseteq b$.  We say that a partition $P$ is finer than a partition $Q$ if $P\preceq Q$, and coarser than $Q$ if $Q\preceq P$.

\begin{prop} $\operatorname{Part}(S)$ is a complete lattice \end{prop}
\begin{proof}
For any set $\mathcal{P}$ of partitions $P_i$ of $S$, the intersection $\bigcap \mathcal{P}$ is a partition of $S$.  Take the meet of $P_i$ to be this intersection.  Next, let $\mathcal{Q}$ be the set of those partitions of $S$ such that each $Q\in \mathcal{Q}$ is coarser than each $P_i$.  This set is non-empty because $S\times S\in \mathcal{Q}$.  Take the meet $P'$ of all these partitions which is again coarser than all partitions $P_i$.  Define the join of $P_i$ to be $P'$ and the proof is complete.
\end{proof}

\textbf{Remarks}.  
\begin{itemize}
\item
The top element of $\operatorname{Part}(S)$ is $S\times S$ and the bottom is the diagonal relation on $S$.
\item
Correspondingly, the partition lattice of $S$ also defines the \emph{lattice of equivalence relations} $\Delta$ on $S$.
\item
Given a family $\lbrace E_i\mid i\in I\rbrace$ of equvialence relations on $S$, we can explicitly describe the join $E:=\bigvee E_i$ of $E_i$, as follows: $a\equiv b\pmod E$ iff there is a finite sequence $a=c_1,\ldots,c_n=b$ such that 
\begin{eqnarray}c_k\equiv c_{k+1} \pmod {E_{i(k)}}\qquad\mbox{ for }k=1,\ldots,n-1.\end{eqnarray}  It is easy to see this definition makes $E$ an equivalence relation.  To see that $E$ is the supremum of the $E_i$, first note that each $E_i\le E$.  Suppose now $F$ is an equivalence relation on $S$ such that $E_i\le F$ and $a\equiv b\pmod E$.  Then we get a finite sequence $c_k$ as described by (1) above, so $c_k\equiv c_{k+1}\pmod F$ for each $k\in \lbrace 1,\ldots,n-1\rbrace$.  Hence $a\equiv b\pmod F$ also.
\end{itemize}
%%%%%
%%%%%
\end{document}
