\documentclass[12pt]{article}
\usepackage{pmmeta}
\pmcanonicalname{Atom}
\pmcreated{2013-03-22 15:20:09}
\pmmodified{2013-03-22 15:20:09}
\pmowner{CWoo}{3771}
\pmmodifier{CWoo}{3771}
\pmtitle{atom}
\pmrecord{13}{37153}
\pmprivacy{1}
\pmauthor{CWoo}{3771}
\pmtype{Definition}
\pmcomment{trigger rebuild}
\pmclassification{msc}{06A06}
\pmclassification{msc}{06B99}
\pmdefines{atomic poset}
\pmdefines{atomic lattice}
\pmdefines{atomistic lattice}
\pmdefines{atomistic}

\usepackage{amssymb,amscd}
\usepackage{amsmath}
\usepackage{amsfonts}

% used for TeXing text within eps files
%\usepackage{psfrag}
% need this for including graphics (\includegraphics)
%\usepackage{graphicx}
% for neatly defining theorems and propositions
%\usepackage{amsthm}
% making logically defined graphics
%%%\usepackage{xypic}

% define commands here
\begin{document}
Let $P$ be a poset, partially ordered by $\leq$.  An element $a\in P$ is called an \emph{atom} if it covers some minimal element of $P$.  As a result, an atom is never minimal.  A poset $P$ is called \emph{atomic} if for every element $p\in P$ that is not minimal has an atom $a$ such that $a\leq p$.


\textbf{Examples}.
\begin{enumerate}
\item Let $A$ be a set and $P=2^A$ its power set.  $P$ is a poset ordered by $\subseteq$ with a unique minimal element $\varnothing$.  Thus, all singleton subsets $\lbrace a \rbrace$ of $A$ are atoms in $P$.
\item $\mathbb{Z}^+$ is partially ordered if we define $a\leq b$ to mean that $a\mid b$.  Then $1$ is a minimal element and any prime number $p$ is an atom.
\end{enumerate}

\textbf{Remark.} Given a lattice $L$ with underlying poset $P$, an element $a\in L$ is called an \emph{atom} (of $L$) if it is an atom in $P$.  A lattice is a called an \emph{atomic lattice} if its underlying poset is atomic.  An \emph{atomistic lattice} is an atomic lattice such that each element that is not minimal is a join of atoms.  If $a$ is an atom in a semimodular lattice $L$, and if $a$ is not under $x$, then $a\vee x$ is an atom in any interval lattice $I$ where $x=\bigwedge I$.

\textbf{Examples}.
\begin{enumerate}
\item $P=2^A$, with the usual intersection and union as the lattice operations meet and join, is atomistic: every subset $B$ of $A$ is the union of all the singleton subsets of $B$.
\item $\mathbb{Z}^+$, partially ordered as above, with lattice binary operations defined by $a\wedge b=\operatorname{gcd}(a,b)$, and $a\vee b= \operatorname{lcm}(a,b)$, is a lattice that is atomic, as we have seen earlier.  But it is not atomistic: $4$ is not a join of $2$'s; $36$ is not a join of $2$ and $3$ are just two counterexamples.
\end{enumerate}
%%%%%
%%%%%
\end{document}
