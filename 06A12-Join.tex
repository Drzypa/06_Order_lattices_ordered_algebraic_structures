\documentclass[12pt]{article}
\usepackage{pmmeta}
\pmcanonicalname{Join}
\pmcreated{2013-03-22 12:27:40}
\pmmodified{2013-03-22 12:27:40}
\pmowner{yark}{2760}
\pmmodifier{yark}{2760}
\pmtitle{join}
\pmrecord{11}{32611}
\pmprivacy{1}
\pmauthor{yark}{2760}
\pmtype{Definition}
\pmcomment{trigger rebuild}
\pmclassification{msc}{06A12}
\pmsynonym{or operator}{Join}
\pmrelated{Meet}
\pmrelated{Semilattice}
\pmdefines{join-semilattice}
\pmdefines{join semilattice}
\pmdefines{upper semilattice}

\usepackage{amssymb}
\usepackage{amsmath}
\usepackage{amsfonts}
\begin{document}
Certain posets $X$ have a binary operation \emph{join} denoted by $\lor$, such that $x \lor y$ is the least upper bound of $x$ and $y$. Such posets are called \emph{join-semilattices}, or \emph{$\lor$-semilattices}, or \emph{upper semilattices}.

If $j$ and $j'$ are both joins of $x$ and $y$, then $j \leq j'$ and $j' \leq j$, and so $j = j'$; thus a join, if it exists, is unique.  The join is also known as the \emph{or operator}.
%%%%%
%%%%%
\end{document}
