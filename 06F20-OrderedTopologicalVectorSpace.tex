\documentclass[12pt]{article}
\usepackage{pmmeta}
\pmcanonicalname{OrderedTopologicalVectorSpace}
\pmcreated{2013-03-22 17:03:23}
\pmmodified{2013-03-22 17:03:23}
\pmowner{CWoo}{3771}
\pmmodifier{CWoo}{3771}
\pmtitle{ordered topological vector space}
\pmrecord{4}{39347}
\pmprivacy{1}
\pmauthor{CWoo}{3771}
\pmtype{Definition}
\pmcomment{trigger rebuild}
\pmclassification{msc}{06F20}
\pmclassification{msc}{46A40}
\pmclassification{msc}{06F30}
\pmsynonym{ordered topological linear space}{OrderedTopologicalVectorSpace}

\usepackage{amssymb,amscd}
\usepackage{amsmath}
\usepackage{amsfonts}
\usepackage{mathrsfs}

% used for TeXing text within eps files
%\usepackage{psfrag}
% need this for including graphics (\includegraphics)
%\usepackage{graphicx}
% for neatly defining theorems and propositions
\usepackage{amsthm}
% making logically defined graphics
%%\usepackage{xypic}
\usepackage{pst-plot}
\usepackage{psfrag}

% define commands here
\newtheorem{prop}{Proposition}
\newtheorem{thm}{Theorem}
\newtheorem{ex}{Example}
\newcommand{\real}{\mathbb{R}}
\newcommand{\pdiff}[2]{\frac{\partial #1}{\partial #2}}
\newcommand{\mpdiff}[3]{\frac{\partial^#1 #2}{\partial #3^#1}}
\begin{document}
Let $k$ be either $\mathbb{R}$ or $\mathbb{C}$ considered as a field.  An \emph{ordered topological vector space} $L$, (\emph{ordered t.v.s} for short) is 
\begin{itemize}
\item a topological vector space over $k$, and 
\item an ordered vector space over $k$, such that 
\item the positive cone $L^+$ of $L$ is a closed subset of $L$.
\end{itemize}

The last statement can be interpreted as follows: if a sequence of non-negative elements $x_i$ of $L$ converges to an element $x$, then $x$ is non-negative.

\textbf{Remark}.  Let $L,M$ be two ordered t.v.s., and $f:L\to M$ a linear transformation that is monotone.  Then if $0\le x\in L$, $0\le f(x)\in M$ also.  Therefore $f(L^+)\subseteq M^+$.  Conversely, a linear map that is invariant under positive cones is monotone.
%%%%%
%%%%%
\end{document}
