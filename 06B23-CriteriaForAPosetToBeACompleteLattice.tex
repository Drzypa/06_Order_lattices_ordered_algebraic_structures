\documentclass[12pt]{article}
\usepackage{pmmeta}
\pmcanonicalname{CriteriaForAPosetToBeACompleteLattice}
\pmcreated{2013-03-22 16:37:53}
\pmmodified{2013-03-22 16:37:53}
\pmowner{CWoo}{3771}
\pmmodifier{CWoo}{3771}
\pmtitle{criteria for a poset to be a complete lattice}
\pmrecord{7}{38832}
\pmprivacy{1}
\pmauthor{CWoo}{3771}
\pmtype{Theorem}
\pmcomment{trigger rebuild}
\pmclassification{msc}{06B23}
\pmclassification{msc}{03G10}
\pmrelated{MeetContinuous}
\pmrelated{IntersectionStructure}

\usepackage{amssymb,amscd}
\usepackage{amsmath}
\usepackage{amsfonts}

% used for TeXing text within eps files
%\usepackage{psfrag}
% need this for including graphics (\includegraphics)
%\usepackage{graphicx}
% for neatly defining theorems and propositions
\usepackage{amsthm}
% making logically defined graphics
%%\usepackage{xypic}
\usepackage{pst-plot}
\usepackage{psfrag}

% define commands here

\begin{document}
\textbf{Proposition}.  Let $L$ be a poset.  Then the following are equivalent.

\begin{enumerate}
\item $L$ is a complete lattice.
\item for every subset $A$ of $L$, $\bigvee A$ exists.
\item for every finite subset $F$ of $L$ and every directed set $D$ of $L$, $\bigvee F$ and $\bigvee D$ exist.
\end{enumerate}

\begin{proof} Implications $1.\Rightarrow 2.\Rightarrow 3.$ are clear.  We will show $3.\Rightarrow 2.\Rightarrow 1.$

$(3.\Rightarrow 2.)$  If $A=\varnothing$, then $\bigvee A=0$ by definition.  So assume $A$ be a non-empty subset of $L$.  Let $A^{\prime}$ be the set of all finite subsets of $A$ and $B=\lbrace \bigvee F\mid F\in A^{\prime}\rbrace$.  By assumption, $B$ is well-defined and $A\subseteq B$.  Next, let $B^{\prime}$ be the set of all directed subsets of $B$, and $C=\lbrace \bigvee D\mid D\in B^{\prime}\rbrace$.  By assumption again, $C$ is well-defined and $B\subseteq C$.  Now, every chain in $C$ has a maximal element in $C$ (since a chain is a directed set), $C$ itself has a maximal element $d$ by Zorn's Lemma.  We will show that $d$ is the least upper bound of elments of $A$.  It is clear that each $a\in A$ is bounded above by $d$ ($A\subseteq B\subseteq C$).  If $t$ is an upper bound of elements of $A$, then it is an upper bound of elements of $B$, and hence an upper bound of elements of $C$, which means $d\le t$.

$(2.\Rightarrow 1.)$  By assumption $\bigvee \varnothing$ exists ($=0$), so that $\bigwedge L=0$.  Now suppose $A$ is a proper subset of $L$.  We want to show that $\bigwedge A$ exists.  If $A=\varnothing$, then $\bigwedge A=\bigvee L=1$ by definition of an arbitrary meet over the empty set.  So assume $A\neq \varnothing$.  Let $A^{\prime}$ be the set of lower bounds of $A$: $A^{\prime}=\lbrace x\in L\mid x\le a\mbox{ for all }a\in A\rbrace$ and let $b=\bigvee A^{\prime}$, the least upper bound of $A^{\prime}$.  $b$ exists by assumption.  Since $A$ is a set of upper bounds of $A^{\prime}$, $b\le a$ for all $a\in A$.  This means that $b$ is a lower bound of elements of $A$, or $b\in A^{\prime}$.  If $x$ is any lower bound of elements of $A$, then $x\le b$, since $x$ is bounded above by $b$ ($b=\bigvee A^{\prime}$).  This shows that $\bigwedge A$ exists and is equal to $b$.
\end{proof}

\textbf{Remarks}.  
\begin{itemize}
\item
Dually, a poset is a complete lattice iff every subset has an infimum iff infimum exists for every finite subset and every directed subset.
\item
The above proposition shows, for example, that every closure system is a complete lattice.
\end{itemize}
%%%%%
%%%%%
\end{document}
