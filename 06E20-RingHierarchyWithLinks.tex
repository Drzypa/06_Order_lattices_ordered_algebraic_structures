\documentclass[12pt]{article}
\usepackage{pmmeta}
\pmcanonicalname{RingHierarchyWithLinks}
\pmcreated{2013-03-22 18:11:17}
\pmmodified{2013-03-22 18:11:17}
\pmowner{SamB}{21010}
\pmmodifier{SamB}{21010}
\pmtitle{ring hierarchy with links?}
\pmrecord{9}{40761}
\pmprivacy{1}
\pmauthor{SamB}{21010}
\pmtype{Example}
\pmcomment{trigger rebuild}
\pmclassification{msc}{06E20}

\endmetadata

% this is the default PlanetMath preamble.  as your knowledge
% of TeX increases, you will probably want to edit this, but
% it should be fine as is for beginners.

% almost certainly you want these
\usepackage{amssymb}
\usepackage{amsmath}
\usepackage{amsfonts}

% used for TeXing text within eps files
%\usepackage{psfrag}
% need this for including graphics (\includegraphics)
%\usepackage{graphicx}
% for neatly defining theorems and propositions
%\usepackage{amsthm}
% making logically defined graphics
%%%\usepackage{xypic}

\usepackage{pstricks}
\usepackage{pst-node}

% define commands here

\begin{document}
This isn't actually a useful diagram; see \PMlinkname{ring hierarchy}{RingHierarchy} for that. This is just an attempt to replicate it with links in the diagram; it isn't working well so far. (It does work in the page images view though.)

\begin{figure}

\psset{linearc=.15}
\psset{arm=0}

\begin{tabular}{ccccccc}
 & & & \rnode{ring}{\PMlinkname{Ring (unital)}{Ring}}
\\\\
 & \rnode{comm}{Commutative ring} \ncline{->}{comm}{ring} & & & \rnode{noeth}{\PMlinkname{Noetherian Ring}{Noetherian}} \ncline{->}{noeth}{ring}
\\\\
\rnode{local}{Local Ring} \ncline{->}{local}{comm}
\end{tabular}

\end{figure}

%%%%%
%%%%%
\end{document}
