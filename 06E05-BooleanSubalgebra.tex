\documentclass[12pt]{article}
\usepackage{pmmeta}
\pmcanonicalname{BooleanSubalgebra}
\pmcreated{2013-03-22 17:57:55}
\pmmodified{2013-03-22 17:57:55}
\pmowner{CWoo}{3771}
\pmmodifier{CWoo}{3771}
\pmtitle{Boolean subalgebra}
\pmrecord{9}{40471}
\pmprivacy{1}
\pmauthor{CWoo}{3771}
\pmtype{Definition}
\pmcomment{trigger rebuild}
\pmclassification{msc}{06E05}
\pmclassification{msc}{03G05}
\pmclassification{msc}{06B20}
\pmclassification{msc}{03G10}
\pmsynonym{dense subalgebra}{BooleanSubalgebra}
\pmdefines{dense Boolean subalgebra}

\usepackage{amssymb,amscd}
\usepackage{amsmath}
\usepackage{amsfonts}
\usepackage{mathrsfs}

% used for TeXing text within eps files
%\usepackage{psfrag}
% need this for including graphics (\includegraphics)
%\usepackage{graphicx}
% for neatly defining theorems and propositions
\usepackage{amsthm}
% making logically defined graphics
%%\usepackage{xypic}
\usepackage{pst-plot}

% define commands here
\newcommand*{\abs}[1]{\left\lvert #1\right\rvert}
\newtheorem{prop}{Proposition}
\newtheorem{thm}{Theorem}
\newtheorem{ex}{Example}
\newtheorem{cor}{Corollary}
\newcommand{\real}{\mathbb{R}}
\newcommand{\pdiff}[2]{\frac{\partial #1}{\partial #2}}
\newcommand{\mpdiff}[3]{\frac{\partial^#1 #2}{\partial #3^#1}}
\begin{document}
\subsubsection*{Boolean Subalgebras}

Let $A$ be a Boolean algebra and $B$ a non-empty subset of $A$.  Consider the following conditions:
\begin{enumerate}
\item if $a\in B$, then $a'\in B$,
\item if $a,b\in B$, then $a\vee b\in B$,
\item if $a,b\in B$, then $a\wedge b\in B$,
\end{enumerate}

It is easy to see that, by de Morgan's laws, conditions 1 and 2 imply conditions 1 and 3, and vice versa.  Also, If $B$ satisfies 1 and 2, then $1=a\vee a'\in B$ by picking an element $a\in B$.  As a result, $0=1'\in B$ as well.

A non-empty subset $B$ of a Boolean algebra $A$ satisfying conditions 1 and 2 (or equivalently 1 and 3) is called a \emph{Boolean subalgebra} of $A$.

\textbf{Examples}.  
\begin{itemize}
\item Every Boolean algebra contains a unique two-element Boolean algebra consisting of just $0$ and $1$.  
\item If $A$ contains a non-trivial element $a$ ($\ne 0$), then $0,1,a,a'$ form a four-element Boolean subalgebra of $A$.
\item Here is a concrete example.  Let $A$ be the \PMlinkname{field of all sets}{RingOfSets} of a set $X$ (the power set of $X$).  Let $B$ be the set of all finite and all cofinite subsets of $A$.  Then $B$ is a subalgebra of $A$.
\item Continuing from the example above, if $X$ is equipped with a topology $\mathcal{T}$, then the set of all clopen sets in $X$ is a Boolean subalgebra of the algebra of all regular open sets of $X$.
\end{itemize}

\textbf{Remarks}.  Let $A$ be a Boolean algebra.
\begin{itemize}
\item Arbitrary intersections of Boolean subalgebras of $A$ is a Boolean subalgebra.  
\item An arbitrary union of an increasing chain of Boolean subalgebras of $A$ is a Boolean subalgebra.
\item A subalgebra $B$ of $A$ is said to be \emph{dense} in $A$ if it is dense as a subset of the underlying poset $A$.  In other words, $B$ is dense in $A$ iff for any $a\in A$, there is a $b\in B$ such that $b\le a$.  It can be easily shown that $B$ is dense in $A$ is equivalent to any one of the following: 
\begin{enumerate}
\item for any $a\in A$, there is $b\in B$ such that $a\le b$;
\item for any $x,y\in A$ with $x\le y$, there is $z\in B$ such that $x\le z\le y$.  
\end{enumerate}
To see the last equivalence, notice first that by picking $x=0$ we see that $B$ is dense, and conversely, if $x\le y$, then there is $r\in B$ such that $r\le y$, so that $x\le x\vee r\le y$.
\end{itemize}

\subsubsection*{Subalgebras Generated by a Set}

Let $A$ be a Boolean algebra and $X$ a subset of $A$.  The intersection of all Boolean subalgebras (of $A$) containing $X$ is called the Boolean subalgebra generated by $X$, and is denoted by $\langle X\rangle$.  As indicated by the remark above, $\langle X\rangle$ is a Boolean subalgebra of $A$, the smallest subalgebra containing $X$.  If $\langle X\rangle = A$, then we say that the Boolean algebra $A$ itself is generated by $X$.

\begin{prop}  Every element of $\langle X\rangle \subseteq A$ has a disjunctive normal form (DNF).  In other words, if $a\in \langle X\rangle$, then $$a=\bigvee_{i=1}^{n} \bigwedge_{j=1}^{\phi(i)} a_{ij}=(a_{11}\wedge \cdots \wedge a_{1\phi(1)})\vee (a_{21}\wedge \cdots \wedge a_{2\phi(2)}) \vee \cdots \vee (a_{n1}\wedge \cdots \wedge a_{n\phi(n)}),$$ where either $a_{ij}$ or $a_{ij}'$ belongs to $X$.
\end{prop}
\begin{proof}
Let $B$ be the set of all elements written in DNF using elements $X$.  Clearly $B\subseteq \langle X\rangle$.  What we want to show is that $\langle X\rangle \subseteq B$.  First, notice that the join of two elements in $B$ is in $B$.  Second, the complement of an element in $B$ is also in $B$.  We prove this in three steps:
\begin{enumerate}
\item If $a\in B$ and either $b$ or $b'$ is in $X$, then $b\wedge a\in B$.  

If $a=(a_{11}\wedge \cdots \wedge a_{1\phi(1)})\vee (a_{21}\wedge \cdots \wedge a_{2\phi(2)}) \vee \cdots \vee (a_{n1}\wedge \cdots \wedge a_{n\phi(n)})$, then 
\begin{eqnarray*}
b\wedge a &=& b\wedge ((a_{11}\wedge \cdots \wedge a_{1\phi(1)})\vee (a_{21}\wedge \cdots \wedge a_{2\phi(2)}) \vee \cdots \vee (a_{n1}\wedge \cdots \wedge a_{n\phi(n)})) \\
&=& (b\wedge a_{11}\wedge \cdots \wedge a_{1\phi(1)})\vee (b\wedge a_{21}\wedge \cdots \wedge a_{2\phi(2)}) \vee \cdots \vee (b\wedge a_{n1}\wedge \cdots \wedge a_{n\phi(n)}) 
\end{eqnarray*}
which is in DNF using elements of $X$.
\item If $a,b\in B$, then $a\wedge b\in B$.

In the last expression of the previous step, notice that each term $b\wedge a_{i1}\wedge \cdots \wedge a_{i\phi(i)}$ can be written in DNF by iteratively using the result of the previous step.  Hence, the join of all these terms is again in DNF (using elements of $X$).
\item If $a\in B$, then $a'\in B$.

If $a=(a_{11}\wedge \cdots \wedge a_{1\phi(1)})\vee (a_{21}\wedge \cdots \wedge a_{2\phi(2)}) \vee \cdots \vee (a_{n1}\wedge \cdots \wedge a_{n\phi(n)})$, then $$a'=(a_{11}'\vee \cdots \vee a_{1\phi(1)}')\wedge (a_{21}'\vee \cdots \vee a_{2\phi(2)}') \wedge \cdots \wedge (a_{n1}'\vee \cdots \vee a_{n\phi(n)}'),$$ which is the meet of $n$ elements in $B$.  Consequently, by step 2, $a'\in B$.
\end{enumerate}
Since $B$ is closed under join and complementation, $B$ is a Boolean subalgebra of $A$.  Since $B$ contains $X$, $\langle X\rangle \subseteq B$.
\end{proof}

Similarly, one can show that $\langle X\rangle$ is the set of all elements in $A$ that can be written in conjunctive normal form (CNF) using elements of $X$.

\begin{cor} Let $\langle B,x\rangle$ be the Boolean subalgebra (of $A$) generated by a Boolean subalgebra $B$ and an element $x\in A$.  Then every element of $\langle B,x\rangle$ has the form $$(b_1\wedge x)\vee (b_2\wedge x')$$ for some $b_1,b_2\in B$. \end{cor}
\begin{proof}
Let $X$ be a generating set for $B$ (pick $X=B$ if necessary).  By the proposition above, every element in $\langle B,x\rangle$ is the join of elements of the form $a_1\wedge a_2\wedge \cdots \wedge a_n$, where each $a_i$ or $a_i'$ is in $X$ or is $x$.  By the absorption laws, the form is reduced to one of the three forms $a$, $a\wedge x$, or $a\wedge x'$, where $a\in B$.  Joins of elements in the form of the first kind is an element of $B$, since $B$ is a subalgebra.  Joins of elements in the form of the second kind is again an element in the form of the second kind (for $(a_1\wedge x)\vee (a_2\wedge x)=(a_1\vee a_2)\wedge x$), and similarly for the last case.  Therefore, any element of $\langle B,x\rangle$ has the form $a\vee (a_1\wedge x)\vee (a_2\wedge x')$.  Since $a=(a\wedge x)\vee (a\wedge x')$, by setting $b_1=a_1\vee a$ and $b_2=a_2\vee a$, we have the desired form.
\end{proof}

\textbf{Remarks}.  
\begin{itemize}
\item
If $a=(b_1\wedge x)\vee (b_2\wedge x')$, then $a'=(b_2'\wedge x)\vee (b_1'\wedge x')$.
\item
Representing $a$ in terms of $(b_1\wedge x)\vee (b_2\wedge x')$ is not unique.  Actually, we have the following fact: $(b_1\wedge x)\vee (b_2\wedge x')=(c_1\wedge x)\vee (c_2\wedge x')$ iff $b_2\Delta c_2\le x\le b_1\leftrightarrow c_1$, where $\Delta$ and $\leftrightarrow$ are the symmetric difference and biconditional operators.
\begin{proof} $(\Rightarrow)$.  The LHS can be rewritten as $(b_1\vee b_2)\wedge (b_2\vee x)\wedge (b_1\vee x')$.  As a result, $c_2\wedge x'\le b_2\vee x$ so that $c_2\vee x=(c_2\wedge x')\vee x\le (b_2\vee x)\vee x=b_2\vee x$.  Therefore, $c_2-b_2 = c_2\wedge b_2' \le (c_2\vee x)\wedge b_2' \le (b_2\vee x)\wedge b_2' = b_2'\wedge x \le x$.  Similarly, $b_2 -c_2\le x$.  Taking the join and we get $b_2\Delta c_2\le x$.  Dually, $b_1\Delta c_1\le x'$, and complementing this to get $x\le (b_1\Delta c_1)'= b_1\leftrightarrow c_1$.

$(\Leftarrow)$.  If $b_2-c_2\le x$, then $b_2\vee c_2= (b_2-c_2)\vee c_2\le x\vee c_2$, so that $(b_2\vee c_2)-x \le c_2 - x$, or $(b_2-x)\vee (c_2-x)\le c_2-x$, which implies that $b_2-x\le c_2-x$.  Similarly, $c_2-x\le b_2-x$.  Putting the two inequalities together and we have $b_2-x=c_2-x$.  Since $x\le b_1\leftrightarrow c_1$ is equivalent to $b_1\Delta c_1 \le x'$ (dual statements), we also have $b_1-x'=c_1-x'$ or $b_1\wedge x= c_1\wedge x$.  Combining (via meet) this equality with the last one, we get $(b_1\wedge x)\wedge (b_2\wedge x')=(c_1\wedge x)\wedge (c_2\wedge x')$.
\end{proof}
\end{itemize}
%%%%%
%%%%%
\end{document}
