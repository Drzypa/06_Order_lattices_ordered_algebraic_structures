\documentclass[12pt]{article}
\usepackage{pmmeta}
\pmcanonicalname{DeMorganAlgebra}
\pmcreated{2013-03-22 16:09:25}
\pmmodified{2013-03-22 16:09:25}
\pmowner{CWoo}{3771}
\pmmodifier{CWoo}{3771}
\pmtitle{De Morgan algebra}
\pmrecord{13}{38238}
\pmprivacy{1}
\pmauthor{CWoo}{3771}
\pmtype{Definition}
\pmcomment{trigger rebuild}
\pmclassification{msc}{06D30}
\pmclassification{msc}{03G10}

\usepackage{amssymb,amscd}
\usepackage{amsmath}
\usepackage{amsfonts}

% used for TeXing text within eps files
%\usepackage{psfrag}
% need this for including graphics (\includegraphics)
%\usepackage{graphicx}
% for neatly defining theorems and propositions
%\usepackage{amsthm}
% making logically defined graphics
%%\usepackage{xypic}
\usepackage{pst-plot}
\usepackage{psfrag}

% define commands here

\begin{document}
A bounded distributive lattice $L$ is called a \emph{De Morgan algebra} if there exists a unary operator $\sim:L\to L$ such that
\begin{enumerate}
\item $\sim(\sim a)=a$ and
\item $\sim(a\vee b)=(\sim a )\wedge (\sim b)$.
\end{enumerate}

From the definition, we have the following properties:
\begin{itemize}
\item $\sim$ is a bijection, since for any $a\in L$, $a=\sim (\sim a)$.
\item $\sim(a\wedge b)=\sim [(\sim (\sim a ))\wedge (\sim (\sim b))]=
\sim [\sim ((\sim a)\vee (\sim b)) ]=(\sim a)\vee (\sim b)$, which is the dual statement of (2) above.  This, together with condition (2), are commonly known as the De Morgan's laws.
\item $\sim 0 =\ \sim (0 \wedge (\sim a))=(\sim 0) \vee a$ for all $a\in L$, so $\sim 0 = 1$.  Dually, $\sim 1=0$.  As a result, a De Morgan algebra is an Ockham algebra.
\item $a \le b$ iff $b=a \vee b$ iff $\sim b=\ \sim(a\vee b)=(\sim a)\wedge (\sim b)$ iff $\sim b \le\ \sim a$.
\item A Boolean algebra is always a De Morgan algebra, where the $\sim$ is the complementation operator $'$.  The converse is not true.  In general, $\sim a$ is not a complement of $a$ (that is, $(\sim a)\wedge a \ne 0$ and $(\sim a)\vee a\ne 1$).  Otherwise, $L$ is a complemented lattice and consequently a Boolean algebra.
\end{itemize}

Furthermore, a Kleene algebra is, by definition, a De Morgan algebra.  But the converse is false.  For example, consider $L=\mathbf{n}\times \mathbf{n}$, where $\mathbf{n}=\lbrace 0,1,\ldots,n\rbrace$ is a chain with the usual ordering.  Define $\sim$ on $L$ by $\sim(a,b)=(n-b,n-a)$.  Then $\sim^2(a,b)=(a,b)$.  The De Morgan's laws follow from the identity $n-(a\vee b)=(n-a)\wedge (n-b)$ applied to each of the two components.  But $L$ is not Kleene in general.  Take $n=3$, then $\sim (1,2)=(1,2)$ and $\sim (2,1)=(2,1)$.  But $(1,2)=\sim (1,2)\wedge (1,2)$ and $(2,1)=\sim (2,1)\vee (2,1)$ are not comparable.

Next, for any $a,b\in L$, define $a-b:=a\wedge (\sim b)$.  Then $-$ is a binary operator.  It has the following properties:

\begin{itemize}
\item $a-0 = a\wedge (\sim 0)=a\wedge 1 =a$.
\item $0-a = 0\wedge (\sim a)=0$.
\item $a-1 = a\wedge (\sim 1)=a\wedge 0=0$.
\item $1-a = 1\wedge (\sim a)=\ \sim a$.
\item $(a-b)-c=(a\wedge (\sim b))\wedge (\sim c)=a\wedge (\sim (b\vee c))=a-(b\vee c)$.
\end{itemize}

Finally, we define for $a,b\in L$, $a+b=(a-b)\vee (b-a)$.  This is again a binary operator, with the following properties:

\begin{itemize}
\item $a+b=b+a$.  This is obvious by the symmetry in the definition of $+$.
\item $a+0=a$.  We have $a+0=(a-0)\vee (0-a)=a\vee 0=a$.
\item $a+1=\ \sim a$, since $a+1=(a-1)\vee (1-a)=0\vee (\sim a)=\ \sim a$.  In particular $1+1=0$.
\item $a+a=(a-a)\vee(a-a)=a-a=a\wedge (\sim a)$.  If we define $2a:=a+a$, then $a+2a=(a-2a)\vee (2a-a)=(a\wedge (a\vee (\sim a))\vee ((a\wedge (\sim a)\wedge (\sim a))=a\vee ((a\wedge (\sim a))=a$.
\item More generally, we have 
\begin{eqnarray*}
a+(a+b) &=& (a-(a+b))\vee ((a+b)-a) \\
&=&(a-((a-b)\vee(b-a)))\vee (((a-b)\vee(b-a))-a) \\ 
&=&(a\wedge (\sim a\vee b)\wedge (\sim b\vee a))\vee (((\sim b \wedge a)\vee (\sim a\wedge b))\wedge (\sim a)) \\
&=& (a\wedge (\sim a\vee b))\vee (((\sim b\wedge a)\wedge (\sim a))\vee (\sim a\wedge b)) \\
&=& ((a\wedge (\sim a\vee b))\vee (\sim a\wedge b)) \vee (\sim b\wedge (\sim a\wedge a)) \\ 
&=& ((\sim a\wedge a) \vee (a\wedge b)\vee (\sim a \wedge b))\vee (\sim b\wedge 2a) \\
&=& (2a \vee ((\sim a \wedge a)\vee b))\vee (\sim b\wedge 2a) \\
&=& (2a \vee (2a \vee b))\vee (\sim b\wedge 2a) \\
&=& (2a \vee b)\vee (\sim b\wedge 2a) \\
&=& 2a \vee (\sim b \wedge 2a) \vee b \\ 
&=& 2a \vee b.
\end{eqnarray*}
\end{itemize}

\textbf{Remark}.  Since a De Morgan algebra is an Ockham algebra, a morphism between any two objects in the category of De Morgan algebras behaves just like an Ockham algebra homomorphism: it preserves $\sim$.

\begin{thebibliography}{6}
\bibitem{gg} G. Gr\"atzer, {\it General Lattice Theory}, 2nd Edition, Birkh\"auser (1998)
\end{thebibliography}
%%%%%
%%%%%
\end{document}
