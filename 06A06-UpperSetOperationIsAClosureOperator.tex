\documentclass[12pt]{article}
\usepackage{pmmeta}
\pmcanonicalname{UpperSetOperationIsAClosureOperator}
\pmcreated{2013-03-22 16:41:43}
\pmmodified{2013-03-22 16:41:43}
\pmowner{rspuzio}{6075}
\pmmodifier{rspuzio}{6075}
\pmtitle{upper set operation is a closure operator}
\pmrecord{8}{38908}
\pmprivacy{1}
\pmauthor{rspuzio}{6075}
\pmtype{Theorem}
\pmcomment{trigger rebuild}
\pmclassification{msc}{06A06}

\endmetadata

% this is the default PlanetMath preamble.  as your knowledge
% of TeX increases, you will probably want to edit this, but
% it should be fine as is for beginners.

% almost certainly you want these
\usepackage{amssymb}
\usepackage{amsmath}
\usepackage{amsfonts}

% used for TeXing text within eps files
%\usepackage{psfrag}
% need this for including graphics (\includegraphics)
%\usepackage{graphicx}
% for neatly defining theorems and propositions
\usepackage{amsthm}
% making logically defined graphics
%%%\usepackage{xypic}

% there are many more packages, add them here as you need them

% define commands here

\newcommand{\up}{\uparrow\!\!}
\newtheorem{definition}{Definition}
\newtheorem{theorem}{Theorem}
\begin{document}
In this entry, we shall prove the assertion made in
the \PMlinkname{main entry}{UpperSet} that $\uparrow$
is a closure operator.  This will be done by checking 
that the defining properties are satisfied.  To begin,
recall the definition of our operation:

\begin{definition}
Let $P$ be a poset and $A$ a subset of $P$. The \emph{upper set} 
of $A$ is defined to be the set
\[
\up A = \lbrace b \in P \mid (\exists \, a \in A) \, 
a \le b \rbrace
\]
\end{definition}

Now, we verify each of the properties which is required
of a closure operator.

\begin{theorem}
$\up \emptyset = \emptyset$
\end{theorem}

\begin{proof}
Any statement of the form ``$(\exists \, a \in \emptyset) P(a)$'' 
is identically false no matter what the predicate $P$ (i.e. it is 
an antitautology) and the set of objects satsfying an identically
false condition is empty, so $\up \emptyset = \emptyset$.
\end{proof}

\begin{theorem}
$A\subseteq \up A$
\end{theorem}

\begin{proof}
This follows from reflexivity --- for every $a \in A$, one has
$a \le a$, hence $a \in \up A$.
\end{proof}

\begin{theorem}
$\uparrow \up A=\up A$
\end{theorem}

\begin{proof}
By the previous result, $\up A \subseteq \uparrow \up A$.  Hence, 
it only remains to show that $\uparrow \up A \subseteq \up A$.
This follows from transitivity.  In order for some $a$ to
be an element of $\uparrow \up A$, there must exist $b$ and $c$
such that $a \ge b \ge C$ and $C \in A$.  By transitivity, $A \ge C$,
so $a \in \up A$, hence $\uparrow \up A \subseteq \up A$ as well.
\end{proof}

\begin{theorem}
If $A$ and $B$ are subsets of a partially ordered set, then
\[
\up (A \cup B) = (\up A) \cup (\up B)
\]
\end{theorem}

\begin{proof}
On the one hand, if $a \in \up (A \cup B)$, then $a \ge b$ for
some $b \in A \cup B$.  It then follows that either $b \in A$
or $b \in B$.  In the former case, $a \in \up A$, in the latter
case, $a \in \up B$ so, either way $a \in (\up A) \cup (\up B)$.
Hence $\up (A \cup B) \subseteq (\up A) \cup (\up B)$.

On the other hand, if $a \in (\up A) \cup (\up B)$, then either
$a \in (\up A)$ or $a \in (\up B)$.  In the former case, there 
exists $b$ such that $a \ge b$ and $b \in A$.  Since $A \subseteq
A \cup B$, we also have $b \in A \cup B$, hence $a \in \up
(A \cup B)$.  Likewise, in the second case, we also conclude
that $a \in \up (A \cup B)$.  Therefore, we have 
$(\up A) \cup (\up B) \subseteq \up (A \cup B)$.
\end{proof}

\begin{theorem}
$\up P = P$
\end{theorem}

\begin{theorem}
$A\subseteq B$, $\up A\subseteq \up B$
\end{theorem}
%%%%%
%%%%%
\end{document}
