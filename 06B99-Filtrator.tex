\documentclass[12pt]{article}
\usepackage{pmmeta}
\pmcanonicalname{Filtrator}
\pmcreated{2013-03-22 19:31:25}
\pmmodified{2013-03-22 19:31:25}
\pmowner{porton}{9363}
\pmmodifier{porton}{9363}
\pmtitle{Filtrator}
\pmrecord{6}{42499}
\pmprivacy{1}
\pmauthor{porton}{9363}
\pmtype{Definition}
\pmcomment{trigger rebuild}
\pmclassification{msc}{06B99}
\pmclassification{msc}{06A06}
\pmclassification{msc}{54A20}
%\pmkeywords{order theory}
%\pmkeywords{partially ordered set}
\pmrelated{Filter}
\pmrelated{Filter2}
\pmdefines{primary filtrator}

% this is the default PlanetMath preamble.  as your knowledge
% of TeX increases, you will probably want to edit this, but
% it should be fine as is for beginners.

% almost certainly you want these
\usepackage{amssymb}
\usepackage{amsmath}
\usepackage{amsfonts}

% used for TeXing text within eps files
%\usepackage{psfrag}
% need this for including graphics (\includegraphics)
%\usepackage{graphicx}
% for neatly defining theorems and propositions
%\usepackage{amsthm}
% making logically defined graphics
%%%\usepackage{xypic}

% there are many more packages, add them here as you need them

% define commands here

\begin{document}
A \emph{filtrator} is a pair $(\mathfrak{A};\mathfrak{Z})$ consisting of a poset $\mathfrak{A}$ (the \emph{base} of the filtrator) and its subset $\mathfrak{Z}$ (the \emph{core} of the filtrator). The set $\mathfrak{Z}$ is considered as a poset with the induced order.

Having fixed a filtrator and an $a\in\mathfrak{A}$, we define:

$$\operatorname{up} a = \{ X\in\mathfrak{Z} | X \ge a \} \quad \operatorname{down} a = \{ X\in\mathfrak{Z} | X \le a \}.$$

Probably the most important example of a filtrator is a \emph{primary filtrator} that is the pair $(\mathfrak{F};\mathfrak{P})$ where $\mathfrak{F}$ is the set of filters on a poset ordered reverse to set-theoretic inclusion of filters and $\mathfrak{P}$ is the set of principal filters on this poset. For a filter $\mathcal{F}\in\mathfrak{F}$ we have $\operatorname{up}\mathcal{F}$ essentially equivalent with the set $\mathcal{F}$.

\begin{thebibliography}{1}
  \bibitem[1]{filters}Victor Porton. \PMlinkexternal{Filters on posets and
  generalizations.}{http://www.mathematics21.org/binaries/filters.pdf} International Journal of Pure and
  Applied Mathematics, 74(1):55--119, 2012.
\end{thebibliography}

%%%%%
%%%%%
\end{document}
