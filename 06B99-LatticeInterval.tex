\documentclass[12pt]{article}
\usepackage{pmmeta}
\pmcanonicalname{LatticeInterval}
\pmcreated{2013-03-22 15:44:56}
\pmmodified{2013-03-22 15:44:56}
\pmowner{CWoo}{3771}
\pmmodifier{CWoo}{3771}
\pmtitle{lattice interval}
\pmrecord{11}{37701}
\pmprivacy{1}
\pmauthor{CWoo}{3771}
\pmtype{Definition}
\pmcomment{trigger rebuild}
\pmclassification{msc}{06B99}
\pmclassification{msc}{06A06}
\pmdefines{prime interval}
\pmdefines{poset interval}
\pmdefines{locally finite lattice}

\endmetadata

\usepackage{amssymb,amscd}
\usepackage{amsmath}
\usepackage{amsfonts}

% used for TeXing text within eps files
%\usepackage{psfrag}
% need this for including graphics (\includegraphics)
%\usepackage{graphicx}
% for neatly defining theorems and propositions
%\usepackage{amsthm}
% making logically defined graphics
%%%\usepackage{xypic}

% define commands here
\begin{document}
\textbf{Definition}.  Let $L$ be a lattice.  A subset $I$ of $L$ is called a \emph{lattice interval}, or simply an \emph{\PMlinkescapetext{interval}} if there exist elements $a,b\in L$ such that $$I=\lbrace t\in L\mid a\le t\le b\rbrace:=[a,b].$$

The elements $a,b$ are called the endpoints of $I$.  Clearly $a,b\in I$.  Also, the endpoints of a lattice interval are unique: if $[a,b]=[c,d]$, then $a=c$ and $b=d$.

\textbf{Remarks}.
\begin{itemize}
\item It is easy to see that the name is derived from that of an interval on a number line.  From this analogy, one can easily define lattice intervals without one or both endpoints.  Whereas an interval on a number line is linearly ordered, a lattice interval in general is not.  Nevertheless, a lattice interval $I$ of a lattice $L$ is a sublattice of $L$.
\item A bounded lattice is itself a lattice interval: $[0,1]$.
\item A \emph{prime interval} is a lattice interval that contains its endpoints and nothing else.  In other words, if $[a,b]$ is prime, then any $c\in [a,b]$ implies that either $c=a$ or $c=b$.  Simply put, $b$ covers $a$.  If a lattice $L$ contains $0$, then for any $a\in L$, $[0,a]$ is a prime interval iff $a$ is an atom.
\item Since no operations of meet and join are used, all of the above discussion can be generalized to define an interval in a poset.
\item Given a lattice $L$, let $\mathcal{B}$ be the collection of all lattice intervals without endpoints, we can form a topolgy on $L$ with $\mathcal{B}$ as the subbasis.  This does not insure that $\wedge$ and $\vee$ are continuous, so that $L$ with this topological structure may not be a topological lattice.
\item \textbf{Locally Finite Lattice}.  A lattice that is derived based on the concept of lattice interval is that of a locally finite lattice.  A lattice $L$ is locally finite iff every one of its interval is finite.  Unless the lattice is finite, a locally finite lattice, if infinite, is either topless or bottomless.
\end{itemize}
%%%%%
%%%%%
\end{document}
