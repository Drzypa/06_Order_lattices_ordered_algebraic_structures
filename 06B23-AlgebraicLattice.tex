\documentclass[12pt]{article}
\usepackage{pmmeta}
\pmcanonicalname{AlgebraicLattice}
\pmcreated{2013-03-22 15:56:31}
\pmmodified{2013-03-22 15:56:31}
\pmowner{CWoo}{3771}
\pmmodifier{CWoo}{3771}
\pmtitle{algebraic lattice}
\pmrecord{15}{37951}
\pmprivacy{1}
\pmauthor{CWoo}{3771}
\pmtype{Definition}
\pmcomment{trigger rebuild}
\pmclassification{msc}{06B23}
\pmclassification{msc}{51D25}
\pmsynonym{compactly-generated lattice}{AlgebraicLattice}
\pmrelated{SumOfIdeals}
\pmdefines{algebraic dcpo}

\endmetadata

\usepackage{amssymb,amscd}
\usepackage{amsmath}
\usepackage{amsfonts}

% used for TeXing text within eps files
%\usepackage{psfrag}
% need this for including graphics (\includegraphics)
%\usepackage{graphicx}
% for neatly defining theorems and propositions
%\usepackage{amsthm}
% making logically defined graphics
%%%\usepackage{xypic}

% define commands here

\begin{document}
A lattice $L$ is said to be an \emph{algebraic lattice} if it is a complete lattice and every element of $L$ can be written as a join of compact elements.

As the name (G. Birkhoff originally coined the term) suggests, algebraic lattices are mostly found in lattices of subalgebras of algebraic systems.  Below are some common examples.

\textbf{Examples}.
\begin{enumerate}
\item Groups.  The lattice $L(G)$ of subgroups of a group $G$ is known to be complete.  Cyclic subgroups are compact elements of $L(G)$.  Since every subgroup $H$ of $G$ is the join of cyclic subgroups, each generated by an element $h\in H$, $L(G)$ is algebraic.
\item Vector spaces.  The lattice $L(V)$ of subspaces of a vector space $V$ is complete.  Since each subspace has a basis, and since each element generates a one-dimensional subspace which is clearly compact, $L(V)$ is algebraic.
\item Rings.  The lattice $L(R)$ of ideals of a ring $R$ is also complete, the join of a set of ideals of $R$ is the ideal generated by elements in each of the ideals in the set.  Any ideal $I$ is the join of cyclic ideals generated by elements $r\in I$.  So $L(R)$ is algebraic.
\item Modules.  The above two examples can be combined and generalized into one, the lattice $L(M)$ of submodules of a module $M$ over a ring.  The arguments are similar.
\item Topological spaces.  The lattice of closed subsets of a topological space is in general \emph{not} algebraic.  The simplest example is $\mathbb{R}$ with the open intervals forming the subbasis.  To begin with, it is not complete: the union of closed subsets $[0,1-\frac{1}{n}]$, $n\in\mathbb{N}$ is $[0,1)$, not a closed set.  In addition, $\mathbb{R}$ itself is a closed subset that is not compact.
\end{enumerate}

\textbf{Remarks}.  
\begin{itemize}
\item
Since every element in an algebraic lattice is a join of compact elements, it is easy to see that every atom is compact: for if $a$ is an atom in an algebraic lattice $L$, and $a=\bigvee S$, where $S\subseteq L$ is a set of compact elements $s\in L$, then each $s$ is either $0$ or $a$.  Therefore, $S$ consists of at most two elements $0$ and $a$.  But $S$ can't be a singleton consisting of $0$ (otherwise $\bigvee S=0\neq a$), so $a\in S$ and therefore $a$ is compact.
\item
The notion of being algebraic in a lattice can be generalized to an arbitrary dcpo:  an \emph{algebraic dcpo} is a dcpo $D$ such that every $a\in D$ can be written as $a=\bigvee C$, where $C$ is a directed set (in $D$) such that each element in $C$ is compact. 
\end{itemize}

\begin{thebibliography}{8}
\bibitem{gb} G. Birkhoff {\em Lattice Theory}, 3rd Edition, AMS Volume XXV, (1967).
\bibitem{gg} G. Gr\"{a}tzer, {\em General Lattice Theory}, 2nd Edition, Birkh\"{a}user (1998).
\bibitem{ghklms} G. Gierz, K. H. Hofmann, K. Keimel, J. D. Lawson, M. W. Mislove, D. S. Scott, {\em Continuous Lattices and Domains}, Cambridge University Press, Cambridge (2003).
\bibitem{sv} S. Vickers, {\em Topology via Logic}, Cambridge University Press, Cambridge (1989).
\end{thebibliography}
%%%%%
%%%%%
\end{document}
