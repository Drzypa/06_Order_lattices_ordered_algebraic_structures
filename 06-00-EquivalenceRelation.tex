\documentclass[12pt]{article}
\usepackage{pmmeta}
\pmcanonicalname{EquivalenceRelation}
\pmcreated{2013-03-22 11:48:27}
\pmmodified{2013-03-22 11:48:27}
\pmowner{CWoo}{3771}
\pmmodifier{CWoo}{3771}
\pmtitle{equivalence relation}
\pmrecord{15}{30349}
\pmprivacy{1}
\pmauthor{CWoo}{3771}
\pmtype{Definition}
\pmcomment{trigger rebuild}
\pmclassification{msc}{06-00}
\pmclassification{msc}{03D20}
\pmrelated{QuotientGroup}
\pmrelated{EquivalenceClass}
\pmrelated{Equivalent}
\pmrelated{EquivalenceRelation}
\pmrelated{Partition}
\pmrelated{MathbbZ_n}
\pmdefines{equivalent}
\pmdefines{equivalence class}

\endmetadata

\usepackage{graphicx}
%%%%%\usepackage{xypic} 
\usepackage{bbm}
\newcommand{\Z}{\mathbbmss{Z}}
\newcommand{\C}{\mathbbmss{C}}
\newcommand{\R}{\mathbbmss{R}}
\newcommand{\Q}{\mathbbmss{Q}}
\newcommand{\mathbb}[1]{\mathbbmss{#1}}
\newcommand{\figura}[1]{\begin{center}\includegraphics{#1}\end{center}}
\newcommand{\figuraex}[2]{\begin{center}\includegraphics[#2]{#1}\end{center}}
\begin{document}
An \emph{equivalence relation} $\sim$ on a set $S$ is a relation that is:
\begin{description}
\item[Reflexive.] $a\sim a$ for all $a\in S$.
\item[Symmetric.] Whenever $a\sim b$, then $b\sim a$.
\item[Transitive.] If $a\sim b$ and $b\sim c$ then $a\sim c$.
\end{description}
If $a$ and $b$ are related this way we say that they are \emph{equivalent} under $\sim$.
If $a\in S$, then the set of all elements of $S$ that are equivalent to $a$ is called the \emph{equivalence class} of $a$.  The set of all equivalence classes under $\sim$ is written $S/\sim$.

An equivalence relation on a set induces a partition on it.  Conversely, any partition induces an equivalence relation. Equivalence relations are important, because often the set $S$ can be 'transformed' into another set (quotient space) by considering each equivalence class as a single unit.

Two examples of equivalence relations:

1. Consider the set of integers $\Z$ and take a positive integer $m$. Then $m$ induces an equivalence relation by $a\sim b$ when $m$ divides $b-a$ (that is, $a$ and $b$ leave the same remainder when divided by $m$).

2. Take a group $(G,\cdot)$ and a subgroup $H$. Define $a\sim b$ whenever $ab^{-1}\in H$. That defines an equivalence relation. Here equivalence classes are called cosets.
%%%%%
%%%%%
%%%%%
%%%%%
\end{document}
