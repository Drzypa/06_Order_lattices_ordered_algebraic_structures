\documentclass[12pt]{article}
\usepackage{pmmeta}
\pmcanonicalname{DilworthsTheorem}
\pmcreated{2013-03-22 15:49:37}
\pmmodified{2013-03-22 15:49:37}
\pmowner{CWoo}{3771}
\pmmodifier{CWoo}{3771}
\pmtitle{Dilworth's theorem}
\pmrecord{14}{37794}
\pmprivacy{1}
\pmauthor{CWoo}{3771}
\pmtype{Theorem}
\pmcomment{trigger rebuild}
\pmclassification{msc}{06A06}
\pmclassification{msc}{06A07}
\pmsynonym{Dilworth chain decomposition theorem}{DilworthsTheorem}
\pmrelated{DualOfDilworthsTheorem}
\pmdefines{chain covering number}

\endmetadata

\usepackage{amssymb,amscd}
\usepackage{amsmath}
\usepackage{amsfonts}

% used for TeXing text within eps files
%\usepackage{psfrag}
% need this for including graphics (\includegraphics)
%\usepackage{graphicx}
% for neatly defining theorems and propositions
\usepackage{amsthm}
% making logically defined graphics
%%%\usepackage{xypic}

% define commands here
\newtheorem*{thm}{Theorem}
\begin{document}
\begin{thm}  If $P$ is a poset with width $w<\infty$, then $w$ is also the smallest integer such that $P$ can be written as the union of $w$ chains.
\end{thm}

\textbf{Remark}.  The smallest cardinal $c$ such that $P$ can be written as the union of $c$ chains is called the \emph{chain covering number} of $P$.  So Dilworth's theorem says that if the width of $P$ is finite, then it is equal to the chain covering number of $P$.  If $w$ is infinite, then statement is not true.  The proof of Dilworth's theorem and its counterexample in the infinite case can be found in the reference below.

\begin{thebibliography}{6}
\bibitem{jbn} J.B. Nation, ``Lattice Theory", \PMlinkexternal{http://www.math.hawaii.edu/~jb/lat1-6.pdf}{http://www.math.hawaii.edu/~jb/lat1-6.pdf}
\end{thebibliography}
%%%%%
%%%%%
\end{document}
