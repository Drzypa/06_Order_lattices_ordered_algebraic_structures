\documentclass[12pt]{article}
\usepackage{pmmeta}
\pmcanonicalname{OrderIdeal}
\pmcreated{2013-03-22 17:01:14}
\pmmodified{2013-03-22 17:01:14}
\pmowner{CWoo}{3771}
\pmmodifier{CWoo}{3771}
\pmtitle{order ideal}
\pmrecord{11}{39305}
\pmprivacy{1}
\pmauthor{CWoo}{3771}
\pmtype{Definition}
\pmcomment{trigger rebuild}
\pmclassification{msc}{06A06}
\pmclassification{msc}{06A12}
\pmsynonym{filter}{OrderIdeal}
\pmsynonym{ideal}{OrderIdeal}
\pmrelated{Filter}
\pmrelated{LatticeFilter}
\pmrelated{LatticeIdeal}
\pmdefines{order filter}
\pmdefines{semilattice ideal}
\pmdefines{semilattice filter}
\pmdefines{subsemilattice}
\pmdefines{principal ideal}
\pmdefines{principal filter}

\endmetadata

\usepackage{amssymb,amscd}
\usepackage{amsmath}
\usepackage{amsfonts}
\usepackage{mathrsfs}

% used for TeXing text within eps files
%\usepackage{psfrag}
% need this for including graphics (\includegraphics)
%\usepackage{graphicx}
% for neatly defining theorems and propositions
\usepackage{amsthm}
% making logically defined graphics
%%\usepackage{xypic}
\usepackage{pst-plot}
\usepackage{psfrag}

% define commands here
\newtheorem{prop}{Proposition}
\newtheorem{thm}{Theorem}
\newtheorem{ex}{Example}
\newcommand{\real}{\mathbb{R}}
\newcommand{\pdiff}[2]{\frac{\partial #1}{\partial #2}}
\newcommand{\mpdiff}[3]{\frac{\partial^#1 #2}{\partial #3^#1}}

\newcommand{\up}{\uparrow\!\!}
\newcommand{\down}{\downarrow\!\!}
\begin{document}
\subsubsection*{Order Ideals and Filters}

Let $P$ be a poset.  A subset $I$ of $P$ is said to be an \emph{order ideal} if 
\begin{itemize}
\item $I$ is a lower set: $\down I=I$, and
\item $I$ is a directed set: $I$ is non-empty, and every pair of elements in $I$ has an upper bound in $I$.
\end{itemize}
An order ideal is also called an ideal for short.  An ideal is said to be \emph{principal} if it has the form $\down x$ for some $x\in P$.

Given a subset $A$ of a poset $P$, we say that $B$ is the ideal generated by $A$ if $B$ is the smallest order ideal (of $P$) containing $A$.  $B$ is denoted by $\langle A\rangle$.  Note that $\langle A\rangle$ exists iff $A$ is a directed set.  In particular, for any $x\in P$, $\down x$ is the ideal generated by $x$.  Also, if $P$ is an upper semilattice, then for any $A\subseteq P$, let $A'$ be the set of finite joins of elements of $A$, then $A'$ is a directed set, and $\langle A\rangle=\down A'$.

Dually, an \emph{order filter} (or simply a \emph{filter}) in $P$ is a non-empty subset $F$ which is both an upper set and a filtered set (every pair of elements in $F$ has a lower bound in $F$).  A \emph{principal filter} is a filter of the form $\up x$ for some $x\in P$.

\textbf{Remark}.
This is a generalization of the notion of a \PMlinkname{filter}{Filter} in a set.  In fact, both ideals and filters are generalizations of ideals and filters in semilattices and lattices.  

\subsubsection*{Examples in a Semilattice}

A subset $I$ in an upper semilattice $P$ is a \emph{semilattice ideal} if 
\begin{enumerate}
\item
if $a,b\in I$, then $a\vee b\in I$ (condition for being an upper subsemilattice)
\item
if $a\in I$ and $b\le a$, then $b\in I$
\end{enumerate}

Then the two definitions are equivalent: if $P$ is an upper semilattice, then $I\subseteq P$ is a semilattice ideal iff $I$ is an order ideal of $P$:  if $I$ is a semilattice ideal, then $I$ is clearly a lower and directed (since $a\vee b$ is an upper bound of $a$ and $b$); if $I$ is an order ideal, then condition 2 of a semilattice ideal is satisfied.  If $a,b\in I$, then there is a $c\in I$ that is an upper bound of $a$ and $b$.  Since $I$ is lower, and $a\vee b\le c$, we have $a\vee b\in I$.

Going one step further, we see that if $P$ is a lattice, then a lattice ideal is exactly an order ideal:  if $I$ is a lattice ideal, then it is clearly an upper subsemilattice, and if $b\le a\in I$, then $b=a\wedge b\in I$ also, so that $I$ is a semilattice ideal.  On the other hand, if $I$ is a semilattice ideal, then $I$ is an upper subsemilattice, as well as a lower subsemilattice, for if $a\in I$, then $a\wedge b\in I$ as well since $a\wedge b\le a$.  This shows that  $I$ is a lattice ideal.

Dually, we can define a \emph{filter} in a lower semilattice, which is equivalent to an order filter of the underly poset.  Going one step futher, we also see that a lattice filter in a lattice is an order filter of the underlying poset.

\textbf{Remark}.  An alternative but equivalent characterization of a semilattice ideal $I$ in an upper semilattice $P$ is the following: $a,b\in I$ iff $a\vee b\in I$.
%%%%%
%%%%%
\end{document}
