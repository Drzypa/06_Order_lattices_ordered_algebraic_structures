\documentclass[12pt]{article}
\usepackage{pmmeta}
\pmcanonicalname{DefinitionOfVectorSpaceNeedsNoCommutativity}
\pmcreated{2015-01-25 12:26:14}
\pmmodified{2015-01-25 12:26:14}
\pmowner{pahio}{2872}
\pmmodifier{pahio}{2872}
\pmtitle{definition of vector space needs no commutativity}
\pmrecord{5}{88224}
\pmprivacy{1}
\pmauthor{pahio}{2872}
\pmtype{Feature}

\endmetadata

% this is the default PlanetMath preamble.  as your knowledge
% of TeX increases, you will probably want to edit this, but
% it should be fine as is for beginners.

% almost certainly you want these
\usepackage{amssymb}
\usepackage{amsmath}
\usepackage{amsfonts}

% need this for including graphics (\includegraphics)
\usepackage{graphicx}
% for neatly defining theorems and propositions
\usepackage{amsthm}

% making logically defined graphics
%\usepackage{xypic}
% used for TeXing text within eps files
%\usepackage{psfrag}

% there are many more packages, add them here as you need them

% define commands here

\begin{document}
In the definition of \PMlinkname{vector space}{VectorSpace} one 
usually lists the needed properties of the vectoral addition 
and the multiplication of vectors by scalars as eight axioms, 
one of them the commutative law
$$u+v = v+u.$$
The latter is however not necessary, because it may be proved 
to be a consequence of the other seven axioms.\, The proof can 
be based on the fact that in defining the \PMlinkname{group}{Group}, 
it suffices to postulate only the existence of a right identity 
element and the right inverses of the elements (see the article 
``\PMlinkname{redundancy of two-sidedness in definition of group}
{RedundancyOfTwoSidednessInDefinitionOfGroup}''). 

Now, suppose the validity of \PMlinkname{the seven other axioms}
{VectorSpace}, but not necessarily the above commutative law of 
addition.\, We will show that the commutative law is in force.

We need the identity\, $(-1)v = -v$\, which is easily justified
(we have $\vec{0} = 0v = (1+(-1))v = \ldots$).\, Then we can 
calculate as follows:
\begin{align*}
v+u &= (v+u)+\vec{0} = (v+u)+[-(u+v)+(u+v)]\\
      &=\; [(v+u)+(-(u+v))]+(u+v) = [(v+u)+(-1)(u+v)]+(u+v)\\
      &= [(v+u)+((-1)u+(-1)v)]+(u+v) = [((v+u)+(-u))+(-v)]+(u+v)\\
      &= [(v+(u+(-u)))+(-v)]+(u+v) = [(v+\vec{0})+(-v)]+(u+v)\\
      &= [v+(-v)]+(u+v) = \vec{0}+(u+v) \\
      &=u+v
\end{align*}
Q.E.D.\\

This proof by {\sc Y. Chemiavsky} and {\sc A. Mouftakhov} is 
found in the 2012 March issue of {\it The American Mathematical 
Monthly}.\\
\end{document}
