\documentclass[12pt]{article}
\usepackage{pmmeta}
\pmcanonicalname{BooleanLattice}
\pmcreated{2013-03-22 12:27:20}
\pmmodified{2013-03-22 12:27:20}
\pmowner{mathcam}{2727}
\pmmodifier{mathcam}{2727}
\pmtitle{Boolean lattice}
\pmrecord{19}{32594}
\pmprivacy{1}
\pmauthor{mathcam}{2727}
\pmtype{Definition}
\pmcomment{trigger rebuild}
\pmclassification{msc}{06E05}
\pmclassification{msc}{03G05}
\pmclassification{msc}{06B20}
\pmclassification{msc}{03G10}
\pmclassification{msc}{06E20}
\pmsynonym{Boolean algebra}{BooleanLattice}
\pmrelated{BooleanRing}

\endmetadata

% this is the default PlanetMath preamble.  as your knowledge
% of TeX increases, you will probably want to edit this, but
% it should be fine as is for beginners.

% almost certainly you want these
\usepackage{amssymb}
\usepackage{amsmath}
\usepackage{amsfonts}

% used for TeXing text within eps files
%\usepackage{psfrag}
% need this for including graphics (\includegraphics)
%\usepackage{graphicx}
% for neatly defining theorems and propositions
%\usepackage{amsthm}
% making logically defined graphics
%%%\usepackage{xypic} 

% there are many more packages, add them here as you need them

% define commands here
\begin{document}
In this entry, the notions of a Boolean lattice, a Boolean algebra, and a Boolean ring are defined, compared and contrasted.

\subsubsection*{Boolean Lattices}

A \emph{Boolean lattice} $B$ is a distributive lattice in which for each element $x\in B$ there exists a complement $x'\in B$ such that
\begin{align*}
x \land x'&=0\\
x \lor x'&=1 \\
(x')'&=x \\
(x \land y)'&=x'\lor y'\\
(x \lor y)'&=x'\land y'
\end{align*}
In other words, a Boolean lattice is the same as a complemented distributive lattice.  A morphism between two Boolean lattices is just a lattice homomorphism (so that $0,1$ and $'$ may not be preserved).

\subsubsection*{Boolean Algebras}

A Boolean algebra is a Boolean lattice such that $'$ and $0$ are considered as operators (unary and nullary respectively) on the algebraic system.  In other words, a morphism (or a Boolean algebra homomorphism) between two Boolean algebras must preserve $0,1$ and $'$.  As a result, the category of Boolean algebras and the category of Boolean lattices are not the same (and the former is a subcategory of the latter).

\subsubsection*{Boolean Rings}

A \emph{Boolean ring} is an (associative) unital ring $R$ such that for any $r\in R$, $r^2=r$.  It is easy to see that 
\begin{itemize}
\item any Boolean ring has characteristic $2$, for $2r=(2r)^2=4r^2=4r$,
\item and hence a commutative ring, for $a+b=(a+b)^2=a^2+ab+ba+b^2=a+ab+ba+b$, so $0=ab+ba$, and therefore $ab=ab+ab+ba=ba$.
\end{itemize}

Boolean rings (with identity, but allowing 0=1) are equivalent to Boolean lattices. To view a Boolean ring as a Boolean lattice, define $$x \land y = xy,\qquad x \lor y = x + y + xy,\qquad\mbox{and}\qquad x'=1+x.$$ To view a Boolean lattice as a Boolean ring, define $$xy = x \land y\qquad\mbox{ and }\qquad x + y = (x' \land y) \lor (x \land y').$$

The category of Boolean algebras is naturally equivalent to the category of Boolean rings.

\begin{thebibliography}{8}
\bibitem{gg} G. Gr\"{a}tzer, {\em General Lattice Theory}, 2nd Edition, Birkh\"{a}user (1998).
\bibitem{rs} R. Sikorski, {\em Boolean Algebras}, 2nd Edition, Springer-Verlag, New York (1964).
\end{thebibliography}
%%%%%
%%%%%
\end{document}
