\documentclass[12pt]{article}
\usepackage{pmmeta}
\pmcanonicalname{UpperSet}
\pmcreated{2013-03-22 15:49:50}
\pmmodified{2013-03-22 15:49:50}
\pmowner{CWoo}{3771}
\pmmodifier{CWoo}{3771}
\pmtitle{upper set}
\pmrecord{20}{37801}
\pmprivacy{1}
\pmauthor{CWoo}{3771}
\pmtype{Definition}
\pmcomment{trigger rebuild}
\pmclassification{msc}{06A06}
\pmsynonym{up set}{UpperSet}
\pmsynonym{down set}{UpperSet}
\pmsynonym{upper closure}{UpperSet}
\pmsynonym{lower closure}{UpperSet}
\pmrelated{LatticeIdeal}
\pmrelated{LatticeFilter}
\pmrelated{Filter}
\pmdefines{lower set}
\pmdefines{upper closed}
\pmdefines{lower closed}

\usepackage{amssymb,amscd}
\usepackage{amsmath}
\usepackage{amsfonts}

% used for TeXing text within eps files
%\usepackage{psfrag}
% need this for including graphics (\includegraphics)
%\usepackage{graphicx}
% for neatly defining theorems and propositions
%\usepackage{amsthm}
% making logically defined graphics
%%%\usepackage{xypic}

% define commands here
\newcommand{\up}{\uparrow\!\!}
\newcommand{\down}{\downarrow\!\!}
\begin{document}
Let $P$ be a poset and $A$ a subset of $P$.  The \emph{upper set} of $A$ is defined to be the set 
$$\lbrace b\in P\mid a\le b \mbox{ for some } a\in A\rbrace,$$
and is denoted by $\up A$.  In other words, $\up A$ is the set of all upper bounds of elements of $A$.

$\uparrow$ can be viewed as a unary operator on the power set $2^P$ sending $A\in 2^P$ to $\up A \in 2^P$.  $\uparrow$ has the following properties 
\begin{enumerate}
\item $\up \varnothing=\varnothing$,
\item $A\subseteq \up A$,
\item $\uparrow \up A=\up A$, and 
\item if $A\subseteq B$, $\up A\subseteq \up B$.
\end{enumerate}
So $\uparrow$ is a closure operator.

An \emph{upper set} in $P$ is a subset $A$ such that its upper set is itself: $\up A=A$.  In other words, $A$ is closed with respect to $\le$ in the sense that if $a\in A$ and $a\le b$, then $b\in A$.  An upper set is also said to be \emph{upper closed}.  For this reason, for any subset $A$ of $P$, the $\up A$ is also called the \emph{upper closure} of $A$.

Dually, the \emph{lower set} (or \emph{lower closure}) of $A$ is the set of all lower bounds of elements of $A$.  The lower set of $A$ is denoted by $\down A$.  If the lower set of $A$ is $A$ itself, then $A$ is a called a \emph{lower set}, or a \emph{lower closed set}.

\textbf{Remarks}.  
\begin{itemize}
\item $\up A$ is \emph{not} the same as the set of upper bounds of $A$, commonly denoted by $A^u$, which is defined as the set $\lbrace b\in P\mid a\le b\mbox{ for \emph{all} }a\in A\rbrace$.  Similarly, $\down A\neq A^{\ell}$ in general, where $A^{\ell}$ is the set of lower bounds of $A$.
\item When $A=\lbrace x\rbrace$, we write $\up x$ for $\up A$ and $\down x$ for $\down A$.  $\up x = \lbrace x\rbrace ^u$ and $\down x=\lbrace x\rbrace ^d$.
\item If $P$ is a lattice and $x\in P$, then $\up x$ is the principal filter generated by $x$, and $\down x$ is the principal ideal generated by $x$.
\item If $A$ is a lower set of $P$, then its set complement $A^{\complement}$ is an upper set: if $a\in A^{\complement}$ and $a\le b$, then $b\in A^{\complement}$ by a contrapositive argument.
\item Let $P$ be a poset.  The set of all lower sets of $P$ is denoted by $\mathcal{O}(P)$.  It is easy to see that $\mathcal{O}(P)$ is a poset (ordered by inclusion), and $\mathcal{O}(P)^{\partial}=\mathcal{O}(P^{\partial})$, where $^{\partial}$ is the dualization operation (meaning that $P^{\partial}$ is the dual poset of $P$).
\end{itemize}
%%%%%
%%%%%
\end{document}
