\documentclass[12pt]{article}
\usepackage{pmmeta}
\pmcanonicalname{QuasiorderIsNotDefinedUniformly}
\pmcreated{2013-03-22 15:35:18}
\pmmodified{2013-03-22 15:35:18}
\pmowner{boute}{11676}
\pmmodifier{boute}{11676}
\pmtitle{Quasi-order is not defined uniformly}
\pmrecord{5}{37499}
\pmprivacy{1}
\pmauthor{boute}{11676}
\pmtype{Definition}
\pmcomment{trigger rebuild}
\pmclassification{msc}{06A99}

% this is the default PlanetMath preamble.  as your knowledge
% of TeX increases, you will probably want to edit this, but
% it should be fine as is for beginners.

% almost certainly you want these
\usepackage{amssymb}
\usepackage{amsmath}
\usepackage{amsfonts}

% used for TeXing text within eps files
%\usepackage{psfrag}
% need this for including graphics (\includegraphics)
%\usepackage{graphicx}
% for neatly defining theorems and propositions
%\usepackage{amsthm}
% making logically defined graphics
%%%\usepackage{xypic}

% there are many more packages, add them here as you need them

% define commands here
\begin{document}
In the literature, some authors define ``quasi order'' as transitive and reflexive, others define it as transitive and irreflexive.

No such discrepancy seems to exist in using ``preorder'' for the former (transitive and reflexive) and ``strict partial order'' for the latter (transitive and irreflexive).

It seems wise to use only the unambiguous terminology, and start any text where the term ``quasi order'' is felt \PMlinkescapetext{necessary} with a proper warning.

Just for completeness: a {\em partial order} is transitive, reflexive and antisymmetric.
%%%%%
%%%%%
\end{document}
