\documentclass[12pt]{article}
\usepackage{pmmeta}
\pmcanonicalname{OrthomodularLattice}
\pmcreated{2013-03-22 16:33:06}
\pmmodified{2013-03-22 16:33:06}
\pmowner{CWoo}{3771}
\pmmodifier{CWoo}{3771}
\pmtitle{orthomodular lattice}
\pmrecord{10}{38735}
\pmprivacy{1}
\pmauthor{CWoo}{3771}
\pmtype{Definition}
\pmcomment{trigger rebuild}
\pmclassification{msc}{06C15}
\pmclassification{msc}{81P10}
\pmclassification{msc}{03G12}
\pmrelated{OrthocomplementedLattice}
\pmrelated{LatticeOfProjections}
\pmdefines{orthomodular poset}
\pmdefines{orthogonal}
\pmdefines{orthogonal complement}
\pmdefines{relative orthogonal complement}

\endmetadata

\usepackage{amssymb,amscd}
\usepackage{amsmath}
\usepackage{amsfonts}

% used for TeXing text within eps files
%\usepackage{psfrag}
% need this for including graphics (\includegraphics)
%\usepackage{graphicx}
% for neatly defining theorems and propositions
%\usepackage{amsthm}
% making logically defined graphics
%%\usepackage{xypic}
\usepackage{pst-plot}
\usepackage{psfrag}

% define commands here

\begin{document}
\subsubsection*{Orthogonality Relations}
Let $L$ be an orthocomplemented lattice and $a,b\in L$.  $a$ is said to be \emph{orthogonal} to $b$ if $a\le b^{\perp}$, denoted by $a\perp b$.  If $a\le b^{\perp}$, then $b=b^{\perp\perp}\le a^{\perp}$, so $\perp$ is a symmetric relation on $L$.  It is easy to see that, for any $a,b\in L$, $a\perp b$ implies $a\wedge b=0$, and $a\perp a^{\perp}$.  

For any $a\in L$, define $M(a):=\lbrace c\in L\mid c\perp a\mbox{ and }1=c \vee a\rbrace$.  An element of $M(a)$ is called an \emph{orthogonal complement} of $a$.  We have $a^{\perp}\in M(a)$, and any orthogonal complement of $a$ is a complement of $a$.

If we replace the $1$ in $M(a)$ by an arbitrary element $b\ge a$, then we have the set $$M(a,b):=\lbrace c\in L\mid c\perp a\mbox{ and }b=c \vee a\rbrace.$$
An element of $M(a,b)$ is called an \emph{orthogonal complement} of $a$ \emph{relative to} $b$.  Clearly, $M(a)=M(a,1)$.  Also, for $a,c\le b$, $c\in M(a,b)$ iff $a\in M(c,b)$.  As a result, we can define a symmetric binary operator $\oplus$ on $[0,b]$, given by $b=a\oplus c$ iff $c\in M(a,b)$.  Note that $b=b\oplus 0$.

Before the main definition, we define one more operation: $b-a:=b\wedge a^{\perp}$.  Some properties: (1) $a-a=0$, $a-0=a$, $0-a=0$, $a-1=0$, and $1-a=a^{\perp}$; (2) $b-a=a^{\perp}-b^{\perp}$; and (3) if $a\le b$, then $a\perp (b-a)$ and $a\oplus (b-a)\le b$.

\subsubsection*{Definition}  
A lattice $L$ is called an \emph{orthomodular lattice} if 
\begin{enumerate}
\item $L$ is orthocomplemented, and
\item (\emph{orthomodular law}) if $x\le y$, then $y=x\oplus (y-x)$.
\end{enumerate}

The orthomodular law can be restated as follows: if $x\le y$, then 
$y=x\vee (y\wedge x^{\perp})$.  Equivalently, $x\le y$ implies $y=(y\wedge x)\vee (y\wedge x^{\perp})$.  Note that the equation is automatically true in an arbitrary distributive lattice, even without the assumption that $x\le y$.

For example, the lattice $\mathbb{C}(H)$ of closed subspaces of a hilbert space $H$ is orthomodular.  $\mathbb{C}(H)$ is modular iff $H$ is finite dimensional.  In addition, if we give the set $\mathbb{P}(H)$ of (bounded) projection operators on $H$ an ordering structure by defining $P\le Q$ iff $P(H)\le Q(H)$, then $\mathbb{P}(H)$ is lattice isomorphic to $\mathbb{C}(H)$, and hence orthomodular.

A simple example of an orthocomplemented lattice that is not orthomodular is the benzene:

\begin{equation*}
\xymatrix {
& 1 \ar@{-}[ld] \ar@{-}[rd] & \\
b \ar@{-}[d] & & a^{\perp} \ar@{-}[d] \\
a \ar@{-}[rd] & & b^{\perp} \ar@{-}[ld] \\
& 0 & }
\end{equation*}
Note that $a\le b$, but $a\vee (b\wedge a^{\perp})=a\vee 0=a\ne b$.

An nice example of an orthomodular lattice that is not modular can be found in the reference below.

\textbf{Remarks}.  
\begin{itemize}
\item Orthomodular lattices were first studied by John von Neumann and Garett Birkhoff, when they were trying to develop the \PMlinkname{logic of quantum mechanics}{QuantumLogic} by studying the structure of the lattice $\mathbb{P}(H)$ of projection operators on a Hilbert space $H$.  However, the term was coined by Irving Kaplansky, when it was realized that $\mathbb{P}(H)$, while orthocomplemented, is not modular.  Rather, it satisfies a variant of the modular law as indicated above.
\item More generally, an \emph{orthomodular poset} $P$ is an orthocomplemented poset such that
\begin{enumerate}
\item given any pair of orthogonal elements $x,y\in P$ ($x\le y^{\perp}$), their greatest lower bound exists ($x\vee y$ exists).  Simply put, $x\perp y$ implies $x\vee y\in P$.
\item for any $x,y\in P$ such that $x\le y$, the orthomodular law holds (the right hand side of the orthomodular law exists via the first condition).
\end{enumerate}
From this definition, we see that an orthomodular lattice is just an orthomodular poset that is also a lattice.
\end{itemize}

\begin{thebibliography}{8}
\bibitem{lb} L. Beran, {\em Orthomodular Lattices, Algebraic Approach}, Mathematics and Its Applications (East European Series), D. Reidel Publishing Company, Dordrecht, Holland (1985).
\end{thebibliography}
%%%%%
%%%%%
\end{document}
