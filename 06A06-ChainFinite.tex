\documentclass[12pt]{article}
\usepackage{pmmeta}
\pmcanonicalname{ChainFinite}
\pmcreated{2013-03-22 16:55:07}
\pmmodified{2013-03-22 16:55:07}
\pmowner{lars_h}{9802}
\pmmodifier{lars_h}{9802}
\pmtitle{chain finite}
\pmrecord{4}{39180}
\pmprivacy{1}
\pmauthor{lars_h}{9802}
\pmtype{Definition}
\pmcomment{trigger rebuild}
\pmclassification{msc}{06A06}
%\pmkeywords{poset}
\pmdefines{chain finite}

% this is the default PlanetMath preamble.  as your knowledge
% of TeX increases, you will probably want to edit this, but
% it should be fine as is for beginners.

% almost certainly you want these
\usepackage{amssymb}
\usepackage{amsmath}
\usepackage{amsfonts}

% used for TeXing text within eps files
%\usepackage{psfrag}
% need this for including graphics (\includegraphics)
%\usepackage{graphicx}
% for neatly defining theorems and propositions
%\usepackage{amsthm}
% making logically defined graphics
%%%\usepackage{xypic}

% there are many more packages, add them here as you need them

% define commands here

\begin{document}
\PMlinkescapeword{relation}
\PMlinkescapeword{information}

A poset is said to be \emph{chain finite} if every chain with both 
\PMlinkname{maximal}{MaximalElement} and minimal element is finite.

$\mathbb{Z}$ with the standard order relation is chain finite, 
since any infinite subset of $\mathbb{Z}$ must be 
\PMlinkname{unbounded}{UpperBound} above or below. 
$\mathbb{Q}$ with the standard order relation is not chain finite, 
since for example 
$\{ x \in\nobreak \mathbb{Q} \,\mid\, 0 \leqslant\nobreak x \leqslant\nobreak 1 \}$ 
is infinite and has both a maximal element $1$ and a minimal element $0$.

Chain finiteness is often used to draw conclusions about an order from information about its covering relation (or equivalently, from its Hasse diagram).
%%%%%
%%%%%
\end{document}
