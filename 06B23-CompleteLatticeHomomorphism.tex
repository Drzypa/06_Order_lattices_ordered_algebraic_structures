\documentclass[12pt]{article}
\usepackage{pmmeta}
\pmcanonicalname{CompleteLatticeHomomorphism}
\pmcreated{2013-03-22 16:58:02}
\pmmodified{2013-03-22 16:58:02}
\pmowner{porton}{9363}
\pmmodifier{porton}{9363}
\pmtitle{complete lattice homomorphism}
\pmrecord{6}{39241}
\pmprivacy{1}
\pmauthor{porton}{9363}
\pmtype{Definition}
\pmcomment{trigger rebuild}
\pmclassification{msc}{06B23}
\pmrelated{CompleteLattice}

\endmetadata

% this is the default PlanetMath preamble.  as your knowledge
% of TeX increases, you will probably want to edit this, but
% it should be fine as is for beginners.

% almost certainly you want these
\usepackage{amssymb}
\usepackage{amsmath}
\usepackage{amsfonts}

% used for TeXing text within eps files
%\usepackage{psfrag}
% need this for including graphics (\includegraphics)
%\usepackage{graphicx}
% for neatly defining theorems and propositions
%\usepackage{amsthm}
% making logically defined graphics
%%%\usepackage{xypic}

% there are many more packages, add them here as you need them

% define commands here

\begin{document}
\emph{Complete lattice homomorphism} is a function from one lattice to an other lattice, which preserves arbitrary (not only finite) meets and joins.

If $\phi:L \to M$ is lattice homomorphism between complete lattices $L$ and $M$ such that
\begin{itemize}
\item
$\phi(\bigvee \lbrace a_i\mid i\in I\rbrace) = \bigvee \lbrace \phi(a_i)\mid
i\in I\rbrace$, and
\item
$\phi(\bigwedge \lbrace a_i\mid i\in I\rbrace) = \bigwedge \lbrace
\phi(a_i)\mid i\in I\rbrace$,
\end{itemize}
then $\phi$ is called a \emph{complete lattice homomorphism}.

Most often are considered \emph{complete lattice homomorphisms} from one complete lattice to an other complete lattice (that is when all meets and joins are defined).

\emph{Complete lattice homomorphism} is a special case of lattice homomorphism.
%%%%%
%%%%%
\end{document}
