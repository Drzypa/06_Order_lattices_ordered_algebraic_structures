\documentclass[12pt]{article}
\usepackage{pmmeta}
\pmcanonicalname{ComplementedLattice}
\pmcreated{2013-03-22 15:02:25}
\pmmodified{2013-03-22 15:02:25}
\pmowner{CWoo}{3771}
\pmmodifier{CWoo}{3771}
\pmtitle{complemented lattice}
\pmrecord{26}{36754}
\pmprivacy{1}
\pmauthor{CWoo}{3771}
\pmtype{Definition}
\pmcomment{trigger rebuild}
\pmclassification{msc}{06C15}
\pmclassification{msc}{06B05}
\pmsynonym{perspective elements}{ComplementedLattice}
\pmsynonym{complemented}{ComplementedLattice}
\pmrelated{Perspectivity}
\pmrelated{OrthocomplementedLattice}
\pmrelated{PseudocomplementedLattice}
\pmrelated{DifferenceOfLatticeElements}
\pmrelated{Pseudocomplement}
\pmdefines{related elements in lattice}
\pmdefines{complement}
\pmdefines{complemented element}

% this is the default PlanetMath preamble.  as your knowledge
% of TeX increases, you will probably want to edit this, but
% it should be fine as is for beginners.

% almost certainly you want these
\usepackage{amssymb,amscd}
\usepackage{amsmath}
\usepackage{amsfonts}

% used for TeXing text within eps files
%\usepackage{psfrag}
% need this for including graphics (\includegraphics)
%\usepackage{graphicx}
% for neatly defining theorems and propositions
%\usepackage{amsthm}
% making logically defined graphics
%%\usepackage{xypic}

% there are many more packages, add them here as you need them

% define commands here
\begin{document}
Let $L$ be a bounded lattice (with $0$ and $1$), and $a\in L$.  A \emph{complement} of $a$ is an element $b\in L$ such that 
\begin{quote}$a\land b=0$ and $a\lor b=1$.\end{quote}

\textbf{Remark}.  Complements may not exist.  If $L$ is a non-trivial chain, then no element (other than $0$ and $1$) has a complement.  This also shows that if $a$ is a complement of a non-trivial element $b$, then $a$ and $b$ form an antichain.

An element in a bounded lattice is \emph{complemented} if it has a complement.  A \emph{complemented lattice} is a bounded lattice in which every element is complemented.


\textbf{Remarks}.
\begin{itemize}
\item In a complemented lattice, there may be more than one complement corresponding to each element.  Two elements are said to be \emph{related}, or \emph{perspective} if they have a common complement.  For example, the following lattice is complemented.

\begin{equation*}
\xymatrix{
& 1 \ar@{-}[ld] \ar@{-}[d] \ar@{-}[rd] & \\
a \ar@{-}[rd] & b \ar@{-}[d] & c \ar@{-}[ld] \\
& 0 &
}
\end{equation*}

Note that none of the non-trivial elements have unique complements.  Any two non-trivial elements are related via the third.
\item If a complemented lattice $L$ is a distributive lattice, then $L$ is uniquely complemented (in fact, a Boolean lattice).  For if $y_1$ and $y_2$ are two complements of $x$, then $$y_2=1\land y_2=(x\lor y_1)\land y_2=
(x\land y_2)\lor(y_1\land y_2)=0\lor(y_1\land y_2)=y_1\land y_2.$$  Similarly, $y_1=y_2\land y_1$. So $y_2=y_1$.
\item In the category of complemented lattices, a morphism between two objects is a $\lbrace 0,1\rbrace$-lattice homomorphism; that is, a lattice homomorphism which preserves $0$ and $1$.
\end{itemize}
%%%%%
%%%%%
\end{document}
