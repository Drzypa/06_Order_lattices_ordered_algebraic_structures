\documentclass[12pt]{article}
\usepackage{pmmeta}
\pmcanonicalname{ExampleOfNoncompleteLatticeHomomorphism}
\pmcreated{2013-03-22 16:58:36}
\pmmodified{2013-03-22 16:58:36}
\pmowner{Algeboy}{12884}
\pmmodifier{Algeboy}{12884}
\pmtitle{example of non-complete lattice homomorphism}
\pmrecord{4}{39253}
\pmprivacy{1}
\pmauthor{Algeboy}{12884}
\pmtype{Example}
\pmcomment{trigger rebuild}
\pmclassification{msc}{06B05}
\pmclassification{msc}{06B99}
\pmrelated{ExtendedRealNumbers}

\usepackage{latexsym}
\usepackage{amssymb}
\usepackage{amsmath}
\usepackage{amsfonts}
\usepackage{amsthm}

%%\usepackage{xypic}

%-----------------------------------------------------

%       Standard theoremlike environments.

%       Stolen directly from AMSLaTeX sample

%-----------------------------------------------------

%% \theoremstyle{plain} %% This is the default

\newtheorem{thm}{Theorem}

\newtheorem{coro}[thm]{Corollary}

\newtheorem{lem}[thm]{Lemma}

\newtheorem{lemma}[thm]{Lemma}

\newtheorem{prop}[thm]{Proposition}

\newtheorem{conjecture}[thm]{Conjecture}

\newtheorem{conj}[thm]{Conjecture}

\newtheorem{defn}[thm]{Definition}

\newtheorem{remark}[thm]{Remark}

\newtheorem{ex}[thm]{Example}



%\countstyle[equation]{thm}



%--------------------------------------------------

%       Item references.

%--------------------------------------------------


\newcommand{\exref}[1]{Example-\ref{#1}}

\newcommand{\thmref}[1]{Theorem-\ref{#1}}

\newcommand{\defref}[1]{Definition-\ref{#1}}

\newcommand{\eqnref}[1]{(\ref{#1})}

\newcommand{\secref}[1]{Section-\ref{#1}}

\newcommand{\lemref}[1]{Lemma-\ref{#1}}

\newcommand{\propref}[1]{Prop\-o\-si\-tion-\ref{#1}}

\newcommand{\corref}[1]{Cor\-ol\-lary-\ref{#1}}

\newcommand{\figref}[1]{Fig\-ure-\ref{#1}}

\newcommand{\conjref}[1]{Conjecture-\ref{#1}}


% Normal subgroup or equal.

\providecommand{\normaleq}{\unlhd}

% Normal subgroup.

\providecommand{\normal}{\lhd}

\providecommand{\rnormal}{\rhd}
% Divides, does not divide.

\providecommand{\divides}{\mid}

\providecommand{\ndivides}{\nmid}


\providecommand{\union}{\cup}

\providecommand{\bigunion}{\bigcup}

\providecommand{\intersect}{\cap}

\providecommand{\bigintersect}{\bigcap}










\begin{document}
The real number line $[-\infty,\infty]=\mathbb{R}\union\{-\infty,\infty\}$ is complete 
in its usual ordering of numbers.  Furthermore, the meet of a subset $S$ of $\mathbb{R}$ is 
the infimum of the set $S$.

Now define the map $f:[-\infty,\infty]\to [-\infty,\infty]$ as
\[f(x)=\left\{\begin{array}{cc} 0 & x\leq 0\\ 1 & x>0.\end{array}\right.\]
First notice that if $x\leq y$ then $f(x)\leq f(y)$, for either $x\leq y\leq 0$ in
which case $f(x)=0=f(y)$, or $x\leq 0< y$ which gives $f(x)=0<1=f(y)$ or
$0<x\leq y$ so $f(x)=1=f(y)$.

In the second place, if $S$ is a finite subset of $\mathbb{R}$ then $S$ contains 
a minimum element $s\in S$.  So $f(s)\in f(S)$ and $f(s)\leq f(t)$ for all $t\in S$, 
so $f(\min S)=f(s)=\min f(S)$.  Hence $f$ is a lattice homomorphism.

However, $f$ is not a complete lattice homomorphism.  To see this let
$S=\{x\in \mathbb{R}: 0< x\}$.  Then $\inf S=0$.  However,
$f(\inf S)=f(0)=0$ while $\inf f(S)=\inf \{1\}=1$.  

%%%%%
%%%%%
\end{document}
