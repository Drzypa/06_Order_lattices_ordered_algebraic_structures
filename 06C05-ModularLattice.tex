\documentclass[12pt]{article}
\usepackage{pmmeta}
\pmcanonicalname{ModularLattice}
\pmcreated{2013-03-22 12:27:26}
\pmmodified{2013-03-22 12:27:26}
\pmowner{yark}{2760}
\pmmodifier{yark}{2760}
\pmtitle{modular lattice}
\pmrecord{17}{32598}
\pmprivacy{1}
\pmauthor{yark}{2760}
\pmtype{Definition}
\pmcomment{trigger rebuild}
\pmclassification{msc}{06C05}
\pmsynonym{Dedekind lattice}{ModularLattice}
\pmrelated{ModularLaw}
\pmrelated{SemimodularLattice}
\pmrelated{NonmodularSublattice}
\pmrelated{ModularInequality}
\pmdefines{modular}

\endmetadata

\usepackage{amssymb}
\usepackage{amsmath}
\usepackage{amsfonts}

\def\meet{\land}
\def\join{\lor}

\begin{document}
\PMlinkescapeword{modular}
\PMlinkescapeword{rank}
\PMlinkescapeword{satisfies}

A lattice $L$ is said to be \emph{modular}
if $x \lor (y \land z) = (x \lor y) \land z$
for all $x,y,z\in L$ such that $x \leq z$.
In fact it is sufficient to show that
$x \lor (y \land z) \ge (x \lor y) \land z$
for all $x,y,z\in L$ such that $x \leq z$,
as the reverse inequality holds in all lattices (see modular inequality).

There are a number of other equivalent conditions for a lattice $L$ to be modular:
\begin{itemize}
\item $(x\meet y)\join(x\meet z)=x\meet(y\join(x\meet z))$
      for all $x,y,z\in L$.
\item $(x\join y)\meet(x\join z)=x\join(y\meet(x\join z))$
      for all $x,y,z\in L$.
\item For all $x,y,z\in L$,
      if $x<z$ then either $x\meet y<z\meet y$ or $x\join y<z\join y$.
\end{itemize}

The following are examples of modular lattices.
\begin{itemize}
\item All \PMlinkname{distributive lattices}{DistributiveLattice}.
\item The lattice of normal subgroups of any group.
\item The lattice of submodules of any \PMlinkname{module}{Module}.
     (See modular law.)
\end{itemize}

A finite lattice $L$ is modular
if and only if it
is graded and its rank function $\rho$ satisfies
$\rho(x)+\rho(y)=\rho(x\land y)+\rho(x\lor y)$ for all $x,y\in L$.
%%%%%
%%%%%
\end{document}
