\documentclass[12pt]{article}
\usepackage{pmmeta}
\pmcanonicalname{DisjunctionPropertyOfWallman}
\pmcreated{2013-03-22 17:53:48}
\pmmodified{2013-03-22 17:53:48}
\pmowner{porton}{9363}
\pmmodifier{porton}{9363}
\pmtitle{disjunction property of Wallman}
\pmrecord{7}{40385}
\pmprivacy{1}
\pmauthor{porton}{9363}
\pmtype{Definition}
\pmcomment{trigger rebuild}
\pmclassification{msc}{06A06}
\pmsynonym{Wallman's disjunction property}{DisjunctionPropertyOfWallman}
%\pmkeywords{partial order}
\pmrelated{Poset}

% this is the default PlanetMath preamble.  as your knowledge
% of TeX increases, you will probably want to edit this, but
% it should be fine as is for beginners.

% almost certainly you want these
\usepackage{amssymb}
\usepackage{amsmath}
\usepackage{amsfonts}

% used for TeXing text within eps files
%\usepackage{psfrag}
% need this for including graphics (\includegraphics)
%\usepackage{graphicx}
% for neatly defining theorems and propositions
%\usepackage{amsthm}
% making logically defined graphics
%%%\usepackage{xypic}

% there are many more packages, add them here as you need them

% define commands here

\begin{document}
A partially ordered set $\mathfrak{A}$ with a least element $0$ has the \emph{disjunction property of Wallman} if for every pair $(a,b)$ of elements of the poset, either $b\leq a$ or there exists an element $c\leq b$ such that $c\ne 0$ and $c$ has no nontrivial common predecessor with $a$. That is, in the latter case, the only $x$ with $x\leq a$ and $x\leq c$ is $x=0$.

For the case if the poset $\mathfrak{A}$ is a $\cap$-semilattice \emph{disjunction property of Wallman} is equivalent to every of the following three formulas:

\begin{enumerate}
\item
$\forall a,b\in\mathfrak{A}:(\{c\in\mathfrak{A}|c\cap a\ne 0\} = \{c\in\mathfrak{A}|c\cap b\ne 0\} \Rightarrow a = b)$;
\item
$\forall a,b\in\mathfrak{A}:(\{c\in\mathfrak{A}|c\cap a\ne 0\} \subseteq \{c\in\mathfrak{A}|c\cap b\ne 0\} \Rightarrow
  a \subseteq b)$;
\item
$\forall a,b\in\mathfrak{A}:(a\subset b \Rightarrow
\{c\in\mathfrak{A}|c\cap a\ne 0\} \subset \{c\in\mathfrak{A}|c\cap b\ne 0\})$.
\end{enumerate}

The proof of this equivalence can be found in \PMlinkexternal{this online article}{http://www.mathematics21.org/binaries/filters.pdf}.
%%%%%
%%%%%
\end{document}
