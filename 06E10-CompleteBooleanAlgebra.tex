\documentclass[12pt]{article}
\usepackage{pmmeta}
\pmcanonicalname{CompleteBooleanAlgebra}
\pmcreated{2013-03-22 18:01:09}
\pmmodified{2013-03-22 18:01:09}
\pmowner{CWoo}{3771}
\pmmodifier{CWoo}{3771}
\pmtitle{complete Boolean algebra}
\pmrecord{11}{40535}
\pmprivacy{1}
\pmauthor{CWoo}{3771}
\pmtype{Definition}
\pmcomment{trigger rebuild}
\pmclassification{msc}{06E10}
\pmrelated{CompleteLattice}
\pmdefines{$\kappa$-complete Boolean algebra}
\pmdefines{countably complete Boolean algebra}

\usepackage{amssymb,amscd}
\usepackage{amsmath}
\usepackage{amsfonts}
\usepackage{mathrsfs}

% used for TeXing text within eps files
%\usepackage{psfrag}
% need this for including graphics (\includegraphics)
%\usepackage{graphicx}
% for neatly defining theorems and propositions
\usepackage{amsthm}
% making logically defined graphics
%%\usepackage{xypic}
\usepackage{pst-plot}

% define commands here
\newcommand*{\abs}[1]{\left\lvert #1\right\rvert}
\newtheorem{prop}{Proposition}
\newtheorem{thm}{Theorem}
\newtheorem{ex}{Example}
\newcommand{\real}{\mathbb{R}}
\newcommand{\pdiff}[2]{\frac{\partial #1}{\partial #2}}
\newcommand{\mpdiff}[3]{\frac{\partial^#1 #2}{\partial #3^#1}}

\begin{document}
A Boolean algebra $A$ is a \emph{complete Boolean algebra} if for every subset $C$ of $A$, the arbitrary join and arbitrary meet of $C$ exist.

By de Morgan's laws, it is easy to see that a Boolean algebra is complete iff the arbitrary join of any subset exists iff the arbitrary meet of any subset exists.  For a proof of this, see \PMlinkname{this link}{PropertiesOfArbitraryJoinsAndMeets}.

For an example of a complete Boolean algebra, let $S$ be any set.  Then the powerset $P(S)$ with the usual set theoretic operations is a complete Boolean algebra.

In a complete Boolean algebra, the infinite distributive and infinite deMorgan's laws hold:
\begin{itemize}
\item $x\wedge \bigvee A = \bigvee (x\wedge A)$ and $x\vee \bigwedge A = \bigwedge (x\vee A)$
\item $(\bigvee A)^* = \bigwedge A^*$ and $(\bigwedge A)^* = \bigvee A^*$, where $A^*:=\lbrace a^* \mid a\in A\rbrace$.
\end{itemize}

In the category of complete Boolean algebras, a morphism between two objects is a Boolean algebra homomorphism that preserves arbitrary joins (equivalently, arbitrary meets), and is called a \emph{complete Boolean algebra homomorphism}.

\textbf{Remark}  There are infinitely many algebras between Boolean algebras and complete Boolean algebras.  Let $\kappa$ be a cardinal.  A Boolean algebra $A$ is said to be $\kappa$-complete if for every subset $C$ of $A$ with $|C|\le \kappa$, $\bigvee C$ (and equivalently $\bigwedge C$) exists.  A $\kappa$-complete Boolean algebra is usually called a $\kappa$-algebra.  If $\kappa=\aleph_0$, the first aleph number, then it is called a \emph{countably complete Boolean algebra}.  

Any complete Boolean algebra is $\kappa$-complete, and any $\kappa$-complete is $\lambda$-complete for any $\lambda\le \kappa$.  An example of a $\kappa$-complete algebra that is not complete, take a set $S$ with $\kappa < |S|$, then the collection $A\subseteq P(S)$ consisting of any subset $T$ such that either $|T|\le \kappa$ or $|S-T|\le \kappa$ is $\kappa$-complete but not complete.

A Boolean algebra homomorphism $f$ between two $\kappa$-algebras $A,B$ is said to be $\kappa$-complete if $$f(\bigvee \lbrace a \mid a\in C\rbrace)= \bigvee \lbrace f(a)\mid a\in C\rbrace $$ for any $C\subseteq A$ with $|C|\le \kappa$.
%%%%%
%%%%%
\end{document}
