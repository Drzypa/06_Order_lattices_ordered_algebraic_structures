\documentclass[12pt]{article}
\usepackage{pmmeta}
\pmcanonicalname{MacNeilleCompletion}
\pmcreated{2013-03-22 16:05:27}
\pmmodified{2013-03-22 16:05:27}
\pmowner{CWoo}{3771}
\pmmodifier{CWoo}{3771}
\pmtitle{MacNeille completion}
\pmrecord{8}{38152}
\pmprivacy{1}
\pmauthor{CWoo}{3771}
\pmtype{Definition}
\pmcomment{trigger rebuild}
\pmclassification{msc}{06B23}
\pmsynonym{Dedekind-MacNeille completion}{MacNeilleCompletion}
\pmsynonym{normal completion}{MacNeilleCompletion}
\pmrelated{DedekindCuts}

\endmetadata

\usepackage{amssymb,amscd}
\usepackage{amsmath}
\usepackage{amsfonts}

% used for TeXing text within eps files
%\usepackage{psfrag}
% need this for including graphics (\includegraphics)
%\usepackage{graphicx}
% for neatly defining theorems and propositions
%\usepackage{amsthm}
% making logically defined graphics
%%\usepackage{xypic}
\usepackage{pst-plot}
\usepackage{psfrag}

% define commands here

\begin{document}
In a first course on real analysis, one is generally introduced to the concept of a Dedekind cut.  It is a way of constructing the set of real numbers from the rationals.  This is a process commonly known as the completion of the rationals.  Three key features of this completion are:
\begin{itemize}
\item the rationals can be embedded in its completion (the reals)
\item every subset with an upper bound has a least upper bound
\item every subset with a lower bound has a greatest lower bound
\end{itemize}

If we extend the reals by adjoining $+\infty$ and $-\infty$ and define the appropriate ordering relations on this new extended set (the extended real numbers), then it is a set where every subset has a least upper bound and a greatest lower bound.

When we deal with the rationals and the reals (and extended reals), we are working with linearly ordered sets.  So the next question is: can the procedure of a completion be generalized to an arbitrary poset?  In other words, if $P$ is a poset ordered by $\le$, does there exist another poset $Q$ ordered by $\le_Q$ such that 
\begin{enumerate}
\item $P$ can be embedded in $Q$ as a poset (so that $\le$ is compatible with $\le_Q$), and
\item every subset of $Q$ has both a least upper bound and a greatest lower bound
\end{enumerate}

In 1937, MacNeille answered this question in the affirmative by the following construction:

\begin{quote}
Given a poset $P$ with order $\le$, define for every subset $A$ of $P$, two subsets of $P$ as follows: 
$$A^u=\lbrace p\in P\mid a\le p\mbox{ for all }a\in A\rbrace\mbox{ and }A^{\ell}=\lbrace q\in P\mid q\le a\mbox{ for all }a\in A\rbrace.$$
Then $M(P):=\lbrace A\in 2^P \mid (A^u)^{\ell}=A\rbrace$ ordered by the usual set inclusion is a poset satisfying conditions (1) and (2) above.
\end{quote}

This is known as the \emph{MacNeille completion} $M(P)$ of a poset $P$.  In $M(P)$, since lub and glb exist for any subset, $M(P)$ is a complete lattice.  So this process can be readily applied to any lattice, if we define a completion of a lattice to follow the two conditions above.

\begin{thebibliography}{8}
\bibitem{hmm} H. M. MacNeille, {\it Partially Ordered Sets}. Trans. Amer. Math. Soc. 42 (1937), pp 416-460
\bibitem{dp} B. A. Davey, H. A. Priestley, {\it Introduction to Lattices and Order}, 2nd edition, Cambridge (2003)
\end{thebibliography}
%%%%%
%%%%%
\end{document}
