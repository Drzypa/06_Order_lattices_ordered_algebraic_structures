\documentclass[12pt]{article}
\usepackage{pmmeta}
\pmcanonicalname{BoundedLattice}
\pmcreated{2013-03-22 15:02:28}
\pmmodified{2013-03-22 15:02:28}
\pmowner{CWoo}{3771}
\pmmodifier{CWoo}{3771}
\pmtitle{bounded lattice}
\pmrecord{13}{36755}
\pmprivacy{1}
\pmauthor{CWoo}{3771}
\pmtype{Definition}
\pmcomment{trigger rebuild}
\pmclassification{msc}{06B05}
\pmclassification{msc}{06A06}
\pmdefines{top}
\pmdefines{bottom}
\pmdefines{bounded poset}

\endmetadata

% this is the default PlanetMath preamble.  as your knowledge
% of TeX increases, you will probably want to edit this, but
% it should be fine as is for beginners.

% almost certainly you want these
\usepackage{amssymb,amscd}
\usepackage{amsmath}
\usepackage{amsfonts}

% used for TeXing text within eps files
%\usepackage{psfrag}
% need this for including graphics (\includegraphics)
%\usepackage{graphicx}
% for neatly defining theorems and propositions
%\usepackage{amsthm}
% making logically defined graphics
%%%\usepackage{xypic}

% there are many more packages, add them here as you need them

% define commands here
\begin{document}
A lattice $L$ is said to be \emph{\PMlinkescapetext{bounded from below}} if there is an element $0\in L$ such that $0\leq x$ for all $x\in L$.  Dually, $L$ is \emph{\PMlinkescapetext{bounded from above}} if there exists an element $1\in L$ such that $x\leq1$ for all $x\in L$.  A \emph{bounded lattice} is one that is \PMlinkescapetext{bounded} both from above and below.

For example, any finite lattice $L$ is bounded, as $\bigvee L$ and $\bigwedge L$, being join and meet of finitely many elements, exist.  $\bigvee L=1$ and $\bigwedge L=0$.

\textbf{Remarks}.
Let $L$ be a bounded lattice with $0$ and $1$ as described above.
\begin{itemize}
\item $0\land x=0$ and $0\lor x=x$ for all $x\in L$.
\item $1\land x=x$ and $1\lor x=1$ for all $x\in L$.
\item As a result, $0$ and $1$, if they exist, are necessarily unique.  For
if there is another such a pair $0^{\prime}$ and $1^{\prime}$, then
$0=0\land 0^{\prime}=0^{\prime}\land 0=0^{\prime}$.  Similarly
$1=1^{\prime}$.
\item $0$ is called the \emph{bottom} of $L$ and $1$ is called the \emph{top} of $L$.
\item $L$ is a lattice interval and can be written as $[0,1]$.
\end{itemize}

\textbf{Remark}.  More generally, a poset $P$ is said to be \emph{bounded} if it has both a greatest element $1$ and a least element $0$.
%%%%%
%%%%%
\end{document}
