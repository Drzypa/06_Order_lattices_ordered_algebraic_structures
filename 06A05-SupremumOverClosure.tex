\documentclass[12pt]{article}
\usepackage{pmmeta}
\pmcanonicalname{SupremumOverClosure}
\pmcreated{2013-03-22 17:08:22}
\pmmodified{2013-03-22 17:08:22}
\pmowner{Wkbj79}{1863}
\pmmodifier{Wkbj79}{1863}
\pmtitle{supremum over closure}
\pmrecord{9}{39446}
\pmprivacy{1}
\pmauthor{Wkbj79}{1863}
\pmtype{Theorem}
\pmcomment{trigger rebuild}
\pmclassification{msc}{06A05}
\pmclassification{msc}{26A15}

\endmetadata

\usepackage{amssymb}
\usepackage{amsmath}
\usepackage{amsfonts}

\usepackage{psfrag}
\usepackage{graphicx}
\usepackage{amsthm}
%%\usepackage{xypic}

\newtheorem{thm*}{Theorem}

\begin{document}
\begin{thm*}
Let $f \colon \mathbb{R} \to \mathbb{R}$ be a continuous function and $A \subseteq \mathbb{R}$.  Then $\displaystyle \sup_{x \in A} f(x)=\sup_{x \in \overline{A}} f(x)$, where $\overline{A}$ denotes the closure of $A$.
\end{thm*}

\begin{proof}
The theorem is clearly true for $A=\emptyset$.  Thus, it will be assumed that $A \neq \emptyset$.

Since $A \subseteq \overline{A}$, we have $\displaystyle \sup_{x \in A} f(x) \le \sup_{x \in \overline{A}} f(x)$.

Suppose first that $\displaystyle \sup_{x \in \overline{A}} f(x)=\infty$.  Let $r \in \mathbb{R}$.  Then there exists $x_0 \in \overline{A}$ with $f(x_0) \ge r+1$.  Since $f$ is continuous, there exists $\delta >0$ such that, for any $x \in \mathbb{R}$ with $-\delta<x-x_0<\delta$, we have $-1<f(x)-f(x_0)<1$.  Since $x_0 \in \overline{A}$, there exists $x_1 \in A$ with $-\delta <x_1-x_0<\delta$.  (Recall that $x \in \overline{A}$ if and only if every neighborhood of $x$ intersects $A$.)  Thus, $f(x_1)-f(x_0)>-1$.  Therefore, $f(x_1)>f(x_0)-1 \ge r+1-1=r$.  Hence, $\displaystyle \sup_{x \in A} f(x)=\infty$.

Now suppose that $\displaystyle \sup_{x \in \overline{A}} f(x)=R$ for some $R \in \mathbb{R}$.  Let $\varepsilon >0$.  Then there exists $x_2 \in \overline{A}$ with $f(x_2) \ge R-\frac{\varepsilon}{2}$.  Since $f$ is continuous, there exists $\delta '>0$ such that, for any $x \in \mathbb{R}$ with $-\delta '<x-x_0<\delta '$, we have $\frac{-\varepsilon}{2}<f(x)-f(x_2)<\frac{\varepsilon}{2}$.  Since $x_2 \in \overline{A}$, there exists $x_3 \in A$ with $-\delta '<x_3-x_2<\delta '$.  Thus, $f(x_3)-f(x_2)>\frac{-\varepsilon}{2}$.  Therefore, $f(x_3)>f(x_2)-\frac{\varepsilon}{2} \ge R-\frac{\varepsilon}{2}-\frac{\varepsilon}{2}=R-\varepsilon$.  Hence, $\displaystyle \sup_{x \in A} f(x) \ge R$.

In either case, it follows that $\displaystyle \sup_{x \in A} f(x)=\sup_{x \in \overline{A}} f(x)$.
\end{proof}

Note that this theorem also holds for continuous functions $f \colon X \to \mathbb{R}$, where $X$ is an arbitrary topological space.  To prove this fact, one would need to slightly adjust the proof supplied here.
%%%%%
%%%%%
\end{document}
