\documentclass[12pt]{article}
\usepackage{pmmeta}
\pmcanonicalname{CompleteLattice}
\pmcreated{2013-03-22 12:56:44}
\pmmodified{2013-03-22 12:56:44}
\pmowner{yark}{2760}
\pmmodifier{yark}{2760}
\pmtitle{complete lattice}
\pmrecord{10}{33304}
\pmprivacy{1}
\pmauthor{yark}{2760}
\pmtype{Definition}
\pmcomment{trigger rebuild}
\pmclassification{msc}{06B23}
\pmclassification{msc}{03G10}
\pmrelated{TarskiKnasterTheorem}
\pmrelated{CompleteLatticeHomomorphism}
\pmrelated{Domain6}
\pmrelated{CompleteSemilattice}
\pmrelated{InfiniteAssociativityOfSupremumAndInfimumRegardingItself}
\pmrelated{CompleteBooleanAlgebra}
\pmrelated{ArbitraryJoin}
\pmdefines{countably complete lattice}
\pmdefines{countably-complete lattice}
\pmdefines{$\kappa$-complete}
\pmdefines{$\kappa$-complete lattice}


\begin{document}
\section*{Complete lattices}

A \emph{complete lattice} is a poset $P$
such that every subset of $P$ has both a supremum and an infimum in $P$.

For a complete lattice $L$,
the supremum of $L$ is denoted by $1$,
and the infimum of $L$ is denoted by $0$.
Thus $L$ is a bounded lattice,
with $1$ as its greatest element and $0$ as its least element.
Moreover, $1$ is the infimum of the empty set,
and $0$ is the supremum of the empty set.

\section*{Generalizations}

A \emph{countably complete lattice} is a poset $P$
such that every countable subset of $P$
has both a supremum and an infimum in $P$.

Let $\kappa$ be an infinite cardinal.
A $\kappa$-complete lattice is a lattice $L$
such that for every subset $A\subseteq L$
with $|A|\le \kappa$, both $\bigvee A$ and $\bigwedge A$ exist.
(Note that an $\aleph_0$-complete lattice
is the same as a countably complete lattice.)

Every complete lattice is a \PMlinkescapetext{$\kappa$-complete lattice}
for every infinite cardinal $\kappa$,
and in particular is a countably complete lattice.
Every countably complete lattice is a bounded lattice.


%%%%%
%%%%%
\end{document}
