\documentclass[12pt]{article}
\usepackage{pmmeta}
\pmcanonicalname{Positive}
\pmcreated{2013-03-22 14:35:05}
\pmmodified{2013-03-22 14:35:05}
\pmowner{pahio}{2872}
\pmmodifier{pahio}{2872}
\pmtitle{positive}
\pmrecord{19}{36147}
\pmprivacy{1}
\pmauthor{pahio}{2872}
\pmtype{Definition}
\pmcomment{trigger rebuild}
\pmclassification{msc}{06F25}
\pmclassification{msc}{11B99}
\pmclassification{msc}{00A05}
\pmsynonym{greater than zero}{Positive}
\pmdefines{negative}

\endmetadata

% this is the default PlanetMath preamble.  as your knowledge
% of TeX increases, you will probably want to edit this, but
% it should be fine as is for beginners.

% almost certainly you want these
\usepackage{amssymb}
\usepackage{amsmath}
\usepackage{amsfonts}

% used for TeXing text within eps files
%\usepackage{psfrag}
% need this for including graphics (\includegraphics)
%\usepackage{graphicx}
% for neatly defining theorems and propositions
%\usepackage{amsthm}
% making logically defined graphics
%%%\usepackage{xypic}

% there are many more packages, add them here as you need them

% define commands here
\begin{document}
\PMlinkescapeword{closed}

The \PMlinkescapetext{word} {\em positive} is usually explained to \PMlinkescapetext{mean} that the number under consideration is greater than zero. \,Without the relation ``$>$'', the positivity of (\PMlinkescapetext{real}) numbers may be defined specifying which numbers of a given number kind are positive, e.g. as follows.

\begin{itemize}
\item In the set $\mathbb{Z}$ of the integers, all numbers obtained from 1 via addition are positive.
\item In the set $\mathbb{Q}$ of the rationals, all numbers obtained from 1 via addition and division are positive.
\item In the set $\mathbb{R}$ of the real numbers, the numbers defined by the equivalence classes of non-zero decimal sequences are positive; these sequences (decimal expansions) consist of natural numbers from 0 to 9 as digits and a single decimal point (where two decimal sequences are equivalent if they are identical, or if one has an infinite tail of 9's, the other has an infinite tail of 0's, and the leading portion of the first sequence is one lower than the leading portion of the second).
\end{itemize}

For example, $1+1+1$ is a positive integer, $\frac{1+1}{1+1+1+1+1}$ is a positive rational and \,$5.15115111511115...$ is a positive real number.

If $a$ is positive and \,$a+b = 0$, then the opposite number $b$ is {\em negative}.

The sets of positive integers, positive rationals, positive (real) algebraic numbers and positive reals are closed under addition and  multiplication, so also the set of positive even numbers.


%%%%%
%%%%%
\end{document}
