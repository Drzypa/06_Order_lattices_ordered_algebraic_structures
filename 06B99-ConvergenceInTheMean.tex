\documentclass[12pt]{article}
\usepackage{pmmeta}
\pmcanonicalname{ConvergenceInTheMean}
\pmcreated{2015-04-08 7:29:35}
\pmmodified{2015-04-08 7:29:35}
\pmowner{pahio}{2872}
\pmmodifier{pahio}{2872}
\pmtitle{convergence in the mean}
\pmrecord{7}{88226}
\pmprivacy{1}
\pmauthor{pahio}{2872}
\pmtype{Definition}

% this is the default PlanetMath preamble.  as your knowledge
% of TeX increases, you will probably want to edit this, but
% it should be fine as is for beginners.

% almost certainly you want these
\usepackage{amssymb}
\usepackage{amsmath}
\usepackage{amsfonts}

% need this for including graphics (\includegraphics)
\usepackage{graphicx}
% for neatly defining theorems and propositions
\usepackage{amsthm}

% making logically defined graphics
%\usepackage{xypic}
% used for TeXing text within eps files
%\usepackage{psfrag}

% there are many more packages, add them here as you need them

% define commands here

\begin{document}
Let
$$b_n := \frac{a_1+a_2+\ldots+a_n}{n} \quad (n = 1,2,3,\ldots)$$
be the arithmetic mean of the numbers $a_1,a_2,\ldots,a_n$. \,
The sequence 
\begin{align}
a_1, a_2, a_3, \ldots
\end{align}
is said to \PMlinkname{{\it converge in the mean}}{ConvergenceInTheMean} iff the 
sequence
\begin{align}
b_1,b_2,b_3,\ldots
\end{align}
converges.\\
On has the

\textbf{Theorem.}\, If the sequence (1) is convergent having the limit $A$, then also the sequence
(2) converges to the limit $A$.\, Thus, a convergent sequence is always convergent in the mean.

{\it Proof.}\, Let $\varepsilon$ be an arbitrary positive number.\, We may write
\begin{align*}
|A-b_n|  &= |A-\frac{1}{n}(a_1+\ldots+a_k)-\frac{1}{n}(a_{k+1}+\ldots+a_n)|\\ 
         &= |\frac{1}{n}[(A-a_1)+\ldots+(A-a_k)]+\frac{1}{n}[(A-a_{k+1})+\ldots+(A-a_n)]|\\
         &\leqq \frac{|(A-a_1)+\ldots+(A-a_k)|}{n}+\frac{|A-a_{k+1}|+\ldots+|A-a_n|}{n}.
\end{align*}
The supposition implies that there is a positive integer $k$ such that 
$$|A-a_i| < \frac{\varepsilon}{2} \quad\mbox{ for all  } i > k.$$
Let's fix the integer $k$.\, Choose the number 
$l$ so great that
$$\frac{|(A-a_1)+\ldots+(A-a_k)|}{n} < \frac{\varepsilon}{2}
 \quad\mbox{ for   } n > l.$$
Let now\, $n > \max\{k,l\}$.\, The three above inequalities yield
$$|A-b_n| \;<\; \frac{\varepsilon}{2}+\frac{1}{n}\!(n-k)\!\!\frac{\varepsilon}{2} \;<\;
\frac{\varepsilon}{2}+\frac{\varepsilon}{2} = \varepsilon,$$
whence we have\, 
$$\lim_{n\to\infty}b_n = A.$$\\

\textbf{Note.}\, The \PMlinkname{converse}{Converse} of the theorem is not 
true.\, For example, if 
$$a_n := \frac{1+(-1)^n}{2}$$
i.e. if the sequence (1) has the form\, $0,1,0,1,0,1,\ldots,$\,
then it is divergent but converges in the mean to the limit 
$\frac{1}{2}$; the corresponding sequence (2) is
$0,\frac{1}{2},\frac{1}{3},\frac{2}{4},\frac{2}{5},
\frac{3}{6},\frac{3}{7},\frac{4}{8},\frac{4}{9},\ldots$\\


\end{document}
