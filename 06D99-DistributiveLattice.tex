\documentclass[12pt]{article}
\usepackage{pmmeta}
\pmcanonicalname{DistributiveLattice}
\pmcreated{2013-03-22 12:27:23}
\pmmodified{2013-03-22 12:27:23}
\pmowner{yark}{2760}
\pmmodifier{yark}{2760}
\pmtitle{distributive lattice}
\pmrecord{20}{32596}
\pmprivacy{1}
\pmauthor{yark}{2760}
\pmtype{Definition}
\pmcomment{trigger rebuild}
\pmclassification{msc}{06D99}
\pmrelated{Distributive}
\pmrelated{Lattice}
\pmrelated{BirkhoffPrimeIdealTheorem}

\endmetadata

\usepackage{amssymb}
\usepackage{amsmath}
\usepackage{amsfonts}
\begin{document}
\PMlinkescapeword{distributive}
\PMlinkescapeword{examples}
\PMlinkescapeword{lattice}
\PMlinkescapeword{lattices}

A \PMlinkname{lattice}{Lattice} is said to be \emph{distributive} if it satisifes either (and therefore both) of the \PMlinkname{distributive laws}{Distributive}:
\begin{itemize}
\item $x \land (y \lor z) = (x \land y) \lor (x \land z)$ 
\item $x \lor (y \land z) = (x \lor y) \land (x \lor z)$
\end{itemize}
Every distributive lattice is \PMlinkname{modular}{ModularLattice}.

Examples of distributive lattices include \PMlinkname{Boolean lattices}{BooleanLattice}, totally ordered sets, and the \PMlinkname{subgroup lattices}{LatticeOfSubgroups} of locally cyclic groups.
%%%%%
%%%%%
\end{document}
