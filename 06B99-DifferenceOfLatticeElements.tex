\documentclass[12pt]{article}
\usepackage{pmmeta}
\pmcanonicalname{DifferenceOfLatticeElements}
\pmcreated{2013-03-22 17:57:44}
\pmmodified{2013-03-22 17:57:44}
\pmowner{porton}{9363}
\pmmodifier{porton}{9363}
\pmtitle{difference of lattice elements}
\pmrecord{10}{40464}
\pmprivacy{1}
\pmauthor{porton}{9363}
\pmtype{Definition}
\pmcomment{trigger rebuild}
\pmclassification{msc}{06B99}
\pmrelated{ComplementedLattice}
\pmrelated{Pseudodifference}
\pmrelated{SectionallyComplementedLattice}

% this is the default PlanetMath preamble.  as your knowledge
% of TeX increases, you will probably want to edit this, but
% it should be fine as is for beginners.

% almost certainly you want these
\usepackage{amssymb}
\usepackage{amsmath}
\usepackage{amsfonts}

% used for TeXing text within eps files
%\usepackage{psfrag}
% need this for including graphics (\includegraphics)
%\usepackage{graphicx}
% for neatly defining theorems and propositions
%\usepackage{amsthm}
% making logically defined graphics
%%%\usepackage{xypic}

% there are many more packages, add them here as you need them

% define commands here

\begin{document}
Let $\mathfrak{A}$ is a lattice with least element $0$.

Let $a,b\in\mathfrak{A}$. A \emph{\PMlinkescapetext{difference}} of $a$ and $b$ is an element $c\in\mathfrak{A}$ that $b\cap c=0$ and $a\cup b=b\cup c$.  When there is only one difference of $a$ and $b$, it is denoted $a\setminus b$.

One immediate property is: $0$ is the unique difference of any element $a$ and itself ($a\setminus a=0$).  For if $c$ is such a difference, then $a\cap c=0$ and $a=a\cup c$.  So $c\le a$ by the second equation, and hence that $c=a\cap c=0$ by the first equation.

For arbitrary lattices \emph{\PMlinkescapetext{differences}} of two given elements do not necessarily exist. For some lattices there may be more than one difference of two given elements.

For a distributive lattice with bottom element $0$, the difference of two elements, if it exists, must be unique.  To see this, let $c$ and $d$ be two differences of $a$ and $b$.  Then
\begin{itemize}
\item $b\cap c=b\cap d=0$, and
\item $a\cup b = b\cup c = b\cup d$.
\end{itemize}
So $c= c\cap (b\cup c)= c\cap (b\cup d)= (c \cap b)\cup (c\cap d)=0\cup (c\cap d)= c\cap d$.  Similarly, $d=d\cap c$.  As a result, $c = c\cap d=d\cap c=d$.
%%%%%
%%%%%
\end{document}
