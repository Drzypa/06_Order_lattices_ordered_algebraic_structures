\documentclass[12pt]{article}
\usepackage{pmmeta}
\pmcanonicalname{BoundedComplete}
\pmcreated{2013-03-22 17:01:08}
\pmmodified{2013-03-22 17:01:08}
\pmowner{CWoo}{3771}
\pmmodifier{CWoo}{3771}
\pmtitle{bounded complete}
\pmrecord{8}{39303}
\pmprivacy{1}
\pmauthor{CWoo}{3771}
\pmtype{Definition}
\pmcomment{trigger rebuild}
\pmclassification{msc}{06A12}
\pmclassification{msc}{06B23}
\pmclassification{msc}{03G10}
\pmrelated{CompletenessPrinciple}
\pmdefines{Dedekind complete}

\endmetadata

\usepackage{amssymb,amscd}
\usepackage{amsmath}
\usepackage{amsfonts}
\usepackage{mathrsfs}

% used for TeXing text within eps files
%\usepackage{psfrag}
% need this for including graphics (\includegraphics)
%\usepackage{graphicx}
% for neatly defining theorems and propositions
\usepackage{amsthm}
% making logically defined graphics
%%\usepackage{xypic}
\usepackage{pst-plot}
\usepackage{psfrag}

% define commands here
\newtheorem{prop}{Proposition}
\newtheorem{thm}{Theorem}
\newtheorem{ex}{Example}
\newcommand{\real}{\mathbb{R}}
\newcommand{\pdiff}[2]{\frac{\partial #1}{\partial #2}}
\newcommand{\mpdiff}[3]{\frac{\partial^#1 #2}{\partial #3^#1}}
\begin{document}
Let $P$ be a poset.  Recall that a subset $S$ of $P$ is called \emph{bounded from above} if there is an element $a\in P$ such that, for every $s\in S$, $s\le a$.

A poset $P$ is said to be \emph{bounded complete} if every subset which is bounded from above has a supremum.

\textbf{Remark}.  Since it is not required that the subset be non-empty, we see that $P$ has a bottom.  This is because the empty set is vacuously bounded from above, and therefore has a supremum.  However, this supremum is less than or equal to every member of $P$, and hence it is the least element of $P$. 

Clearly, any complete lattice is bounded complete.  An example of a non-complete bounded complete poset is any closed subset of $\mathbb{R}$ of the form $[a,\infty)$, where $a\in \mathbb{R}$.  In addition, arbitrary products of bounded complete posets is also bounded complete.  

It can be shown that a poset is a bounded complete dcpo iff it is a complete semilattice.

\textbf{Remark}.  A weaker concept is that of \emph{Dedekind completeness}: A poset $P$ is \emph{Dedekind complete} if every \emph{non-empty} subset bounded from above has a supremum.  An obvious example is $\mathbb{R}$, which is Dedekind complete but not bounded complete (as it has no bottom).  Dedekind completeness is more commonly known as the least upper bound property.
%%%%%
%%%%%
\end{document}
