\documentclass[12pt]{article}
\usepackage{pmmeta}
\pmcanonicalname{Poset}
\pmcreated{2013-03-22 11:43:41}
\pmmodified{2013-03-22 11:43:41}
\pmowner{mps}{409}
\pmmodifier{mps}{409}
\pmtitle{poset}
\pmrecord{22}{30132}
\pmprivacy{1}
\pmauthor{mps}{409}
\pmtype{Definition}
\pmcomment{trigger rebuild}
\pmclassification{msc}{06A99}
\pmsynonym{partially ordered set}{Poset}
%\pmkeywords{Relation}
%\pmkeywords{Partial Order}
%\pmkeywords{Set}
\pmrelated{Relation}
\pmrelated{PartialOrder}
\pmrelated{Semilattice}
\pmrelated{StarProduct}
\pmrelated{HasseDiagram}
\pmrelated{GreatestLowerBound}
\pmrelated{NetsAndClosuresOfSubspaces}
\pmrelated{OrderPreservingMap}
\pmrelated{DisjunctionPropertyOfWallman}
\pmdefines{comparable}
\pmdefines{incomparable}
\pmdefines{cover}
\pmdefines{covering}
\pmdefines{order-preserving function}
\pmdefines{monotone}
\pmdefines{monotonic}
\pmdefines{order morphism}
\pmdefines{morphism of posets}
\pmdefines{dual poset}
\pmdefines{consecutive}

\endmetadata

% this is the default PlanetMath preamble.  as your knowledge
% of TeX increases, you will probably want to edit this, but
% it should be fine as is for beginners.

% almost certainly you want these
\usepackage{amssymb}
\usepackage{amsmath}
\usepackage{amsfonts}

% used for TeXing text within eps files
%\usepackage{psfrag}
% need this for including graphics (\includegraphics)
%\usepackage{graphicx}
% for neatly defining theorems and propositions
\usepackage{amsthm}
% making logically defined graphics
%%%%%%%\usepackage{xypic}
\xyoption{all}

% there are many more packages, add them here as you need them

% define commands here
\newtheorem*{proposition*}{Proposition}
\theoremstyle{definition}
\newtheorem{example}{Example}
\begin{document}
\PMlinkescapeword{implies}
\PMlinkescapeword{structure}
\PMlinkescapeword{represents}
\PMlinkescapeword{satisfies}
\PMlinkescapeword{properties}
\PMlinkescapeword{property}
\PMlinkescapeword{associate}
\PMlinkescapeword{weaker}
\PMlinkescapeword{satisfy}
\PMlinkescapeword{equality}
\PMlinkescapeword{transitive closure}
\PMlinkescapeword{completes}
\PMlinkescapeword{graph}

A \emph{poset}, or \emph{partially ordered set}, consists of a set $P$
and a binary relation $\le$ on $P$ which satisfies the following
properties:
\begin{itemize}
\item
$\le$ is \PMlinkname{\emph{reflexive}}{Reflexive}, so $a\le a$ always
holds;

\item
$\le$ is \emph{antisymmetric}, so if $a\le b$ and $b\le a$ hold, then
$a=b$; and

\item
$\le$ is \PMlinkname{\emph{transitive}}{Transitive3}, so if $a\le b$
and $b\le c$ hold, then $a\le c$ also holds.

\end{itemize}

The relation $\le$ is called a \emph{partial order} on $P$.  In
practice, $(P,\le)$ is usually conflated with $P$; if a distinction is
needed, $P$ is called the \emph{ground set} or \emph{underlying set} of $(P,\le)$.  
The binary relation $<$ defined by removing the diagonal from $\le$ 
(i.e.\, $a<b$ iff $a\leq b$ and $a\neq b$) satisfies the following properties:
\begin{itemize}
\item
$<$ is \emph{irreflexive}, so if $a<b$ holds, then $b<a$ does not
hold; and

\item
$<$ is \emph{transitive}.
\end{itemize}
Since $\le$ is reflexive, it can be uniquely recovered from $<$ by adding 
the diagonal.  For this reason, an irreflexive and transitive binary
relation $<$ (called a \emph{strict partial order}) also defines a poset, by means
of the associated relation $\le$ described above (which is called \emph{weak partial order}).

Since every partial order is reflexive and transitive, every poset is
a preorder.  The notion of partial order is stricter than that of
preorder, Let $Q$ be the structure with ground set $Q=\{a,b\}$ and
binary relation $\preceq\, = \{(a,a),(a,b),(b,a),(b,b)\}$.  A diagram
of this structure, omitting loops, is displayed below.
\[\xymatrix{
b\ar@/^/[d] \\
a\ar@/^/[u]
}\]
Observe that the binary relation on $Q$ is reflexive and transitive,
so $Q$ is a preorder.  On the other hand, $a\preceq b$ and $b\preceq
a$, while $a\ne b$.  So the binary relation on $Q$ is not
antisymmetric, implying that $Q$ is not a poset.

Since every total order is reflexive, antisymmetric, and transitive,
every total order is a poset.  The notion of partial order is weaker
than that of total order.  A total order must obey the trichotomy law,
which states that for any $a$ and $b$ in the order, either $a\le b$ or
$b\le a$.  Let $P$ be the structure with ground set $\{a,b,c\}$ and
binary relation $\le\, = \{(a,a),(a,b),(a,c),(b,b),(c,c)\}$.  A
diagram of this structure, omitting loops, is displayed below.
\[\xymatrix{
b &                 & c \\
  & a\ar[ul]\ar[ur] &
}\]
Observe that the binary relation on $P$ is reflexive, antisymmetric,
and transitive, so $P$ is a poset.  On the other hand, neither $b\le
c$ nor $c\le b$ holds in $P$.  Thus $P$ fails to satisfy the
trichotomy law and is not a total order.

The failure of the trichotomy law for posets motivates the following
terminology.  Let $P$ be a poset.  If $a\le b$ or $b\le a$ holds in
$P$, we say that $a$ and $b$ are \emph{comparable}; otherwise, we say
they are \emph{incomparable}.  We use the notation $a\shortparallel b$
to indicate that $a$ and $b$ are incomparable.

If $(P,\le_P)$ and $(Q,\le_Q)$ are posets, then a function
$\varphi\colon P\to Q$ is said to be \emph{order-preserving}, or
\emph{monotone}, provided that it preserves inequalities.  That is,
$\varphi$ is order-preserving if whenever $a\le_P b$ holds, it follows
that $\varphi(a)\le_Q\varphi(b)$ also holds.  The identity function on
the ground set of a poset is order-preserving.  If $(P,\le_P)$,
$(Q,\le_Q)$, and $(R,\le_R)$ are posets and $\varphi\colon P\to Q$ and
$\psi\colon Q\to R$ are order-preserving functions, then the
composition $\psi\circ\varphi\colon P\to R$ is also order-preserving.

Posets together with order-preserving functions form a category, which
we denoted by $\mathbf{Poset}$.  Thus an order-preserving function
between the ground sets of two posets is sometimes also called a
\emph{morphism of posets}.  The category of posets has
\PMlinkname{arbitrary products}{ProductofPosets}.  Moreover, every
poset can itself be viewed as a category, and it turns out that a
morphism of posets is the same as a functor between the two posets.
% We discuss this in more detail below.

\section*{Examples of posets}

The two extreme posets are the chain, in which any two elements are
comparable, and the antichain, in which no two elements are
comparable.  A poset with a singleton underlying set is necessarily
both a chain and an antichain, but a poset with a larger underlying
set cannot be both.

\begin{example} 
Let $\mathbb{N}$ be the set of natural numbers.  Inductively define a
binary relation $\le$ on $\mathbb{N}$ by the following rules:
\begin{itemize} 
\item 
for any $n\in\mathbb{N}$, the relation $0\le n$ holds; and

\item
whenever $m\le n$, the relation $m+1\le n+1$ also holds.
\end{itemize}
Then $(\mathbb{N},\le)$ is a chain, hence a poset.  This structure can
be naturally embedded in the larger chains of the integers, the
rational numbers, and the real numbers.  
\end{example}

The next example shows that nontrivial antichains exist.

\begin{example}
Let $P$ be a set with cardinality greater than $1$.  Let $\le$ be the
diagonal of $P$.  Thus $\le$ represents equality, which is trivially a
partial order relation (which is also the intersection of all partial
orderings on $P$).  By construction, $a\le b$ in $P$ if and only
$a=b$.  Thus no two elements of $P$ are comparable.
\end{example}

So far the only posets we have seen are chains and antichains.  Most
posets are neither.  The following construction gives many such
examples.

\begin{example}
If $X$ is any set, the powerset $P=P(X)$ of $X$ is partially ordered
by inclusion, that is, by the relation $A\le B$ if and only if
$A\subseteq B$.
\end{example}

There are important structure theorems for posets concerning chains
and antichains.  One of the foundational results is Dilworth's
theorem.  This theorem was massively generalized by Greene and
Kleitman.

A final example shows that one can manufacture a poset from an existing one.

\begin{example}
Let $P$ be a poset ordered by $\le$.   The \emph{dual poset} of $P$ is defined as 
follows: it has the same underlying set as $P$, whose order is defined by $a\le'b$ 
iff $b\le a$.  It is easy to see that $\le'$ is a partial order.  The dual of $P$ 
is usually denoted by $P^{\partial}$.
\end{example}

\section*{Graph-theoretical view of posets}

Let $P$ be a poset with strict partial order $<$.  Then $P$ can be
viewed as a directed graph with vertex set the ground set of $P$ and
edge set $<$.  For example, the following diagram displays the Boolean
algebra $B_2$ as a directed graph.
\[\xymatrix{
             & \{0,1\}                        &              \\
\{0\}\ar[ur] &                                & \{1\}\ar[ul] \\
             & \emptyset\ar[ul]\ar[uu]\ar[ur] &
}\]
If $P$ is a sufficiently complicated poset, then drawing all of the
edges of $P$ can obscure rather than reveal the structure of $P$.  For
this reason it is convenient to restrict attention to a subrelation of
$<$ from which $<$ can be uniquely recovered.  

We describe a method of constructing a canonical subgraph of $P$ from
which the partial order can be recovered as long as every interval of
$P$ has finite height.  If $a$ and $b$ are elements of $P$, then we
say that $b$ \emph{covers} $a$ if $a<b$ and there are no elements of
$P$ strictly larger than $a$ but strictly smaller than $b$, that is, if
$[a,b]=\{a,b\}$.  Two elements are said to be \emph{consecutive} if 
one covers another.  Define a binary relation $<:$ on $P$ by
\[
a <: b \text{\ if and only if $b$ covers $a$.}
\]
By construction, the binary relation $<:$ is a subset of $<$.  Since
$<$ is transitive, the \PMlinkname{transitive
closure}{ClosureOfASetViaRelations} of $<:$ is also contained in $<$.

\begin{proposition*}
Suppose every interval of $P$ has finite height.  Then $<$ is the
transitive closure of $<:$.
\end{proposition*}

\begin{proof}
We prove this by induction on height.  By definition of $<:$, if $a<b$
and the height of $[a,b]$ is 1, then $a<:b$.

Assume for induction that whenever $a<b$ and the height of $[a,b]$ is
at most $n$, then $(a,b)$ is in the transitive closure of $<:$.
Suppose that $a<b$ and that the height of $[a,b]$ is $n+1$.  Since
every chain in $[a,b]$ is finite, it contains an element $c$ which is
strictly larger than $a$ and \PMlinkname{minimal}{MaximalElement} with
respect to this property.  Therefore $[a,c]=\{a,c\}$, from which we
conclude that $a<:c$.  Since the interval $[c,b]$ is a proper
subinterval of $[a,b]$, it has height at most $n$, so by the induction
assumption we conclude that $(c,b)$ is in the transitive closure of
$<:$.  Since $(a,c)$ and $(c,b)$ are in the transitive closure of
$<:$, so is $(a,b)$.  Hence whenever $a<b$ and the height of $[a,b]$
is at most $n+1$, then $(a,b)$ is in the transitive closure of $<:$.

This completes the proof.
\end{proof}

In the same way we associated a graph to $<$ we can associate a graph
to $<:$.  The graph is usually called the Hasse diagram of the poset.
Below we display the graph associated to the cover relation $<:$ of
$B_2$.
\[\xymatrix{
             & \{0,1\}                 &              \\
\{0\}\ar[ur] &                         & \{1\}\ar[ul] \\
             & \emptyset\ar[ul]\ar[ur] &              
}\]
For simplicity, the Hasse diagram of a poset is usually drawn as an
undirected graph.  Elements which are higher in the partial order
relation are drawn physically higher.  Since a strict partial order is
acyclic, this can be done uniquely and the partial order can be
recovered from the drawing.

\begin{thebibliography}{1}
\bibitem{Gr1998}
Gr\"atzer, G., \emph{General lattice theory}, 2nd ed., Birkh\"auser, 1998.

\bibitem{St1996}
Stanley, R., \emph{Enumerative Combinatorics}, vol. 1, 2nd ed., Cambridge
University Press, Cambridge, 1996.
\end{thebibliography}

%%%%%
%%%%%
%%%%%
%%%%%
%%%%%
%%%%%
%%%%%
\end{document}
