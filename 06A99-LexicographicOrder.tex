\documentclass[12pt]{article}
\usepackage{pmmeta}
\pmcanonicalname{LexicographicOrder}
\pmcreated{2013-03-22 15:14:05}
\pmmodified{2013-03-22 15:14:05}
\pmowner{matte}{1858}
\pmmodifier{matte}{1858}
\pmtitle{lexicographic order}
\pmrecord{13}{37005}
\pmprivacy{1}
\pmauthor{matte}{1858}
\pmtype{Definition}
\pmcomment{trigger rebuild}
\pmclassification{msc}{06A99}
\pmdefines{dictionary order}

% this is the default PlanetMath preamble.  as your knowledge
% of TeX increases, you will probably want to edit this, but
% it should be fine as is for beginners.

% almost certainly you want these
\usepackage{amssymb}
\usepackage{amsmath}
\usepackage{amsfonts}
\usepackage{amsthm}

\usepackage{mathrsfs}

% used for TeXing text within eps files
%\usepackage{psfrag}
% need this for including graphics (\includegraphics)
%\usepackage{graphicx}
% for neatly defining theorems and propositions
%
% making logically defined graphics
%%%\usepackage{xypic}

% there are many more packages, add them here as you need them

% define commands here

\newcommand{\sR}[0]{\mathbb{R}}
\newcommand{\sC}[0]{\mathbb{C}}
\newcommand{\sN}[0]{\mathbb{N}}
\newcommand{\sZ}[0]{\mathbb{Z}}

 \usepackage{bbm}
 \newcommand{\Z}{\mathbbmss{Z}}
 \newcommand{\C}{\mathbbmss{C}}
 \newcommand{\F}{\mathbbmss{F}}
 \newcommand{\R}{\mathbbmss{R}}
 \newcommand{\Q}{\mathbbmss{Q}}



\newcommand*{\norm}[1]{\lVert #1 \rVert}
\newcommand*{\abs}[1]{| #1 |}



\newtheorem{thm}{Theorem}
\newtheorem{defn}{Definition}
\newtheorem{prop}{Proposition}
\newtheorem{lemma}{Lemma}
\newtheorem{cor}{Corollary}

\newcommand{\wspace}[0]{'\ \ '}
\begin{document}
Let $A$ be a set equipped with a total order $<$, and let $A^n=A\times \cdots \times A$ be the $n$-fold Cartesian product of $A$.  Then the \emph{lexicographic order} $<$ on $A^n$ is defined as follows:
 
If $a=(a_1, \ldots, a_n)\in A^n$ and $b=(b_1, \ldots, b_n)\in A^n$, 
then $a<b$ if $a_1<b_1$ or
\begin{eqnarray*}
   a_1 &=& b_1, \\   
       &\vdots & \\
   a_k &=& b_k, \\
   a_{k+1} &<& b_{k+1} \\
\end{eqnarray*}
for some $k=1,\ldots, n-1$.

\subsubsection*{Examples}
\begin{itemize}
\item The lexicographic order yields a total order on the field of complex numbers.
\item The lexicographic order of words of finite length consisting of letters $\wspace$ (space) $<a<b<\cdots<y<z$ is the \emph{dictionary order}.  To compare words of different length, one simply pads the shorter with $\wspace$s from the right. For example, $\operatorname{prove} < \operatorname{proved} < \operatorname{proven}$.
\end{itemize}

\subsubsection*{Properties}
\begin{itemize}
\item 
The lexicographic order is a total order.
\item 
If the original set is well-ordered, the lexicographic ordering on the product is also a well-ordering.
\end{itemize}
%%%%%
%%%%%
\end{document}
