\documentclass[12pt]{article}
\usepackage{pmmeta}
\pmcanonicalname{RepresentingABooleanAlgebraByFieldOfSets}
\pmcreated{2013-03-22 19:08:27}
\pmmodified{2013-03-22 19:08:27}
\pmowner{CWoo}{3771}
\pmmodifier{CWoo}{3771}
\pmtitle{representing a Boolean algebra by field of sets}
\pmrecord{13}{42040}
\pmprivacy{1}
\pmauthor{CWoo}{3771}
\pmtype{Theorem}
\pmcomment{trigger rebuild}
\pmclassification{msc}{06E20}
\pmclassification{msc}{06E05}
\pmclassification{msc}{03G05}
\pmclassification{msc}{06B20}
\pmclassification{msc}{03G10}
\pmrelated{FieldOfSets}
\pmrelated{RepresentingADistributiveLatticeByRingOfSets}
\pmrelated{LatticeHomomorphism}
\pmrelated{RepresentingACompleteAtomicBooleanAlgebraByPowerSet}
\pmrelated{StoneRepresentationTheorem}
\pmrelated{MHStonesRepresentationTheorem}

\usepackage{amssymb,amscd}
\usepackage{amsmath}
\usepackage{amsfonts}
\usepackage{mathrsfs}

% used for TeXing text within eps files
%\usepackage{psfrag}
% need this for including graphics (\includegraphics)
%\usepackage{graphicx}
% for neatly defining theorems and propositions
\usepackage{amsthm}
% making logically defined graphics
%%\usepackage{xypic}
\usepackage{pst-plot}

% define commands here
\newcommand*{\abs}[1]{\left\lvert #1\right\rvert}
\newtheorem{prop}{Proposition}
\newtheorem{thm}{Theorem}
\newtheorem{ex}{Example}
\newcommand{\real}{\mathbb{R}}
\newcommand{\pdiff}[2]{\frac{\partial #1}{\partial #2}}
\newcommand{\mpdiff}[3]{\frac{\partial^#1 #2}{\partial #3^#1}}
\begin{document}
In this entry, we show that every Boolean algebra is isomorphic to a field of sets, originally noted by Stone in 1936.  The bulk of the proof has actually been carried out in \PMlinkname{this entry}{RepresentingADistributiveLatticeByRingOfSets}, which we briefly state:
\begin{quote}
if $L$ is a distributive lattice, and $X$ the set of all prime ideals of $L$, then the map $F:L\to P(X)$ defined by $F(a)=\lbrace P\mid a\notin P\rbrace$ is an embedding.
\end{quote}

Now, if $L$ is a Boolean lattice, then every element $a\in L$ has a complement $a'\in L$.  $a'$ is in fact uniquely determined by $a$.

\begin{prop} The embedding $F$ above preserves $'$ in the following sense: $$F(a')=X-F(a).$$ \end{prop}
\begin{proof} $P\in F(a')$ iff $a'\notin P$ iff $a\in P$ iff $P\notin F(a)$ iff $P\in X-F(a)$.  \end{proof}

\begin{thm} Every Boolean algebra is isomorphic to a field of sets. \end{thm}
\begin{proof} From what has been discussed so far, $F$ is a Boolean algebra isomorphism between $L$ and $F(L)$, which is a ring of sets first of all, and a field of sets, because $X-F(a)=F(a')$.  \end{proof}

\textbf{Remark}.  There are at least two other ways to characterize a Boolean algebra as a field of sets: let $L$ be a Boolean algebra:
\begin{itemize}
\item Every prime ideal is the kernel of a homomorphism into $\boldsymbol{2}:=\lbrace 0,1\rbrace$, and vice versa.  So for an element $a$ to be not in a prime ideal $P$ is the same as saying that $\phi(a)=1$ for some homomorphism $\phi:L \to \boldsymbol{2}$.  If we take $Y$ to be the set of all homomorphisms from $L$ to $\boldsymbol{2}$, and define $G:L\to P(Y)$ by $G(a)=\lbrace \phi \mid \phi(a)=1 \rbrace$, then it is easy to see that $G$ is an embedding of $L$ into $P(Y)$.
\item Every prime ideal is a maximal ideal, and vice versa.  Furthermore, $P$ is maximal iff $P'$ is an ultrafilter.  So if we define $Z$ to be the set of all ultrafilters of $L$, and set $H:L\to P(Z)$ by $H(a)=\lbrace U\mid a\in U\rbrace$, then it is easy to see that $H$ is an embedding of $L$ into $P(Z)$.
\end{itemize}
If we appropriately topologize the sets $X,Y$, or $Z$, then we have the content of the Stone representation theorem.
%%%%%
%%%%%
\end{document}
