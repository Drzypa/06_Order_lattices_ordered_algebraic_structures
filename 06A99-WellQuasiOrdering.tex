\documentclass[12pt]{article}
\usepackage{pmmeta}
\pmcanonicalname{WellQuasiOrdering}
\pmcreated{2013-03-22 13:54:15}
\pmmodified{2013-03-22 13:54:15}
\pmowner{CWoo}{3771}
\pmmodifier{CWoo}{3771}
\pmtitle{well quasi ordering}
\pmrecord{13}{34653}
\pmprivacy{1}
\pmauthor{CWoo}{3771}
\pmtype{Definition}
\pmcomment{trigger rebuild}
\pmclassification{msc}{06A99}
\pmclassification{msc}{06A07}
\pmsynonym{well quasi order}{WellQuasiOrdering}
\pmsynonym{wqo}{WellQuasiOrdering}
%\pmkeywords{Order}
%\pmkeywords{Orderings}
%\pmkeywords{QuasiOrdering}
%\pmkeywords{PartialOrderings}
%\pmkeywords{WellQuasiOrdering}
\pmrelated{quasiOrder}
\pmrelated{QuasiOrder}

\endmetadata

% this is the default PlanetMath preamble.  as your knowledge
% of TeX increases, you will probably want to edit this, but
% it should be fine as is for beginners.

% almost certainly you want these
\usepackage{amssymb}
\usepackage{amsmath}
\usepackage{amsfonts}

% used for TeXing text within eps files
%\usepackage{psfrag}
% need this for including graphics (\includegraphics)
%\usepackage{graphicx}
% for neatly defining theorems and propositions
\usepackage{amsthm}
% making logically defined graphics
%%%\usepackage{xypic}

% there are many more packages, add them here as you need them

% define commands here
\newtheorem{prop}{Proposition}
\newcommand{\NN}{{\mathbf N}}
\begin{document}
Let $Q$ be a set and $\precsim$ a quasi-order on $Q$.  An
infinite sequence in $Q$ is referred to as \emph{bad} if for
all $i<j \in \NN$, $a_i \not\precsim a_j$ holds; otherwise it is called
\emph{good}. Note that an antichain is obviously a bad sequence. 

A quasi-ordering $\precsim$ on $Q$ is a \emph{well-quasi-ordering} (\emph{wqo}) if for every infinite sequence is good.  Every well-ordering is a well-quasi-ordering.

The following proposition gives equivalent definitions for
well-quasi-ordering:
\begin{prop}
Given a set $Q$ and a binary relation $\precsim$ over $Q$, the
following conditions are equivalent:
\begin{itemize}
  \item $(Q, \precsim)$ is a well-quasi-ordering;
  \item $(Q, \precsim)$ has no infinite ($\omega$-) strictly decreasing chains
and no infinite antichains.
  \item Every linear extension of $Q/_\approx$ is a well-order, where $\approx$ is the equivalence relation and $Q/_\approx$ is the set of equivalence classes induced by $\approx$.
  \item Any infinite ($\omega$-) $Q$-sequence contains an
increasing chain.
\end{itemize}
\end{prop}

The equivalence of WQO to the second and the fourth conditions is proved by the infinite version of Ramsey's theorem.
%%%%%
%%%%%
\end{document}
