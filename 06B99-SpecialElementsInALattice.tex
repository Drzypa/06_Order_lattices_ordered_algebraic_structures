\documentclass[12pt]{article}
\usepackage{pmmeta}
\pmcanonicalname{SpecialElementsInALattice}
\pmcreated{2013-03-22 16:42:29}
\pmmodified{2013-03-22 16:42:29}
\pmowner{CWoo}{3771}
\pmmodifier{CWoo}{3771}
\pmtitle{special elements in a lattice}
\pmrecord{6}{38923}
\pmprivacy{1}
\pmauthor{CWoo}{3771}
\pmtype{Definition}
\pmcomment{trigger rebuild}
\pmclassification{msc}{06B99}
%\pmkeywords{distributive}
%\pmkeywords{standard}
%\pmkeywords{neutral}
\pmdefines{distributive element}
\pmdefines{standard element}
\pmdefines{neutral element}
\pmdefines{dually distributive}
\pmdefines{dually standard}

\usepackage{amssymb,amscd}
\usepackage{amsmath}
\usepackage{amsfonts}

% used for TeXing text within eps files
%\usepackage{psfrag}
% need this for including graphics (\includegraphics)
%\usepackage{graphicx}
% for neatly defining theorems and propositions
\usepackage{amsthm}
% making logically defined graphics
%%\usepackage{xypic}
\usepackage{pst-plot}
\usepackage{psfrag}

% define commands here

\begin{document}
Let $L$ be a lattice and $a\in L$ is said to be
\begin{itemize}
\item \emph{distributive} if $a\vee (b\wedge c)=(a\vee b)\wedge (a\vee c)$,
\item \emph{standard} if $b\wedge (a\vee c)=(b\wedge a)\vee (b\wedge c)$, or
\item \emph{neutral} if $(a\wedge b)\vee (b\wedge c)\vee (c\wedge a) = (a\vee b)\wedge (b\vee c)\wedge (c\vee a)$
\end{itemize}
for all $b,c\in L$.  There are also dual notions of the three types mentioned above, simply by exchanging $\vee$ and $\wedge$ in the definitions.  So a \emph{dually distributive} element $a\in L$ is one where $a\wedge (b\vee c)=(a\wedge b)\vee (a\wedge c)$ for all $b,c\in L$, and a \emph{dually standard element} is similarly defined.  However, a \emph{dually neutral} element is the same as a neutral element.

\textbf{Remarks}  For any $a\in L$, suppose $P$ is the property in $L$ such that $a\in P$ iff $a\vee b=a\vee c$ and $a\wedge b=a\wedge c$ imply $b=c$ for all $b,c\in L$.
\begin{itemize}
\item A standard element is distributive.  Conversely, a distributive satisfying  $P$ is standard.
\item A neutral element is distributive (and consequently dually distributive).  Conversely, a distributive and dually distributive element that satisfies $P$ is neutral.
\end{itemize}

\begin{thebibliography}{8}
\bibitem{gb} G. Birkhoff {\em Lattice Theory}, 3rd Edition, AMS Volume XXV, (1967).
\bibitem{gg} G. Gr\"{a}tzer, {\em General Lattice Theory}, 2nd Edition, Birkh\"{a}user (1998).
\end{thebibliography}
%%%%%
%%%%%
\end{document}
