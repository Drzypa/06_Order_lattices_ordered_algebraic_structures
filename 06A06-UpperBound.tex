\documentclass[12pt]{article}
\usepackage{pmmeta}
\pmcanonicalname{UpperBound}
\pmcreated{2013-03-22 11:52:15}
\pmmodified{2013-03-22 11:52:15}
\pmowner{djao}{24}
\pmmodifier{djao}{24}
\pmtitle{upper bound}
\pmrecord{9}{30450}
\pmprivacy{1}
\pmauthor{djao}{24}
\pmtype{Definition}
\pmcomment{trigger rebuild}
\pmclassification{msc}{06A06}
\pmclassification{msc}{11A07}
\pmdefines{bound}
\pmdefines{lower bound}
\pmdefines{bounded}
\pmdefines{bounded from above}
\pmdefines{bounded from below}

\endmetadata

\usepackage{amssymb}
\usepackage{amsmath}
\usepackage{amsfonts}
\usepackage{graphicx}
%%%%\usepackage{xypic}
\begin{document}
Let $S$ be a set with a partial ordering $\leq$, and let $T$ be a subset of $S$. 
An {\em upper bound} for $T$ is an element $z \in S$ such that $x \leq z$ for all $x \in T$. We say that $T$ is {\em bounded from above} if there exists an upper bound for $T$.

{\em Lower bound}, and \emph{bounded from below} are defined in a similar manner.
%%%%%
%%%%%
%%%%%
%%%%%
\end{document}
