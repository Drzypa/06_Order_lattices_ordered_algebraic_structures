\documentclass[12pt]{article}
\usepackage{pmmeta}
\pmcanonicalname{ExampleOfBooleanAlgebras}
\pmcreated{2013-03-22 17:52:33}
\pmmodified{2013-03-22 17:52:33}
\pmowner{CWoo}{3771}
\pmmodifier{CWoo}{3771}
\pmtitle{example of Boolean algebras}
\pmrecord{15}{40356}
\pmprivacy{1}
\pmauthor{CWoo}{3771}
\pmtype{Example}
\pmcomment{trigger rebuild}
\pmclassification{msc}{06B20}
\pmclassification{msc}{03G05}
\pmclassification{msc}{06E05}
\pmclassification{msc}{03G10}

\endmetadata

\usepackage{amssymb,amscd}
\usepackage{amsmath}
\usepackage{amsfonts}
\usepackage{mathrsfs}

% used for TeXing text within eps files
%\usepackage{psfrag}
% need this for including graphics (\includegraphics)
%\usepackage{graphicx}
% for neatly defining theorems and propositions
\usepackage{amsthm}
% making logically defined graphics
%%\usepackage{xypic}
\usepackage{pst-plot}

% define commands here
\newcommand*{\abs}[1]{\left\lvert #1\right\rvert}
\newtheorem{prop}{Proposition}
\newtheorem{thm}{Theorem}
\newtheorem{ex}{Example}
\newcommand{\real}{\mathbb{R}}
\newcommand{\pdiff}[2]{\frac{\partial #1}{\partial #2}}
\newcommand{\mpdiff}[3]{\frac{\partial^#1 #2}{\partial #3^#1}}
\begin{document}
Below is a list of examples of Boolean algebras.  Note that the phrase ``usual set-theoretic operations'' refers to the operations of union $\cup$, intersection $\cap$, and set complement $'$.

\begin{enumerate}
\item Let $A$ be a set.  The power set $P(A)$ of $A$, or the collection of all the subsets of $A$, together with the operations of union, intersection, and set complement, the empty set $\varnothing$ and $A$, is a Boolean algebra.  This is the canonical example of a Boolean algebra.
\item In $P(A)$, let $F(A)$ be the collection of all finite subsets of $A$, and $cF(A)$ the collection of all cofinite subsets of $A$.  Then $F(A)\cup cF(A)$ is a Boolean algebra.
\item More generally, any field of sets is a Boolean algebra.  In particular, any sigma algebra $\sigma$ in a set is a Boolean algebra.
\item (product of algebras) Let $A$ and $B$ be Boolean algebras.  Then $A\times B$ is a Boolean algebra, where 
\begin{eqnarray}
(a,b)\vee (c,d)&:=&(a\vee c, b\vee d),\\ 
(a,b)\wedge (c,d)&:=& (a\wedge c,b\wedge d),\\
(a,b)'&:=&(a',b').
\end{eqnarray}
\item More generally, if we have a collection of Boolean algebras $A_i$, indexed by a set $I$, then $\prod_{i\in I} A_i$ is a Boolean algebra, where the Boolean operations are defined componentwise.  
\item In particular, if $A$ is a Boolean algebra, then set of functions from some non-empty set $I$ to $A$ is also a Boolean algebra, since $A^I=\prod_{i\in I} A$.
\item (subalgebras) Let $A$ be a Boolean algebra, any subset $B\subseteq A$ such that $0\in B$, $a'\in B$ whenever $a\in B$, and $a\vee b\in B$ whenever $a,b\in B$ is a Boolean algebra.  It is called a \emph{Boolean subalgebra} of $A$.  In particular, the homomorphic image of a Boolean algebra homomorphism is a Boolean algebra.
\item (quotient algebras) Let $A$ be a Boolean algebra and $I$ a Boolean ideal in $A$.  View $A$ as a Boolean ring and $I$ an ideal in $A$.  Then the quotient ring $A/I$ is Boolean, and hence a Boolean algebra.
\item Let $A$ be a set, and $R_n(A)$ be the set of all $n$-ary relations on $A$.  Then $R_n(A)$ is a Boolean algebra under the usual set-theoretic operations.  The easiest way to see this is to realize that $R_n(A)=P(A^n)$, the powerset of the $n$-fold power of $A$.
\item The set of all clopen sets in a topological space is a Boolean algebra.
\item Let $X$ be a topological space and $A$ be the collection of all regularly open sets in $X$.  Then $A$ has a Boolean algebraic structure.  The meet and the constant operations follow the usual set-theoretic ones: $U\wedge V=U\cap V$, $0=\varnothing$ and $1=X$.  However, the join $\wedge$ and the complementation $'$ on $A$ are different.  Instead, they are given by
\begin{eqnarray}
U'&:=&X-\overline{U},\\
U\vee V&:=& (U\cup V)''.
\end{eqnarray}
\end{enumerate}

%%%%%
%%%%%
\end{document}
