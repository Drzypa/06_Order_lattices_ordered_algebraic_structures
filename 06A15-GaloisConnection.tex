\documentclass[12pt]{article}
\usepackage{pmmeta}
\pmcanonicalname{GaloisConnection}
\pmcreated{2013-03-22 15:08:15}
\pmmodified{2013-03-22 15:08:15}
\pmowner{CWoo}{3771}
\pmmodifier{CWoo}{3771}
\pmtitle{Galois connection}
\pmrecord{13}{36881}
\pmprivacy{1}
\pmauthor{CWoo}{3771}
\pmtype{Definition}
\pmcomment{trigger rebuild}
\pmclassification{msc}{06A15}
\pmsynonym{Galois correspondence}{GaloisConnection}
\pmsynonym{Galois connexion}{GaloisConnection}
\pmrelated{InteriorAxioms}
\pmrelated{AdjointFunctor}
\pmdefines{upper adjoint}
\pmdefines{lower adjoint}

% this is the default PlanetMath preamble.  as your knowledge
% of TeX increases, you will probably want to edit this, but
% it should be fine as is for beginners.

% almost certainly you want these
\usepackage{amssymb,amscd}
\usepackage{amsmath}
\usepackage{amsfonts}

% used for TeXing text within eps files
%\usepackage{psfrag}
% need this for including graphics (\includegraphics)
%\usepackage{graphicx}
% for neatly defining theorems and propositions
%\usepackage{amsthm}
% making logically defined graphics
%%%\usepackage{xypic}

% there are many more packages, add them here as you need them

% define commands here
\begin{document}
\PMlinkescapeword{root}

The notion of a Galois connection has its root in Galois theory.  By the \PMlinkname{fundamental theorem of Galois theory}{FundamentalTheoremOfGaloisTheory}, there is a one-to-one correspondence between the intermediate fields between a field $L$ and its subfield $F$ (with appropriate conditions imposed on the extension $L/F$), and the subgroups of the Galois group $\operatorname{Gal}(L/F)$ such that the bijection is inclusion-reversing:
$$\operatorname{Gal}(L/F)\supseteq H\supseteq\langle e \rangle\quad  \mbox{ iff }\quad F\subseteq L^H\subseteq L,\mbox{ and}$$ 
$$F\subseteq K\subseteq L\quad\mbox{ iff }\quad \operatorname{Gal}(L/F)\supseteq \operatorname{Gal}(L/K)\supseteq\langle e \rangle.$$

If the language of Galois theory is distilled from the above paragraph, what remains reduces to a more basic and general concept in the theory of ordered-sets:

\textbf{Definition}.  Let $(P, \le_P)$ and $(Q, \le_Q)$ be two posets. A \emph{Galois connection} between $(P,\le_P)$ and $(Q,\le_Q)$ is a pair of functions $f:=(f^*,f_*)$ with $f^*\colon P\to Q$ and $f_*\colon Q\to P$, such that, for all $p\in P$ and $q\in Q$, we have $$f^*(p)\leq_Q q\quad \mbox{ iff }\quad p\leq_P f_*(q).$$ We denote a Galois
connection between $P$ and $Q$ by $P\stackrel{f}{\multimap}Q$, or simply $P\multimap Q$.

If we define $\le_P^{\prime}$ on $P$ by $a\le_P^{\prime}b$ iff $b\le_P a$, and define $\le_Q^{\prime}$ on $Q$ by $c\le_Q^{\prime}d$ iff $d\le_Q c$, then $(P,\le_P^{\prime})$ and $(Q,\le_Q^{\prime})$ are posets, (the duals of $(P,\le_P)$ and $(Q,\le_Q)$).  The existence of a Galois connection between $(P,\le_P)$ and $(Q,\le_Q)$ is the same as the existence of a Galois connection between $(Q,\le_Q^{\prime})$ and $(P,\le_P^{\prime})$.  In short, we say that there is a Galois connection between $P$ and $Q$ if there is a Galois connection between two posets $S$ and $T$ where $P$ and $Q$ are the underlying sets (of $S$ and $T$ respectively).  With this, we may say without confusion that ``a Galois connection exists between $P$ and $Q$ iff a Galois connection exists between $Q$ and $P$''.

\textbf{Remarks.}
\begin{enumerate}
\item Since $f^*(p)\leq_Q f^*(p)$ for all $p\in P$, then by definition, $p\leq_P f_*f^*(p)$.  Alternatively, we can write \begin{eqnarray}1_P\leq_P f_*f^*, \end{eqnarray} where $1_P$ stands for the identity map on $P$. Similarly, if $1_Q$ is the identity map on $Q$, then \begin{eqnarray}f^*f_*\leq_Q 1_Q. \end{eqnarray}
\item Suppose $a\leq_P b$.  Since $b\leq_P f_*f^*(b)$ by the remark above, $a\leq_P f_*f^*(b)$ and so by definition, $f^*(a)\leq_Q f^*(b)$. This shows that $f^*$ is monotone.  Likewise, $f_*$ is also monotone.
\item Now back to Inequality (1), $1_P\leq_P f_*f^*$ in the first remark.  Applying the second remark, we obtain
\begin{eqnarray}f^*\leq_Q f^*f_*f^*. \end{eqnarray} Next, according to Inequality (2), $f^*f_*(q)\leq_Q q$ for any $q\in Q$, it is true, in particular, when $q=f^*(p)$. Therefore, we also have
\begin{eqnarray}f^*f_*f^*\leq_Q f^*.\end{eqnarray}  Putting Inequalities (3) and (4) together we have
\begin{eqnarray}f^*f_*f^*=f^*.\end{eqnarray}  Similarly, \begin{eqnarray}f_*f^*f_*=f_*.\end{eqnarray}
\item If $(f,g)$ and $(f,h)$ are Galois connections between $(P,\le_P)$ and $(Q,\le_Q)$, then $g=h$.  To see this, observe that $p\le_P g(q)$ iff $f(p)\le_Q q$ iff $p \le_P h(q)$, for any $p\in P$ and $q\in Q$.  In particular, setting $p=g(q)$, we get $g(q)\le_P h(q)$ since $g(q)\le_P g(q)$.  Similarly, $h(q)\le_P g(q)$, and therefore $g=h$.  By a similarly argument, if $(g,f)$ and $(h,f)$ are Galois connections between $(P,\le_P)$ and $(Q,\le_Q)$, then $g=h$.  Because of this uniqueness property, in a Galois connection $f=(f^*,f_*)$, $f^*$ is called \emph{the upper adjoint} of $f_*$ and $f_*$ \emph{the lower adjoint} of $f^*$.
\end{enumerate}

\textbf{Examples.}
\begin{itemize}
\item The most famous example is already mentioned in the first paragraph above: let $L$ is a finite-dimensional Galois extension of a field $F$, and $G:=\operatorname{Gal}(L/F)$ is the Galois group of $L$ over $F$.  If we define
\begin{itemize}
\item[a.] $P=\lbrace K\mid K\mbox{ is a field such that }F\subseteq K\subseteq L \rbrace,$ with $\leq_P=\subseteq$,
\item[b.] $Q=\lbrace H\mid H\mbox{ is a subgroup of }G \rbrace,$ with $\leq_Q=\supseteq$,
\item[c.] $f^*:P\to Q$ by $f^*(K)=\operatorname{Gal}(L/K)$, and \item[d.] $f_*:Q\to P$ by $f_*(H)=L^H$, the fixed field of $H$ in $L$.
\end{itemize}
Then, by the fundamental theorem of Galois theory, $f^*$ and $f_*$ are bijections, and $(f^*,f_*)$ is a Galois connection between $P$ and $Q$.
\item Let $X$ be a topological space.  Define $P$ be the set of all open subsets of $X$ and $Q$ the set of all closed subsets of $X$. Turn $P$ and $Q$ into posets with the usual set-theoretic inclusion. Next, define $f^*:P\to Q$ by $f^*(U)=\overline{U}$, the closure of $U$, and $f_*:Q\to P$ by $f_*(V)=\operatorname{int}(V)$, the interior of $V$.  Then $(f^*,f_*)$ is a Galois connection between $P$ and $Q$.  Incidentally, those elements fixed by $f_*f^*$ are precisely the regular open sets of $X$, and those fixed by $f^*f_*$ are the regular closed sets.
\end{itemize}

\textbf{Remark}.  The pair of functions in a Galois connection are order preserving as shown above.  One may also define a Galois connection as a pair of maps $f^*:P\to Q$ and $f_*:Q\to P$ such that $f^*(p)\le_Q q$ iff $f_*(q)\le_P p$, so that the pair $f^*,f_*$ are order reversing.  In any case, the two definitions are equivalent in that one may go from one definition to another, (simply exchange $Q$ with $Q^{\partial}$, the \PMlinkname{dual}{DualPoset} of $Q$).

\begin{thebibliography}{6}
\bibitem{tsb} T.S. Blyth, {\em Lattices and Ordered Algebraic Structures}, Springer, New York (2005).
\bibitem{dp} B. A. Davey, H. A. Priestley, {\it Introduction to Lattices and Order}, 2nd Edition, Cambridge (2003)
\end{thebibliography}
%%%%%
%%%%%
\end{document}
