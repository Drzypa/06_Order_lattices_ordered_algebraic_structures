\documentclass[12pt]{article}
\usepackage{pmmeta}
\pmcanonicalname{DescendingChainCondition}
\pmcreated{2013-03-22 12:01:14}
\pmmodified{2013-03-22 12:01:14}
\pmowner{antizeus}{11}
\pmmodifier{antizeus}{11}
\pmtitle{descending chain condition}
\pmrecord{7}{30984}
\pmprivacy{1}
\pmauthor{antizeus}{11}
\pmtype{Definition}
\pmcomment{trigger rebuild}
\pmclassification{msc}{06A99}
\pmsynonym{DCC}{DescendingChainCondition}

\endmetadata

\usepackage{amssymb}
\usepackage{amsmath}
\usepackage{amsfonts}
\usepackage{graphicx}
%%%\usepackage{xypic}
\begin{document}
A partially ordered set $S$ (for example, a collection of subsets of a set $X$, ordered by inclusion) satisfies the {\it descending chain condition} or {\it DCC} if there does not exist an infinite descending chain $s_1 > s_2 > \cdots$ of elements of $S$.

See also the ascending chain condition (ACC).
%%%%%
%%%%%
%%%%%
\end{document}
