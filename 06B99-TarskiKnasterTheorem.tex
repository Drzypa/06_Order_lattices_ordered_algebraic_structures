\documentclass[12pt]{article}
\usepackage{pmmeta}
\pmcanonicalname{TarskiKnasterTheorem}
\pmcreated{2013-03-22 15:30:19}
\pmmodified{2013-03-22 15:30:19}
\pmowner{kompik}{10588}
\pmmodifier{kompik}{10588}
\pmtitle{Tarski-Knaster theorem}
\pmrecord{18}{37366}
\pmprivacy{1}
\pmauthor{kompik}{10588}
\pmtype{Theorem}
\pmcomment{trigger rebuild}
\pmclassification{msc}{06B99}
\pmsynonym{Tarski theorem}{TarskiKnasterTheorem}
\pmsynonym{Knaster-Tarski theorem}{TarskiKnasterTheorem}
\pmrelated{lattice}
\pmrelated{completelattice}
\pmrelated{fixedpoint}
\pmrelated{ProofOfSchroederBernsteinTheoremUsingTarskiKnasterTheorem}
\pmrelated{Lattice}
\pmrelated{CompleteLattice}

% almost certainly you want these
\usepackage{amssymb}
\usepackage{amsmath}
\usepackage{amsfonts}
\usepackage{amsthm}

% used for TeXing text within eps files
%\usepackage{psfrag}
% need this for including graphics (\includegraphics)
%\usepackage{graphicx}
% for neatly defining theorems and propositions
\usepackage{amsthm}
% making logically defined graphics
%%%\usepackage{xypic}

% there are many more packages, add them here as you need them

% define commands here
\newtheorem{acknowledgement}{Acknowledgement}
\newtheorem{algorithm}{Algorithm}
\newtheorem{axiom}{Axiom}
\newtheorem{case}{Case}
\newtheorem{claim}{Claim}
\newtheorem{conclusion}{Conclusion}
\newtheorem{condition}{Condition}
\newtheorem{conjecture}{Conjecture}
\newtheorem{corollary}{Corollary}
\newtheorem{criterion}{Criterion}
\newtheorem{definition}{Definition}
\newtheorem{example}{Example}
\newtheorem{exercise}{Exercise}
\newtheorem{lemma}{Lemma}
\newtheorem{notation}{Notation}
\newtheorem{problem}{Problem}
\newtheorem{proposition}{Proposition}
\newtheorem{remark}{Remark}
\newtheorem{solution}{Solution}
\newtheorem{summary}{Summary}
\newtheorem{theorem}{Theorem}
\numberwithin{equation}{section}
\begin{document}
The aim of this article is to prove
\begin{theorem}
[LATTICE-THEORETICAL FIXPOINT THEOREM]Let

\begin{enumerate}
\item[(i)] $\mathfrak{A}=(A,\ \leq)$ \ be a complete lattice,

\item[(ii)] $f$ be an increasing function on $A$ to $A,$

\item[(iii)] $P$ the set of all fixpoints of $f.$
\end{enumerate}

Then the set $P$ is not empty and the system $(P,\ \leq)$ is a complete
lattice; in particular we have
\[
\cup P=\cup E_{x}[f(x)\geq x]\in P
\]
and
\[
\cap P=\cap E_{x}[f(x)\leq x]\in P.
\]

\end{theorem}

This article follows, with clarification, Tarski's original
1955 article \cite{tarski}, which may not be available to some readers.

\section{Definitions}

The symbol $\emptyset$ denotes the empty set. The symbol $\forall$ is
shorthand for "for all".

\begin{definition}
A \textbf{poset} (partially ordered set) $\mathfrak{A}=(A,\ \leq)$ consists of
a nonempty set $A$ and a relation
\[
\leq:\ A\times A\longrightarrow\{\text{False},\ \text{True}\}
\]
that is reflexive ($a\leq a$ $\forall a\in A$), antisymmetric ($\forall
\ a,\ b\in A,$ if $a\leq b$ and $b\leq a,$ then $a=b$). and transitive
\ ($\forall\ a,\ b,\ c\in A,$ if $a\leq b$ and $b\leq c,$ then $a\leq c$).
\end{definition}

\begin{definition}
If $\mathfrak{A}=(A,\ \leq)$ is a poset and $B$\ is a nonempty subset of $A,$
then $\sup(B)$ is an element of $A,$ if it exists, with these properties:

\begin{enumerate}
\item $\forall\ b\in B,~b\leq\sup(B)$, (in shorthand, $B\leq\sup(B)$).

\item If $a\in A$ and $B\leq a,$ then $\sup(B)\leq a.$
\end{enumerate}

Similarly $\inf(B)$ is an element of $A,$ if it exists, with these properties:

\begin{enumerate}
\item[(3)] $\inf(B)\leq B,$

\item[(4)] If $a\in A$ and $a\leq B,$ then $a\leq\inf(B).$
\end{enumerate}
\end{definition}

\begin{theorem}
If $\sup(B)$ exists, then it is unique; if $\inf(B)$ exists, it is unique.
\end{theorem}

\begin{definition}
Let $\mathfrak{A}=(A,\ \leq)$ be a poset. Then $\mathfrak{A}$ is a
\textbf{lattice} if $\forall\ a,\ b\in A,$ the set $\{a,\ b\}$ has a $\sup$
and an $\inf.$ We write
\[
a\cup b=\sup(\{a,\ b\})\quad\text{and}\quad a\cap b=\inf(\{a,\ b\}).
\]

\end{definition}

As an exercise, prove the following:

\begin{theorem}
Let $\mathfrak{A}=(A,\ \leq)$ be a lattice.\ Then

\begin{enumerate}
\item $a\cup b=b\cup a,\qquad a\cap b=b\cap a$

\item $(a\cup b)\cup c=a\cup(b\cup c),\qquad(a\cap b)\cap c=a\cap(b\cap c)$

\item $a\cup(b\cap c)\leq(a\cup b)\cap(a\cup c).\qquad(a\cap b)\cup(a\cap
c)\leq a\cap(b\cup c).$
\end{enumerate}
\end{theorem}

\begin{definition}
A poset $\mathfrak{A}=(A,\ \leq)$ is a \textbf{complete lattice} if for each
nonempty $B\subseteq A,$ both $\sup(B)$ and $\inf(B)$ exist.
\end{definition}

In particular, this is the case when $B=\{a,\ b\},$ so a complete lattice is a lattice.

\section{The Tarski-Knaster Theorem}

Statement of the theorem, as it is in \cite{tarski}:

\begin{theorem}
[LATTICE-THEORETICAL FIXPOINT THEOREM]Let

\begin{enumerate}
\item[(i)] $\mathfrak{A}=(A,\ \leq)$ \ be a complete lattice,

\item[(ii)] $f$ be an increasing function on $A$ to $A,$

\item[(iii)] $P$ the set of all fixpoints of $f.$
\end{enumerate}

Then the set $P$ is not empty and the system $(P,\ \leq)$ is a complete
lattice; in particular we have
\[
\cup P=\cup E_{x}[f(x)\geq x]\in P
\]
and
\[
\cap P=\cap E_{x}[f(x)\leq x]\in P.
\]

\end{theorem}

The symbols $\cup P$ and $\cap P$ are what I am calling $\sup(P)$ and
$\inf(C).$ Now let me restate the theorem:

\begin{theorem}
Let $\mathfrak{A}=(A,\ \leq)$ be a complete lattice. Let $f$ be an increasing
function on $A$ to $A,$ that is, whenever $a\leq b,$ then $f(a)\leq f(b).$ Let
$P$ be the set of all fixed points of $f:$
\[
P=\{\ a\in A\ |\ f(a)=a\ \}.
\]
Then

\begin{enumerate}
\item $P\neq\emptyset$.

\item The system $\mathfrak{P}=(P,$ $\leq)$ is a complete lattice.

\item Set $B=\{\ b\in A\ |\ b\leq f(b)\ \}$. Then $B\neq\emptyset$ and
$\sup(B)=\sup(P)\in P.$

\item Set $C=\{\ c\in A\ |\ f(c)\leq c\ \}$. Then $C\neq\emptyset\ $and
$\inf(C)=\inf(P)\in P.$
\end{enumerate}
\end{theorem}

\begin{lemma}
Let $\mathfrak{A}=(A,\ \leq)$ be a complete lattice and $\emptyset\neq
A_{0}\subseteq A.$ Let
\[
D=\{d\in A\ |\ d\leq A_{0}\},\qquad E=\{e\in A\ |\ A_{0}\leq e\}.
\]
($d\leq A_{o}$ means $d\leq a_{0}$ $\forall a_{0}\in A_{0}.$) Then both $B$
and $C$ are complete lattices, with $\leq$ \ inherited from that of $A.$
\end{lemma}

\begin{proof}
[Proof (of Lemma)]We'll do the case of $D;$ that for $E$ is similar (dual).
First, $D\neq\emptyset$ because $\inf(A)\in D.$ Suppose $\emptyset\neq
G\subseteq D.$ Obviously
\[
\inf(G)\leq g\leq A_{0}
\]
for any $g\in G,$ so $\inf(G)\in D.$ As $g\leq A_{0}$ for all $g\in G,$ we
have $\sup(G)\leq A_{0}$ by (2) in the definition of $\sup.$ So $\sup(G)\in
D.$ As $G$ was any nonempty subset of $D,$ these statements say that $D$ is a
complete lattice.
\end{proof}

\begin{proof}
[Proof (of Theorem)]$B\neq\emptyset$ because $\inf(A)\in B.$ Hence $b_{1}
=\sup(B)$ exists. If $b\in B,$ then $b\leq b_{1},$ which implies
\[
f(b)\leq f(b_{1}).
\]
because $f$ is increasing., so
\[
b\leq f(b)\leq f(b_{1}).
\]
This is true $\forall$ $b\in B.$ Hence
\[
b_{1}=\sup(B)\leq f(b_{1}).
\]
Therefore $b_{1}\in B.$ Because $f$ is increasing, this implies
\[
f(b_{1})\leq f(f(b_{1})),
\]
from which $f(b_{1})\in B.$ Hence $f(b_{1})\leq\sup(B)=b_{1}.$ From
\[
b_{1}\leq f(b_{1})\leq b_{1}
\]
we have $f(b_{1})=b_{1},$ that is, $b_{1}\in P.$ This settles (1):\ $P\neq
\emptyset,$ and incidentally proves $\sup(B)\in P.\smallskip$

From this,\ $\sup(B)\leq\sup(P).$ But obviously $P\subseteq B,$ so
$\sup(P)\leq\sup(B).$ Therefore $\sup(P)=\sup(B),$ completing the proof of
(3). Statement (4) is proved similarly (or dually).$\smallskip$

Next we prove (2). Let $\emptyset\neq P_{0}\subseteq P.$ Set
\[
Q=\{\ q\in A\ |\ q\leq P_{0}\ \}.
\]
By the Lemma, $Q$ is a complete lattice. If $q\in Q,$\ then $\forall\ p_{0}\in
P_{0},$ we have $q\leq p_{0},$ hence
\[
f(q)\leq f(p_{0})=p_{0}\ ,
\]
that is, $f(q)\in Q.$ In other words,
\[
f:\ Q\longrightarrow Q.
\]
By what we have already proved for $A,$ applied to $Q,$ we have $\sup(Q)\in
Q,$ and $\sup(Q)$ is a fixed point of $f.$ That is, $\sup(Q)\in P.$ But
obviously $\sup(Q)=\inf(P_{0}),$ so $\inf(P_{0})\in P$ Similarly $\sup
(P_{0})\in P,$ completing the proof of (2):\ $P$ is a complete lattice.
\end{proof}

This theorem was proved by A. Tarski \cite{tarski}. A special case of
this theorem (for lattices of sets) appeared in a paper of B. Knaster
\cite{knaster}. Kind of converse of this theorem was proved by Anne C.
Davis \cite{davis}: If every order-preserving function $f\colon L\to
L$ has a fixed point, then $L$ is a complete lattice.

\begin{thebibliography}{1}
\bibitem{davis}
Anne~C. Davis, \emph{{A characterization of complete lattices.}}, Pac. J. Math. \textbf{5} (1955), 311--319.

\bibitem{forster}
Thomas Forster, \emph{{Logic, induction and sets}}, {Cambridge University Press}, Cambridge, 2003.

\bibitem{knaster}
B.~Knaster, \emph{{Un th\'eor\`eme sur les fonctions d'ensembles.}}, Annales Soc. Polonaise \textbf{6} (1928), 133--134.

\bibitem{kolibiar}
M.~Kolibiar, A.~Leg\'e\v{n}, T.~\v{S}al\'at, and \v{S}. Zn\'am, \emph{Algebra a pr\'\i buzn\'e discipl\'\i ny}, Alfa, Bratislava, 1992 (Slovak).

\bibitem{nation}
J.~B. Nation: \PMlinkexternal{\emph{Notes on lattice theory}}{http://bigcheese.math.sc.edu/~mcnulty/alglatvar/}

\bibitem{tarski}
Alfred Tarski, \emph{{A lattice-theoretical fixpoint theorem and its applications.}}, Pac. J. Math. \textbf{5} (1955), 285--309.

\bibitem{wiki}
Wikipedia's entry on \PMlinkexternal{Knaster-Tarski theorem}{http://en.wikipedia.org/wiki/Knaster-Tarski_theorem}
\end{thebibliography}
%%%%%
%%%%%
\end{document}
