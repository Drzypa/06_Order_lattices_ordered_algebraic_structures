\documentclass[12pt]{article}
\usepackage{pmmeta}
\pmcanonicalname{RepresentingADistributiveLatticeByRingOfSets}
\pmcreated{2013-03-22 19:08:24}
\pmmodified{2013-03-22 19:08:24}
\pmowner{CWoo}{3771}
\pmmodifier{CWoo}{3771}
\pmtitle{representing a distributive lattice by ring of sets}
\pmrecord{7}{42039}
\pmprivacy{1}
\pmauthor{CWoo}{3771}
\pmtype{Theorem}
\pmcomment{trigger rebuild}
\pmclassification{msc}{06D99}
\pmclassification{msc}{06D05}
\pmrelated{RingOfSets}
\pmrelated{RepresentingABooleanLatticeByFieldOfSets}

\usepackage{amssymb,amscd}
\usepackage{amsmath}
\usepackage{amsfonts}
\usepackage{mathrsfs}

% used for TeXing text within eps files
%\usepackage{psfrag}
% need this for including graphics (\includegraphics)
%\usepackage{graphicx}
% for neatly defining theorems and propositions
\usepackage{amsthm}
% making logically defined graphics
%%\usepackage{xypic}
\usepackage{pst-plot}

% define commands here
\newcommand*{\abs}[1]{\left\lvert #1\right\rvert}
\newtheorem{prop}{Proposition}
\newtheorem{thm}{Theorem}
\newtheorem{lem}{Lemma}
\newtheorem{ex}{Example}
\newcommand{\real}{\mathbb{R}}
\newcommand{\pdiff}[2]{\frac{\partial #1}{\partial #2}}
\newcommand{\mpdiff}[3]{\frac{\partial^#1 #2}{\partial #3^#1}}
\begin{document}
In this entry, we present the proof of a fundamental fact that every distributive lattice is lattice isomorphic to a ring of sets, originally proved by Birkhoff and Stone in the 1930's.  The proof uses the \PMlinkname{prime ideal theorem of Birkhoff}{BirkhoffPrimeIdealTheorem}.  First, a simple results from the prime ideal theorem:

\begin{lem} Let $L$ be a distributive lattice and $a,b\in L$ with $a\ne b$.  Then there is a prime ideal containing one or the other. \end{lem}
\begin{proof}  Let $I=\langle a\rangle$ and $J=\langle b\rangle$, the principal ideals generated by $a,b$ respectively.  If $I=J$, then $b\le a$ and $a\le b$, or $a=b$, contradicting the assumption.  So $I\ne J$, which means either $a\notin J$ or $b\notin I$.  In either case, apply the prime ideal theorem to obtain a prime ideal containing $I$ (or $J$) not containing $b$ (or $a$).
\end{proof}

Before proving the theorem, we have one more concept to introduce:

\textbf{Definition}.  Let $L$ be a distributive lattice, and $X$ the set of all prime ideals of $L$.  Define $F: L\to P(X)$, the powerset of $X$, by $$F(a):=\lbrace P\mid a\notin P\rbrace.$$

\begin{prop} $F$ is an injection. \end{prop}
\begin{proof} If $a\ne b$, then by the lemma there is a prime ideal $P$ containing one but not another, say $a\in P$ and $b\notin P$.  Then $P\notin F(a)$ and $P\in F(b)$, so that $F(a)\ne F(b)$. \end{proof}

\begin{prop} $F$ is a lattice homomorphism. \end{prop}
\begin{proof} There are two things to show:
\begin{itemize}
\item $F$ preserves $\wedge$: If $P\in F(a\wedge b)$, then $a\wedge b\notin P$, so that $a\notin P$ and $b\notin P$, since $P$ is a sublattice.  So $P\in F(a)$ and $P\in F(b)$ as a result.  On the other hand, if $P\in F(a)\cap F(b)$, then $a\notin P$ and $b\notin P$.  Since $P$ is prime, $a\wedge b\notin P$, so that $P\in F(a\wedge b)$.  Therefore, $F(a\wedge b)=F(a)\cap F(b)$.
\item $F$ preserves $\vee$: If $P\in F(a\vee b)$, then $a\vee b\notin P$, which implies that $a\notin P$ or $b\notin P$, since $P$ is a sublattice of $L$.  So $P\in F(a)\cup F(b)$.  On the other hand, if $P\in F(a)\cup F(b)$, then $a\vee b\notin P$, since $P$ is a lattice ideal.  Hence $F(a\vee b)=F(a)\cup F(b)$.
\end{itemize}
Therefore, $F$ is a lattice homomorphism.
\end{proof}

The function $F$ is called the \emph{canonical embedding} of $L$ into $P(X)$.

\begin{thm} Every distributive lattice is isomorphic to a ring of sets.  \end{thm}
\begin{proof}  Let $L,X,F$ be as above.  Since $F:L \to P(X)$ is an embedding, $L$ is lattice isomorphic to $F(L)$, which is a ring of sets.  \end{proof}

\textbf{Remark}.  Using the result above, one can show that if $L$ is a Boolean algebra, then $L$ is isomorphic to a field of sets.  See link below for more detail.
%%%%%
%%%%%
\end{document}
