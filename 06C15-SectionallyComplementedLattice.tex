\documentclass[12pt]{article}
\usepackage{pmmeta}
\pmcanonicalname{SectionallyComplementedLattice}
\pmcreated{2013-03-22 17:58:46}
\pmmodified{2013-03-22 17:58:46}
\pmowner{CWoo}{3771}
\pmmodifier{CWoo}{3771}
\pmtitle{sectionally complemented lattice}
\pmrecord{6}{40488}
\pmprivacy{1}
\pmauthor{CWoo}{3771}
\pmtype{Definition}
\pmcomment{trigger rebuild}
\pmclassification{msc}{06C15}
\pmclassification{msc}{06B05}
\pmrelated{DifferenceOfLatticeElements}
\pmdefines{sectionally complemented}
\pmdefines{dually sectionally complemented lattice}

\endmetadata

\usepackage{amssymb,amscd}
\usepackage{amsmath}
\usepackage{amsfonts}
\usepackage{mathrsfs}

% used for TeXing text within eps files
%\usepackage{psfrag}
% need this for including graphics (\includegraphics)
%\usepackage{graphicx}
% for neatly defining theorems and propositions
\usepackage{amsthm}
% making logically defined graphics
%%\usepackage{xypic}
\usepackage{pst-plot}

% define commands here
\newcommand*{\abs}[1]{\left\lvert #1\right\rvert}
\newtheorem{prop}{Proposition}
\newtheorem{thm}{Theorem}
\newtheorem{ex}{Example}
\newcommand{\real}{\mathbb{R}}
\newcommand{\pdiff}[2]{\frac{\partial #1}{\partial #2}}
\newcommand{\mpdiff}[3]{\frac{\partial^#1 #2}{\partial #3^#1}}
\begin{document}
\begin{prop} Let $L$ be a lattice with the least element $0$.  Then the following are equivalent:
\begin{enumerate}
\item Every pair of elements have a \PMlinkname{difference}{DifferenceOfLatticeElements}.
\item for any $a\in L$, the lattice interval $[0,a]$ is a complemented lattice.
\end{enumerate}
\end{prop}
\begin{proof}
Suppose first that every pair of elements have a difference.  Let $b\in [0,a]$ and let $c$ be a difference between $a$ and $b$.  So $0=b\wedge c$ and $c\vee b=b\vee a=a$, since $b\le a$.  This shows that $c$ is a complement of $b$ in $[0,a]$.

Next suppose that $[0,a]$ is complemented for every $a\in L$.  Let $x,y\in L$ be any two elements in $L$.  Let $a=x\vee y$.  Since $[0,a]$ is complemented, $y$ has a complement, say $z\in [0,a]$.  This means that $y\wedge z=0$ and $y\vee z=a= x\vee y$.  Therefore, $z$ is a difference of $x$ and $y$.
\end{proof}

\textbf{Definition}.  A lattice $L$ with the least element $0$ satisfying either of the two equivalent conditions above is called a \emph{sectionally complemented lattice}.

Every relatively complemented lattice is sectionally complemented.  Every sectionally complemented distributive lattice is relatively complemented.

Dually, one defines a \emph{dually sectionally complemented lattice} to be a lattice $L$ with the top element $1$ such that for every $a\in L$, the interval $[a,1]$ is complemented, or, equivalently, the lattice dual $L^{\partial}$ is sectionally complemented.

\begin{thebibliography}{6}
\bibitem{gg} G. Gr\"atzer, {\it General Lattice Theory}, 2nd Edition, Birkh\"auser (1998)
\end{thebibliography}
%%%%%
%%%%%
\end{document}
