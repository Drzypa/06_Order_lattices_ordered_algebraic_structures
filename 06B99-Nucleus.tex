\documentclass[12pt]{article}
\usepackage{pmmeta}
\pmcanonicalname{Nucleus}
\pmcreated{2014-12-18 15:34:14}
\pmmodified{2014-12-18 15:34:14}
\pmowner{porton}{9363}
\pmmodifier{porton}{9363}
\pmtitle{Nucleus}
\pmrecord{1}{}
\pmprivacy{1}
\pmauthor{porton}{9363}
\pmtype{Definition}
\pmclassification{msc}{06B99}

% this is the default PlanetMath preamble.  as your knowledge
% of TeX increases, you will probably want to edit this, but
% it should be fine as is for beginners.

% almost certainly you want these
\usepackage{amssymb}
\usepackage{amsmath}
\usepackage{amsfonts}

% need this for including graphics (\includegraphics)
\usepackage{graphicx}
% for neatly defining theorems and propositions
\usepackage{amsthm}

% making logically defined graphics
%\usepackage{xypic}
% used for TeXing text within eps files
%\usepackage{psfrag}

% there are many more packages, add them here as you need them

% define commands here

\begin{document}
In order theory, a \emph{nucleus} is a 
function $F$ on a meet-semilattice $\mathfrak{A}$ such that (for every $p$ in 
$\mathfrak{A}$):

\begin{enumerate}
\item $p \le F(p)$
\item $F(F(p)) = F(p)$
\item $F(p \wedge q) = F(p) \wedge F(q)$
\end{enumerate}

Usually, the term \emph{nucleus}} is used in frames and locales theory (when the 
semilattice $\mathfrak{A}$ is a frame).

\section{Some well known results about nuclei}

\textbf{Proposition} If $F$ is a nucleus on a frame $\mathfrak{A}$, then the poset 
$\operatorname{Fix}(F)$ of fixed points of $F$, with order inherited from 
$\mathfrak{A}$, is also a frame.
\end{document}
