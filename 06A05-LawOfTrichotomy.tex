\documentclass[12pt]{article}
\usepackage{pmmeta}
\pmcanonicalname{LawOfTrichotomy}
\pmcreated{2013-03-22 14:13:46}
\pmmodified{2013-03-22 14:13:46}
\pmowner{yark}{2760}
\pmmodifier{yark}{2760}
\pmtitle{law of trichotomy}
\pmrecord{9}{35668}
\pmprivacy{1}
\pmauthor{yark}{2760}
\pmtype{Definition}
\pmcomment{trigger rebuild}
\pmclassification{msc}{06A05}
\pmclassification{msc}{03E20}
\pmdefines{trichotomy}
\pmdefines{trichotomous}

\endmetadata

\usepackage{amssymb}
\usepackage{amsmath}
\usepackage{amsfonts}
\begin{document}
\PMlinkescapeword{case}
\PMlinkescapeword{equivalent}
\PMlinkescapeword{property}
\PMlinkescapeword{restricted}

The \emph{law of trichotomy} for a binary relation $R$ on a set $S$ is the property that
\begin{itemize}
\item for all $x,y\in S$, exactly one of the following holds: $xRy$ or $yRx$ or $x=y$.
\end{itemize}
A binary relation satisfying the law of trichotomy is sometimes said to be \emph{trichotomous}.
Trichotomous binary relations are equivalent to tournaments,
although the study of tournaments is usually restricted to the finite case.

A transitive trichotomous binary relation is called a total order, and is typically written $<$.

The law of trichotomy for cardinal numbers is equivalent (in ZF) to the \PMlinkname{axiom of choice}{AxiomOfChoice}.
%%%%%
%%%%%
\end{document}
